\subsubsection{Orientado a columnas}
Este modelo de base de datos se originó con el BigTable de Google. Otras bases de datos orientados a columnas incluyen HBase, Hypertable y Cassandra. 


Cassandra comenzó como un proyecto en Facebook, pero Facebook lo lanzó a la comunidad de código abierto, que ha seguido desarrollando a Cassandra en una de las bases de datos orientadas a columnas más populares.




Afortunadamente, las bases de datos de la familia de columnas son conceptualmente simples y conceptualmente lo suficientemente cercanas al modelo relacional para que su comprensión del modelo relacional pueda ayudarlo a comprender el modelo familiar de columnas.


El término base de datos orientada a columnas puede referirse a dos conjuntos diferentes de tecnologías que a menudo se confunden entre sí. 


En cierto sentido, el término de base de datos orientada a columnas puede usarse para tecnologías de bases de datos relacionales tradicionales que usan almacenamiento enfocado en columnas en lugar de almacenamiento en filas. 


Por otro lado, en un sistema de base de datos NoSQL describe un tipo de base de datos que organiza datos en pares nombre-valor donde el nombre actua también como la clave.


\begin{figure}[h!t] 
    \centering
    \includegraphics[width=0.75\textwidth]{noSQL/columna.png}
    \caption{Familia de columnas}
    \label{img:documentos-columna}
\end{figure}


Cada par de clave-valor representa una columna y siempre contiene una fecha que sirve para resolver confiltos de escritura, datos expirados entre otras cosas.



Al ser una bases de datos NoSQL, no requiere que los datos se ajusten a estructuras predefinidas ni admite SQL para consultas.


