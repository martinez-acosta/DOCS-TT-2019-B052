\section{Conclusiones}


Se ha elegido Flask por ser un \textit{framework} web conocido por el equipo, porque es ideal para el prototipado rápido de proyectos y al equipo le ha dado resultados en proyectos anteriores.

De acuerdo con el sitio Stack Overflow\cite{noauthor_stack_nodate}, JavaScript es el lenguaje más popular de 2019 y aunque TypeScript es de los lenguajes que tienen un mayor nivel de aceptación, se usará JavaScript no solo por ser el lenguaje más popular y, en consecuencia, con más compatibilidad y material de ayuda, sino también porque el equipo está acostumbrado a este lenguaje y tiene experiencia con web \textit{frameworks} escritos en JavaScript.

De acuerdo con Mosquera\cite{martinez-mosquera_modeling_2020}, MongoDB es la base de datos NoSQL orientada a documentos más popular y usada en los \textit{papers} de investigación; por ello el equipo ha decidido usar MongoDB como base de datos.

Se llevó a la práctica en el prototipo funcional las tres opciones antes expuestas junto con algunas otras y se decidió usar GoJS para el proyecto por ser la biblioteca JavaScript más completa para diagramado y que genera los diagramas con un JSON simple para parsear esos datos y realizar la conversión.


Se ha elegido usar en una primera instancia Vuetify porque es un CSS \textit{framework} que está integrado en las tecnologías asociadas de Vue, como Vue Router, Vue Meta; asimismo, sus componentes son simples de entender y de implementar.


Tomando en cuenta la experiencia del equipo con CSS, además de que el proyecto no se enfocará en hacer muchas hojas de estilo para cada componente, página o vista de la aplicación web, se usará CSS en lugar de Sass.

Se usará Vue/Nuxt por ser el \textit{framework} con el que el equipo está más acostumbrado, además de ser el máx flexible de las opciones expuestas.

Se ha elegido Python para desarrollar los algoritmos del proyecto dado que es multiplataforma y es de fácil integración con el \textit{framework} web elegido para el \textit{back end}.
