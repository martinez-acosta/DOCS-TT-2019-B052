\section{Conclusiones}


Los modelos de datos que han sido presentados en este capítulo son el fundamento para diseñar bases de datos; tradicionalmente, el modelado de bases de datos relacionales empieza con el modelo entidad-relación (para el modelo conceptual), se transforma al modelo relacional para normalizar cada relación y por medio de SQL se obtiene el esquema de la base de datos. 


Sin embargo, este manera de diseñar bases de datos no es la más apropiada para el modelado NoSQL, porque como comenta Chebotko\cite{chebotko_big_2015}, es clave conocer desde el modelo conceptual cómo se realizarán las consultas; además, por la naturaleza de los modelos de datos NoSQL, operaciones como la normalización pierden su propósito, ya que en lugar de optimizar, perjudican el desempeño de las bases de datos NoSQL, porque la normalización es contraria a un concepto fundamental de los modelos de datos NoSQL: la agregación de datos anidados.

Asimismo, de las tecnologías expuestas para desarrollar la propuesta de solución se ha elegido Flask para el \textit{back end} por ser un \textit{framework} web conocido por el equipo, es ideal para el prototipado rápido de proyectos y el equipo tiene experiencia con este \textit{framework}.


Se ha elegido JavaScript porque es el lenguaje más popular de 2019, tiene compatibilidad, material de ayuda y también porque el equipo tiene experiencia con web \textit{frameworks} escritos en JavaScript.


Como ya se había mencionado, de acuerdo con Mosquera\cite{martinez-mosquera_modeling_2020}, MongoDB es la base de datos NoSQL orientada a documentos más popular y usada en los \textit{papers} de investigación; por ello el equipo ha decidido usar MongoDB como base de datos.


La elección de GoJS es por ser la biblioteca JavaScript más completa para diagramado de todas las que se probaron en el prototipo funcional.



La decisión de usar CSS es porque la propuesta de solución no está enfocada en diseñar estilos para los diferentes componentes de Vue; además, se ha decidido delegar el aspecto visual a un CSS \textit{framework}, Vuetify, que está integrado en las tecnologías asociadas de Vue.


Finalmente, se han elegido Vue/Nuxt por ser los \textit{frameworks} con los que el equipo tiene más experiencia y para implementar los algoritmos de la propuesta de solución se ha optado por usar Python, dado que es multiplataforma y es de fácil integración con el \textit{framework} web elegido para el \textit{back end}.
