\subsection{Ingeniería Dirigida por Modelos}

De acuerdo con Scherp \cite{scherp_framework_2013}, la ingeniería dirigida por modelos (en inglés \textit{Model-driven Software Development}) es el desarrollo de \textit{software} mediante modelos y transformaciones entre modelos con el objetivo de automatizar el mapeo entre modelos a código fuente.

Asimismo, de acuerdo con Reussner \cite{reussner_handbuch_2006}, los conceptos principales en la ingeniería dirigida por modelos son:

\begin{enumerate}
  \item \textbf{Modelo}: es una vista simplificada y abstracta de un sistema real. 
  \item \textbf{Metamodelo}: define elementos y reglas para generar modelos; consiste en una sintaxis abstracta, sintaxis concreta y una semántica.
  \item \textbf{Transformación entre modelos}: es un mapeo computable que toma como entrada una instancia de un tipo de modelo y como salida genera una instancia de otro tipo de modelo.
  \item \textbf{Lenguaje específico de dominio}: es un lenguaje definido por un metamodelo que contiene conceptos para un dominio específico.
\end{enumerate}