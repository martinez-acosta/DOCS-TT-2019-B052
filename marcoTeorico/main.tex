
Del estado del arte se ha concluido que el modelo entidad-relación es el modelo conceptual más usado para describir modelos de datos NoSQL; también es de notar que ninguna herramienta de las estudiadas ofrece validación estructural de su modelo conceptual, obtención del esquema relacional o de las sentencias SQL.


En este apartado se muestran los diferentes conceptos con los que el lector debe estar relacionado para comprender la solución que se propone y el resto del capítulo está organizado de la siguiente manera: primero se muestran los modelos de datos para bases de datos, empezando con el modelo entidad-relación, donde está cómo interactúan entre sí sus elementos (entidades, atributos y relaciones) para representar una base de datos; después se expone el modelo relacional y una descripción de SQL (su lenguaje de consultas); también se muestra de forma concisa y breve los cuatro modelos de datos NoSQL (clave-valor, orientado a columnas, orientado a documentos y orientado a grafos); se explica cada tecnología usada en la propuesta de solución y el porqué ha sido elegida; finalmente, se termina con las conclusiones del capítulo.


\section{Modelos de datos para bases de datos}
De acuerdo a la bibliografía de Elmasri \cite{ramez_elmasri_fundamentos_nodate}, un modelo de datos es una colección de conceptos que describen una estructura de una base de datos. 


Los modelos de datos de alto nivel o conceptuales ofrecen conceptos visuales simples que representan un modelo de datos, mientras que los modelos de datos de bajo nivel o físicos ofrecen conceptos que describen los detalles de cómo se implementa el almacenamiento de los datos en el sistema de la base de datos.


Los modelos de datos conceptuales utilizan conceptos como entidades, atributos y relaciones o, en el caso de ser bases de datos no relacionales, no hay un estándar definido para modelar conceptualmente este tipo de bases de datos.


Una entidad representa un objeto o concepto del mundo real, un atributo representa alguna propiedad que describe a una entidad y una relación es una asociación entre entidades.


A continuación se presentan varios de los modelos de datos existentes.

\subsection{Modelo entidad-relación}

De acuerdo a Elmasri \cite{ramez_elmasri_fundamentos_nodate}, el modelo entidad-relación, que fue creado y formalizado por Peter Chen en 1976\cite{chen_entity-relationship_nodate}, se utiliza con frecuencia para el diseño conceptual de las aplicaciones de base de datos. 


En esta sección se describen los conceptos básicos y las restricciones del modelo ER. 

\subsubsection{Entidades}
El objeto básico representado por el modelo ER es una entidad, que es una cosa del mundo real con una existencia independiente.


Una entidad puede ser un objeto con una existencia física (por ejemplo, una persona en particular, un coche, una casa o un empleado) o puede ser un objeto con una existencia conceptual (por ejemplo, una empresa, un trabajo o un curso universitario).

\subsubsection*{Tipo de entidad}

Un tipo de entidad define una colección (o conjunto) de entidades que tienen los mismos atributos.


La colección de todas las entidades de un tipo de entidad en particular de la base de datos se denomina conjunto de entidades; se usa el mismo nombre del tipo de entidad para hacer referencia al conjunto de entidades. 

\subsubsection*{Tipos de entidades débiles}

Los tipos de entidad que no tienen atributos clave propios se denominan tipos de entidad débiles. En contraposición, los tipos de entidad regulares que tienen un atributo clave se denominan tipos de entidad fuertes.


Las entidades que pertenecen a un tipo de entidad débil, se identifican como relacionadas con entidades específicas de otro tipo de entidad ( llamada entidad identificado o propietario) en combinación con uno de sus valores de atributo.


Este tipo de relación que relaciona un tipo de entidad débil con su propietario lo podemos llamar relación identificativa del tipo de entidad débil. 


Un tipo de entidad débil siempre tiene una restricción de participación total (dependencia de existencia) respecto a su relación identificativa, porque una entidad débil no puede identificarse sin una entidad propietaria. No obstante, no toda dependencia de existencia produce un tipo de entidad débil.


Un tipo de entidad débil normalmente tiene una clave parcial, que es el conjunto de atributos que pueden identificar sin lugar a dudas las entidades débiles que están relacionadas con la misma entidad propietaria. 


En los diagramas ER, tanto el tipo de la entidad débil como la relación identificativa, se distinguen rodeando sus cuadros y rombos mediante unas líneas dobles.


El atributo de clave parcial aparece subrayado con una línea discontinua o punteada. Los tipos de entidades débiles se puede representar a veces como atributos complejos (compuestos, multivalor). 


En general, se puede definir cualquier cantidad de niveles de tipos de entidad débil; un tipo de entidad propietaria puede ser ella misma un tipo de entidad débil. 


Además, un tipo de entidad débil puede tener más de un tipo de entidad identificativa y un tipo de relación identificativa de grado superior a dos.

\subsubsection{Atributos}
Cada entidad tiene atributos, que son propiedades particulares que la describen y en el modelo ER hay varios tipos: simple frente a compuesto, monovalor frente a multivalor, almacenado frente a derivado y nulo.


\paragraph*{Atributos compuestos frente a atributos simples} Los atributos compuestos se pueden dividir en subpartes más pequeñas que representan atributos más básicos con significados independientes.


Los atributos que no son divisibles se denominan atributos simples o atómicos, mientras que los atributos compuestos pueden forma una jerarquía. 


El valor de un atributo compuesto es la concatenación de los valores de sus atributos simples.

\paragraph*{Atributos monovalor y multivalor}  La mayoría de los atributos tienen un solo valor para una entidad en particular; dichos atributos reciben el nombre de monovalor o de un solo valor. 


En algunos casos, un atributo puede tener un conjunto de valores para la misma entidad y se denominan multivalor.


Un atributo multivalor puede tener límites superior e inferior para restringir el número de valores permitidos para cada entidad individual.


\paragraph*{Atributos almacenados y derivados}
El atributo derivado puede calcularse u obtenerse a partir de otro atributo, que se denomina almacenado.


\paragraph*{Atributos complejos}

Los atributos complejos son los atributos compuestos y multivalor que se anidan arbitrariamente.

Podemos representar el anidamiento arbitrario agrupando componentes de un atributo compuesto entre paréntesis () separando los componentes con comas, y mostrando los atributos multivalor entre llaves \{\}. 

Por ejemplo, si una persona puede tener más de una residencia y cada
residencia puede tener una sola dirección y varios teléfonos, el atributo TlfDir de una persona se puede especificarse Como


\{TlfDir(\{Tlf(CodÁrea,NumTlf)\},\\
Dir(DirCalle(Número,Calle,NumApto),\\
Ciudad,Provincia,CP))\}

Los atributos Tlf y Dir son compuestos.
\paragraph*{Atributos clave de un tipo de entidad}
Una restricción importante de las entidades de un tipo de entidad es la clave o restricción de unicidad de los atributos.


Un tipo de entidad normalmente tiene un atributo cuyos valores son distintos para cada entidad del conjunto de entidades.


Los valores de un atributo en que se pueden utilizar para identificar cada entidad inequívocamente se denomina atributo clave.


En la notación diagramática ER cada atributo clave tiene su nombre subrayado dentro del óvalo y algunos tipos de entidad tienen más de un atributo clave. 


Un tipo de entidad que carece de clave se le denomina tipo de entidad débil (que se explicará más adelante).


\paragraph*{Conjuntos de valores (dominios) de atributos} Cada atributo simple de un tipo de entidad está asociado con un conjunto de valor (o dominio de valores) que especifica el conjunto de los valores que se pueden asignar a ese atributo por cada entidad individual. 


Los conjuntos de valores no se muestran en los diagramas ER; normalmente se especifican mediante los tipos de datos básicos disponibles en la mayoría de los lenguajes de programación, como entero, cadena, booleano, flotante, tipo enumerado, subrango, etcétera. 


También se emplean otros tipos de datos adicionales para representar la fecha, la hora y otros.

\paragraph*{Atributos de los tipos de relación}


Los tipos de relación también pueden tener atributos; los atributos de los tipos de relación 1:1 o 1:N se pueden trasladar a uno de los tipos de entidad participantes.


En el caso de un tipo de relación 1:N, un atributo de relación solo se puede migrar al tipo de entidad que se encuentra en el lado N de la relación. 


Para los tipos de relación M:N, algunos atributos pueden determinarse mediante la combinación de entidades participantes en una instancia de relación, no mediante una sola relación. Dichos atributos deben especificarse como atributos de relación.

\subsubsection{Relaciones}

Un tipo de relación R entre n tipos de entidades $E_1, E_2, ..., E_n$ define un conjunto de asociaciones (o un conjunto de relaciones) entre las entidades de esos tipos de entidades. 


Como en el caso de los tipos de entidades y los conjuntos de entidades, normalmente se hace referencia a un tipo de relación y su correspondiente conjunto de relaciones con el mismo nombre $R$.


Matemáticamente, el conjunto de relaciones $R$ es un conjunto de instancias de relación $r_i$, donde cada $r_i$ asocia $n$ entidades individuales ($e_1, e_2,..., e_n$) y cada entidad $e_j$ de $r_i$ es un miembro del tipo de entidad $E_j, 1 \leq j \leq n$. Por tanto, un tipo de relación es una relación matemática en $E_1, E_2,..., E_n$.


De forma alternativa, se puede definir como un subconjunto del producto cartesiano $E_1 \times E_2,\times ... 	\times E_n$. Se dice que cada uno de los tipos de entidad $E_1, E_2,..., E_n$ participa en el tipo de relación $R$; de forma parecida, cada una de las entidades individuales $e_1, e_2,..., e_n$ se dice que participa en la instancia de relación $r_i = (e_1, e_2,..., e_n)$.


En los diagramas ER, los tipos de relaciones se muestran mediante rombos, conectados a su vez mediante líneas a los rectángulos que representan los tipos de entidad participantes y el nombre de la relación se muestra dentro del rombo.


\paragraph*{Grado de relación, nombres de rol y relaciones recursivas}
\paragraph*{Grado de un tipo de relación}
El grado de un tipo de relación es el número de tipos de entidades participantes. Un tipo de relación de grado dos se denomina binario y uno de grado tres ternario. Las relaciones pueden ser generalmente de cualquier grado, pero las más comunes son las relaciones binarias.

\paragraph*{Nombres de rol y relaciones recursivas}
Cada tipo de entidad que participa en un tipo de relación juega un papel o rol particular en la relación.


El nombre de rol hace referencia al papel que una entidad participante del tipo de entidad juega en cada instancia de relación y ayuda a explicar el significado de la relación.


Los nombres de rol no son técnicamente necesarios en los tipos de relación donde todos los tipos de entidad participantes son distintos, puesto que cada nombre de tipo de entidad participante se puede utilizar como participación.

Cuando un tipo de entidad se relaciona consigo misma, se tiene una relación recursiva y es necesario indicar los roles que juegan los miembros en la relación.


\paragraph*{Restricciones en los tipos de relaciones}


Los tipos de relaciones normalmente tienen ciertas restricciones que limitan las posibles combinaciones entre las entidades que pueden participar en el conjunto de relaciones correspondiente.


Estas restricciones están determinadas por la situación del minimundo representado por las relaciones. 


Podemos distinguir dos tipos principales de restricciones de relación: razón de cardinalidad y participación.


\paragraph*{Razones de cardinalidad para las relaciones binarias}
La razón de cardinalidad de una relación binaria especifica el número máximo de instancias de relación en las que una entidad puede participar.


Las posibles razones de cardinalidad para los tipos de relación binaria son 1:1, 1:N, N:1 y M:N.
\begin{enumerate}
    \item uno a uno: una relación R de X a Y es uno a uno si cada entidad en X se asocia con cuando mucho una entidad en Y e, inversamente, cada entidad en Y se asocia con cuando mucho una entidad en X.
    \item uno a muchos: una relación R de X a Y es uno a muchos si cada entidad en X se puede asociar con muchas entidades en Y, pero cada entidad en Y se asocia con cuando mucho una entidad en X. 
    \item muchos a uno: una relación R de X a Y es muchos a uno si cada entidad en X se asocia con cuando mucho una entidad en Y, pero cada entidad en Y se puede asociar con muchas entidades en X. 
    \item muchos a muchos: una relación R de X a Y es muchos a muchos si cada entidad en X se puede asociar con muchas entidades en Y y cada entidad en Y se puede asociar con muchas entidades en X. 
\end{enumerate}

\paragraph*{Restricciones de participación y dependencias de existencia}
La restricción de participación especifica si la existencia de una entidad depende de si está relacionada con otra entidad a través de un tipo de relación.


Esta restricción especifica el número mínimo de instancias de relación en las que puede participar cada entidad y en ocasiones recibe el nombre de restricción de cardinalidad mínima.


Hay dos tipos de restricciones de participación, total y parcial; la participación total también se conoce como dependencia de existencia.

\begin{enumerate}
    \item Participación total: si todo miembro de un conjunto de entidades debe participar en una relación, es una participación total del conjunto de entidades en la relación. Esto se denota al dibujar una línea doble desde el rectángulo de entidades hasta el rombo de relación.
    \item Participación parcial: una línea sencilla indica que algunos miembros del conjunto de entidades no deben participar en la relación.
    \end{enumerate}

Nos referiremos a la razón de cardinalidad y a las restricciones de participación, en conjunto, como restricciones estructurales de un tipo de relación.





\subsubsection{Resumen de la notación para los diagramas ER}

\begin{longtable}[l]{ c p{7cm} }

    \caption{Notación para los diagramas ER\label{long}}\\
    
    \hline
    \multicolumn{2}{ c }{Notación para los diagramas ER}\\
    \hline
    \multicolumn{1}{c}{Símbolo} & \multicolumn{1}{c}{Significado}\\
    \hline
    \endfirsthead
    
    \hline
    \multicolumn{2}{|l|}{Continuación de Tabla \ref{long}}\\
    \hline
    Continuación la notación para los diagramas ER\\
    \hline
    \endhead
    
    \hline
    \endfoot
    
    \hline
    \multicolumn{2}{ c }{Fin de Tabla}\\
    \hline%\ 
    \endlastfoot
    
    \parbox[c]{7em}{\includegraphics[width=\linewidth]{modeloEntidadRelacion/entidad.png}} & \multicolumn{1}{c}{Entidad}\\
    \parbox[c]{7em}{\includegraphics[width=\linewidth]{modeloEntidadRelacion/entidadDebil.png}} & \multicolumn{1}{c}{Entidad débil}\\
    \multicolumn{1}{c}{\parbox[c]{7em}{\includegraphics[width=0.5\linewidth]{modeloEntidadRelacion/relacion.png}}} & \multicolumn{1}{c}{Relación}\\
    \parbox[c]{7em}{\includegraphics[width=0.5\linewidth]{modeloEntidadRelacion/relacionDeIdentificacion.png}} & \multicolumn{1}{c}{Relación de identificación}\\
    \parbox[c]{7em}{\includegraphics[width=\linewidth]{modeloEntidadRelacion/atributo.png}} & \multicolumn{1}{c}{Atributo}\\
    \parbox[c]{7em}{\includegraphics[width=\linewidth]{modeloEntidadRelacion/atributoClave.png}} & \multicolumn{1}{c}{Atributo clave}\\
    \parbox[c]{7em}{\includegraphics[width=\linewidth]{modeloEntidadRelacion/atributoMultivalor.png}} & \multicolumn{1}{c}{Atributo multivalor}\\
    \parbox[c]{7em}{\includegraphics[width=\linewidth]{modeloEntidadRelacion/atributoCompuesto.png}} & \multicolumn{1}{c}{Atributo compuesto}\\
    \parbox[c]{7em}{\includegraphics[width=\linewidth]{modeloEntidadRelacion/atributoDerivado.png}} & \multicolumn{1}{c}{Atributo derivado}\\
    \parbox[c]{7em}{\includegraphics[width=\linewidth]{modeloEntidadRelacion/participacionTotal.png}} & \multicolumn{1}{c}{Participación total}\\
    \parbox[c]{7em}{\includegraphics[width=\linewidth]{modeloEntidadRelacion/razonCardinalidad.png}} & \multicolumn{1}{c}{Razón de cardinalidad}\\
    \parbox[c]{7em}{\includegraphics[width=\linewidth]{modeloEntidadRelacion/restriccionEstructural.png}} & \multicolumn{1}{c}{Restricción estructural}\\
\end{longtable}  
\subsection{Modelo relacional}

De acuerdo a la bibliografia de Elmasri\cite{ramez_elmasri_fundamentos_nodate}, el modelo relacional introducido por Ted Codd en 1970\cite{codd_relational_nodate} utiliza el concepto de una relación matemática como bloque de construcción básico y tiene su base teórica en la teoría de conjuntos y la lógica del predicado.

La lógica de predicado, utilizada ampliamente en matemáticas, proporciona un marco en el que una afirmación (declaración de hecho) se verifica como verdadera o falsa.


La teoría de conjuntos es una ciencia matemática que trata con conjuntos o grupos de cosas y se utiliza como base para la manipulación de datos en el modelo relacional.


El modelo relacional representa la base de datos como una colección de relaciones. Cuando una relación está pensada como una tabla de valores, cada fila representa una colección de valores relacionados.


Asimismo, cada fila de la tabla representa un hecho que, por lo general, corresponde con una relación o entidad real. El nombre de la tabla y de las columnas se utiliza para ayudar a interpretar el significado de cada uno de los valores de las filas.


En terminología formal, una fila recibe el nombre de tupla, una cabecera de columna es un atributo y el nombre de la tabla una relación. El tipo de dato que describe los valores en cada columna está representado por un dominio de posibles valores. 


Basado en estos conceptos, el modelo relacional tiene tres componentes bien definidos:
\begin{enumerate}
    \item Una estructura de datos lógica representada por relaciones.
    \item Un conjunto de reglas de integridad para garantizar que los datos sean consistentes.
    \item Un conjunto de operaciones que define cómo se manipulan los datos.
\end{enumerate}
\subsubsection{Relaciones}
\paragraph*{Estructuras de datos relacionales}
Su usan tablas con relación entre ellas.
\paragraph*{Tablas}
En este modelo, las tablas se usan para contener información acerca de los objetos a representar en la base de datos. Al usar los términos del modelo entidad-relación, los conjuntos de entidades y de relaciones se muestran usando tablas.


Una relación o esquema de relación se representa como una tabla bidimensional en la que las filas de la tabla corresponden a registros individuales y las columnas corresponden a atributos.

Formalmente, un esquema de relación $R$, denotado por $R(A_1, A_2,..., A_n)$ está constituido por un nombre de relación $R$ y una lista de atributos $A_1, A_2,..., A_n$. 


Cada atributo $A_i$ es el nombre de un papel jugado por algún dominio $D$ en el esquema de relación $R$. Se dice que $D$ es el dominio de $A_i$ y se especifica como $dom(A_i)$. 


Un esquema de relación se utiliza para describir una relación; se dice que $R$ es el nombre de la misma. El grado de una relación es el número de atributos $n$ de la misma.


La figura~\ref{img:modeloRelacional-Tabla} muestra de manera visual el modelo relacional; una tabla en el modelo relacional está compuesto de atributos, tiene un nombre de relación y tiene $n$ tuplas.


\begin{figure}[H]
    \centering
    \includegraphics[width=\textwidth]{modeloRelacional/tabla.png}
    \caption{Tabla en el modelo relacional}
    \label{img:modeloRelacional-Tabla}
\end{figure} 
Cada fila de la tabla corresponde a un registro individual o instancia de entidad. En el modelo relacional cada fila se llama tupla y la tabla que representa una relación tiene las siguientes características:
\begin{itemize}
    \item Cada celda de la tabla contiene solo un valor.
    \item Cada columna tiene un nombre distinto, que es el nombre del atributo que representa.
    \item Todos los valores en una columna provienen del mismo dominio, pues todos son valores del atributo correspondiente.
    \item Cada tupla o fila es distinta; no hay tuplas duplicadas.
    \item El orden de las tuplas o filas es irrelevante.
\end{itemize}
\paragraph*{Relaciones y tablas de bases de datos}   
Una relación (o estado de relación) $r$ del esquema $R(A_1, A_2,..., A_n)$, también especificado como $r(R)$, es un conjunto de n-tuplas $r={t_1, t_2,..., t_m}$.


Cada tupla $t$ es una lista ordenada de $n$ valores $t=<v_1, v_2,...,v_n>$, donde $v_i$, $1 \leq i \leq n$, es un elemento de $dom(A_i)$ o un valor especial NULL.


El i-ésimo valor de la tupla $t$, que se corresponde con el atributo $A_i$ , se referencia como $t[A_i]$ o $t[i]$ si utilizamos una notación posicional.

\subsubsection{Claves}

En el modelo relacional, las claves son importantes porque aseguran que cada fila en una tabla sea unívocamente identificable. 

También son usadas para establecer relaciones entre tablas y asegurar la integridad de los datos.

Una clave es un atributo o grupo de atributos que identifican los valores de otros atributos. 

\paragraph*{Clave compuesta}
Una clave compuesta es una clave que se compone de más de un atributo. Un atributo que forma parte de una clave se denomina atributo clave.

\paragraph*{Superclave}
Un atributo o atributos que identifican de manera única cualquier fila de una tabla

\subsubsection{Restricciones de integridad}
\paragraph*{Integridad de dominio}
La integridad de dominio es la validez de las restricciones que debe cumplir una determinada columna de la tabla.
\paragraph*{Integridad de entidad}
Todas las claves principales son únicas, y ninguna clave primaria debe ser nula.
\paragraph*{Integridad referencial}
Una clave externa es ser nula siempre que no sea parte de la clave principal de su tabla, o tiene el valor que coincida con el valor de la clave primaria en una tabla con la que está relacionada (cada valor de clave externa no nula debe hacer referencia a un valor de clave primaria existente).

\subsubsection{Propiedades de las relaciones}
\paragraph*{Grado}
El número de columnas en una tabla se llama grado de la relación. Una relación con una sola columna es de grado uno y se llama relación unaria. Una relación con dos columnas se llama binaria, una con tres columnas se llama ternaria y, después de ella, por lo general se usa el término n-aria. El grado de una relación es parte de la intensión de la relación y nunca cambia.


\paragraph*{Cardinalidad}
La cardinalidad de una relación es el número de entidades a las que otra entidad mapea dicha relación.

\subsubsection{ACID}
El modelo relacional en las transacciones cumple con las propiedades de ACID, que es el acrónimo de \textit{Atomicity} (atomicidad), \textit{Consistency} (consistencia), \textit{Isolation} (aislamiento) y \textit{Durability} (durabilidad). 

\paragraph*{Atomicidad}
Requiere que se completen todas las operaciones (solicitudes SQL) de una transacción;
si no, la transacción se cancela. 


Si una transacción $T_{1}$ tiene cuatro solicitudes SQL, las cuatro solicitudes deben completarse con éxito; de lo contrario, se anula toda la transacción.


En otras palabras, una transacción se trata como una unidad de trabajo única, indivisible y lógica.
\paragraph*{Consistencia}
Indica la permanencia del estado consistente de la base de datos. Una transacción lleva una base de datos de un estado consistente a otro. 


Cuando se completa una transacción, la base de datos debe estar en un estado coherente. Si alguna de las partes de la transacción viola una restricción de integridad, se anula la transacción completa.
\paragraph*{Aislamiento}
Significa que los datos utilizados durante la ejecución de una transacción no es utilizada por una segunda transacción hasta que se complete la primera. 


En otras palabras, si la transacción $T_{1}$ se está ejecutando y está utilizando el elemento de datos $X$, ninguna otra transacción accede a ese elemento de datos ($T_{2}...T_{n}$) hasta que finalice $T_{1}$.


Esta propiedad es particularmente útil en entornos de bases de datos multiusuario porque varios usuarios acceden y actualizan la base de datos al mismo tiempo.
\paragraph*{Durabilidad}

Garantiza que una vez que se realizan y confirman los cambios en la transacción, no se deshacen ni pierden, incluso en el caso de una falla del sistema.
\subsubsection{Structured Query Language}
\textit{Structured Query Language} o SQL está basado en el álgebra relacional, en el cálculo relacional y es un lenguaje de manipulación de datos, un lenguaje de definición de datos, un lenguaje de control de transacciones y un lenguaje de control de datos.

\paragraph*{Lenguaje de manipulación de datos (DML)}
Un \textit{Data Manipulation Language} o DML incluye comandos para insertar, actualizar, eliminar y recuperar datos dentro de las tablas de la base de datos. 

\paragraph*{Lenguaje de definición de datos (DDL)}
 Un \textit{Data Definition Language} o DDL incluye comandos para crear objetos de base de datos como tablas, índices y vistas, así como comandos para definir accesos a objetos de la base de datos. 

\paragraph*{Lenguaje de control de transacciones (TCL)}
Los comandos de un \textit{Transaction Control Language} o TCL se ejecutan dentro del contexto de una transacción, que es una unidad lógica de trabajo compuesta por una o más instrucciones SQL. 


SQL proporciona comandos para controlar el procesamiento de estas transacciones atómicas.

\paragraph*{Lenguaje de control de datos (DCL)}
Los comandos de un \textit{Data Control Language} o DCL se utilizan para controlar el acceso a los objetos de datos, como otorgar a un usuario permiso para ver solo una tabla y otorgar a otro usuario permiso para cambiar los datos de la mista tabla.


%\renewcommand*{\arraystretch}{1.4}
\begin{longtable}[l]{ l p{7cm} }

    \caption{Comandos de SQL.\label{long}}\\
    
    \hline
    \multicolumn{2}{ c }{Comandos de manipulación de datos}\\
    \hline
    \multicolumn{1}{c}{Comando} & \multicolumn{1}{c}{Descripción}\\
    \hline
    \endfirsthead
    
    \hline
    \multicolumn{2}{|l|}{Continuación de Tabla \ref{long}}\\
    \hline
    Continuación de comandos SQL\\
    \hline
    \endhead
    
    \hline
    \endfoot
    
    \hline
    \multicolumn{2}{ c }{Fin de Tabla}\\
    \hline%\hline
    \endlastfoot
    
    \textbf{SELECT} & \textbf{Selecciona atributos de filas en una o más tablas o vistas}\\
    \qquad FROM  & Especifica las tablas de las que se deben recuperar los datos\\
    \qquad WHERE  & Restringe la selección de filas en función de una expresión condicional\\
    \qquad GROUP BY & Agrupa las filas seleccionadas en función de uno o más atributos\\
    \qquad HAVING & Restringe la selección de filas agrupadas en función de una condición\\
    \qquad ORDER BY & Ordena las filas seleccionadas en función de uno o más atributos\\
     \textbf{INSERT} & \textbf{Inserta filas en una tabla}\\
     \textbf{UPDATE} & \textbf{Modifica los valores de un atributo en una o más filas de la tabla}\\
    \textbf{DELETE} & \textbf{Elimina una o más filas de una tabla}\\
    \textbf{Operadores de comparación} \\
    \qquad $=,<,>,\leq,\geq,<>,!=$ & Usados en expresiones condicionales\\
    \textbf{Operadores lógicos} \\
    \qquad AND, OR, NOT & Usados en expresiones condicionales\\
    \textbf{Operadores especiales} \\
    \qquad BETWEEN & Comprueba si un valor de atributo está dentro de un rango\\
    \qquad IN & Comprueba si un valor de atributo coincide con algún valor dentro de una lista de valores\\
    \qquad LIKE & Comprueba si un valor de atributo coincide con un patrón de cadena dado\\
    \qquad IS NULL & Comprueba si un valor de atributo es nulo\\
    \qquad EXIST & Comprueba si una subconsulta devuelve alguna fila\\
    \qquad DISTINCT & Limita que los valores sean únicos\\
    \textbf{Comandos de definición de datos}\\
    \textbf{CREATE SCHEMA} & \textbf{Crea el esquema de la base de datos}\\
    \textbf{CREATE TABLE} & Crea una nueva tabla en el esquema de la base de datos del usuario\\
    \qquad NOT NULL & Asegura que una columna no tendrá valores nulos\\
    \qquad UNIQUE & Asegura que una columna no tendrá valores duplicados\\
    \qquad PRIMARY KEY & Define una clave primaria para una tabla\\
    \qquad FOREIGN KEY & Define una clave externa para una tabla\\
    \qquad DEFAULT & Define un valor predeterminado para una columna (cuando no se proporciona ningún valor)\\
    \qquad CHECK & Valida datos en un atributo\\
    \textbf{CREATE INDEX} & \textbf{Crea un índice para una tabla}\\
    \textbf{CREATE VIEW} & \textbf{Crea un subconjunto dinámico de filas y columnas a partir de una o más tablas}\\
    \textbf{ALTER TABLE} & \textbf{Modifica la definición de una tabla (agrega, modifica o elimina atributos o restricciones)}\\
    \textbf{CREATE TABLE AS} & \textbf{Crea una nueva tabla basada en una consulta en el esquema de la base de datos del usuario}\\
    \textbf{DROP TABLE} & \textbf{Elimina permanentemente una tabla (y sus datos)}\\
    \textbf{DROP INDEX} & \textbf{Elimina permanentemente un índice}\\
    \textbf{DROP VIEW} & \textbf{Elimina permanentemente una vista}\\
    \textbf{Lenguaje de control de transacciones}\\
    \qquad COMMIT & Guarda de manera permanente los cambios en los datos\\
    \qquad ROLLBACK & Restaura los datos a sus valores anteriores\\
    \textbf{Lenguaje de control de datos}\\
    \qquad GRANT & Le da un usuario permiso de hacer una acción de sistema o de acceder a datos de un objeto\\
    \qquad REVOKE & Le quita el privilegio a un usario de hacer algunas operaciones\\
\end{longtable} 
\subsection{Modelos NoSQL}
De acuerdo a la bibliografia de Catherine \cite{cristina_marta_bender_topicos_nodate}, el término NoSQL significa \textit{not only SQL} y se usa para agrupar sistemas de bases de datos diferentes a los relacionales.


Por los nuevos requerimientos en la época actual como disponibilidad total, tolerancia a fallos, almacenamiento de penta bytes de información distribuida en miles de servidores, la necesidad de nodos con escalabilidad horizontal, entre otros, surge la necesidad de sistemas de bases de datos no relacionales.


Estos tipos de sistemas no requieren esquemas fijos, son fáciles y rápidos en la instalación, usan lenguajes no declarativos, ofrecen alto rendimiento y disponibilidad, evitan operaciones de junturas, soportan paralelismo y escalan principalmente en forma horizontal soportando estructuras distribuidas que no necesariamente cumplen las propiedades ACID\cite{cristina_marta_bender_topicos_nodate}, sino que se enfocan en el modelo de consistencia de datos BASE.

\subsubsection{Teorema CAP}
En el  Simposio de Principios de Computación Distribuida (PODC, en inglés) en el año 2000\cite{brewer_towards_2000}, el Dr. Eric Brewer declaró en su presentación que ``en cualquier sistema de datos altamente distribuido hay tres propiedades comúnmente deseables: \textit{Consistency} (consistencia), \textit{Availability} (disponibilidad) y \textit{Partition tolerance} (tolerancia al particionado). Sin embargo, es imposible que un sistema proporcione las tres propiedades al mismo tiempo".


El acrónimo CAP representa las tres propiedades deseables:

\paragraph*{Consistencia}

En una base de datos distribuida, la consistencia tiene el papel más importante. Todos los nodos deben ver los mismos datos al mismo tiempo, lo que significa que las réplicas deben actualizarse inmediatamente. Sin embargo, esto implica lidiar con la latencia y los atrasos de la red.

No hay que confundir la consistencia en la gestión de transacciones con la consistencia del teorema CAP. La consistencia de la gestión de transacciones se refiere al resultado cuando la ejecución de una transacción en una base de datos cumple con todas las restricciones de integridad.


La consistencia en CAP se basa en la suposición de que todas las transacciones tienen lugar al mismo tiempo en todos los nodos, como si se estuvieran ejecutando en una base de datos de un solo nodo (todos los nodos ven los mismos datos al mismo tiempo).

\paragraph*{Disponibilidad}
En términos simples, el sistema siempre cumple una solicitud. Ninguna solicitud recibida se pierde y este es un requisito fundamental para todas las organizaciones centradas en la web.

\paragraph*{Tolerancia al particionado}
El sistema continúa funcionando incluso en caso de falla de un nodo. Esto es equivalente a la transparencia de fallas en bases de datos distribuidas. El sistema fallará solo si fallan todos los nodos.

Aunque el teorema CAP se centra en sistemas basados ​​en la web altamente distribuidos, sus implicaciones están muy extendidas para todos los sistemas distribuidos, incluidas las bases de datos.


En los sistemas de bases de datos, las propiedades ACID aseguran que todas las transacciones exitosas den como resultado un estado de base de datos consistente, uno en el que todas las operaciones de datos siempre devuelven los mismos resultados. 


Para bases de datos distribuidas centralizadas y pequeñas, la latencia no es un problema, pero para una base de datos altamente distribuida el garantizar transacciones ACID sin pagar un alto precio en latencia de red o en conflictos de datos.


La relación entre consistencia y disponibilidad ha generado un nuevo tipo de sistemas de datos distribuidos, diferente al ACID, denominados BASE, \textit{Basically Available} (básicamente disponibles), \textit{Soft state} (estado suave), \textit{Eventually consistent} (eventualmente consistente).

\paragraph*{BASE}

BASE se refiere a un modelo de consistencia de datos en el que los cambios de datos no son inmediatos, sino que se propagan lentamente a través del sistema hasta que todas las réplicas sean consistentes. 


En la práctica, la aparición de bases de datos distribuidas NoSQL proporciona un espectro de consistencia que va desde lo altamente consistente (ACID) hasta lo eventualmente consistente (BASE).

\subsubsection{Clave-Valor}
De acuerdo con Coronel\cite{coronel_database_nodate}, una base de datos de clave-valor es un paradigma de modelo de datos diseñado para almacenar, recuperar y administrar arreglos asociativos.


Comúnmente se usa un diccionario o tabla \textit{hash} que contiene una colección de registros anidados, secuencias de bits que se almacenan y se recuperan utilizando una clave que identifica de manera única el registro y se utiliza para encontrar rápidamente los datos dentro de la base de datos.


No obstante, es responsabilidad de las aplicaciones que hagan uso de este tipo de base de datos interpretar el significado de los datos; no hay claves foráneas y las relaciones no son rastreables entre claves, lo que permite que el DBMS sea rápido y escalable.


La figura~\ref{img:claveValor-bucket} es una representación visual de un \textit{bucket} usado en las bases de datos NoSQL de clave-valor.

\begin{figure}[H]
    \centering
    \includegraphics[width=0.75\textwidth]{noSQL/bucket.png}
    \caption{\textit{Bucket} en el almacenamiento clave-valor}
    \label{img:claveValor-bucket}
\end{figure} 


Los pares de clave-valor generalmente se organizan en \textit{buckets}; todas las claves dentro un \textit{bucket} deben ser únicas, pero está permitido que se repitan en otros \textit{buckets} y todas las operaciones se basan en el \textit{bucket} + la clave.


En este tipo de bases de datos se usan las operaciones de \textit{get}, \textit{store} y \textit{delete}; la operación \textit{get} o \textit{fetch} es usada para obteber el valor de un par; el operador de \textit{store} almacena datos en una clave. 


Si la combinación de \textit{bucket} + clave no existe, se añade como un nuevo par de clave-valor; en cambio, si existe la combinación de \textit{bucket} + clave, el valor es reemplazado por el nuevo; el operador de \textit{delete} es para eliminar un par de clave-valor.


De acuerdo con Sadalge\cite{sadalage_nosql_nodate}, algunas de las bases de datos de clave-valor populares son Riak, Redis, Memcached DB, Berkeley DB, HamsterDB, Amazon DynamoDB y Project Voldemort (una implementación de código abierto de Amazon DynamoDB).



\subsubsection{Orientado a documentos}
Una base de datos orientada a documentos es una base de datos NoSQL que almacena datos en documentos etiquetados en pares clave-valor.


A diferencia de una base de datos clave-valor donde el componente de valor contiene cualquier tipo de datos, una base de datos de documentos siempre almacena un documento en el componente de valor y puede estar en cualquier formato codificado, como XML, JSON o BSON.


La figura~\ref{img:documentos-documento} es la representación visual de un \textit{bucket}, donde tiene una clave como identificador único y documentos anidados.

\begin{figure}[H]
    \centering
    \includegraphics[width=0.75\textwidth]{noSQL/documento.png}
    \caption{Documento en el almacenamiento de documentos}
    \label{img:documentos-documento}
\end{figure}
Otra diferencia importante es que si bien las bases de datos clave-valor no intentan comprender el contenido del componente de valor, las bases de datos de documentos sí lo hacen.% Las etiquetas representan porciones de un documento.


Por ejemplo, hay documentos con etiquetas para identificar qué texto en el documento representa el título, el autor y el cuerpo del documento.


Dentro del cuerpo del documento, existen etiquetas adicionales para indicar capítulos y secciones. A pesar del uso de etiquetas en los documentos, las bases de datos de documentos se consideran sin esquema, es decir, no imponen una estructura predefinida en los datos almacenados.


Para una base de datos de documentos, no tener esquemas significa que aunque todos los documentos tienen etiquetas, no todos tienen las mismas etiquetas, por lo es posible que cada documento tenga su propia estructura.


Las etiquetas en una base de datos de documentos son extremadamente importantes porque son la base de la mayoría de las capacidades adicionales que tienen las bases de datos de documentos sobre las bases de datos clave-valor.


Las etiquetas dentro del documento son accesibles para el DBMS, lo que hace posible consultas complejas. Al igual que las bases de datos clave-valor agrupan pares clave-valor en grupos lógicos llamados \textit{buckets}, las bases de datos de documentos agrupan documentos en grupos lógicos llamados colecciones.


Si bien es posible recuperar un documento especificando la colección y la clave, también es posible realizar consultas en función del contenido de las etiquetas.


Las bases de datos de documentos tienden a funcionar bajo el supuesto de que un documento es independiente, o sea que no está en diferentes tablas como en una base de datos relacional.


Una base de datos de documentos asume que todos los datos relacionados de una orden estén en un solo documento; por lo tanto, cada orden en una colección contendría datos sobre el cliente, el pedido en sí y los productos comprados en esa orden.


Las bases de datos de documentos no almacenan relaciones como se hace en el modelo relacional y generalmente no tienen soporte para operaciones como la unión.
\subsubsection{Orientado a columnas}
De acuerdo con Coronel\cite{coronel_database_nodate}, el modelo de base de datos NoSQL orientado a columnas se originó con el BigTable de Google. 

La figura~\ref{img:familia-columna} representa una familia de columnas; la imagen de arriba es la partición de las diferentes familias de columnas y en la imagen de abajo se nota cada partición individual.

\begin{figure}[H] 
    \centering
    \includegraphics[width=0.75\textwidth]{noSQL/columna.png}
    \caption{Familia de columnas}
    \label{img:familia-columna}
\end{figure}


Las bases de datos de la familia de columnas son parecidas a las relaciones del modelo relacional y organizan datos en pares nombre-valor donde el nombre actúa también como la clave; como se nota en la figura~\ref{img:familia-columna}, un par de clave-valor representa una columna y siempre contiene una fecha que sirve para resolver conflictos de escritura o datos expirados.

\subsubsection{Orientado a grafos}
De acuerdo con Coronel\cite{coronel_database_nodate}, una base de datos NoSQL orientada a grafos está basada en la teoría de grafos para almacenar datos con muchas relaciones.



La figura~\ref{img:nosql-grafo} representa un grafo de una bilioteca, donde cada rectángulo es un nodo y están asociados entre sí por relaciones.
\begin{figure}[H] 
    \centering
    \includegraphics[width=0.75\textwidth]{noSQL/grafo.png}
    \caption{modelo conceptual orientado a grafos}
    \label{img:nosql-grafo}
\end{figure}


Como se muestra en la figura~\ref{img:nosql-grafo}, los componentes principales de las bases de datos de grafos son nodos, aristas y propiedades; el nodo es una instancia específica de algo sobre lo que queremos mantener datos.


Las propiedades son como atributos; son los datos que necesitamos almacenar sobre el nodo; todos los nodos tienen propiedades como nombre y apellido, pero no todos los nodos deben tener las mismas propiedades.


Un borde es una relación entre nodos, está representada por una flecha en la figura~\ref{img:nosql-grafo} y es posible que estén en una dirección o ser bidireccionales.


Para hacer una consulta se atraviesa el grafo y los recorridos se enfocan en las relaciones entre nodos, como la ruta más corta y el grado de conexión.


\subsection{Ingeniería Dirigida por Modelos}

De acuerdo con Scherp \cite{scherp_framework_2013}, la ingeniería dirigida por modelos (en inglés \textit{Model-driven Software Development}) es el desarrollo de \textit{software} mediante modelos y transformaciones entre modelos con el objetivo de automatizar el mapeo entre modelos a código fuente.

Asimismo, de acuerdo con Reussner \cite{reussner_handbuch_2006}, los conceptos principales en la ingeniería dirigida por modelos son:

\begin{enumerate}
  \item \textbf{Modelo}: es una vista simplificada y abstracta de un sistema real. 
  \item \textbf{Metamodelo}: define elementos y reglas para generar modelos; consiste en una sintaxis abstracta, sintaxis concreta y una semántica.
  \item \textbf{Transformación entre modelos}: es un mapeo computable que toma como entrada una instancia de un tipo de modelo y como salida genera una instancia de otro tipo de modelo.
  \item \textbf{Lenguaje específico de dominio}: es un lenguaje definido por un metamodelo que contiene conceptos para un dominio específico.
\end{enumerate}
\subsection{Lenguaje específico de dominio}
De acuerdo con Fowler \cite{fowler_domain-specific_2010}, un lenguaje específico de dominio (en inglés \textit{domain specific language}) es un lenguaje de programación dedicado a un dominio en partícular (un dominio puede ser en el contexto de un banco, o de una aplicación web, o de consultas de bases de datos), porque representa un problema específico y provee una técnica para solucionar una situación particular. 

Hay dos tipos de lenguajes en un lenguaje DSL:

\begin{enumerate}
  \item DSL: el lenguaje en el que es escrito un DSL.
  \item Lenguaje del host: el lenguaje en el que es ejecutado y procesado el DSL.
\end{enumerate}


Asimismo, un DSL es externo si es un lenguaje diferente al lenguaje del host, pero es interno si el lenguaje DSL es solo un subconjunto de instrucciones del lenguaje del host.
\subsubsection*{Metodología de diseño}

Debido a que un DSL puede ser externo se necesitan de herramientas existentes o desarrollar las herramientas que permitan al lenguaje host interpretar el DSL externo.

De acuerdo con Deursen\cite{van_deursen_domain-specific_2000}, un DSL se puede diseñar en los siguientes pasos:
\begin{enumerate}
  \item Identificar el problema de dominio
  \item Reunir información relevante sobre el problema de dominio escogido.
  \item Crear una gramática que exprese semánticamente el problema de dominio.
  \item Desarrollar un compilador que traduzca programas del lenguaje DSL al lenguaje host.
\end{enumerate} 

\subsubsection*{Implementación}

Desarrollar un intérprete o un compilador es el enfoque clásico para implementar un nuevo lenguaje, aunque se pueden utilizar herramientas de compilación estándar o herramientas dedicadas a la implementación de DSL como Draco, ASF + SDF o Xtext.

La principal ventaja de desarrollar un compilador o intérprete es que la implementación está completamente adaptada al DSL y no es necesario hacer concesiones en cuanto a notación o tipos de datos. Por otra parte, un problema importante es el costo de desarrollar un compilador o intérprete desde cero, y la falta de reutilización de otras implementaciones\cite{van_deursen_domain-specific_2000}.

\subsubsection{Transformación entre modelos}
\paragraph*{Transformación modelo a modelo}
\paragraph*{Transformación modelo a texto}

\section{Tecnologías a usar}
Para seleccionar las tecnologías a usar en las propuesta de solución se ha optado por investigar tecnologías similares y esta sección está organizado de la siguiente manera: primero se muestra cada tecnología usar; en caso de que sea la única opción se describirá qué es y en caso de que haya varias opciones, se explicará cada opción y se tendrá un apartado al final de cada comparación sobre la tecnología que se ha elegido.
\subsection{Hypertext Transfer Protocol}
De acuerdo con la W3C\cite{noauthor_http_nodate}, Hypertext Transfer Protocol (HTTP, o protocolo de transferencia de hipertexto), es el protocolo de comunicación que permite transferir información en la World Wide Web.


HTTP fue desarrollado por el World Wide Web Consortium y la Internet Engineering Task Force, colaboración que culminó en 1999 con la publicación de varios RFC, siendo el más importante el RFC 2616 que especifica la versión 1.1 del protocolo.


HTTP es un protocolo sin estado, es decir, no guarda ninguna información sobre conexiones anteriores; sin embargo, el desarrollo de aplicaciones web necesita frecuentemente mantener un estado, por lo que se usan \textit{cookies}, que son archivos generados en un servidor que son almacenados en el sistema cliente.


Es un protocolo orientado a transacciones y sigue el esquema petición-respuesta entre un cliente y un servidor; al cliente se le suele llamar ``agente de usuario'' (o \textit{user agent}), que realiza una petición enviando un mensaje con cierto formato al servidor, mientras que al servidor se le suele llamar servidor web y envía un mensaje de respuesta. 


HTTP tiene métodos de petición flexibles que permiten añadir nuevos métodos o funcionalidades; el número de métodos ha ido en aumento según se avanza en las versiones del protocolo donde los más importantes son:


\begin{enumerate}
    \item Método \textit{get}: solicita una representación del recurso especificado; solo deben recuperar datos. 
    \item Método \textit{head}: pide una respuesta idéntica a la que correspondería a una petición \textit{get}, pero en la respuesta no se devuelve el cuerpo; esto es útil para poder recuperar los metadatos de los encabezados de respuesta, sin tener que transportar todo el contenido.
    \item Método \textit{post}: envía datos para que sean procesados por un recurso identificado que se incluirán en el cuerpo de la petición. 
    \item Método \textit{put}: sube o carga un recurso especificado (archivo o fichero) y es más eficiente que el método \textit{post}, porque permite escribir un archivo en una conexión socket establecida con el servidor.
\end{enumerate}
\subsection{HTML 5 }

De acuerdo a la documentación de Mozilla\cite{noauthor_html_nodate}, HyperText Markup Language, abreviado HTML, o lenguaje de marcado de hipertextos, es la pieza más básica para la construcción de la web y se usa para definir el sentido y estructura del contenido en una página web. 


Es un estándar a cargo del World Wide Web Consortium (W3C) o Consorcio WWW, organización dedicada a la estandarización de casi todas las tecnologías ligadas a la web. 


HTML hace uso de enlaces que conectan las páginas web entre sí, ya sea dentro de un mismo sitio web o entre diferentes sitios web.


Un elemento HTML se separa de otro texto en un documento por medio de ``etiquetas", las cuales consisten en elementos rodeados por ``<,>".


Ejemplos de estas etiquetas son <head>, <title>, <body>, <header>, <p>, <ul>, <ol>, <li>, y otros más.


HTML 5 (HyperText Markup Language, versión 5) es la quinta revisión importante del lenguaje básico de la World Wide Web, HTML.


HTML5 establece elementos y atributos que reflejan el uso típico de los sitios web modernos. Algunos de ellos son técnicamente similares a las etiquetas <div> y <span>, pero tienen un significado semántico, como por ejemplo <nav> (bloque de navegación del sitio web) y <footer>.


Características:

\begin{enumerate}
    \item Incorpora etiquetas: \textit{canvas} 2D y 3D, audio, vídeo con codecs para mostrar los contenidos multimedia. Actualmente hay una lucha entre imponer codecs libres (WebM + VP8) o privados (H.264/MPEG-4 AVC).
    \item Etiquetas para manejar grandes conjuntos de datos: Datagrid, Details, Menu y Command. Permiten generar tablas dinámicas que pueden filtrar, ordenar y ocultar contenido en cliente.
    \item Mejoras en los formularios: Nuevos tipos de datos (\textit{email, number, url, datetime})  y facilidades para validar el contenido sin JavaScript.
    \item Visores: MathML (fórmulas matemáticas) y SVG (gráficos vectoriales); en general se deja abierto a poder interpretar otros lenguajes XML.
    \item Drag \& Drop: nueva funcionalidad para arrastrar objetos como imágenes.
\end{enumerate}

Respecto a la compatibilidad con los navegadores, la mayoría de elementos de HTML5 son compatibles con Firefox 19, Chrome 25, Safari 6 y Opera 12 en adelante.


\subsection{CSS3 vs SCSS}

De acuerdo a la documentación de la W3C\cite{noauthor_what_nodate}, Cascading Style Sheets, CSS, o hojas de estilo en cascada, es un lenguaje de diseño gráfico para definir y crear la presentación de un documento estructurado escrito en un lenguaje de marcado.


CSS está diseñado principalmente para marcar la separación del contenido del documento y la presentación del mismo, con características como las capas o layouts, los colores y las fuentes.


La separación entre el contenido del documento y la presentación busca mejorar la accesibilidad, proveer más flexibilidad y control, permitir que varios documentos HTML compartan un mismo estilo usando una sola hoja de estilos separada en un archivo .css y reducir la complejidad y la repetición de código en la estructura del documento.


La especificación CSS describe un esquema prioritario para determinar qué reglas de estilo se aplican si más de una regla coincide para un elemento en particular. Estas reglas son aplicadas con un sistema llamado de cascada, de modo que las prioridades son calculadas y asignadas a las reglas, así que los resultados son predecibles.


La especificación CSS es mantenida por el World Wide Web Consortium (W3C). El MIME type text/css está registrado para su uso por CSS descrito en el RFC 23185​. El W3C proporciona una herramienta de validación de CSS gratuita para los documentos CSS.



CSS se ha creado en varios niveles y perfiles. Cada nivel de CSS se construye sobre el anterior, generalmente añadiendo funciones al nivel previo.


Los perfiles son, generalmente, parte de uno o varios niveles de CSS definidos para un dispositivo o interfaz particular. Actualmente, pueden usarse perfiles para dispositivos móviles, impresoras o televisiones.


La última versión del estándar, CSS3.1, está dividida en varios documentos separados, llamados ``módulos".


Cada módulo añade nuevas funcionalidades a las definidas en CSS2, de manera que se preservan las anteriores para mantener la compatibilidad.


Los trabajos en el CSS3.1 comenzaron a la vez que se publicó la recomendación oficial de CSS2 y los primeros borradores de CSS3.1 fueron liberados en junio de 1999.


Debido a la modularización del CSS3.1, diferentes módulos pueden encontrarse en diferentes estados de su desarrollo,​ de forma que hay alrededor de cincuenta módulos publicados,​ tres de ellos se convirtieron en recomendaciones oficiales de la W3C en 2011: ``Selectores", ``Espacios de nombres", y ``Color".


Respecto al soporte de los navegadores web, cada navegador web usa un motor de renderizado para renderizar páginas web, y el soporte de CSS no es exactamente igual en ninguno de los motores de renderizado. Ya que los navegadores no aplican el CSS correctamente, muchas técnicas de programación han sido desarrolladas para ser aplicadas por un navegador específico (comúnmente conocida esta práctica como \textit{CSS hacks} o \textit{CSS filters}).


Además, CSS3 definen muchas queries, entre las cuales se provee la directiva @supports que permite a los desarrolladores especificar navegadores con soporte para alguna función en específico directamente en el CSS3.1​. 


\subsection{JavaScript vs TypeScript}\label{ref:sec-javascript}

De acuerdo con Stack Overflow\cite{noauthor_most_nodate}, los web \textit{frameworks} más populares están escritos en JavaScript/TypeScript, por ello se realizó una investigación del lenguaje más apto para el proyecto.

\subsubsection*{JavaScript}
De acuerdo con la documentación de Mozilla\cite{noauthor_javascript_nodate}, JavaScript es una marca registrada con licencia de Sun Microsystems (ahora Oracle) que se usa para describir la implementación del lenguaje de programación JavaScript.


Debido a problemas de registro de marcas en la Asociación Europea de Fabricantes de Computadoras, la versión estandarizada del lenguaje tiene el nombre de ECMAScript, sin embargo, en la práctica se conoce como lenguaje JavaScript. 


Abreviado como JS, es un lenguaje ligero e interpretado, orientado a objetos, basado en prototipos, imperativo, débilmente tipado y dinámico; es usado en node.js, Apache CouchDB y Adobe Acrobat.


El núcleo del lenguaje JavaScript está estandarizado por el Comité ECMA TC39 como un lenguaje llamado ECMAScript y la última versión de la especificación es ECMAScript 6.0 que define: 

\begin{enumerate}
    \item Sintaxis: reglas de análisis, palabras clave, flujos de control, inicialización literal de objetos.
    \item Mecanismos de control de errores: \textit{throw, try/catch}, habilidad para crear tipos de errores definidos por el usuario.
    \item Tipos: \textit{boolean, number, string, function, object}.
    \item  Objetos globales: en un navegador, los objetos globales son los objetos de la ventana, pero ECMAScript solo define una API no específica para navegadores, como \textit{parseInt, parseFloat, decodeURI o encodeURI}.
    \item Mecanismo de herencia basada en prototipos.
    \item Objetos y funciones incorporadas.
    \item Modo estricto.
\end{enumerate}

La sintaxis básica es similar a Java y C++ con la intención de reducir el número de nuevos conceptos necesarios para aprender el lenguaje; las construcciones del lenguaje, como sentencias \textit{if}, bucles \textit{for}, \textit{while}, bloques \textit{switch} y \textit{try catch} funcionan de la misma manera que en estos lenguajes (o casi).


JavaScript funciona como lenguaje procedimental y como lenguaje orientado a objetos; los objetos se crean añadiendo métodos y propiedades a lo que de otra forma serían objetos vacíos en tiempo de ejecución, en contraposición a las definiciones sintácticas de clases comunes en los lenguajes compilados como C++ y Java.


Las capacidades dinámicas de JavaScript incluyen construcción de objetos en tiempo de ejecución, listas variables de parámetros, variables que contienen funciones, creación de scripts dinámicos (mediante eval), introspección de objetos (mediante \textit{for ... in}), y recuperación de código fuente.


Desde 2012 todos los navegadores modernos soportan completamente ECMAScript 5.1 y el 17 de Julio de 2015 ECMA International publicó la sexta versión de ECMAScript, oficialmente llamada ECMAScript 2015 y fue inicialmente nombrada como ECMAScript 6 o ES6. Desde entonces, los estándares ECMAScript están en ciclos de lanzamiento anuales.


\subsubsection*{TypeScript}
De acuerdo con TypeScript Publishing\cite{publishing_typescript_2019}, TypeScript es por definición JavaScript para el desarrollo de aplicaciones, siendo también un superconjunto del mismo.


TypeScript es un lenguaje compilado orientado a objetos, fue diseñado por Anders Hejlsberg (diseñador de C\#) en Microsoft;es tanto un lenguaje como un conjunto de herramientas y es un superconjunto de JavaScript porque genera código JavaScript.


\paragraph*{Características}
\begin{itemize}
    \item Compilación: cuenta con un transpilador para la verificación de errores si hay errores de compilación, cosa que no es posible con JavaScript.
    \item Tipeo estático fuerte: provee un sistema opcional de tipeo estático y de inferencia de tipos a traves del TypeScript Language Service, lo que permite inferir el tipo de una variabla declara sin tipo en función de su valor.
    \item Definiciones de tipo: permite la extensión del lenguaje con bibliotecas externas JavaScript.
    \item Programación orientaba a objetos: admite conceptos como clases, interfaces, herencia, etc.
\end{itemize}



\subsubsection*{Elección}

De acuerdo con el sitio Stack Overflow\cite{noauthor_stack_nodate}, JavaScript es el lenguaje más popular de 2019 y aunque TypeScript es de los lenguajes que tienen un mayor nivel de aceptación, se usará JavaScript no solo por ser el lenguaje más popular y, en consecuencia, con más compatibilidad y material de ayuda, sino también porque el equipo está acostumbrado a este lenguaje y tiene experiencia con web \textit{frameworks} escritos en JavaScript.


\subsection{JavaScript Web Frameworks: Vue/Nuxt vs React vs Angular}

De acuerdo con \textit{Wired}\cite{wired_wired_2020}, un web \textit{framework} es un conjunto de \textit{software} que permite el desarrollo de una aplicación web y en el lenguaje JavaScript hay varias opciones, incluidas los más populares Vue, React y Angular.

\subsubsection*{React}
De acuerdo con el sitio web de React\cite{react_react_2020}, es una biblioteca JavaScript de código abierto diseñada para crear interfaces de usuario con el objetivo de facilitar el desarrollo de \textit{single page applications}.

\paragraph*{Características}
\begin{enumerate}
    \item Virtual DOM: React usa un virtual DOM propio en lugar del navegador.
    \item Props: son definidos como atributos de configuración para cada componente.
    \item Estado de cada componente: lleva un registro de las propiedades y atributos del componente.
    \item Ciclos de vida: son la serie de estados por los cuales pasan los componentes \textit{statefull} a lo largo de su existencia. 
\end{enumerate}

\subsubsection*{Angular}
De acuerdo con el sitio web de Angular\cite{angular_angular_2020}, Angular es un web \textit{framework} desarrollado en TypeScript de código abierto mantenido por Google que se utiliza para crear y mantener \textit{single page aplications}. 

\begin{enumerate}
    \item Generación de código.
    \item Componentes.
    \item Ciclos de vida de componentes.
    
\end{enumerate}

\subsubsection*{Vue}

De acuerdo con la documentación de Vue.js\cite{noauthor_que_nodate}, Vue es un \textit{framework} progresivo para desarrollar interfaces de usuario; a diferencia de otros \textit{frameworks}, Vue está diseñado desde para ser utilizado incrementalmente.


La biblioteca central está enfocada solo en la capa de visualización y es fácil de utilizar e integrar con otras bibliotecas o proyectos existentes; por otro lado, Vue también es capaz de impulsar sofisticadas \textit{single-page applications} cuando se utiliza en combinación con bibliotecas de apoyo.


\paragraph*{Comparación con React}

React y Vue comparten muchas similitudes; ambos utilizan un DOM virtual, proporcionan componentes de vista reactivos, enrutamiento y la gestión global del estado manejado por bibliotecas asociadas.


Tanto React como Vue ofrecen un rendimiento comparable en los casos de uso más comunes, con Vue normalmente un poco por delante debido a su implementación más ligera del DOM virtual.


En Vue las dependencias de un componente se rastrean automáticamente durante su renderizado, por lo que el sistema sabe con precisión qué componentes deben volver a renderizarse cuando cambia el estado; se considera que cada componente tiene un \textit{shouldComponentUpdate} automáticamente implementado.


\paragraph*{Comparación con Angular}
En términos de rendimiento, ambos \textit{frameworks} son excepcionalmente rápidos y no hay suficientes datos de casos de uso en el mundo real para hacer un veredicto.


Vue es mucho menos intrusivo en las decisiones del desarrollador que Angular, ofreciendo soporte oficial para una variedad de sistemas de desarrollo, sin restricciones sobre cómo estructurar su aplicación; muchos desarrolladores disfrutan de esta libertad, mientras que algunos prefieren tener solo una forma correcta de desarrollar cualquier aplicación.


Para empezar con Vue, todo lo que se necesita es familiarizarse con HTML y ES5 JavaScript, mientras que la curva de aprendizaje de Angular es mucho más pronunciada. 


La complejidad de Angular se debe en su enfoque para diseñar aplicaciones grandes y complejas, pero eso hace que el \textit{framework} sea mucho más difícil de entender.



\subsubsection*{Nuxt.js}
De acuerdo a la documentación de Nuxt\cite{noauthor_what_nodate-1}, el objetivo de Nuxt.js es hacer que el desarrollo web en Vue sea eficaz con herramientas de desarrollo como Webpack, Babel y PostCSS; 

\paragraph*{Características}
\begin{enumerate}
    \item Manejo de archivos Vue (*.vue).
    \item División automática de código.
    \item Representación del lado del servidor.
    \item Potente sistema de enrutamiento con datos asincrónicos.
    \item Servicio de archivos estáticos.
    \item Soporte sintaxis ES2015+ (Javascript ES6).
    \item Gestión de elementos <head> <title>, <meta> y similares.
    \item Preprocesador: Sass, Less, Stylus, etc..
\end{enumerate}



\subsubsection*{Elección}

Se usará Vue/Nuxt por ser el \textit{framework} con el que el equipo está más acostumbrado, además de ser el máx flexible de las opciones expuestas.
\subsection{CSS Frameworks: Vuetify vs Bootstrap}
De acuerdo con Wikipedia\cite{wikipedia_contributors_css_2020}, un CSS \textit{framework} es una biblioteca de estilos genéricos usada para implementar diseños web y aportan una serie de utilidades que son aprovechadas frecuentemente en los distintos diseños web.

\subsubsection*{Vuetify}
De acuerdo con la documentación de Google\cite{noauthor_introduction_nodate}, Material Design es un lenguaje visual que sintetiza los principios clásicos del buen diseño respecto a las ideas de Google y en estos principios está basado Vuetify.


El objetivo de Material Design es crear un lenguaje visual que sintetice los principios clásicos del buen diseño, unificar el desarrollo de un único sistema subyacente para la experiencia del usuario en plataformas y dispositivos, así como personalizar el lenguaje visual de Material Design.


Asimismo, de acuerdo con la documentación de Vuetify\cite{noauthor_vuetify_nodate}, este CSS \textit{framework} está integrado para ser usado en los componentes de Vue/Nuxt como botones, barras de navegación, \textit{layouts} y demás.

\subsubsection*{Bootstrap}
De acuerdo con su documentacion oficial\cite{noauthor_documentation_nodate-1}, Bootstrap es un CSS \textit{framework} orientado al diseño responsivo de una aplicación web. 


Tiene \textit{templates} para botones, barras de navegación, estilos de tipografía entre otros; es de fácil integración con React, Angular o Vue y tiene una comunidad extensa por los años y popularidad que tiene.

\subsubsection*{Elección}

Se ha elegido usar en una primera instancia Vuetify porque es un CSS \textit{framework} que está integrado en las tecnologías asociadas de Vue, como Vue Router, Vue Meta; asimismo, sus componentes son simples de entender y de implementar.

\subsection{MySQL vs MongoDB}\label{ref:databases}
\subsubsection*{MongoDB}
De acuerdo con su documentación oficial\cite{mongodb_mongodb_2020}, MongoDB (del inglés humongous, ``enorme'') es un sistema de base de datos NoSQL, orientado a documentos y de código abierto.


En lugar de guardar los datos en tablas, tal y como se hace en las bases de datos relacionales, MongoDB guarda estructuras de datos BSON (una especificación similar a JSON) con un esquema dinámico, haciendo que la integración de los datos en ciertas aplicaciones sea más fácil y rápida.

\paragraph*{Características}
\begin{enumerate}
    \item Consultas \textit{ad hoc}: MongoDB soporta la búsqueda por campos, consultas de rangos y expresiones regulares.
    \item Indexación: es posible que cualquier campo en un documento de MongoDB sea indexado, al igual que es posible hacer índices secundarios. 
    \item Replicación: MongoDB soporta el tipo de replicación primario-secundario. 
    \item Balanceo de carga: MongoDB escala de forma horizontal usando el concepto de \textit{sharding}.
    \item Almacenamiento de archivos: MongoDB es utilizado como un sistema de archivos, aprovechando su capacidad para el balanceo de carga y la replicación de datos en múltiples servidores. 
    \item Agregación: MongoDB proporciona un framework de agregación que permite realizar operaciones similares a la operación \textit{group by} de SQL.
\end{enumerate}


\subsubsection*{MySQL}

De acuerdo con su documentación oficial\cite{mysql_mysql_2020}, MySQL es un gestor de base de datos relacionales de código abierto con un modelo cliente-servidor.


Archiva datos en tablas separadas en lugar de guardar todos los datos en un gran archivo, permitiendo tener mayor velocidad y flexibilidad; estas tablas están relacionadas de formas definidas, por lo que se hace posible combinar distintos datos en varias tablas y conectarlos.

\paragraph*{Características}
\begin{enumerate}
    \item Permite escojer múltiples motores de almacenamiento para cada tabla.
    \item Agrupación de transacciones, pudiendo reunirlas de forma múltiple desde varias conexiónes con el fin de incrementar el número de transacciones por segundo.
    \item Conectividad segura.
    \item Ejecución de transacciones y uso de claves foráneas.
    \item Presenta un amplio subconjunto del lenguaje SQL.
\end{enumerate}

\subsubsection*{Elección}

De acuerdo con Mosquera\cite{martinez-mosquera_modeling_2020}, MongoDB es la base de datos NoSQL orientada a documentos más popular y usada en los \textit{papers} de investigación; por ello el equipo ha decidido usar MongoDB como base de datos.


\subsection{Bibliotecas JavaScript para diagramado: GoJS vs Fabric.js vs D3.js}\label{ref:sec-gojs}
\subsubsection*{GoJS}
De acuerdo con la documentación de GoJS\cite{noauthor_gojs_nodate}, GoJS es una biblioteca de JavaScript y TypeScript para crear diagramas interactivos; permite crear todo tipo de diagramas, desde diagramas de flujo y organigramas hasta diagramas industriales altamente específicos, diagramas SCADA y BPMN, diagramas médicos como genogramas y diagramas de modelos de brotes. 


Ofrece muchas funciones para la interactividad del usuario, como arrastrar y soltar, copiar y pegar, edición de texto, información sobre herramientas, menús contextuales, diseños automáticos, plantillas, enlace de datos y modelos, gestión de estado, paletas, descripciones generales, controladores de eventos, comandos, herramientas extensibles para operaciones personalizadas y animaciones personalizables.


Está escrita en TypeScript y puede usarse como una biblioteca de JavaScript o incorporarse a su proyecto desde fuentes de TypeScript; normalmente se ejecuta completamente en el navegador, renderizando a un elemento \textit{canvas} HTML o SVG sin ningún requisito del lado del servidor.


\subsubsection*{Fabric.js}

De acuerdo con la documentación de Fabric\cite{noauthor_fabric_2020}, Fabric.js es una biblioteca de JavaScript que proporciona un modelo para trabajar sobre un \textit{canvas} HTML5 para poder agregar objetos como rectas, circunferencias, rectángulos, etc.

\paragraph*{Caracteristicas}
\begin{enumerate}
    \item Drag \& Drop integrado en cada objeto de Fabric.js
    \item Permite la especialización de clases para crear objetos personalizados.
\end{enumerate}


\subsubsection*{D3.js}

De acuerdo con la documentación de D3\cite{d3_d3js_2020}, D3.js es una biblioteca de JavaScript para producir infogramas dinámicos e interactivos en navegadores web. 

\paragraph*{Caracteristicas}
\begin{enumerate}
    \item Selecciones: es posible la selección de elementos del documento HTML y asignarle propiedades.
    \item Transiciones:  permiten interpolar en el tiempo valores de atributos, lo que produce cambios visuales en los infogramas.
    \item Asociación de datos: se asocia a cada elemento un objeto SVG con propiedades (forma, colores, valores) y comportamientos (transiciones, eventos).
\end{enumerate}

\subsubsection*{Elección}

Se llevó a la práctica en el prototipo funcional las tres opciones antes expuestas junto con algunas otras y se decidió usar GoJS para el proyecto por ser la biblioteca JavaScript más completa para diagramado y que genera los diagramas con un JSON simple para parsear esos datos y realizar la conversión.




\subsection{Python 3 vs Java}
De acuerdo con Mark Lutz\cite{lutz_learning_2013}, Python es un lenguaje de programación interpretado, interactivo y orientado a objetos. Incorpora módulos, excepciones, tipeo dinámico, tipos de datos dinámicos y clases.


Tiene sintaxis clara, interfaces para muchas llamadas de sistema y bibliotecas, así como para varios sistemas de ventanas. Además, es extensible en C o C++. 


También se puede usar como un lenguaje de extensión para aplicaciones que necesitan una interfaz programable y es portátil: se ejecuta en muchas variantes de Unix, en Mac y en Windows 2000 y versiones posteriores.


La biblioteca estándar del lenguaje, cubre áreas como el procesamiento de cadenas (expresiones regulares, Unicode, cálculo de diferencias entre archivos), protocolos de Internet (HTTP, FTP, SMTP, XML-RPC, POP, IMAP, programación CGI), ingeniería de software (pruebas unitarias, registro, creación de perfiles, análisis del código Python) e interfaces del sistema operativo (llamadas al sistema, sistemas de archivos, sockets TCP/IP).

\subsubsection*{Fortalezas}

\paragraph*{Es orientado a objetos y funcional}
Python es un lenguaje orientado a objetos; su modelo de clase admite nociones avanzadas como el polimorfismo, la sobrecarga del operador y la herencia múltiple; sin embargo, en el contexto de la simple sintaxis y escritura de Python, la programación orientada a objetos es fácil de aplicar.


Además de servir para la estructuración y reutilización de código, la naturaleza orientada a objetos de Python lo hace ideal como herramienta de secuencias de comandos para otros lenguajes de sistemas orientados a objetos. Por ejemplo, con el código apropiado, los programas Python pueden especializar clases implementadas en C++, Java y C\#.


No obstante, la programación orientada a objetos es una opción en Python. Al igual que C++, Python admite modos de programación tanto procedimentales como orientados a objetos. Las herramientas orientadas a objetos se pueden aplicar siempre que las restricciones lo permitan. 


Además de sus paradigmas originales de procedimientos (basados ​​en declaraciones) y orientados a objetos (basados ​​en clases), Python en los últimos años ha adquirido soporte incorporado para la programación funcional, un conjunto que incluye generadores, comprensiones, cerraduras, mapas, decoradores , funciones anónimas lambdas.

\paragraph*{Es extensible}
Su conjunto de herramientas lo ubica entre los lenguajes de \textit{scripting} tradicionales como Tcl, Scheme y Perl; y los lenguajes de desarrollo de sistemas como C, C ++ y Java.


Python proporciona toda la simplicidad y facilidad de uso de un lenguaje de programación, junto con herramientas de ingeniería de software más avanzadas que normalmente se encuentran en lenguajes compilados.

A diferencia de algunos lenguajes de secuencias de comandos, esta combinación hace que Python sea útil para proyectos de desarrollo a gran escala. Algunas de las herramientas de Python son:
\paragraph*{Escritura dinámica}
Python realiza un seguimiento de los tipos de objetos que utiliza su programa cuando se ejecuta; eso no requiere declaraciones complicadas de tipo y tamaño en su código. De hecho, no existe una declaración de tipo o variable en Python. 


Debido a que el código Python no restringe los tipos de datos, también se aplica automáticamente a toda una gama de objetos.

\paragraph*{Gestión automática de la memoria}
Python asigna automáticamente objetos y los reclama el recolector de basura cuando ya no se usan y la mayoría puede crecer y reducirse según la demanda. Es decir, Python realiza un seguimiento de los detalles de la memoria de bajo nivel.


\paragraph*{Tipos de objetos incorporados}
Python proporciona estructuras de datos de uso común como listas, diccionarios y cadenas como partes intrínsecas del lenguaje. Son flexibles y fáciles de usar. Por ejemplo, los objetos integrados pueden crecer y reducirse según demanda, pueden anidarse arbitrariamente para representar información compleja, y más.
\paragraph*{Herramientas incorporadas}
Para procesar todos esos tipos de objetos, Python viene con operadores potentes y estándar, que incluyen concatenación (unir colecciones), segmentar (extraer secciones), ordenar, mapear y más.
\paragraph*{Utilidades de biblioteca}
Para tareas más específicas, Python también viene con una gran colección de herramientas de biblioteca precodificadas que admiten todo, desde la coincidencia de expresiones regulares hasta la creación de redes. Una vez que aprende el lenguaje en sí, las herramientas de la biblioteca de Python son donde ocurre gran parte de la acción a nivel de aplicación.
\paragraph*{Utilidades de terceros}
Debido a que Python es de código abierto, los desarrolladores pueden contribuir con herramientas precodificadas que admitan tareas que aún no son herramientas estándar; en la Web, encontrará soporte gratuito para COM, imágenes, programación numérica, XML y acceso a bases de datos.
\subsection{Back end: Django vs Flask} \label{sec:flask}
Para elegir el lenguaje a usar para el \textit{back end} se realizó un estudio de lenguajes apropiados para usar con un JavaScript web \textit{framework}.

\subsubsection*{Flask}
De acuerdo con su documentación oficial\cite{projects_flask_2020}, Flask es un \textit{micro web framework} escrito en Python; se clasifica como micro porque no requiere herramientas o bibliotecas particulares; está basado en la especificación WSGI de Werkzeug, el motor de templates Jinja2 y tiene una licencia BSD.


No tiene capa de abstracción de base de datos, validación de formularios ni ningún otro componente donde las bibliotecas de terceros preexistentes brinden funciones comunes; sin embargo, Flask admite extensiones que agregan características de la aplicación como si se implementaran en el propio Flask.


Existen extensiones para mapear relaciones de objetos, validación de formularios, manejo de carga, varias tecnologías de autenticación de licencia libre.


\paragraph*{Características}

\begin{enumerate}
    \item Es un \textit{framework} que se destaca en instalar extensiones o complementos de acuerdo al tipo de proyecto que se va a desarrollar, es decir, es perfecto para el prototipado rápido de proyectos.
    \item Incluye un servidor web, así podemos evitamos instalar uno como Apache o Nginx; además, ofrece soporte para pruebas unitarias y para \textit{cookies}, apoyándose en el motor de plantillas ​Jinja2​.
    \item Su velocidad es mejor a comparación de Django; generalmente el desempeño que tiene Flask es superior debido a su diseño minimalista que tiene en su estructura.
    \item  Flask permite combinarse con herramientas para potenciar su funcionamiento, por ejemplo: Jinja2, SQLAlchemy, Mako y Peewee.
\end{enumerate}


\subsubsection*{Django}

De acuerdo con su documentación oficial\cite{projects_flask_2020}, Django es un \textit{framework} de desarrollo web de código abierto escrito en Python, que sigue el patrón de diseño conocido como MVC (Modelo–Vista–Controlador).

\paragraph*{Características}
\begin{enumerate}
    \item Aplicaciones ``enchufables'' que pueden instalarse en cualquier página gestionada con Django.
    \item Una API de base de datos robusta.
    \item Un sistema incorporado de ``vistas genéricas'' que ahorra tener que escribir la lógica de ciertas tareas comunes.
    \item Un sistema extensible de plantillas basado en etiquetas, con herencia de plantillas.
    \item Un despachador de URL basado en expresiones regulares.
    \item Un sistema \textit{middleware} para desarrollar características adicionales.
    \item Documentación incorporada accesible a través de la aplicación administrativa.
\end{enumerate}

\subsubsection*{Elección}

Se ha elegido Flask por ser un \textit{framework} web conocido por el equipo, porque es ideal para el prototipado rápido de proyectos y al equipo le ha dado resultados en proyectos anteriores.
\section{Conclusiones}


Se ha elegido Flask por ser un \textit{framework} web conocido por el equipo, porque es ideal para el prototipado rápido de proyectos y al equipo le ha dado resultados en proyectos anteriores.

De acuerdo con el sitio Stack Overflow\cite{noauthor_stack_nodate}, JavaScript es el lenguaje más popular de 2019 y aunque TypeScript es de los lenguajes que tienen un mayor nivel de aceptación, se usará JavaScript no solo por ser el lenguaje más popular y, en consecuencia, con más compatibilidad y material de ayuda, sino también porque el equipo está acostumbrado a este lenguaje y tiene experiencia con web \textit{frameworks} escritos en JavaScript.

De acuerdo con Mosquera\cite{martinez-mosquera_modeling_2020}, MongoDB es la base de datos NoSQL orientada a documentos más popular y usada en los \textit{papers} de investigación; por ello el equipo ha decidido usar MongoDB como base de datos.

Se llevó a la práctica en el prototipo funcional las tres opciones antes expuestas junto con algunas otras y se decidió usar GoJS para el proyecto por ser la biblioteca JavaScript más completa para diagramado y que genera los diagramas con un JSON simple para parsear esos datos y realizar la conversión.


Se ha elegido usar en una primera instancia Vuetify porque es un CSS \textit{framework} que está integrado en las tecnologías asociadas de Vue, como Vue Router, Vue Meta; asimismo, sus componentes son simples de entender y de implementar.


Tomando en cuenta la experiencia del equipo con CSS, además de que el proyecto no se enfocará en hacer muchas hojas de estilo para cada componente, página o vista de la aplicación web, se usará CSS en lugar de Sass.

Se usará Vue/Nuxt por ser el \textit{framework} con el que el equipo está más acostumbrado, además de ser el máx flexible de las opciones expuestas.

Se ha elegido Python para desarrollar los algoritmos del proyecto dado que es multiplataforma y es de fácil integración con el \textit{framework} web elegido para el \textit{back end}.
