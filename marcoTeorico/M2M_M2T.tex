\subsection{Transformación entre modelos}
De acuerdo con Kapferer\cite{kapferer_model_2019}, hay diferentes tipos de transformaciones entre modelos, siendo las más importantes endógena/exógena, \textit{in-place/out-place} y horizontal/vertical. 

\subsubsection{Endógena vs exógena}
Para comparar las transformaciones del modelo, se debe hacer una primera distinción importante entre las transformaciones de un mismo lenguaje y las transformaciones entre diferentes lenguajes; una transformación se llama endógena si un modelo se transforma en otro modelo en el mismo lenguaje o metamodelo. Por otro lado, una transformación se denomina exógena si el modelo de origen y de destino no están representados en el mismo lenguaje.


Usualmente, Las transformaciones endógenas son optimizaciones o refactorizaciones donde se mejoran ciertos atributos de calidad de un modelo manteniendo el lenguaje de representación y la semántica; un ejemplo de transformación exógena podría ser la migración de un programa de un lenguaje a otro.

\subsubsection{In-place vs out-place}

Esta distinción se refiere únicamente a las transformaciones endógenas. Una transformación endógena se denomina \textit{in-place} si el modelo de origen y el de destino son el mismo, lo que significa que la transformación opera directamente en el modelo de entrada. Si una transformación endógena utiliza un modelo como fuente, pero crea o cambia otro modelo, implicando que hay más de un modelo en juego se denomina \textit{out-place}. Las transformaciones exógenas siempre son \textit{out-place}.

\subsubsection{Horizontal vs vertical}

Esta distinción entre transformaciones de modelos se refiere al nivel de abstracción. Si el modelo de origen y destino de una transformación se encuentran en el mismo nivel de abstracción, se denomina horizontal, mientras que las transformaciones entre diferentes niveles de abstracción se denominan verticales.

\subsubsection{Transformación modelo a modelo y modelo a texto}


De acuerdo con Scherp\cite{scherp_framework_2013}, una transformación de modelo se define con un lenguaje de transformación que generalmente proporciona una definición de reglas de transformación.

Una regla de transformación define la asignación de elementos de metamodelo de origen particulares a elementos de metamodelo de destino (transformación M2M) o texto (transformación M2T).

Los lenguajes de transformación se pueden distinguir entre lenguajes de transformación imperativos/operacionales y lenguajes de transformación declarativos/relacionales.


