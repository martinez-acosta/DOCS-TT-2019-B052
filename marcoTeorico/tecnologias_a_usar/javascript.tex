\subsection{JavaScript vs TypeScript}

\subsubsection*{JavaScript}
De acuerdo con la documentación de Mozilla\cite{noauthor_javascript_nodate}, JavaScript es una marca registrada con licencia de Sun Microsystems (ahora Oracle) que se usa para describir la implementación del lenguaje de programación JavaScript.


Debido a problemas de registro de marcas en la Asociación Europea de Fabricantes de Computadoras, la versión estandarizada del lenguaje tiene el nombre de ECMAScript, sin embargo, en la práctica se conoce como lenguaje JavaScript. 


Abreviado como JS, es un lenguaje ligero e interpretado, orientado a objetos, basado en prototipos, imperativo, débilmente tipado y dinámico; es usado en node.js, Apache CouchDB y Adobe Acrobat.


EL núcleo del lenguaje JavaScript está estandarizado por el Comité ECMA TC39 como un lenguaje llamado ECMAScript. La última versión de la especificación es ECMAScript 6.0.


El estándar ECMAScript define 

\begin{enumerate}
    \item Sintaxis: reglas de análisis, palabras clave, flujos de control, inicialización literal de objetos.
    \item Mecanismos de control de errores: \textit{throw, try/catch}, habilidad para crear tipos de Errores definidos por el usuario.
    \item Tipos: \textit{boolean, number, string, function, object}.
    \item  Objetos globales: en un navegador, los objetos globales son los objetos de la ventana, pero ECMAScript solo define API no especificas para navegadores, como \textit{parseInt, parseFloat, decodeURI, encodeURI}.
    \item Mecanismo de herencia basada en prototipos.
    \item Objetos y funciones incorporadas.
    \item Modo estricto.
\end{enumerate}

La sintaxis básica es similar a Java y C++ con la intención de reducir el número de nuevos conceptos necesarios para aprender el lenguaje. Las construcciones del lenguaje, tales como sentencias \textit{if}, bucles \textit{for} y \textit{while}, bloques \textit{switch} y \textit{try catch} funcionan de la misma manera que en estos lenguajes (o casi).


JavaScript puede funcionar como lenguaje procedimental y como lenguaje orientado a objetos. Los objetos se crean programáticamente añadiendo métodos y propiedades a lo que de otra forma serían objetos vacíos en tiempo de ejecución, en contraposición a las definiciones sintácticas de clases comunes en los lenguajes compilados como C++ y Java. Una vez se ha construido un objeto, puede usarse como modelo (o prototipo) para crear objetos similares.


Las capacidades dinámicas de JavaScript incluyen construcción de objetos en tiempo de ejecución, listas variables de parámetros, variables que pueden contener funciones, creación de scripts dinámicos (mediante eval), introspección de objetos (mediante \textit{for ... in}), y recuperación de código fuente (los programas de JavaScript pueden decompilar el cuerpo de funciones a su código fuente original).


Desde el 2012, todos los navegadores modernos soportan completamente ECMAScript 5.1. Los navegadores más antiguos soportan por lo menos ECMAScript 3. 


El 17 de Julio de 2015, ECMA International publicó la sexta versión de ECMAScript, la cual es oficialmente llamada ECMAScript 2015 y fue inicialmente nombrada como ECMAScript 6 o ES6. Desde entonces, los estándares ECMAScript están en ciclos de lanzamiento anuales.


\subsubsection*{TypeScript}
De acuerdo con TypeScript Publishing\cite{typescript_publishing_typescript_2019}, TypeScript es por definición es JavaScript para el desarrollo de aplicaciones, siendo también un superconjunto del mismo.


TypeScript es un lenguaje compilado orientado a objetos. Fue diseñado por Anders Hejlsberg (diseñador de C\#) en Microsoft. TypeScript es tanto un lenguaje como un conjunto de herramientas. TypeScript es un superconjunto de JavaScript que genera código JavaScript.


\paragraph*{Características}
\begin{itemize}
    \item Compilación: cuenta con un transpilador para la verificación de errores si hay errores de compilación, cosa que no es posible con JavaScript.
    \item Tipeo estático fuerte: provee un sistema opcional de tipeo estático y de inferencia de tipos a traves del TypeScript Language Service, lo que permite inferir el tipo de una variabla declara sin tipo en función de su valor.
    \item Definiciones de tipo: permite la extensión del lenguaje con bibliotecas externas JavaScript.
    \item Programación orientaba a objetos: admite conceptos como clases, interfaces, herencia, etc.
\end{itemize}



\subsubsection*{Elección}

De acuerdo con el sitio stack overflow\cite{noauthor_stack_nodate}, JavaScript es el lenguaje más popular de 2019 y aunque TypeScript es de los lenguajes que tienen un mayor nivel de aceptación, se usará JavaScript no solo por ser el lenguaje más popular y, en consecuencia, con más compatibilidad y material de ayuda, sino también porque el equipo está acostumbrado a este lenguaje.

