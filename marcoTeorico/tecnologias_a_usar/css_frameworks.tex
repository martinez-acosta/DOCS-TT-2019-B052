\subsection{CSS Frameworks: Vuetify vs Bootstrap}
De acuerdo con Wikipedia\cite{wikipedia_contributors_css_2020}, un CSS \textit{framework} es una biblioteca de estilos genéricos que puede ser usada para implementar diseños web y aportan una serie de utilidades que pueden ser aprovechadas frecuentemente en los distintos diseños web.
\subsubsection*{Vuetify}
De acuerdo a la documentación de Google\cite{noauthor_introduction_nodate}, Material Design es es un lenguaje visual que sintetiza los principios clásicos del buen diseño respecto a las ideas de Google y en estos principios está basado Vuetify.


El objetivo de Material Design es crear un lenguaje visual que sintetice los principios clásicos del buen diseño, unificar el desarrollo de un único sistema subyacente que unifique la experiencia del usuario en plataformas y dispositivos, así como personalizar el lenguaje visual de Material Design.


Asimismo, de acuerdo con la documentación de Vuetify\cite{noauthor_vuetify_nodate}, este CSS \textit{framework} está integrado para ser usado en los componentes de Vue/Nuxt, desde botones, barras de navegación, \textit{layouts} y demás.

\subsubsection*{Bootstrap}

De acuerdo a la documentacion de Bootstrap\cite{noauthor_documentation_nodate-1}, es un CSS \textit{framework} orientado al diseño responsivo de una aplicación web. 


Tiene \textit{templates} para botones, barras de navegación, estilos de tipografía entre otros. Es de fácil integración con React, AngularJS o Vue y tiene una comunidad extensa por los años y popularidad que tiene.

\subsubsection*{Elección}

Se ha elegido usar en una primera instancia Vuetify porque es un CSS \textit{framework} que está integrado en las tecnologías asociadas de Vue, como Vue Router, Vue Meta. Asimismo, sus componentes son simples de entender y de implementar.
