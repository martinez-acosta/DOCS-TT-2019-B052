\subsection{Bibliotecas JavaScript para diagramado: GoJS vs Fabric.js vs D3.js}
\subsubsection*{GoJS}
De acuerdo a la documentación de GoJS\cite{noauthor_gojs_nodate}, GoJS es una biblioteca de JavaScript y TypeScript para crear diagramas y gráficos interactivos.


GoJS le permite crear todo tipo de diagramas y gráficos para sus usuarios, desde simples diagramas de flujo y organigramas hasta diagramas industriales altamente específicos, diagramas SCADA y BPMN, diagramas médicos como genogramas y diagramas de modelos de brotes, y más. 


GoJS facilita la construcción de diagramas JavaScript de nodos complejos, enlaces y grupos con plantillas y diseños personalizables.


GoJS ofrece muchas funciones avanzadas para la interactividad del usuario, como arrastrar y soltar, copiar y pegar, edición de texto en el lugar, información sobre herramientas, menús contextuales, diseños automáticos, plantillas, enlace de datos y modelos, gestión de estado y deshacer transaccional, paletas , descripciones generales, controladores de eventos, comandos, herramientas extensibles para operaciones personalizadas y animaciones personalizables.


GoJS se implementa en TypeScript y puede usarse como una biblioteca de JavaScript o incorporarse a su proyecto desde fuentes de TypeScript. 


GoJS normalmente se ejecuta completamente en el navegador, renderizando a un elemento HTML Canvas o SVG sin ningún requisito del lado del servidor.


\subsubsection*{Fabric.js}

De acuerdo con la documentación de Fabric\cite{noauthor_fabric_2020}, Fabric.js es una biblioteca de JavaScript que proporciona un modelo para trabajar sobre un canvas HTML5 para poder agregar objetos como rectas, circunferencias, rectángulos, etc.

\paragraph*{Caracteristicas}
\begin{enumerate}
    \item Drag \& Drop integrado en cada objeto de Fabric.js
    \item Permite la especialización de clases para crear objetos personalizados.
\end{enumerate}


\subsubsection*{D3.js}

De acuerdo con Wikipedia\cite{wikipedia_contributors_d3js_2020}, D3.js es es una biblioteca de JavaScript para producir a partir de datos infogramas dinámicos e interactivos en navegadores web. 

\paragraph*{Caracteristicas}
\begin{enumerate}
    \item Selecciones: se pueden seleccionar elementos del documento HTML y asignarle propiedades.
    \item Transiciones:  permiten interpolar en el tiempo valores de atributos, lo que produce cambios visuales en los infogramas.
    \item Asociación de datos: se asocia a cada elemento un objeto SVG con propiedades (forma, colores, valores) y comportamientos (transiciones, eventos).
\end{enumerate}

\subsubsection*{Elección}

Se llevó a la práctica en el prototipo funcional las tres opciones antes expuestas junto con algunas otras y se decidió usar GoJS para el proyecto por ser la biblioteca JavaScript más completa para diagramado y que genere los diagramas con un JSON simple para parsear esos datos y realizar la conversión.



