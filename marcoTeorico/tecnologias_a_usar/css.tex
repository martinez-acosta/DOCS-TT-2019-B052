\subsection{Style Sheet Language: CSS vs SASS}
De acuerdo con Lie\cite{lie_cascading_2005}, un \textit{style sheet language} es un lenguaje que representa los estilos o elementos visuales de documentos estructurados.

\subsubsection*{CSS}
De acuerdo con la documentación de la W3C\cite{noauthor_what_nodate}, Cascading Style Sheets (CSS, o hojas de estilo en cascada) es un lenguaje de diseño gráfico para definir y crear la presentación de un documento estructurado escrito en un lenguaje de marcado.


La separación entre el contenido del documento y la presentación busca mejorar la accesibilidad, proveer más flexibilidad y control, permitir que varios documentos HTML compartan un mismo estilo usando una sola hoja de estilos separada en un archivo .css, reducir la complejidad y la repetición de código.


La especificación CSS describe un esquema prioritario para determinar qué reglas de estilo se aplican si más de una regla coincide para un elemento en particular; estas reglas son aplicadas con un sistema llamado de cascada, de modo que las prioridades son calculadas y asignadas a las reglas, así que los resultados son predecibles.


La especificación CSS es mantenida por el World Wide Web Consortium; el tipo MIME \textit{text/css} está registrado para su uso por CSS descrito en el RFC 23185​. 


CSS se ha creado en varios niveles y perfiles, donde cada nivel de CSS se construye sobre el anterior, generalmente añadiendo funciones al nivel previo.


La última versión del estándar, CSS3.1, está dividida en varios documentos separados, llamados ``módulos''; cada módulo añade nuevas funcionalidades a las definidas en CSS2, de manera que se preservan las anteriores para mantener la compatibilidad.


Los trabajos en CSS3.1 comenzaron a la vez que se publicó la recomendación oficial de CSS2 y su primeros borradores fueron liberados en junio de 1999.


Debido a la modularización de CSS3.1, diferentes módulos están en diferentes estados de su desarrollo,​ hay alrededor de cincuenta módulos publicados,​ tres de ellos se convirtieron en recomendaciones oficiales de la W3C en 2011: ``selectores'', ``espacios de nombres'' y ``color''.


Respecto al soporte de los navegadores web, cada navegador web usa un motor de renderizado para renderizar páginas web y el soporte de CSS no es exactamente igual en ninguno de los motores de renderizado; ya que los navegadores no aplican el CSS correctamente, muchas técnicas de programación han sido desarrolladas para ser aplicadas por un navegador específico (comúnmente conocida esta práctica como \textit{CSS hacks} o \textit{CSS filters}).


\subsubsection*{Sass}

De acuerdo con su documentación oficial\cite{noauthor_documentation_nodate}, Sass es un lenguaje de preprocesado que genera hojas de estilo en cascada (CSS) y consta de dos sintaxis.

\subsubsection*{Características}

\paragraph*{Variables}
Las variables comienzan con un signo de dólar y la asignación de valor se realiza con dos puntos y permite 4 tipos de datos:

\begin{enumerate}
    \item Números (incluyendo las unidades).
    \item Strings (con comillas o sin ellas).
    \item Colores (código o nombre).
    \item Booleanos.
\end{enumerate}
Las variables son resultados o argumentos de varias funciones disponibles.

\paragraph*{Anidamiento}
CSS soporta anidamiento lógico, pero los bloques de código no son anidados; Sass permite que el código anidado sea insertado dentro de cualquier otro bloque.


\paragraph*{Mixins}

Como CSS no soporta \textit{mixins}, cualquier código duplicado debe ser repetido en cada lugar; un \textit{mixin} en Sass es una sección de código que contiene código Sass. 


Cada vez que se llama un \textit{mixin} en el proceso de conversión, el contenido del mismo es insertado en el lugar de la llamada; los \textit{mixin} permiten una solución limpia a las repeticiones de código, así como una forma fácil de alterar el mismo.

\subsubsection*{Elección}

Tomando en cuenta la experiencia del equipo con CSS, además de que el proyecto no se enfocará en hacer muchas hojas de estilo para cada componente, página o vista de la aplicación web, se usará CSS en lugar de Sass.