\subsection{MongoDB vs Apache CouchDB}
\subsubsection*{MongoDB}
De acuerdo a la documentación de Wikipedia\cite{wikipedia_mongodb_2020}, MongoDB (del inglés humongous, ``enorme") es un sistema de base de datos NoSQL, orientado a documentos y de código abierto.


En lugar de guardar los datos en tablas, tal y como se hace en las bases de datos relacionales, MongoDB guarda estructuras de datos BSON (una especificación similar a JSON) con un esquema dinámico, haciendo que la integración de los datos en ciertas aplicaciones sea más fácil y rápida.

\paragraph*{Características}
\begin{enumerate}
    \item Consultas ad hoc: MongoDB soporta la búsqueda por campos, consultas de rangos y expresiones regulares.
    \item Indexación: cualquier campo en un documento de MongoDB puede ser indexado, al igual que es posible hacer índices secundarios. 
    \item Replicación: MongoDB soporta el tipo de replicación primario-secundario. 
    \item Balanceo de carga; MongoDB puede escalar de forma horizontal usando el concepto de \textit{sharding}.
    \item Almacenamiento de archivos: MongoDB puede ser utilizado como un sistema de archivos, aprovechando la capacidad de MongoDB para el balanceo de carga y la replicación de datos en múltiples servidores. 
    \item Agregación: MongoDB proporciona un framework de agregación que permite realizar operaciones similares al GROUP BY de SQL.
\end{enumerate}


\subsubsection*{Apache CouchDB}
