\subsection{Back end: Django vs Flask} \label{sec:flask}
Para elegir el lenguaje a usar para el \textit{back end} se realizó un estudio de lenguajes apropiados para usar con un JavaScript web \textit{framework}.

\subsubsection*{Flask}
De acuerdo con su documentación oficial\cite{projects_flask_2020}, Flask es un \textit{micro web framework} escrito en Python; se clasifica como micro porque no requiere herramientas o bibliotecas particulares; está basado en la especificación WSGI de Werkzeug, el motor de templates Jinja2 y tiene una licencia BSD.


No tiene capa de abstracción de base de datos, validación de formularios ni ningún otro componente donde las bibliotecas de terceros preexistentes brinden funciones comunes; sin embargo, Flask admite extensiones que agregan características de la aplicación como si se implementaran en el propio Flask.


Existen extensiones para mapear relaciones de objetos, validación de formularios, manejo de carga, varias tecnologías de autenticación de licencia libre.


\paragraph*{Características}

\begin{enumerate}
    \item Es un \textit{framework} que se destaca en instalar extensiones o complementos de acuerdo al tipo de proyecto que se va a desarrollar, es decir, es perfecto para el prototipado rápido de proyectos.
    \item Incluye un servidor web, así podemos evitamos instalar uno como Apache o Nginx; además, ofrece soporte para pruebas unitarias y para \textit{cookies}, apoyándose en el motor de plantillas ​Jinja2​.
    \item Su velocidad es mejor a comparación de Django; generalmente el desempeño que tiene Flask es superior debido a su diseño minimalista que tiene en su estructura.
    \item  Flask permite combinarse con herramientas para potenciar su funcionamiento, por ejemplo: Jinja2, SQLAlchemy, Mako y Peewee.
\end{enumerate}


\subsubsection*{Django}

De acuerdo con su documentación oficial\cite{projects_flask_2020}, Django es un \textit{framework} de desarrollo web de código abierto escrito en Python, que sigue el patrón de diseño conocido como MVC (Modelo–Vista–Controlador).

\paragraph*{Características}
\begin{enumerate}
    \item Aplicaciones ``enchufables'' que pueden instalarse en cualquier página gestionada con Django.
    \item Una API de base de datos robusta.
    \item Un sistema incorporado de ``vistas genéricas'' que ahorra tener que escribir la lógica de ciertas tareas comunes.
    \item Un sistema extensible de plantillas basado en etiquetas, con herencia de plantillas.
    \item Un despachador de URL basado en expresiones regulares.
    \item Un sistema \textit{middleware} para desarrollar características adicionales.
    \item Documentación incorporada accesible a través de la aplicación administrativa.
\end{enumerate}

\subsubsection*{Elección}

Se ha elegido Flask por ser un \textit{framework} web conocido por el equipo, porque es ideal para el prototipado rápido de proyectos y al equipo le ha dado resultados en proyectos anteriores.