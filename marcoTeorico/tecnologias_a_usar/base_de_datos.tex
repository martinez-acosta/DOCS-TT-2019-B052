\subsection{MySQL vs MongoDB}\label{ref:databases}
\subsubsection*{MongoDB}
De acuerdo con su documentación oficial\cite{mongodb_mongodb_2020}, MongoDB (del inglés humongous, ``enorme'') es un sistema de base de datos NoSQL, orientado a documentos y de código abierto.


En lugar de guardar los datos en tablas, tal y como se hace en las bases de datos relacionales, MongoDB guarda estructuras de datos BSON (una especificación similar a JSON) con un esquema dinámico, haciendo que la integración de los datos en ciertas aplicaciones sea más fácil y rápida.

\paragraph*{Características}
\begin{enumerate}
    \item Consultas \textit{ad hoc}: MongoDB soporta la búsqueda por campos, consultas de rangos y expresiones regulares.
    \item Indexación: es posible que cualquier campo en un documento de MongoDB sea indexado, al igual que es posible hacer índices secundarios. 
    \item Replicación: MongoDB soporta el tipo de replicación primario-secundario. 
    \item Balanceo de carga: MongoDB escala de forma horizontal usando el concepto de \textit{sharding}.
    \item Almacenamiento de archivos: MongoDB es utilizado como un sistema de archivos, aprovechando su capacidad para el balanceo de carga y la replicación de datos en múltiples servidores. 
    \item Agregación: MongoDB proporciona un framework de agregación que permite realizar operaciones similares a la operación \textit{group by} de SQL.
\end{enumerate}


\subsubsection*{MySQL}

De acuerdo con su documentación oficial\cite{mysql_mysql_2020}, MySQL es un gestor de base de datos relacionales de código abierto con un modelo cliente-servidor.


Archiva datos en tablas separadas en lugar de guardar todos los datos en un gran archivo, permitiendo tener mayor velocidad y flexibilidad; estas tablas están relacionadas de formas definidas, por lo que se hace posible combinar distintos datos en varias tablas y conectarlos.

\paragraph*{Características}
\begin{enumerate}
    \item Permite escojer múltiples motores de almacenamiento para cada tabla.
    \item Agrupación de transacciones, pudiendo reunirlas de forma múltiple desde varias conexiónes con el fin de incrementar el número de transacciones por segundo.
    \item Conectividad segura.
    \item Ejecución de transacciones y uso de claves foráneas.
    \item Presenta un amplio subconjunto del lenguaje SQL.
\end{enumerate}

\subsubsection*{Elección}

De acuerdo con Mosquera\cite{martinez-mosquera_modeling_2020}, MongoDB es la base de datos NoSQL orientada a documentos más popular y usada en los \textit{papers} de investigación; por ello el equipo ha decidido usar MongoDB como base de datos.

