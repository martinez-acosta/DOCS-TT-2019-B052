\subsection{HTML 5 }

De acuerdo con la documentación de Mozilla\cite{noauthor_html_nodate}, HyperText Markup Language (HTML o lenguaje de marcado de hipertextos) es la pieza más básica en la construcción de la web, usada para definir el sentido y estructura del contenido en una página web. 


Es un estándar a cargo del World Wide Web Consortium (W3C o Consorcio WWW), organización dedicada a la estandarización de casi todas las tecnologías ligadas a la web. 


HTML hace uso de enlaces que conectan las páginas web entre sí, ya sea dentro de un mismo sitio web o entre diferentes sitios web.


Un elemento HTML se separa de otro texto en un documento por medio de ``etiquetas'', las cuales consisten en elementos rodeados por ``<,>''.


HTML 5 (HyperText Markup Language, versión 5) es la última revisión importante del lenguaje HTML en el que establece elementos y atributos que reflejan el uso de sitios web modernos. 


Características:

\begin{enumerate}
    \item Incorpora etiquetas: \textit{canvas} 2D \& 3D, audio y vídeo con códecs para mostrar los contenidos multimedia; actualmente hay una lucha entre imponer códecs libres (WebM + VP8) o privados (H.264/MPEG-4 AVC).
    \item Etiquetas para manejar grandes conjuntos de datos: permiten generar tablas dinámicas para filtrar, ordenar y ocultar contenido en cliente.
    \item Mejoras en los formularios: nuevos tipos de datos como \textit{email, number, url, datetime}  y facilidades para validar contenido sin JavaScript.
    \item Visores: MathML (fórmulas matemáticas) y SVG (gráficos vectoriales).
    \item Drag \& Drop: nueva funcionalidad para arrastrar objetos como imágenes.
\end{enumerate}

Respecto a la compatibilidad con los navegadores, la mayoría de elementos de HTML5 son compatibles con Firefox 19, Chrome 25, Safari 6 y Opera 12 en adelante.

