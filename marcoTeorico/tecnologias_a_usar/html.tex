\subsection{HTML 5 }

De acuerdo a la documentación de Mozilla\cite{noauthor_html_nodate}, HyperText Markup Language, abreviado HTML, o lenguaje de marcado de hipertextos, es la pieza más básica para la construcción de la web y se usa para definir el sentido y estructura del contenido en una página web. 


Es un estándar a cargo del World Wide Web Consortium (W3C) o Consorcio WWW, organización dedicada a la estandarización de casi todas las tecnologías ligadas a la web. 


HTML hace uso de enlaces que conectan las páginas web entre sí, ya sea dentro de un mismo sitio web o entre diferentes sitios web.


Un elemento HTML se separa de otro texto en un documento por medio de ``etiquetas", las cuales consisten en elementos rodeados por ``<,>".


Ejemplos de estas etiquetas son <head>, <title>, <body>, <header>, <p>, <ul>, <ol>, <li>, y otros más.


HTML 5 (HyperText Markup Language, versión 5) es la quinta revisión importante del lenguaje básico de la World Wide Web, HTML.


HTML5 establece elementos y atributos que reflejan el uso típico de los sitios web modernos. Algunos de ellos son técnicamente similares a las etiquetas <div> y <span>, pero tienen un significado semántico, como por ejemplo <nav> (bloque de navegación del sitio web) y <footer>.


Características:

\begin{enumerate}
    \item Incorpora etiquetas: \textit{canvas} 2D y 3D, audio, vídeo con codecs para mostrar los contenidos multimedia. Actualmente hay una lucha entre imponer codecs libres (WebM + VP8) o privados (H.264/MPEG-4 AVC).
    \item Etiquetas para manejar grandes conjuntos de datos: Datagrid, Details, Menu y Command. Permiten generar tablas dinámicas que pueden filtrar, ordenar y ocultar contenido en cliente.
    \item Mejoras en los formularios: Nuevos tipos de datos (\textit{email, number, url, datetime})  y facilidades para validar el contenido sin JavaScript.
    \item Visores: MathML (fórmulas matemáticas) y SVG (gráficos vectoriales); en general se deja abierto a poder interpretar otros lenguajes XML.
    \item Drag \& Drop: nueva funcionalidad para arrastrar objetos como imágenes.
\end{enumerate}

Respecto a la compatibilidad con los navegadores, la mayoría de elementos de HTML5 son compatibles con Firefox 19, Chrome 25, Safari 6 y Opera 12 en adelante.

