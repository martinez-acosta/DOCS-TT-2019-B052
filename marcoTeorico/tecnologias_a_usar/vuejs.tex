\subsection{JavaScript Web Frameworks: Vue/Nuxt vs React vs Angular}

De acuerdo con Wikipedia\cite{wikipedia_contributors_web_2020}, un web \textit{framework} es un conjunto de \textit{software} que permite el desarrollo de una aplicación web y en el lenguaje JavaScript hay varias opciones, incluidas los más populares Vue, React y Angular.

\subsubsection*{React}
De acuerdo con Wikipedia\cite{wikipedia_react_2020}, React es una biblioteca Javascript de código abierto diseñada para crear interfaces de usuario con el objetivo de facilitar el desarrollo de \textit{single page applications}.

\paragraph*{Características}
\begin{enumerate}
    \item Virtual DOM: React usa un virtual DOM propio en lugar del navegador.
    \item Props: son definidos como los atributos de configuración para dicho componente.
    \item Estado de cada componente: lleva un registro de las propiedades y atributos del componente.
    \item Ciclos de vida: son la serie de estados por los cuales pasan los componentes statefull a lo largo de su existencia. 
\end{enumerate}

\subsubsection*{Angular}
De acuerdo con Wikipedia\cite{wikipedia_angular_2020}, Angular es un web \textit{framework} desarrollado en TypeScript de código abierto mantenido por Google que se utiliza para crear y mantener \textit{single page aplications}. 

\begin{enumerate}
    \item Generación de código
    \item Componentes
    \item Ciclos de vida de componentes
    
\end{enumerate}

\subsubsection*{Vue/Nuxt}

De acuerdo con la documentación de Vue.js\cite{noauthor_que_nodate}, Vue es un \textit{framework} progresivo para desarrollar interfaces de usuario. A diferencia de otros \textit{frameworks} monolíticos, Vue está diseñado desde cero para ser utilizado incrementalmente.


La librería central está enfocada solo en la capa de visualización y es fácil de utilizar e integrar con otras librerias o proyectos existentes. Por otro lado, Vue también es perfectamente capaz de impulsar sofisticadas \textit{single-page applications} cuando se utiliza en combinación con librerías de apoyo.


\paragraph*{Comparación con React}

React y Vue comparten muchas similitudes; ambos utilizan un DOM virtual, proporcionan componentes de vista reactivos y componentes, mantienen el enfoque en la librería central, con temas como el enrutamiento y la gestión global del estado manejadas por librerías asociadas.


Tanto React como Vue ofrecen un rendimiento comparable en los casos de uso más comunes, con Vue normalmente un poco por delante debido a su implementación más ligera del DOM virtual.


En Vue, las dependencias de un componente se rastrean automáticamente durante su renderizado, por lo que el sistema sabe con precisión qué componentes deben volver a renderizarse cuando cambia el estado. Se puede considerar que cada componente tiene un \textit{shouldComponentUpdate} automáticamente implementado.


En general, esto elimina la necesidad de toda una clase de optimizaciones de rendimiento a los desarrolladores y les permite centrarse más en la construcción de la aplicación en sí misma a medida que va escalando.

\paragraph*{Comparación con Angular}
En términos de rendimiento, ambos \textit{frameworks} son excepcionalmente rápidos y no hay suficientes datos de casos de uso en el mundo real para hacer un veredicto.


Vue es mucho menos intrusivo en las decisiones del desarrollador que Angular, ofreciendo soporte oficial para una variedad de sistemas de desarrollo, sin restricciones sobre cómo estructurar su aplicación. Muchos desarrolladores disfrutan de esta libertad, mientras que algunos prefieren tener solo una forma correcta de desarrollar cualquier aplicación.


Para empezar con Vue, todo lo que se necesita es familiarizarse con HTML y ES5 JavaScript; con estas habilidades básicas, se puede empezar a desarrollar aplicaciones no triviales.


La curva de aprendizaje de Angular es mucho más pronunciada. La superficie de la API del framework es enorme y como usuario necesitará familiarizarse con muchos más conceptos antes de ser productivo. La complejidad de Angular se debe en gran medida a su objetivo de diseño de apuntar solo a aplicaciones grandes y complejas - pero eso hace que el \textit{framework} sea mucho más difícil de entender para los desarrolladores con menos experiencia.

\subsubsection*{Nuxt.js}
De acuerdo a la documentación de Nuxt\cite{noauthor_what_nodate-1}, Nuxt.js es un \textit{framework} progresivo basado en Vue.js para crear aplicaciones web. Se basa en Vue.js y herramientas de desarrollo como webpack, Babel y PostCSS. El objetivo de Nuxt es hacer que el desarrollo web en Vue.js sea eficaz.

\paragraph*{Características}
\begin{enumerate}
    \item Manejo de archivos Vue (*.vue)
    \item División automática de código
    \item Representación del lado del servidor
    \item Potente sistema de enrutamiento con datos asincrónicos
    \item Servicio de archivos estáticos
    \item Soporte sintaxis ES2015+ (Javascript ES6)
    \item Gestión del elemento <head> (<title>, <meta>, etc.)
    \item Preprocesador: Sass, Less, Stylus, etc.
\end{enumerate}



\subsubsection*{Elección}

Se usará Vue/Nuxt por ser el \textit{framework} con el que el equipo está más acostumbrado, además de ser el máx flexible de las opciones expuestas.