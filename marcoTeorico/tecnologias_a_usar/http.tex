\subsection{Hypertext Transfer Protocol}
De acuerdo con la W3C\cite{noauthor_http_nodate}, Hypertext Transfer Protocol (HTTP, o protocolo de transferencia de hipertexto), es el protocolo de comunicación que permite transferir información en la World Wide Web.


HTTP fue desarrollado por el World Wide Web Consortium y la Internet Engineering Task Force, colaboración que culminó en 1999 con la publicación de varios RFC, siendo el más importante el RFC 2616 que especifica la versión 1.1 del protocolo.


HTTP es un protocolo sin estado, es decir, no guarda ninguna información sobre conexiones anteriores; sin embargo, el desarrollo de aplicaciones web necesita frecuentemente mantener un estado, por lo que se usan \textit{cookies}, que son archivos generados en un servidor que son almacenados en el sistema cliente.


Es un protocolo orientado a transacciones y sigue el esquema petición-respuesta entre un cliente y un servidor; al cliente se le suele llamar ``agente de usuario'' (o \textit{user agent}), que realiza una petición enviando un mensaje con cierto formato al servidor, mientras que al servidor se le suele llamar servidor web y envía un mensaje de respuesta. 


HTTP tiene métodos de petición flexibles que permiten añadir nuevos métodos o funcionalidades; el número de métodos ha ido en aumento según se avanza en las versiones del protocolo donde los más importantes son:


\begin{enumerate}
    \item Método \textit{get}: solicita una representación del recurso especificado; solo deben recuperar datos. 
    \item Método \textit{head}: pide una respuesta idéntica a la que correspondería a una petición \textit{get}, pero en la respuesta no se devuelve el cuerpo; esto es útil para poder recuperar los metadatos de los encabezados de respuesta, sin tener que transportar todo el contenido.
    \item Método \textit{post}: envía datos para que sean procesados por un recurso identificado que se incluirán en el cuerpo de la petición. 
    \item Método \textit{put}: sube o carga un recurso especificado (archivo o fichero) y es más eficiente que el método \textit{post}, porque permite escribir un archivo en una conexión socket establecida con el servidor.
\end{enumerate}