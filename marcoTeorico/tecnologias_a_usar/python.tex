\subsection{Python 3 vs Java}
De acuerdo con Mark Lutz\cite{lutz_learning_2013}, Python es un lenguaje de programación interpretado, interactivo y orientado a objetos. Incorpora módulos, excepciones, tipeo dinámico, tipos de datos dinámicos y clases.


Tiene sintaxis clara, interfaces para muchas llamadas de sistema y bibliotecas, así como para varios sistemas de ventanas. Además, es extensible en C o C++. 


También se puede usar como un lenguaje de extensión para aplicaciones que necesitan una interfaz programable y es portátil: se ejecuta en muchas variantes de Unix, en Mac y en Windows 2000 y versiones posteriores.


La biblioteca estándar del lenguaje, cubre áreas como el procesamiento de cadenas (expresiones regulares, Unicode, cálculo de diferencias entre archivos), protocolos de Internet (HTTP, FTP, SMTP, XML-RPC, POP, IMAP, programación CGI), ingeniería de software (pruebas unitarias, registro, creación de perfiles, análisis del código Python) e interfaces del sistema operativo (llamadas al sistema, sistemas de archivos, sockets TCP/IP).

\subsubsection*{Fortalezas}

\paragraph*{Es orientado a objetos y funcional}
Python es un lenguaje orientado a objetos; su modelo de clase admite nociones avanzadas como el polimorfismo, la sobrecarga del operador y la herencia múltiple; sin embargo, en el contexto de la simple sintaxis y escritura de Python, la programación orientada a objetos es fácil de aplicar.


Además de servir para la estructuración y reutilización de código, la naturaleza orientada a objetos de Python lo hace ideal como herramienta de secuencias de comandos para otros lenguajes de sistemas orientados a objetos. Por ejemplo, con el código apropiado, los programas Python pueden especializar clases implementadas en C++, Java y C\#.


No obstante, la programación orientada a objetos es una opción en Python. Al igual que C++, Python admite modos de programación tanto procedimentales como orientados a objetos. Las herramientas orientadas a objetos se pueden aplicar siempre que las restricciones lo permitan. 


Además de sus paradigmas originales de procedimientos (basados ​​en declaraciones) y orientados a objetos (basados ​​en clases), Python en los últimos años ha adquirido soporte incorporado para la programación funcional, un conjunto que incluye generadores, comprensiones, cerraduras, mapas, decoradores , funciones anónimas lambdas.

\paragraph*{Es extensible}
Su conjunto de herramientas lo ubica entre los lenguajes de \textit{scripting} tradicionales como Tcl, Scheme y Perl; y los lenguajes de desarrollo de sistemas como C, C ++ y Java.


Python proporciona toda la simplicidad y facilidad de uso de un lenguaje de programación, junto con herramientas de ingeniería de software más avanzadas que normalmente se encuentran en lenguajes compilados.

A diferencia de algunos lenguajes de secuencias de comandos, esta combinación hace que Python sea útil para proyectos de desarrollo a gran escala. Algunas de las herramientas de Python son:
\paragraph*{Escritura dinámica}
Python realiza un seguimiento de los tipos de objetos que utiliza su programa cuando se ejecuta; eso no requiere declaraciones complicadas de tipo y tamaño en su código. De hecho, no existe una declaración de tipo o variable en Python. 


Debido a que el código Python no restringe los tipos de datos, también se aplica automáticamente a toda una gama de objetos.

\paragraph*{Gestión automática de la memoria}
Python asigna automáticamente objetos y los reclama el recolector de basura cuando ya no se usan y la mayoría puede crecer y reducirse según la demanda. Es decir, Python realiza un seguimiento de los detalles de la memoria de bajo nivel.


\paragraph*{Tipos de objetos incorporados}
Python proporciona estructuras de datos de uso común como listas, diccionarios y cadenas como partes intrínsecas del lenguaje. Son flexibles y fáciles de usar. Por ejemplo, los objetos integrados pueden crecer y reducirse según demanda, pueden anidarse arbitrariamente para representar información compleja, y más.
\paragraph*{Herramientas incorporadas}
Para procesar todos esos tipos de objetos, Python viene con operadores potentes y estándar, que incluyen concatenación (unir colecciones), segmentar (extraer secciones), ordenar, mapear y más.
\paragraph*{Utilidades de biblioteca}
Para tareas más específicas, Python también viene con una gran colección de herramientas de biblioteca precodificadas que admiten todo, desde la coincidencia de expresiones regulares hasta la creación de redes. Una vez que aprende el lenguaje en sí, las herramientas de la biblioteca de Python son donde ocurre gran parte de la acción a nivel de aplicación.
\paragraph*{Utilidades de terceros}
Debido a que Python es de código abierto, los desarrolladores pueden contribuir con herramientas precodificadas que admitan tareas que aún no son herramientas estándar; en la Web, encontrará soporte gratuito para COM, imágenes, programación numérica, XML y acceso a bases de datos.