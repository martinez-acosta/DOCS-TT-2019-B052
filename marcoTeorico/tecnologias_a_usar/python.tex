\subsection{Python 3 }
De acuerdo con Mark Lutz\cite{lutz_learning_2013}, Python es un lenguaje de programación interpretado, interactivo y orientado a objetos; incorpora módulos, excepciones, tipeo dinámico, tipos de datos dinámicos y clases.


La biblioteca estándar del lenguaje es extensible en C o C++, cubre áreas como el procesamiento de cadenas (expresiones regulares, Unicode, cálculo de diferencias entre archivos), protocolos de Internet (HTTP, FTP, SMTP, XML-RPC, POP, IMAP, programación CGI), ingeniería de software (pruebas unitarias, registro, creación de perfiles, análisis del código Python) e interfaces del sistema operativo (llamadas al sistema, sistemas de archivos, \textit{sockets TCP/IP}).

\subsubsection*{Características}
\begin{enumerate}
    \item Es extensible.
    \item Tiene escritura dinámica.
    \item Gestión automática de la memoria.
    \item Tipos de objetos incorporados.
    \item Herramientas incorporadas.
    \item Utilidades de biblioteca.
    \item Utilidades de terceros.
    \item Es orientado a objetos y funcional.
\end{enumerate}


\subsubsection*{Elección}\label{ref:python}
Se ha elegido Python para desarrollar los algoritmos del proyecto dado que es multiplataforma y es de fácil integración con el \textit{framework} web elegido para el \textit{back end}.