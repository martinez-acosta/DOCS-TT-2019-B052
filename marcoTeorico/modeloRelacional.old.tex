\subsection{Modelo Relacional}


\subsubsection{Dominios, atributos, tuplas y relaciones}

Un dominio D es un conjunto de valores atómicos. Por atómico queremos decir que cada valor de un dominio es indivisible en lo que al modelo relacional se refiere. Una forma habitual de especificar un dominio es indicar un tipo de dato desde el que se dibujan los valores del mismo.


Lo expuesto anteriormente se conoce como definiciones lógicas de dominios y cada dominio cuenta con un nombre, un tipo de dato y un formato.


Un esquema de relación $R$, denotado por $R(A_1, A_2,..., A_n)$ está constituido por un nombre de relación $R$ y una lista de atributos $A_1, A_2,..., A_n$. 


Cada atributo $A_i$ es el nombre de un papel jugado por algún dominio $D$ en el esquema de relación $R$. Se dice que $D$ es el dominio de $A_i$ y se especifica como $dom(A_i)$. 


Un esquema de relación se utiliza para describir una relación; se dice que $R$ es el nombre de la misma. El grado de una relación es el número de atributos $n$ de la misma.




Los términos intensidad de la relación para el esquema $R$ y extensión de relación del estado $r(R)$ son también muy utilizados.


\begin{figure}[ht]
    \centering
    \includegraphics[width=\textwidth]{modeloRelacional/tabla.png}
    \caption{Atributos y tuplas en una relación}
    \label{img:modeloRelacional-Tabla}
\end{figure} 


Tomando en cuenta la figura \ref{img:modeloRelacional-Tabla}, la relación como una tabla en la que cada tupla aparece como una fila y cada atributo como un encabezamiento de columna que indica la interpretación que habrá que dar a cada uno de los valores de la misma.


Los valores NULL representan atributos cuyos valores no se conocen, o no existen, para una tupla ESTUDIANTE individual.


Formalmente, una relación (o estado de relación) $r(R)$ es una relación matemática de grado $n$ en los dominios $dom(A_1), dom(A_2),..., dom(A_n)$ que es un subconjunto del producto cartesiano de los dominios que definen $R$:


$r(R) \subseteq (dom(A_1) \times dom(A_2) \times ... \times dom(A_n))$


El producto cartesiano especifica todas las posibles combinaciones de valores de los dominios, por tanto, si especificamos el número total de valores, o cardinalidad, de un dominio $D$ como $|D|$ (asumiendo que todos ellos son finitos), el número total de tuplas del producto cartesiano es:


$|dom(A_1)| \times |dom(A_2)| \times .... \times |dom(A_n)|$


Este producto de cardinalidades de todos los dominios representa el número total de posibles instancias, o
tuplas, que pueden existir en la relación $r(R)$.


De todas estas posibles combinaciones, un estado de relación en un momento dado (el estado de relación actual) solo refleja las tuplas válidas que representan un estado particular del mundo real. 


\subsubsection{Características de las relaciones}

\paragraph{Ordenación de tuplas en una relación}


Una relación está definida como un conjunto de tuplas. Matemáticamente, los elementos de un conjunto no guardan un orden entre ellos; por tanto, las tuplas en una relación tampoco la tienen. En otras palabras, una relación no es sensible al ordenamiento de las tuplas.


\paragraph{Ordenación de los valores dentro de una tupla y definición alternativa de una relación}

Una n-tupla es una lista ordenada de n valores, por lo que el orden de valores dentro de una de ellas (y por consiguiente de los atributos de un esquema de relación) es importante. Sin embargo, a nivel lógico, el orden de los atributos y sus valores no es tan importante mientras se mantenga la correspondencia entre ellos.

\paragraph{Valores y NULL en las tuplas}


Cada valor en una tupla es un valor atómico, es decir, no es divisible en componentes dentro del esqueleto del modelo relacional básico. Por tanto, no están permitidos los atributos compuestos y multivalor.


Este modelo suele recibir a veces el nombre de modelo relacional plano. Una gran parte de la teoría que se esconde tras el modelo relacional fue desarrollada con este principio en mente, el cual recibe el nombre de principio de primera forma normal.


Así pues, los atributos multivalor deben representarse en relaciones separadas, mientras que los compuestos lo están sólo por sus atributos de componente simple en el modelo relacional básico.


Un concepto importante es el de los valores NULL (nulo) que se utilizan para representar los valores de atributos que pueden ser desconocidos o no ser aplicables a una tupla. Para estos casos, existe un valor especial llamado NULL. 


\paragraph{Interpretación (significado) de una relación}


El esquema de relación puede interpretarse como una declaración o un tipo de aserción; cada tupla de la relación puede ser interpretada entonces como un hecho o una instancia particular de una aserción.


\subsubsection{Notación del modelo relacional}

\begin{itemize}
    \item Un esquema de relación $R$ de grado $n$ se designa como $R(A_1, A_2,..., A_n)$.
    \item Las letras $Q, R, S$ especifican nombres de relación.
    \item Las letras $q, r, s$ especifican estados de relación.
    \item Las letras $t, u, v$ indican tuplas.
    \item En general, el nombre de una relación como ESTUDIANTE indica también el conjunto real de tuplas de la misma (el estado actual de la relación) mientras que ESTUDIANTE(Nombre, Ine,...) se refiere solo a su esquema.
    \item Un atributo $A$ puede cualificarse con el nombre de relación $R$ al cual pertenece usando la notación de
    punto, $R.A$: por ejemplo, ESTUDIANTE.Nombre o ESTUDIANTE.Edad. Esto es así porque dos atributos en relaciones diferentes pueden usar el mismo nombre. Sin embargo, todos los nombres de atributo en una relación particular deben ser distintos.
    \item Una n-tupla $t$ en una relación $r(R)$ está designada por $t=<v_1, v_2,..., v_n>$, donde $v_i$ es el valor correspondiente al atributo $A_i$. La siguiente notación se refiere a los valores componente de las tuplas:
    \begin{itemize}
        \item Tanto $t[A_i]$ como $t.A_i$ (y, a veces, $t[i]$) hacen referencia al valor $v_i$ de $t$ del atributo $A_i$ .
        \item Tanto $t[A_u, A_w ,..., A_z]$ como $t.(A_u, A_w,..., A_z)$, donde $Au, A_w,..., A_z$ es una lista de atributos de $R$, hacen referencia a la subtupla de valores $<v_u, v_w,..., v_z>$ de $t$ correspondientes a los atributos especificados en la lista.
    \end{itemize}
    
\end{itemize}

\subsubsection{Restricciones del modelo relacional y esquemas de bases de datos relacionales}
Generalmente, existen muchas restricciones, o constraints, en los valores de un estado de base de datos y pueden dividirse en tres categorías:
\begin{enumerate}
    \item Restricciones que son inherentes al modelo de datos y que reciben el nombre de restricciones implícitas o inherentes basadas en el modelo.
    \item Restricciones que pueden expresarse directamente en los esquemas del modelo de datos, por lo general especificándolas en el DDL. Las llamaremos restricciones explícitas o basadas en el esquema.
    \item Restricciones que no pueden expresarse directamente en los esquemas del modelo de datos y que
    por consiguiente deben ser expresadas e implementadas por los programas. Las llamaremos restricciones semánticas, basadas en aplicación o reglas de negocio.
\end{enumerate}


Otra categoría importante de restricciones son las dependencias de datos, que incluyen las dependencias funcionales y las dependencias multivalor. Suelen emplearse para comprobar la corrección del diseño de una base de datos relacional y en un proceso llamado normalización.


\paragraph{Restricciones de dominio}
Las restricciones de dominio especifican que dentro de cada tupla, el valor de un atributo $A$ debe ser un valor atómico del dominio $dom(A)$. 


Los tipos de datos asociados a ellos suelen incluir valores numéricos estándar para datos enteros (como entero corto, entero o entero largo) y reales (de coma flotante de simple y doble precisión).


También están disponibles tipos de datos para el almacenamiento de caracteres, valores lógicos, cadenas de longitud fija y variable, fechas, horas y moneda. Es posible describir otros dominios como un subrango de valores de un tipo de dato, o como un tipo de dato enumerado en el que todos sus posibles valores están explícitamente listados.



\paragraph{Restricciones de clave y restricciones en valores NULL}


Una relación está definida como un conjunto de tuplas. Por definición, todos los elementos de un conjunto son distintos; por tanto, todas las tuplas en una relación también deben serlo. 


Esto significa que dos tuplas no pueden tener la misma combinación de valores para todos sus atributos. Habitualmente existen otros subconjuntos de atributos de una relación $R$ con la propiedad de que dos tuplas en cualquier relación $r$ de $R$ no deben tener la misma combinación de valores para estos atributos.


En general, un esquema de relación puede contar con más de una clave. En este caso, cada una de ellas reci-
be el nombre de clave candidata. Es común designar una de ellas como la clave principal de la relación y será la que se utilice para identificar las tuplas en la relación.


Usamos la convención de que los atributos que forman la clave principal de un esquema de relación están subrayados.


\subsubsection{Esquemas de datos relacional}

Un esquema de base de datos relacional $S$ es un conjunto de esquemas de relación $S={R_1, R_2,..., R_m}$ y de restricciones de integridad $RI$. 


Un estado de base de datos relacional $DB$ de $S$ es un conjunto de estado de relación $DB={r_1, r_2,..., r_m}$ en el que cada $r_i$ es un estado de $R_i$ y satisface las restricciones de integridad especificadas en $RI$. 



Cuando nos referimos a una base de datos relacional, incluimos implícitamente tanto su esquema como su estado actual. Un estado de base de datos que no cumple todas sus restricciones de integridad se dice que está en un estado incorrecto, mientras que aquél que sí las cumple está en un estado correcto.


\subsubsection{Integridad de entidad, integridad referencial y claves foráneas} 
Las restricciones de integridad de entidad declaran que el valor de ninguna clave principal puede ser NULL, porque dicha clave se emplea para identificar tuplas individuales en una relación. 

Si se permitiera el valor NULL, significaría que no se podrían identificar ciertas tuplas. Por ejemplo, si dos o más tuplas tuvieran NULL en sus claves primarias, no seríamos capaces de diferenciarlas si intentásemos hacer referencia a ellas desde otras relaciones.


Las restricciones de clave y las de integridad de entidad se especifican en relaciones individuales. Las de integridad referencial están especificadas entre dos relaciones y se utilizan para mantener la consistencia entre las tuplas de dos relaciones. 


Una clave foránea especifica una restricción de integridad referencial entre dos esquemas de relación $R_1$ y $R_2$. Un conjunto de atributos FK en una relación $R_1$ es una clave foránea de $R_1$ que referencia a la relación $R_2$ si satisface las siguientes reglas:

\begin{enumerate}
    \item  Los atributos en FK tienen el mismo dominio, o dominios, que los atributos de clave principal PK de
    $R_2$; se dice que los atributos FK referencian o hacen referencia a la relación $R_2$.
    \item Un valor de FK en una tupla $t_1$ del estado actual $r_1(R_1)$ tampoco aparece como valor de PK en alguna
    tupla $t_2$ del estado actual $r_2(R_2)$ o es NULL. En el caso anterior, tenemos que $t_1[FK] = t_2 [PK]$ y la tupla $t_1$ hace referencia a la tupla $t_2$.
\end{enumerate}
En esta definición, $R_1$ recibe el nombre de relación de referencia y $R_2$ es relación referenciada. Si se mantienen ambas condiciones, se establece una restricción de integridad referencial de $R_1$ a $R_2$.


