\subsubsection{Clave-Valor}
De acuerdo a la bibliografía de Coronel\cite{coronel_database_nodate} Una base de datos de clave-valor, o almacén de clave-valor, es un paradigma de almacenamiento de datos diseñado para almacenar, recuperar y administrar arreglos asociativos.


Comúnmente se usa un diccionario o tabla hash que contienen una colección de registros anidados, secuenciaa de bits, que se almacenan y se recuperan utilizando una clave que identifica de manera única el registro y se utiliza para encontrar rápidamente los datos dentro de la base de datos.


No obstante, es responsabilidad de las aplicaciones que hagan uso de este tipo de base de datos interpretar el significado de los datos. No hay claves foráneas y las relaciones no pueden rastrearse entre las claves. Esta simplificación en el RDBMS de clave-valor permite que sean rápidas y escalables.


Los pares de clave-valor generalmente se organizan en \textit{buckets}. Todas las claves dentro un \textit{bucket} deben ser únicas, pero pueden repetirse en otros \textit{buckets} y todas las operaciones, incluidas las consultas, se basan en el \textit{bucket} + la clave.


\begin{figure}[h!t]
    \centering
    \includegraphics[width=0.75\textwidth]{noSQL/bucket.png}
    \caption{\textit{Bucket} en el almacenamiento clave-valor}
    \label{img:claveValor-bucket}
\end{figure} 


Respecto a las operaciones de este tipo de bases de datos, se usan las operaciones de \textit{get}, \textit{store} y \textit{delete}. La operación \textit{get} o \textit{fetch} es usada para obteber el valor de un par. El operador de \textit{store} almacena datos en una clave. Si la combinación de \textit{bucket} + clave no existe, se añade como un nuevo par de clave-valor. En cambio, en caso de existir la combinación de \textit{bucket} + clave, el valor es reemplazado por el nuevo. El operador de \textit{delete} es para eliminar un par de clave-valor.


Algunas de las bases de datos de clave-valor populares son Riak, Redis, Memcached DB, Berkeley DB, HamsterDB, Amazon DynamoDB y Project Voldemort (una implementación de código abierto de Amazon DynamoDB)\cite{sadalage_nosql_nodate}.


