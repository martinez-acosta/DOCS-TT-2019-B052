\subsubsection{Clave-Valor}
De acuerdo con Coronel\cite{coronel_database_nodate}, una base de datos de clave-valor es un paradigma de modelo de datos diseñado para almacenar, recuperar y administrar arreglos asociativos.


Comúnmente se usa un diccionario o tabla \textit{hash} que contiene una colección de registros anidados, secuencias de bits que se almacenan y se recuperan utilizando una clave que identifica de manera única el registro y se utiliza para encontrar rápidamente los datos dentro de la base de datos.


No obstante, es responsabilidad de las aplicaciones que hagan uso de este tipo de base de datos interpretar el significado de los datos; no hay claves foráneas y las relaciones no son rastreables entre claves, lo que permite que el DBMS sea rápido y escalable.


La figura~\ref{img:claveValor-bucket} es una representación visual de un \textit{bucket} usado en las bases de datos NoSQL de clave-valor.

\begin{figure}[H]
    \centering
    \includegraphics[width=0.75\textwidth]{noSQL/bucket.png}
    \caption{\textit{Bucket} en el almacenamiento clave-valor}
    \label{img:claveValor-bucket}
\end{figure} 


Los pares de clave-valor generalmente se organizan en \textit{buckets}; todas las claves dentro un \textit{bucket} deben ser únicas, pero está permitido que se repitan en otros \textit{buckets} y todas las operaciones se basan en el \textit{bucket} + la clave.


En este tipo de bases de datos se usan las operaciones de \textit{get}, \textit{store} y \textit{delete}; la operación \textit{get} o \textit{fetch} es usada para obteber el valor de un par; el operador de \textit{store} almacena datos en una clave. 


Si la combinación de \textit{bucket} + clave no existe, se añade como un nuevo par de clave-valor; en cambio, si existe la combinación de \textit{bucket} + clave, el valor es reemplazado por el nuevo; el operador de \textit{delete} es para eliminar un par de clave-valor.


De acuerdo con Sadalge\cite{sadalage_nosql_nodate}, algunas de las bases de datos de clave-valor populares son Riak, Redis, Memcached DB, Berkeley DB, HamsterDB, Amazon DynamoDB y Project Voldemort (una implementación de código abierto de Amazon DynamoDB).


