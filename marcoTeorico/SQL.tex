\subsubsection{Structured Query Language}
\textit{Structured Query Language} o SQL está basado en el álgebra relacional, en el cálculo relacional y es un lenguaje de manipulación de datos, un lenguaje de definición de datos, un lenguaje de control de transacciones y un lenguaje de control de datos.

\paragraph*{Lenguaje de manipulación de datos (DML)}
Un \textit{Data Manipulation Language} o DML incluye comandos para insertar, actualizar, eliminar y recuperar datos dentro de las tablas de la base de datos. 

\paragraph*{Lenguaje de definición de datos (DDL)}
 Un \textit{Data Definition Language} o DDL incluye comandos para crear objetos de base de datos como tablas, índices y vistas, así como comandos para definir accesos a objetos de la base de datos. 

\paragraph*{Lenguaje de control de transacciones (TCL)}
Los comandos de un \textit{Transaction Control Language} o TCL se ejecutan dentro del contexto de una transacción, que es una unidad lógica de trabajo compuesta por una o más instrucciones SQL. 


SQL proporciona comandos para controlar el procesamiento de estas transacciones atómicas.

\paragraph*{Lenguaje de control de datos (DCL)}
Los comandos de un \textit{Data Control Language} o DCL se utilizan para controlar el acceso a los objetos de datos, como otorgar a un usuario permiso para ver solo una tabla y otorgar a otro usuario permiso para cambiar los datos de la mista tabla.
