\subsubsection{Orientado a grafos}
De acuerdo con Coronel\cite{coronel_database_nodate}, una base de datos NoSQL orientada a grafos está basada en la teoría de grafos para almacenar datos con muchas relaciones.



La figura~\ref{img:nosql-grafo} representa un grafo de una bilioteca, donde cada rectángulo es un nodo y están asociados entre sí por relaciones.
\begin{figure}[H] 
    \centering
    \includegraphics[width=0.75\textwidth]{noSQL/grafo.png}
    \caption{modelo conceptual orientado a grafos}
    \label{img:nosql-grafo}
\end{figure}


Como se muestra en la figura~\ref{img:nosql-grafo}, los componentes principales de las bases de datos de grafos son nodos, aristas y propiedades; el nodo es una instancia específica de algo sobre lo que queremos mantener datos.


Las propiedades son como atributos; son los datos que necesitamos almacenar sobre el nodo; todos los nodos tienen propiedades como nombre y apellido, pero no todos los nodos deben tener las mismas propiedades.


Un borde es una relación entre nodos, está representada por una flecha en la figura~\ref{img:nosql-grafo} y es posible que estén en una dirección o ser bidireccionales.


Para hacer una consulta se atraviesa el grafo y los recorridos se enfocan en las relaciones entre nodos, como la ruta más corta y el grado de conexión.

