\subsubsection{Orientado a grafos}
Una base de datos de grafos es una base de datos NoSQL basada en la teoría de grafos para almacenar datos sobre entornos ricos en relaciones.


La teoría de grafos es un campo matemático y de ciencias de la computación que modela relaciones, o aristas, entre objetos llamados nodos. 

\begin{figure}[h!t] 
    \centering
    \includegraphics[width=0.75\textwidth]{noSQL/grafo.png}
    \caption{modelo conceptual orientado a grafos}
    \label{img:nosql-grafo}
\end{figure}


Modelar y almacenar datos sobre relaciones es el enfoque de las bases de datos de grafos y la teoría en la que se basa es un campo de estudio bien establecido que se remonta a cientos de años. 


Como resultado, la creación de un modelo de base de datos basado en la teoría de grafos proporciona de inmediato una fuente rica de algoritmos y aplicaciones que han ayudado a las bases de datos de grafos a ganar terreno.


Los componentes principales de las bases de datos de grafos son nodos, aristas y propiedades, como se muestra en la figura ~\ref{img:nosql-grafo}. Un nodo corresponde a la idea de una instancia de entidad relacional.


El nodo es una instancia específica de algo sobre lo que queremos mantener datos. Cada nodo (círculo) en la figura ~\ref{img:nosql-grafo} representa un solo agente. 


Las propiedades son como atributos; son los datos que necesitamos almacenar sobre el nodo. Todos los nodos de agente pueden tener propiedades como nombre y apellido, pero no todos los nodos deben tener las mismas propiedades.


Un borde es una relación entre nodos. Los bordes (mostrados como flechas en la figura ~\ref{img:nosql-grafo}) pueden estar en una dirección, o pueden ser bidireccionales.


Para las consultas, la terminología correcta sería atravesar el grafo. Las bases de datos de grafos los recorridos se enfocan en las relaciones entre nodos, como la ruta más corta y el grado de conexión.


La base de datos de grafos comparte algunas características con otras bases de datos NoSQL en que las bases de datos de grafos no obligan a los datos a ajustarse a estructuras predefinidas, no admiten SQL y están optimizados para proporcionar velocidad de procesamiento, al menos para datos intensivos en relaciones.


Sin embargo, otras características clave no se aplican a las bases de datos de grafos. Las bases de datos de grafos no se escalan muy bien a los clústeres debido a las diferencias en los datos agregados.