
\subsection{Modelo entidad-relación}

De acuerdo con Elmasri \cite{ramez_elmasri_fundamentos_nodate}, el modelo entidad-relación (o modelo ER) fue creado por Peter Chen en 1976\cite{chen_entity-relationship_nodate} para el diseño conceptual de bases de datos y está conformado de entidades, atributos y relaciones. 

\subsubsection{Entidades}
El objeto básico representado por el modelo ER es una entidad, que es una cosa del mundo real con una existencia independiente; una entidad es un objeto con existencia física o conceptual.

\paragraph*{Tipo de entidad débil vs regular}

Un tipo de entidad define un conjunto de entidades con los mismos atributos; un tipo de entidad sin atributo clave se denomina tipo de entidad débil; en contraposición, un tipo de entidad regular con atributo clave se denomina tipo de entidad fuerte.


Las entidades que pertenecen a un tipo de entidad débil se identifican como relacionadas con un tipo de entidad propietaria en combinación con uno de sus valores de atributo.


El tipo de relación que relaciona un tipo de entidad débil con su propietaria se llama relación identificativa del tipo de entidad débil. 


Un tipo de entidad débil siempre tiene una restricción de participación total (dependencia de existencia) respecto a su relación identificativa, porque una entidad débil no se identifica sin una entidad propietaria; no obstante, no toda dependencia de existencia produce un tipo de entidad débil.


Un tipo de entidad débil posee una clave parcial, que es el conjunto de atributos que identifican inequívocamente las entidades débiles que están relacionadas con la misma entidad propietaria. 


En los diagramas ER tanto el tipo de la entidad débil como la relación identificativa se distinguen rodeando sus rectángulos y rombos mediante líneas dobles.


El atributo de clave parcial aparece subrayado con una línea discontinua o punteada; los tipos de entidades débiles se representan a veces como atributos complejos (compuestos o multivalor). 


En general, se define cualquier cantidad de niveles de tipos de entidad débil; un tipo de entidad propietaria puede ser un tipo de entidad débil. 


Además, es posible que un tipo de entidad débil tenga más de un tipo de entidad identificativa y un tipo de relación identificativa de grado superior a dos.

\subsubsection{Atributos}
Cada entidad posee atributos, que son propiedades particulares que la describen y en el modelo ER hay varios tipos: simple, compuesto, monovalor, multivalor, almacenado, derivado y nulo.


\paragraph*{Atributos compuestos vs atributos simples} 


Los atributos compuestos se dividen en subpartes más pequeñas que representan atributos más básicos con significados independientes.


Los atributos que no son divisibles se denominan atributos simples o atómicos, mientras que los atributos compuestos forman una jerarquía. 


El valor de un atributo compuesto es la concatenación de los valores de sus atributos simples.

\paragraph*{Atributos monovalor vs multivalor}  


La mayoría de los atributos poseen un solo valor para una entidad en particular; dichos atributos reciben el nombre de monovalor o de un solo valor. 


En algunos casos, un atributo es un conjunto de valores para la misma entidad y se denominan multivalor.


Un atributo multivalor tiene un límite superior y uno inferior para restringir el número de valores permitidos para cada entidad individual.


\paragraph*{Atributos almacenados vs derivados}
El atributo derivado se calcula u obtiene a partir de un atributo almacenado.


\paragraph*{Atributos complejos}

Los atributos complejos son atributos compuestos y multivalor que se anidan arbitrariamente.

\paragraph*{Atributos clave de un tipo de entidad}

Un atributo clave identifica inequívocamente cada tipo de entidad, cuenta con un nombre subrayado dentro del óvalo y algunos tipos de entidad poseen más de un atributo clave y si carece de clave, se le denomina tipo de entidad débil.

\paragraph*{Atributos de los tipos de relación}


Los atributos de los tipos de relación 1:1 o 1:N se trasladan a uno de los tipos de entidad participantes.


En el caso de un tipo de relación 1:N, un atributo de relación solo se migra al tipo de entidad que se encuentra en el lado N de la relación. 


Para los tipos de relación M:N, algunos atributos se determinan mediante la combinación de entidades participantes en una instancia de relación, no mediante una sola relación; dichos atributos deben especificarse como atributos de relación.

\subsubsection{Relaciones}

Un tipo de relación $R$ entre $n$ tipos de entidades $E_1, E_2, ..., E_n$ define un conjunto de relaciones entre las entidades de esos tipos de entidades. 


Como en el caso de los tipos de entidades y los conjuntos de entidades, normalmente se hace referencia a un tipo de relación y su correspondiente conjunto de relaciones con el mismo nombre $R$.


Matemáticamente, el conjunto de relaciones $R$ es un conjunto de instancias de relación $r_i$, donde cada $r_i$ asocia $n$ entidades individuales ($e_1, e_2,..., e_n$) y cada entidad $e_j$ de $r_i$ es un miembro del tipo de entidad $E_j, 1 \leq j \leq n$. Por tanto, un tipo de relación es una relación matemática en $E_1, E_2,..., E_n$.


En los diagramas ER, los tipos de relación se muestran mediante rombos, conectados a su vez mediante líneas a los rectángulos que representan los tipos de entidad participantes y el nombre de la relación se muestra dentro del rombo.


\paragraph*{Grado de un tipo de relación}
El grado de un tipo de relación es el número de tipos de entidades participantes; un tipo de relación de grado dos se denomina binario, de grado tres ternario y de grado $n$ n-ario. 

\paragraph*{Nombres de rol y relaciones recursivas}
Cada tipo de entidad que participa en un tipo de relación juega un papel o rol particular en la relación.


El nombre de rol hace referencia al papel que una entidad participante del tipo de entidad juega en cada instancia de relación y ayuda a explicar el significado de la relación.


Los nombres de rol no son técnicamente necesarios en los tipos de relación donde todos los tipos de entidad participantes son distintos, puesto que cada nombre de tipo de entidad participante se utiliza como participación.

Cuando un tipo de entidad se relaciona consigo misma, se tiene una relación recursiva y es necesario indicar los roles que juegan los miembros en la relación.


\paragraph*{Restricciones en los tipos de relación}


Los tipos de relación normalmente tienen ciertas restricciones que limitan las posibles combinaciones entre las entidades que participan en el conjunto de relaciones correspondiente y están determinadas por la situación del minimundo representado; se distinguen dos tipos principales de restricciones de relación: razón de cardinalidad y participación.


\paragraph*{Razones de cardinalidad para las relaciones binarias}
La razón de cardinalidad de una relación binaria especifica el número máximo de instancias de relación en las que una entidad participa.


Las posibles razones de cardinalidad para los tipos de relación binaria son 1:1, 1:N, N:1 y M:N.
\begin{enumerate}
    \item Uno a uno: una relación $R$ de $X$ a $Y$ es uno a uno si cada entidad en $X$ se asocia con cuando mucho una entidad en $Y$ e, inversamente, cada entidad en $Y$ se asocia con cuando mucho una entidad en $X$.
    \item Uno a muchos: una relación $R$ de $X$ a $Y$ es uno a muchos si cada entidad en $X$ se asocia con muchas entidades en $Y$, pero cada entidad en $Y$ se asocia con cuando mucho una entidad en $X$. 
    \item Muchos a uno: una relación $R$ de $X$ a $Y$ es muchos a uno si cada entidad en $X$ se asocia con cuando mucho una entidad en $Y$, pero cada entidad en $Y$ se asocia con muchas entidades en $X$. 
    \item Muchos a muchos: una relación $R$ de $X$ a $Y$ es muchos a muchos si cada entidad en $X$ se asocia con muchas entidades en $Y$ y cada entidad en $Y$ se asocia con muchas entidades en $X$. 
\end{enumerate}

\paragraph*{Restricciones de participación y dependencias de existencia}
Hay dos tipos de restricciones de participación, total y parcial; la restricción de participación determina si la existencia de una entidad depende de su relación con otra entidad y especifica el número mínimo de instancias de relación en las que participa cada entidad.

\begin{enumerate}
    \item Participación total: si todo miembro de un conjunto de entidades debe participar en una relación, es una participación total del conjunto de entidades en la relación. Esto se denota al dibujar una línea doble desde el rectángulo de entidades hasta el rombo de relación.
    \item Participación parcial: una línea sencilla indica que algunos miembros del conjunto de entidades no deben participar en la relación.
\end{enumerate}

La razón de cardinalidad y las restricciones de participación, en conjunto, son restricciones estructurales de un tipo de relación.

\subsubsection{Resumen de la notación para los diagramas ER}


Finalmente, la tabla~\ref{tab:notacion-ER} es un resumen de la simbología usada en los diagramas entidad-relación en la que se muestra en la primera columna la representación gráfica y a su lado su significado en el modelo.

% Please add the following required packages to your document preamble:
% \usepackage{booktabs}
\begin{table}[H]
    \centering
    \begin{tabular}{@{}ll@{}}
    \toprule
    \multicolumn{2}{l}{Notación del modelo entidad-relación}                 \\ \midrule
    Símbolo & Signficado                                                           \\
            &                                                                      \\
    \parbox[c]{4em}{\includegraphics[width=\linewidth]{modeloEntidadRelacion/01.png}}       & Entidad                                                              \\
    \parbox[c]{4em}{\includegraphics[width=\linewidth]{modeloEntidadRelacion/02.png}}       & Entidad débil                                                        \\
    \parbox[c]{2em}{\includegraphics[width=\linewidth]{modeloEntidadRelacion/03.png}}       & Relación                                                             \\
    \parbox[c]{2em}{\includegraphics[width=\linewidth]{modeloEntidadRelacion/04.png}}       & \begin{tabular}[c]{@{}l@{}}Relación de\\ identificación\end{tabular} \\
    \parbox[c]{4em}{\includegraphics[width=\linewidth]{modeloEntidadRelacion/05.png}}       & Atributo                                                             \\
    \parbox[c]{4em}{\includegraphics[width=\linewidth]{modeloEntidadRelacion/06.png}}       & \begin{tabular}[c]{@{}l@{}}Atributo \\ clave\end{tabular}            \\
    \parbox[c]{4em}{\includegraphics[width=\linewidth]{modeloEntidadRelacion/07.png}}       & \begin{tabular}[c]{@{}l@{}}Atributo\\ multivalor\end{tabular}        \\
    \parbox[c]{4em}{\includegraphics[width=\linewidth]{modeloEntidadRelacion/08.png}}       & \begin{tabular}[c]{@{}l@{}}Atributo\\ compuesto\end{tabular}         \\
    \parbox[c]{4em}{\includegraphics[width=\linewidth]{modeloEntidadRelacion/09.png}}       & \begin{tabular}[c]{@{}l@{}}Atributo\\ derivado\end{tabular}          \\
    \parbox[c]{4em}{\includegraphics[width=\linewidth]{modeloEntidadRelacion/10.png}}       & \begin{tabular}[c]{@{}l@{}}Participación\\ total\end{tabular}        \\
    \parbox[c]{4em}{\includegraphics[width=\linewidth]{modeloEntidadRelacion/11.png}}       & \begin{tabular}[c]{@{}l@{}}Razón de\\ cardinalidad\end{tabular}      \\
    \parbox[c]{4em}{\includegraphics[width=\linewidth]{modeloEntidadRelacion/12.png}}       & \begin{tabular}[c]{@{}l@{}}Restricción\\ estructural\end{tabular}   
    \end{tabular}
    \caption{Notación del modelo entidad-relación}
    \label{tab:notacion-ER}
    \end{table}