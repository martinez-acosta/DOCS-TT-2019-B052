\subsection{Modelo relacional}

De acuerdo con Elmasri\cite{ramez_elmasri_fundamentos_nodate}, el modelo relacional introducido por Ted Codd en 1970\cite{codd_relational_nodate} utiliza el concepto de una relación matemática como bloque de construcción básico y tiene su base teórica en la teoría de conjuntos y la lógica del predicado.

La lógica de predicado, utilizada ampliamente en matemáticas, proporciona un marco en el que una afirmación (declaración de hecho) se verifica como verdadera o falsa.


La teoría de conjuntos es una ciencia matemática que trata con conjuntos o grupos de cosas y se utiliza como base para la manipulación de datos en el modelo relacional.

La figura~\ref{img:modeloRelacional-Tabla} muestra de manera visual el modelo relacional; una tabla en el modelo relacional está compuesto de atributos, tiene un nombre de relación y tiene $n$ tuplas.


\begin{figure}[H]
    \centering
    \includegraphics[width=\textwidth]{modeloRelacional/tabla.png}
    \caption{Tabla en el modelo relacional}
    \label{img:modeloRelacional-Tabla}
\end{figure} 
Como se observa en la imagen~\ref{img:modeloRelacional-Tabla}, el modelo relacional representa la base de datos como una colección de relaciones, siendo cada relación una tabla de valores, donde cada fila representa una colección de valores relacionados.


Asimismo, cada fila de la tabla representa un hecho que, por lo general, corresponde con una relación o entidad real; el nombre de la tabla y de las columnas se utiliza para ayudar a interpretar el significado de cada uno de los valores de las filas.


En terminología formal, una fila recibe el nombre de tupla, una cabecera de columna es un atributo y el nombre de la tabla una relación; el tipo de dato que describe los valores en cada columna está representado por un dominio de posibles valores. 


Basado en estos conceptos, el modelo relacional tiene tres componentes bien definidos:
\begin{enumerate}
    \item Una estructura de datos lógica representada por relaciones.
    \item Un conjunto de reglas de integridad para garantizar que los datos sean consistentes.
    \item Un conjunto de operaciones que define cómo se manipulan los datos.
\end{enumerate}
\subsubsection{Relaciones}

En este modelo, las tablas se usan para contener información acerca de los objetos a representar en la base de datos.


Una relación se representa como una tabla bidimensional en la que las filas de la tabla corresponden a registros individuales y las columnas corresponden a atributos.

Formalmente, un esquema de relación $R$, denotado por $R(A_1, A_2,..., A_n)$ está constituido por un nombre de relación $R$ y una lista de atributos $A_1, A_2,..., A_n$. 


Un esquema de relación se utiliza para describir una relación; se dice que $R$ es el nombre de la misma y el grado de una relación es su número de atributos $n$.


En el modelo relacional cada fila se llama tupla y la tabla que representa una relación tiene las siguientes características:
\begin{itemize}
    \item Cada celda de la tabla contiene solo un valor.
    \item Cada columna tiene el nombre del atributo que representa.
    \item Todos los valores en una columna provienen del mismo dominio, pues todos son valores del atributo correspondiente.
    \item Cada tupla o fila es distinta; no hay tuplas duplicadas.
    \item El orden de las tuplas o filas es irrelevante.
\end{itemize}
\paragraph*{Relaciones y tablas de bases de datos}   
Una relación $r$ del esquema $R(A_1, A_2,..., A_n)$, también especificado como $r(R)$ es un conjunto de n-tuplas $r={t_1, t_2,..., t_m}$.


Cada tupla $t$ es una lista ordenada de $n$ valores $t=<v_1, v_2,...,v_n>$, donde $v_i$, $1 \leq i \leq n$ es un elemento de $dom(A_i)$ o un valor especial nulo.


\subsubsection{Claves}

En el modelo relacional, las claves son importantes porque aseguran que cada fila en una tabla sea unívocamente identificable; son usadas para establecer relaciones entre tablas y asegurar la integridad de los datos.

Una clave es un atributo o grupo de atributos que identifican los valores de otros atributos y puede ser compuesta, clave o superclave. 

\paragraph*{Clave compuesta}
Una clave compuesta es una clave que se compone de más de un atributo y si forma parte de una clave se denomina atributo clave.

\paragraph*{Superclave}
Un atributo o atributos que identifican de manera única cualquier fila de una tabla.

\subsubsection{Restricciones de integridad}
\paragraph*{Integridad de dominio}
La integridad de dominio es la validez de las restricciones que debe cumplir una determinada columna de la tabla.
\paragraph*{Integridad de entidad}
Todas las claves principales son únicas y ninguna clave primaria debe ser nula.
\paragraph*{Integridad referencial}
Una clave externa puede ser nula siempre que no sea parte de la clave principal de su tabla o tiene el valor que coincida con el valor de la clave primaria en una tabla con la que está relacionada.

\subsubsection{Propiedades de las relaciones}
\paragraph*{Grado}
El número de columnas en una tabla se llama grado de la relación, es parte de la intensión de la relación y nunca cambia; una relación con una sola columna es de grado uno y se llama relación unaria; con dos columnas se llama binaria; con tres columnas se llama ternaria y con más columnas n-aria.


\paragraph*{Cardinalidad}
La cardinalidad de una relación es el número de entidades a las que otra entidad mapea dicha relación.
