\section{Modelos de datos para bases de datos}
De acuerdo a la bibliografía de Elmasri \cite{ramez_elmasri_fundamentos_nodate}, un modelo de datos es una colección de conceptos que describen una estructura de una base de datos. 


Los modelos de datos de alto nivel o conceptuales ofrecen conceptos visuales simples que representan un modelo de datos, mientras que los modelos de datos de bajo nivel o físicos ofrecen conceptos que describen los detalles de cómo se implementa el almacenamiento de los datos en el sistema de la base de datos.


Los modelos de datos conceptuales utilizan conceptos como entidades, atributos y relaciones o, en el caso de ser bases de datos no relacionales, no hay un estándar definido para modelar conceptualmente este tipo de bases de datos.


Una entidad representa un objeto o concepto del mundo real, un atributo representa alguna propiedad que describe a una entidad y una relación es una asociación entre entidades.


A continuación se presentan varios de los modelos de datos existentes.