\section{Modelos de datos para bases de datos}
De acuerdo con Elmasri \cite{ramez_elmasri_fundamentos_nodate}, un modelo de datos es una colección de conceptos que describen la estructura de una base de datos. 


Los modelos de datos conceptuales o de alto nivel ofrecen conceptos visuales simples que representan un modelo de datos, mientras que los modelos de datos físicos o de bajo nivel son detalles de cómo se implementa el almacenamiento de los datos en el sistema de la base de datos.


Una entidad representa un objeto o concepto del mundo real; un atributo representa alguna propiedad que describe una entidad y una relación es una asociación entre entidades.


A continuación se presentan los modelos de datos que se utilizarán en el desarrollo de este trabajo.