\section{Alcance y limitaciones}\label{sec:alcance}
De acuerdo con Dullea\cite{dullea_analysis_2003}, un diagrama ER es válido solo si es estructural y semánticamente válido; sin embargo, hasta donde se sabe, en los últimos 17 años no hay estudios sobre la validez semántica de un diagrama ER, porque como expresa Dullea en su trabajo, la validez semántica depende del minimundo que se quiere representar en el diagrama y es imposible definir una métrica generalizada; por ello la validez de un diagrama ER será solo estructuralmente.


Se usará la notación oficial en los modelos de datos conceptuales entidad-relación y relacional, es decir, la notación de Chen y la notación de Codd, respectivamente.


Para la generación de \textit{scripts} del esquema de datos relacional se usará el lenguaje de consultas SQL.


Asimismo, los \textit{scripts} generados se probarán con MySQL porque es uno de los DBMS que se usa en la asignatura de Base de Datos y están familiarizados los estudiantes de ESCOM.


Para la transformación del modelo ER al esquema del modelo relacional no se realizará ninguna normalización, porque está fuera del alcance para el equipo programar la lógica necesaria y se necesita un diagrama no normalizado para la transformación al modelo conceptual NoSQL.


Dado que hay varias propuestas en la literatura sobre modelos conceptuales NoSQL, se optará por usar la propuesta de Alfonso de la Vega\cite{de_la_vega_mortadelo_2020}, ya que el modelo que propone, conocido como \textit{Generic Data Metamodel}, es un metamodelo conceptual que describe un modelo de datos NoSQL independientemente de si es de clave-valor, orientado a documentos, a columnas o grafos.

No se implementará la opción de marcar una entidad en el \textit{Generic Data Metamodel} como \textit{muy usada}.

Respecto a las transformaciones modelo a modelo y modelo a texto, se necesita implementar una gramática del modelo conceptual NoSQL, otra gramática para el modelo lógico orientado a documentos y la transformación entre ambas gramáticas. 


Debido a la difícultad de implementar de cero un compilador por cada gramática y porque de acuerdo con Kahani\cite{kahani_survey_2019}, el 75 \% de las herramientas para transformaciones dirigidas a modelo están escritas en Java, además de que las herramientas en Java son las únicas que tienen la generación automática del analizador léxico, analizador sintántico y analizador morfológico que se necesita para manipular cada gramática.


Por las razones antes mencionadas y porque el propósito del proyecto es mostrar un diagrama de un modelo orientado a documentos a partir de un diagrama entidad-relación, no implementar compiladores para gramáticas, se optará por usar el código fuente de Mortadelo para las transformaciones modelo a modelo (que están implementadas usando el Eclipse Modeling Framework\cite{steinberg_emf_2008} de Java).


De ser posible, como un trabajo a futuro se realizará una implementación propia para la transformaciones entre modelos. 


Los \textit{scripts} NoSQL generados por la aplicación solo serán probados y ejecutados en un único NoSQL DBMS.


Se eligirá el modelo de datos orientado a documentos y se usará MongoDB para probar los \textit{scripts} NoSQL generados por la aplicación, porque de acuerdo a Mosquera\cite{martinez-mosquera_modeling_2020}, las de bases de datos orientadas a documentos son las más estudiadas y MongoDB es el NoSQL DBMS más probado para modelado físico en la literatura del tema.


Respecto al registro de los usuarios, se podrá acceder a la aplicación con un correo válido o no válido, teniendo en cuenta de que si no crea su cuenta con un correo válido, no le llegará notificación de correo que le indica que se ha registrado exitosamente junto con su contraseña.


El diagramador ER solo podrá editar elementos del modelo entidad-relación, no se podrá diagramar elementos del modelo entidad-relación extendido ni de ningún otro modelo.


No se mostrará el diagrama generado del modelo relacional, porque solo se crea para generar las sentencias SQL para ser probadas en MySQL.


La propuesta de solución se podrá ejecutar en el navegador web Google Chrome versión 81 en adelante.


Si un diagrama entidad-relación básico no contiene consultas, solo se podrá obtener el esquema del modelo relacional (sentencias SQL); en caso de que exista al menos una consulta, se puede obtener el modelo conceptual NoSQL.
%El grado de partición en el modelo entidad-relación será máximo de dos entidades.
    