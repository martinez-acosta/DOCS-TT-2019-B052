\section{Objetivos}

\subsection*{Objetivo general}
Desarrollar una aplicación web que brinde la funcionalidad de una herramienta CASE que permita la edición de un diagrama de bases de datos bajo el modelo entidad-relación y realice la validación del mismo; el diagrama podrá ser transformado al modelo relacional con la posibilidad de obtener el esquema de la base de datos en sentencias SQL, o bien, obtendrá el modelo conceptual en un modelo de datos no relacional y tendrá la posibilidad de implementar el modelo no relacional en un NoSQL DBMS (\textit{NoSQL Database Management System}).

\subsection*{Objetivos específicos}

La edición del diagrama entidad-relación básico (o diagrama ER) se implementará con alguna biblioteca para diagramado, en donde al usuario se le permitirá arrastar los entidades, atributos y relaciones del diagrama ER desde una ``paleta''  a la zona para diagramar, conocida también como ``canvas''.


La validación del diagrama ER se realizará con eventos de escucha en el ``canvas'' o zona de diagramado y será de acuerdo a reglas de validación que se expondrán más adelante.


La transformación del diagrama ER al diagrama del esquema relacional se realizará con alguno de los algoritmos conocidos en la literatura del tema y se generará la definición del esquema de sentencias SQL desde el diagrama del esquema relacional.


Se implementará algún modelo conceptual NoSQL que exista en la literatura del tema.


Se generará la definición del esquema de una base de datos NoSQL desde el modelo conceptual elegido y se probará en un único NoSQL DBMS.