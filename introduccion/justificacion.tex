\section{Justificación}

Las pocas herramientas para modelado conceptual NoSQL no ofrecen un modelado desde un modelo conceptual relacional, porque se enfocan en un modelo conceptual específico para cada tipo sistemas NoSQL (clave-valor, orientado a documentos, columnas o grafos). Tampoco ofrecen validaciones para sus diagramas ni dan la posiblidad de obtener el esquema SQL, ya que no están enfocados en sistemas relacionales.


Asimismo, como se mencionó en la introducción, por los pocos años que han pasado desde que el interés y el uso de los sistemas NoSQL aumentó, estos sistemas todavía no están reflejados en los programas de estudio.


En consecuencia, para abordar esta problemática en la que el estudiante diseñe un sistema NoSQL sin haber aprendido ninguna metodología de desarrollo para dichos sistemas y sin que logre visualizar en primera instancia el diseño de una base de datos no relacional a partir de lo que ve en la asignatura de Bases de Datos, se propone una herramienta CASE que apoye al estudiante a diseñar una base de datos NoSQL a partir de los conocimientos adquiridos del estudiante.


Se usará como modelo conceptual el modelo entidad-relación por ser un modelo bien conocido para el estudiante de licenciatura en sistemas, que está probado en el área de las bases de datos y de acuerdo con Mosquera\cite{martinez-mosquera_modeling_2020}, es también el más usado en las investigaciones sobre el modelado conceptual de sistemas NoSQL.


%Respecto al modelo físico que obtenga la herramienta a desarrollar, se probará en el esquema orientado a documentos en una base de datos NoSQL orientada a documentos.

