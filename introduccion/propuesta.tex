\section{Propuesta de solución}

Desarrollar una aplicación web que permita:

\begin{itemize}
    \item $1.^{o}$: la edición de un diagrama entidad-relación o diagrama ER.
    \item $2.^{o}$: realizar la validación del diagrama ER.
    \item $3.^{o}$: transformar el diagrama ER a un diagrama del modelo relacional o diagrama Relacional.
    \item $4.^{o}$: obtener el esquema de la base de datos en sentencias SQL del diagrama Relacional.
    \item $5.^{o}$: obtener el esquema conceptual de un modelo de datos NoSQL orientado a documentos
    \item $6.^{o}$: obtener el esquema de datos NoSQL en un NoSQL RDBMS (\textit{NoSQL Database Management System}).
\end{itemize}

Respecto al primer punto, la edición del diagama ER se implementará con alguna biblioteca para diagramado, en donde al usuario se le permitirá arrastar los elementos del diagrama ER desde una ``paleta" (como una entidad, un atributo, etc. ) a la zona para diagramar, conocida también como ``canvas".


Para el segundo punto, la validación se realizará con eventos de escucha en el ``canvas'' o zona de diagramado y será de acuerdo a reglas de validación que se expondrán más adelante.


Para el tercer punto, la tranformación del diagrama ER al diagrama Relacional se realizará con alguno de los algoritmos bien conocidos en la literatura del tema y así como en el segundo punto, también se expondrá más adelante el algoritmo a usar.


Asimismo, para el cuarto punto se generarán las sentencias SQL desde el diagrama Relacional.


Para el quinto punto, dado que hay varias propuestas en la literatura sobre modelos conceptuales NoSQL, se optará por usar la propuesta de Alfonso de la Vega\cite{de_la_vega_mortadelo_2020}, ya que el modelo que propone, conocido como Generic Data Metamodel, es un modelo conceptual que sirve para describir una base de datos orientada a documentos, a columnas, de clave-valor y orientada a grafos. Además, es una notación que está probada con varios casos de estudio.


Finalmente, para el sexto punto se generará el esquema de una base de datos NoSQL desde el Generic Data Metamodel del punto anterior.