\section{Descripción del problema}
Como se ha visto en la introducción, son pocas las herramientas propuestas en la literatura para el modelado de sistemas NoSQL en sus tres niveles de abstracción (nivel conceptual, lógico y físico).


Solo en el modelado conceptual, el modelo entidad-relación puede considerarse como una tendencia, sin embargo, el modelo entidad-relación por sí mismo no es suficiente para representar cómo se consultarán los datos ni tampoco con qué frecuencia se accederán a ellos; por eso, para modelar sistemas NoSQL es necesario conocer enfoques de desarrollo como el \textit{query-driven design}, el \textit{domain-driven design}, el \textit{data-driven design} o el \textit{workload-driven design}.


En general, para los modelos de datos NoSQL no hay un modelo estándar ni tampoco existe un acuerdo sobre la mejor definición de reglas de transformación entre sus tres niveles de abstracción; por ejemplo, en la literatura sobre el tema se pueden encontrar unos 36 estudios que proponen diferentes enfoques para las transformaciones.


Por los pocos años que han pasado desde que el interés sobre el tema aumentó, que son usados estos sistemas de bases de datos y los diferentes enfoques de tranformación entre modelos, aún no se ven reflejados en los programas de estudio o solo se da una introducción del tema a los estudiantes de licenciatura en ingeniería en sistemas o carreras afines. 


Asimismo, en varias ocasiones el estudiante para verificar la validez de los diagramas entidad-relación que desarrolla tiene que confiar en sus conocimientos recientemente adquiridos (sin tener certeza de la validez) o consultarlo con el maestro de turno.


En consecuencia, para el estudiante es un problema tener que enfrentarse a diseñar un sistema NoSQL, porque no tiene la certeza de desarrollar diagramas entidad-relación válidos, no se le enseña ninguna metodología de desarrollo para sistemas NoSQL, tampoco conoce los distintos modelos de datos NoSQL que hay, mucho menos sus ventajas o desventajas para poder elegir el modelo más apropiado para su aplicación y, como resultado, no pueden visualizar el diseño de una base de datos no relacional a partir de los conocimientos adquiridos en la asignatura de Bases de Datos.


Además, la escasez de herramientas CASE para el diagramado de esquemas no relacionales o herramientas que apoyen la migración de un modelo entidad-relación o relacional a uno no relacional aumenta la complejidad para que el estudiante haga uso de una base de datos no relacional.
