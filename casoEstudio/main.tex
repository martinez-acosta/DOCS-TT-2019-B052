
De acuerdo con De la Vega\cite{de_la_vega_mortadelo_2020}, el modelo entidad-relación más usado en la literatura NoSQL es el modelo Venues que puede encontrar en el \textit{paper} de Chebotko~\cite{chebotko_big_2015}.

\section{Diagramado modelo entidad-relación básico}

El diagramado de Venues se ha hecho con la propuesta de solución desarrollada y en la figura~\ref{img:venues-er}

\begin{figure}[H]
    \centering
    \includegraphics[width=0.75\textwidth]{casoEstudio/venues-er.png}
    \caption{Diagrama entidad-relación de Venues}
    \label{img:venues-er}
\end{figure}

\section{Validación modelo entidad-relación básico}

 Una vez que el usuario cree tener un diagrama válido o quiere conocer si su diagrama actual presenta algún error, puede hacer click en el boton ``validar diagrama'' con lo que se le mostrará una ventana con los errores que la aplicación detecta. La figura~\ref{img:ex_diagrama1} muestra un ejemplo de diagrama sencillo mientras que en la figura~\ref{img:errors_diagrama1} se aprecian los errores estructurales del mismo.

 \begin{figure}[H]
  \begin{subfigure}[b]{0.49\textwidth}
      \includegraphics[width=\textwidth]{casoEstudio/ex_diagrama1.png}
      \caption{Ejemplo de un diagram er básico.}
      \label{img:ex_diagrama1}
    \end{subfigure}
    \hfill
    \begin{subfigure}[b]{0.49\textwidth}
      \includegraphics[width=\textwidth]{casoEstudio/errors_diagrama1.png}
      \caption{Ventana de errores del diagrama er.}
      \label{img:errors_diagrama1}
    \end{subfigure}
\end{figure}

 Siguiendo con el ejemplo del modelo Venues la figura~\ref{img:venues_validationER} muestra la validación del diagrma y como se espera este es estructuralmente válido.

 \begin{figure}[H]
  \centering
  \includegraphics[width=0.75\textwidth]{casoEstudio/vuenues_validation.png}
  \caption{Validación del modelo Venues.}
  \label{img:venues_validationER}
\end{figure}

\section{Generación sentencias SQL}
 Una vez validado el diagrama entidad-relación básico se pueden generar las sentencias SQL que se muestra en la figura~\ref{img:venues-sql} 

 \begin{figure}[H]
    \centering
    \includegraphics[width=0.75\textwidth]{casoEstudio/venues-sql.png}
    \caption{Sentencias SQL de Venues.}
    \label{img:venues-sql}
\end{figure}

\subsection*{Prueba de sentencias SQL en MySQL}

Como se muestra en la figura~\ref{img:mysql_workbench} se ha probado la ejecución de las sentencias sql en el gestor gráfico MySQL Workbench en su versión 8.0.22 y al no apreciar en la parte inferior de la figura ningún error se concluye que la ejecución ha sido exitosa.

\begin{figure}[H]
  \centering
  \includegraphics[width=0.75\textwidth]{casoEstudio/mysql_workbench.png}
  \caption{Ejecución de las sentencias SQL en MySQL Workbench.}
  \label{img:mysql_workbench}
\end{figure}

\section{Modelo conceptual NoSQL}

Con un diagrama entidad-relación básico válido es posible generar las entidades del modelo conceptual NoSQL como se muestra en la figura~\ref{img:venues-conceptual}.

\begin{figure}[H]
    \centering
    \includegraphics[width=0.75\textwidth]{casoEstudio/venues-conceptual.png}
    \caption{Entidades del modelo conceptual GDM}
    \label{img:venues-conceptual}
\end{figure}

Una vez generadas las entidades del modelo conceptual GDM es necesario que el usuario defina las consultas en la sección de consultas del GDM como se muestra en la figura~\ref{img:venues-conceptual-queries}.

\begin{figure}[H]
    \centering
    \includegraphics[width=0.75\textwidth]{casoEstudio/venues-conceptual-queries.png}
    \caption{Entidades y consultas del modelo conceptual GDM}
    \label{img:venues-conceptual-queries}
\end{figure}

El modelo Venues de  Chebotko contiene nueve consultas, que en el GDM son las siguientes:

\begin{verbatim}
    query Q1_artifactsByVenue:
    select venue.venueName, venue.year,
           artifacts.artifactId, artifacts.artifactTitle, artifacts.avgRating,
           artifacts.authors, artifacts.keywords
    from Venue as venue
    including venue.featuresArtifact as artifacts
    where venue.venueName = "?"
      and venue.year > "?"
  
  query Q2_artifactsByAuthor:
    select artifact.artifactId, artifact.artifactTitle, artifact.avgRating,
           artifact.authors, artifact.keywords, artifact.authors as author
    from Artifact as artifact
    where artifact.authors = "?"
  
  query Q3_usersByLikedArtifact:
    select user.userId, user.userName, user.userEmail
    from User as user
    including user.likesArtifact as artifacts
    where artifacts.artifactId = "?"
  
  query Q4_expertsByLikedArtifact:
    select user.userId, user.areasOfExpertise, user.userName, user.userEmail,
           user.areasOfExpertise as areaOfExpertise
    from User as user
    including user.likesArtifact as artifacts
    where artifacts.artifactId = "?"
      and user.areasOfExpertise = "?"
  
  query Q5_ratingByArtifact:
    select artifact.artifactId, artifact.avgRating
    from Artifact as artifact
    where artifact.artifactId = "?"
  
  query Q6_venuesLikedByUser:
    select user.userId, venues.venueName, venues.year,
           venues.country, venues.homepage
    from User as user
    including user.likesVenue as venues
    where user.userId = "?"
  
  query Q7_artifactsLikedByUser:
    select user.userId,
           likesArtifacts.artifactId,
           likesArtifacts.artifactTitle, likesArtifacts.authors,
           venue.venueName
    from User as user
    including user.likesArtifact as likesArtifacts,
              user.likesArtifact.venue as venue
    where user.userId = "?"
      and venue.year > "?"
  
  query Q8_reviewsByUser:
    select user.userId,
           reviews.reviewId, reviews.reviewTitle, reviews.body,
           artifact.artifactId, artifact.artifactTitle
    from User as user
    including user.postsReview as reviews,
              user.postsReview.artifact as artifact
    where user.userId = "?"
      and reviews.rating > "?"
  
  query Q9_artifacts:
    select artifact.artifactId, artifact.artifactTitle, artifact.avgRating,
           artifact.authors, artifact.keywords,
           venue.venueName
    from Artifact as artifact
    including artifact.venue as venue
    where artifact.artifactId = "?"
\end{verbatim}

\section{Modelo lógico NoSQL}

A partir de un modelo conceptual GDM, que está conformado por entidades y consultas, se puede generar el modelo lógido orientado a documentos. En la figura~\ref{img:venues-logical} se muestra el diagrama correspondiente al modelo lógico orientado a documentos de Venues.

\begin{figure}[H]
    \centering
    \includegraphics[width=0.75\textwidth]{casoEstudio/venues-logical.png}
    \caption{Modelo lógico orientado a documentos de Venues}
    \label{img:venues-logical}
\end{figure}

\section{Modelo físico NoSQL}

Tambien se pueden obtener las sentencias para MongoDB. Para el modelo Venues se muestra a continuación las sentencias correspondientes a su modelo lógico.

\begin{minted}[linenos,tabsize=2,breaklines, fontsize=\small]{json}
    db.createCollection("Venue",
{
  validator: {
    $jsonSchema: {    bsonType: "object",
    properties: {
venueName: { bsonType: "string"
        },year: { bsonType: "double"
        },venueId: { bsonType: "objectId"
        },homepage: { bsonType: "string"
        },country: { bsonType: "string"
        },topics: { bsonType: "string"
        },artifactRefArray: {
    bsonType: [
            "array"
          ],
    items: {
         properties: {
artifactRef: { bsonType: "objectId"
              }
            }
          }
        }
      }
    }
  }
})
db.createCollection("Artifact",
{
  validator: {
    $jsonSchema: {    bsonType: "object",
    properties: {
artifactId: { bsonType: "objectId"
        },authors: { bsonType: "string"
        },numRatings: { bsonType: "double"
        },avgRating: { bsonType: "double"
        },sumRatings: { bsonType: "double"
        },artifactTitle: { bsonType: "string"
        },keywords: { bsonType: "string"
        },venueRef: { bsonType: "objectId"
        }
      }
    }
  }
})
db.createCollection("User",
{
  validator: {
    $jsonSchema: {    bsonType: "object",
    properties: {
userId: { bsonType: "objectId"
        },userName: { bsonType: "string"
        },userEmail: { bsonType: "string"
        },areasOfExpertise: { bsonType: "string"
        },artifactRefArray: {
    bsonType: [
            "array"
          ],
    items: {
         properties: {
artifactRef: { bsonType: "objectId"
              }
            }
          }
        },venueRefArray: {
    bsonType: [
            "array"
          ],
    items: {
         properties: {
venueRef: { bsonType: "objectId"
              }
            }
          }
        },reviewArray: {
    bsonType: [
            "array"
          ],
    items: {
         properties: {
rating: { bsonType: "double"
              },body: { bsonType: "string"
              },reviewTitle: { bsonType: "string"
              },reviewId: { bsonType: "objectId"
              },artifactRef: { bsonType: "objectId"
              }
            }
          }
        }
      }
    }
  }
})
\end{minted}


\subsection*{Prueba de sentencias mongo en MongoDB}

Como se muestra en la figura~\ref{img:venues-mongo} se ha probado crear las colecciones en el manejador MongoDB y se solicita a MongoDB la información sobre la colección creada. Posteriormente, MongoDB muestra correctamente las reglas de validación para la colección Venue.

\begin{figure}[H]
    \centering
    \includegraphics[width=0.75\textwidth]{casoEstudio/venues-mongo.png}
    \caption{Prueba de sentencias del modelo físico orientado a documentos de Venues en MongoDB}
    \label{img:venues-mongo}
\end{figure}

\section{Conclusiones}
Como se muestra en este caso de estudio, la aplicación diagrama correctamente el modelo Venues, genera sus sentencias SQL que están probadas en MySQL, también genera correctamente el modelo conceptual, lógico y físico de un modelo NoSQL. Se diagrama el modelo lógico y se prueban las sentencias del modelo físico en MongoDB.

