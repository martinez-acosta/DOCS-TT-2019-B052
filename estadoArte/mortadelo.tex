\section{Mortadelo}
De acuerdo con de la Vega\cite{de_la_vega_mortadelo_2020}, Mortadelo está basado en el \textit{model driven}; es decir que para diseñar bases de datos NoSQL necesita de modelos conceptuales definidos.


Lo anterior implica usar dos modelos diferentes pero interrelacionados que añaden complejidad al modelado de la base datos; por esta razón de la Vega propone el GDM (\textit{Generic Data Metamodel}), que es un modelo conceptual NoSQL donde la estructura (entidades, atributos, relaciones entre entidades) y patrones de acceso (cómo se consultarán los datos) están integradas en un mismo modelo conceptual.


Cabe destacar que el GDM es un modelo conceptual independiente del paradigma NoSQL, por lo que puede representar una base de datos NoSQL de clave-valor, orientado a documentos, orientado a columnas u orientado a grafos.

La figura~\ref{img:mortadelo-process} muestra los tres pasos principales de Mortadelo, empezando desde la izquierda con su propuesta de modelo conceptual NoSQL, el \textit{Generic Data Metamodel}, siguiendo con una representación de los modelos lógicos orientado a columnas o documentos en los que realiza ``modelo a modelo'' (M2M) para generar el modelo lógico, para finalizar una transformación ``modelo a texto'' (M2T) con el modelo físico de cada modelo lógico, respectivamente.


\begin{figure}[H] 
    \centering
    \includegraphics[width=0.75\textwidth]{mortadelo/01.png}
    \caption{Mortadelo}
    \label{img:mortadelo-process}
\end{figure}
