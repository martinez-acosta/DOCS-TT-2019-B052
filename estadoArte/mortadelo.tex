\subsection{Mortadelo}
De acuerdo a Alfonso de la Vega\cite{de_la_vega_mortadelo_2020}, Mortadelo es un modelo conceptual para el diseño de bases de datos NoSQL. 


Asimismo, define un proceso de transformación desde su propuesta de modelo conceptual denotado como Generic Data Model o GDM para generar \textit{scripts} del modelo físco de alguno de los modelos de datos NoSQL (clave-valor, orientado a documentos, orientado a columnas o orientado a grafos). 


El GDM está compuesto por dos bloques, el primero que contiene la información sobre el dominio de las entidades y sus relaciones (modelo de estructura) y el segundo bloque tiene la definición de cómo se harán las consultas de los datos (consultas de acceso).


Para la comprobación de su validez, se ha implementado una herramienta que admite la generación de almacenes de datos de columnas con transformaciones concretas para Cassandra, así mismo ofrece soporte preliminar de los basados ​​en documentos.
