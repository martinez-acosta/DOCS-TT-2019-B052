De acuerdo a varios autores\cite{chebotko_big_2015,de_lima_workload-driven_2015,mior_nose_2017-1}, un modelo conceptual que especifique solo qué entidades conforman el sistema no es suficiente, porque para los sistemas NoSQL es clave conocer cómo las entidades del modelo conceptual serán consultadas; por eso estos autores complementan los modelos conceptuales tradicionales como el modelo entidad-relación o el modelo relacional con propuestas sobre cómo conocer los patrones de acceso; sin embargo, ¿qué herramientas usan estas propuestas?


Para responder esta pregunta en este capítulo se presenta una investigación de herramientas similares, empezando por dos inspiradas por el modelo conceptual propuesto por Chebotko en 2015; además, se muestran otras tres herramientas, todas basadas en diferentes propuestas sobre modelado conceptual, explicando a detalle Mortadelo, la herramienta elegida en el proyecto para representar conceptualmente una base de datos NoSQL.


Lo que resta de este capítulo está organizado de la siguiente manera: se presenta la descripción de cada oferta y se finaliza en las conclusiones con una tabla comparativa.

%\section{Herramientas similares}
\section{Kashliev Data Modeler}
Es una herramienta de modelado desarrollada a partir del 2015 por Andrii Kashliev, profesor asistente del Departamento de informática de la Universidad del Oeste Michigan, que automatiza el diseño de esquemas para Apache Cassandra, una base de datos NoSQL orientada a columnas. 


Por medio de una aplicación web, KDM ayuda al usuario en el modelado de datos, comenzando con un modelo conceptual de datos (de una notación familiar al modelo entidad-relación) y generando un modelo físico de datos o \textit{script} CQL (Cassandra Query Language). 


KDM automatiza: 

\begin{enumerate}
    \item El mapeo conceptual a lógico
    \item El mapeo lógico a físico
    \item La generación de \textit{script} CQL 
\end{enumerate}

El modelo de datos usado en KDM es el propuesto por Chebotko\cite{chebotko_big_2015} y por medio de un algoritmo, KDM automatiza el proceso de pasar del modelo de datos a un modelo de datos físicos. 


Además, KDM cuenta con una versión de prueba de tiempo indefinido con características limitadas que permite la generación de un modelo lógico y guardar los proyectos de KDM.

\section{NoSQL Workbench for Amazon DynamoDB}

De acuerdo al sitio web de Amazon\cite{amazon_nosql_2020}, NoSQL Workbench for Amazon DynamoDB es una herramienta desarrollada por Amazon que proporciona funciones de desarrollo de consultas, modelado y visualización de datos para diseñar, crear, consultar y administrar bases de datos NoSQL orientadas a columnas.


NoSQL Workbench for Amazon DynamoDB usa el modelo conceptual propuesto por Chebotko\cite{chebotko_big_2015}.


El visualizador de modelo de datos proporciona un lienzo donde se asignan consultas y visualizan las facetas (parte de la base de datos) de la aplicación sin tener que escribir código.


Cada faceta corresponde a un patrón de acceso diferente en DynamoDB, donde cada patrón se agrega manualmente; cuenta con un generador de operaciones para ver, explorar y consultar conjuntos de datos.


Por último, admite la proyección, la declaración de expresiones condicionales y permite generar código de muestra en varios idiomas.


\section{HacKolade}
De acuerdo al sitio web de Hackolade\cite{hackolade_hackolade_2020}, es un \textit{software} con capacidad de representar objetos JSON anidados; la aplicación combina la representación gráfica de colecciones (término usado en las bases de datos orientados a documentos) y vistas en un diagrama.


La aplicación está basada en la desnormalización, el polimorfismo y matrices anidadas JSON; la representación gráfica de la definición del esquema JSON de cada colección está en una vista de árbol jerárquica. 


Hackolade genera dinámicamente \textit{scripts} MongoDB a medida que construye un modelo de datos, deriva modelos de datos por medio de ingeniería inversa de instancias MongoDB, dando facilidad para obtener descripciones, propiedades y restricciones.


Lamentablemente, en el sitio web de HacKolade no hay información sobre qué modelo conceptual está basado su aplicación, así que se asume que usan una propuesta propia para el modelo conceptual del que hacen uso.


Es una herramienta de modelado de datos para MongoDB, Neo4j, Cassandra, Couchbase, Cosmos DB, DynamoDB, Elasticsearch, HBase, Hive, Google BigQuery, Firebase/Firestore, MarkLogic, entre otros.
\section{Mortadelo}
De acuerdo con de la Vega\cite{de_la_vega_mortadelo_2020}, Mortadelo está basado en el \textit{model driven}; es decir que para diseñar bases de datos NoSQL necesita de modelos conceptuales definidos.


Lo anterior implica usar dos modelos diferentes pero interrelacionados que añaden complejidad al modelado de la base datos; por esta razón de la Vega propone el GDM (\textit{Generic Data Metamodel}), que es un modelo conceptual NoSQL donde la estructura (entidades, atributos, relaciones entre entidades) y patrones de acceso (cómo se consultarán los datos) están integradas en un mismo modelo conceptual.


Cabe destacar que el GDM es un modelo conceptual independiente del paradigma NoSQL, por lo que puede representar una base de datos NoSQL de clave-valor, orientado a documentos, orientado a columnas u orientado a grafos.

La figura~\ref{img:mortadelo-process} muestra los tres pasos principales de Mortadelo, empezando desde la izquierda con su propuesta de modelo conceptual NoSQL, el \textit{Generic Data Metamodel}, siguiendo con una representación de los modelos lógicos orientado a columnas o documentos en los que realiza ``modelo a modelo'' (M2M) para generar el modelo lógico, para finalizar una transformación ``modelo a texto'' (M2T) con el modelo físico de cada modelo lógico, respectivamente.


\begin{figure}[H] 
    \centering
    \includegraphics[width=0.75\textwidth]{mortadelo/01.png}
    \caption{Mortadelo}
    \label{img:mortadelo-process}
\end{figure}
\subsubsection*{Mortadelo: modelo conceptual (Generic Data Metamodel)}

La figura~\ref{img:mortadelo-gdm} muestra el \textit{Generic Data Metamodel}, que está compuesto por clases interrelacionadas entre sí con notación UML\footnote{La simbología usada en los diagramas de clase UML está en el apéndice.} y consta de dos secciones principales: los elementos de la estructura (\textit{structure model elements}) y el cómo se realizarán las consultas (\textit{access queries elements}).


%Como se aprecia en la figura~\ref{img:mortadelo-gdm}, el modelo conceptual que propone de la Vega contiene en un mismo modelo con notación UML los elementos de la estructura (\textit{structure model elements}) y el cómo se realizarán las consultas (\textit{access queries elements}).


\begin{figure}[H] 
    \centering
    \includegraphics[width=0.75\textwidth]{mortadelo/GDM.png}
    \caption{Generic Data Metamodel}
    \label{img:mortadelo-gdm}
\end{figure}


A continuación se da una explicación de cada elemento de la figura~\ref{img:mortadelo-gdm} donde la clase \textit{model} es para indicar que un modelo GDM tiene $n$ entidades y $n$ consultas donde $n=0,1,...,n$.


\paragraph*{Structure model elements}


\begin{itemize}    
    
    \item Clase \textit{entity}: contiene \textit{features} y solo es referenciada directamente desde las clases \textit{from} y \textit{reference}.
    \item Clase \textit{feature}: es una clase abstracta, una \textit{reference}, un \textit{attribute}, o hereda de la clase \textit{annotatable element} para que la instancia de la clase sea comentada con indicadores de texto que proveen información extra para generar el modelo lógico. 
    \item Clase \textit{reference}: empieza con la palabra clave ``ref'', un nombre de tipo de entidad, una cardinalidad y un nombre de referencia; por ejemplo: ``ref Category[*] categories'' define una referencia llamada \textit{categories}, del tipo de entidad \textit{category} con una cardinalidad de cero a varios.
    \item Clase \textit{attribute}: contiene un tipo y un identificador de unicidad.
    \item Clase \textit{annotatable element}: es una clase abstracta para permitir que una clase contenga anotaciones.
    \item Clase \textit{annotation}: es un indicador de texto que provee información extra para generar el modelo lógico.
    
\end{itemize}

\paragraph*{Access queries elements}


\begin{itemize}
    
    \item Clase \textit{query}: tiene solo un elemento de la clase \textit{from}, $n$ elementos de la clase \textit{attribute selection}, $n$ elementos de la clase \textit{inclusion} y tiene o no un único elemento de la clase \textit{boolean expression}.
    \item Clase \textit{from}: asocia una clase \textit{query} con la clase \textit{entity}; es la clase que permite referenciar un tipo de entidad.
    \item Clase \textit{attribute selection}: accede a los atributos de la \textit{entity} referenciada por la clase \textit{from} o la clase \textit{inclusion}.
    \item Clase \textit{boolean expression}: expresa una expresión booleana para declarar alguna restricción.
    \item Clase \textit{inclusion}: permite acceder en una \textit{query} a los atributos de otros tipos de entidad.
    
\end{itemize}

La figura~\ref{img:mortadelo-gdm.textual.notation} es una instancia del modelo conceptual GDM en su notación textual en la que se nota que, por ejemplo, en la tercera consulta se accede a los elementos de la entidad \textit{category} a través del elemento ref de la entidad \textit{product}.


\begin{figure}[H] 
    \centering
    \includegraphics[width=0.65\textwidth]{mortadelo/GDM-textual-notation.png}
    \caption{Notación textual del GDM}
    \label{img:mortadelo-gdm.textual.notation}
\end{figure}


Para más detalles sobre su modelo lógico orientado a documentos y algoritmos asociados, visite la sección~\ref{alg:gdm-to-logic}.
%Como se ha mostrado, Mortadelo es quizá la herramienta más completa de las estudiadas y Alfonso de la Vega expone en su \textit{paper} no solo el modelado conceptual, lógico y físico de una base de datos NoSQL, sino también explica a detalle los algoritmos usados con casos de uso.
\section{NoSE: Schema Design for NoSQL Applications}
De acuerdo con Michael J. Mior\cite{mior_nose_2017}, NoSE usa un modelo conceptual junto con el \textit{workload} para describir cómo se accederán a los datos y así genera un modelo físico de una base de datos NoSQL orientado a columnas.


NoSE debe tener un modelo conceptual que describa la información que se almacenará, por ello NoSE espera este modelo conceptual en forma de un grafo de entidad.


Los grafos de entidad son un tipo restringido del modelo de entidad-relación; cada cuadro representa un tipo de entidad, tienen atributos en los que uno más sirven como clave para identificarla, cada borde es una relación entre entidades y la cardinalidad asociada de la relación (uno a muchos, uno a uno o muchos a muchos). 


Respecto al \textit{workload}, se describe como un conjunto de consultas parametrizadas y declaraciones de ``actualización''; cada consulta y actualización está asociada con un peso que indica su frecuencia relativa en la carga de trabajo.
\section{Conclusiones}

Como se ha podido ver en este capítulo, son pocas las herramientas para modelado conceptual NoSQL; de ellas, dos usan la propuesta de Chebotko para modelar conceptualmente bases de datos orientadas a columnas y las demás herramientas modelan bases de datos NoSQL con base en diferentes propuestas.


Asimismo, es de notar que ninguna herramienta permite la validación estructural del modelo conceptual que usan, ni tampoco generan el esquema del modelo relacional, porque no están enfocadas a modelar bases de datos relacionales y por la misma razón no producen los \textit{scripts} de sentencias SQL del esquema relacional.


Para finalizar este capítulo, la tabla~\ref{tab:tabla-comparativa} muestra una comparación de las diferentes características de cada herrmienta expuesta en este capítulo.

% Please add the following required packages to your document preamble:
% \usepackage{booktabs}
% \usepackage{graphicx}
\begin{table}[H]
	\centering
	\resizebox{\textwidth}{!}{%
	\begin{tabular}{@{}llllllllllllll@{}}
	\toprule
	Herramienta                                                                         & Creador                                                       & Objetivo      & Licencia  & \begin{tabular}[c]{@{}l@{}}Sistema\\ Operativo\end{tabular}     & \begin{tabular}[c]{@{}l@{}}Fecha de\\ publicación\end{tabular} & \multicolumn{3}{c}{Modelo}                                                                                                                                                                  & \begin{tabular}[c]{@{}l@{}}Patrones \\ de Acceso\end{tabular} & Metodología                                                & \begin{tabular}[c]{@{}l@{}}Validación\\ Estructural\end{tabular} & \begin{tabular}[c]{@{}l@{}}Modelado\\ Relacional\end{tabular} & \begin{tabular}[c]{@{}l@{}}Esquema\\ SQL\end{tabular} \\ \midrule
																						&                                                               &               &           &                                                                 &                                                                & \multicolumn{1}{c}{Conceptual}                                & \multicolumn{1}{c}{Lógico}                                    & \multicolumn{1}{c}{Físico}                                  &                                                               &                                                            &                                                                  &                                                               &                                                       \\ \cmidrule(lr){7-9}
																						&                                                               &               &           &                                                                 &                                                                &                                                               &                                                               &                                                             &                                                               &                                                            &                                                                  &                                                               &                                                       \\
	KDM                                                                                 & \begin{tabular}[c]{@{}l@{}}Andrii \\ Kashliev\end{tabular}    & Comercial     & Privativa & Web                                                             & 2015                                                           & \begin{tabular}[c]{@{}l@{}}entidad-\\ relación\end{tabular}   & Columnas                                                      & Cassandra                                                   & Querys                                                        & \begin{tabular}[c]{@{}l@{}}query-\\ driven\end{tabular}    & N/A                                                              & N/A                                                           & N/A                                                   \\
	HacKolade                                                                           & \begin{tabular}[c]{@{}l@{}}IntegrIT \\ SA / NV\end{tabular}   & Comercial     & Privativa & \begin{tabular}[c]{@{}l@{}}Windows,\\ Mac,\\ Linux\end{tabular} & 2016                                                           & \begin{tabular}[c]{@{}l@{}}entidad-\\ relación\end{tabular}   & \begin{tabular}[c]{@{}l@{}}Multipara-\\ digma\end{tabular}    & \begin{tabular}[c]{@{}l@{}}Multipara-\\ digma\end{tabular}  & N/A                                                           & N/A                                                        & N/A                                                              & N/A                                                           & N/A                                                   \\
	\begin{tabular}[c]{@{}l@{}}NoSQL \\ Workbench\\ for Amazon\\  DynamoDB\end{tabular} & Amazon                                                        & Comercial     & Privativa & \begin{tabular}[c]{@{}l@{}}Windows,\\ Mac\end{tabular}          & 2019                                                           & \begin{tabular}[c]{@{}l@{}}entidad-\\ relación\end{tabular}   & Columnas                                                      & MongoDB                                                     & Querys                                                        & \begin{tabular}[c]{@{}l@{}}query-\\ driven\end{tabular}    & N/A                                                              & N/A                                                           & N/A                                                   \\
	Mortadelo                                                                           & \begin{tabular}[c]{@{}l@{}}Alfonso \\ de la Vega\end{tabular} & Investigación & Libre     & Web                                                             & 2018                                                           & GDM                                                           & \begin{tabular}[c]{@{}l@{}}Columnas\\ Documentos\end{tabular} & \begin{tabular}[c]{@{}l@{}}Cassandra\\ MongoDB\end{tabular} & Querys                                                        & \begin{tabular}[c]{@{}l@{}}query-\\ driven\end{tabular}    & N/A                                                              & N/A                                                           & N/A                                                   \\
	NoSE                                                                                & \begin{tabular}[c]{@{}l@{}}Michael\\  J. Mior\end{tabular}    & Investigación & Libre     & Web                                                             & 2016                                                           & \begin{tabular}[c]{@{}l@{}}Grafos de\\ entidades\end{tabular} & Documentos                                                    & MongoDB                                                     & Workload                                                      & \begin{tabular}[c]{@{}l@{}}workload-\\ driven\end{tabular} & N/A                                                              & N/A                                                           & N/A                                                   \\
	\begin{tabular}[c]{@{}l@{}}Propuesta de\\  solución\end{tabular}                    & \begin{tabular}[c]{@{}l@{}}TT \\ 2019-B052\end{tabular}       & Investigación & Libre     & Web                                                             & 2020                                                           & \begin{tabular}[c]{@{}l@{}}entidad-\\ relación\end{tabular}   & Documentos                                                    & MongoDB                                                     & Querys                                                        & \begin{tabular}[c]{@{}l@{}}query-\\ driven\end{tabular}    & Sí                                                               & Sí                                                            & Sí                                                   
	\end{tabular}%
	}
	\caption{Tabla comparativa de las herramientas estudiadas y la propuesta de solución.}
	\label{tab:tabla-comparativa}
	\end{table}
