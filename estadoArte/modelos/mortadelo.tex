\subsection{Mortadelo}
De acuerdo con de la Vega\cite{de_la_vega_mortadelo_2020}, Mortadelo está basado en el \textit{model driven}; es decir, que para diseñar bases de datos NoSQL necesita de modelos conceptuales bien definidos que conformen un metamodelo.


Como menciona este autor, un modelo conceptual que especifique solo qué entidades conforman el sistema no es suficiente, porque para los sistemas NoSQL, es clave conocer cómo estas entidades serán consultadas y actualizadas en el día a día de la implementación de la base de datos.


Por ello los modelos conceptuales tradicionales han sido complementados con lenguajes específicos para los patrones de acceso a los datos.


Lo anterior implica usar dos modelos diferentes pero interrelacionados que añade complejidad al modelado de la base datos. Por esta razón de la Vega propone el Generic Data Model (GDM), que es un metamodelo para el modelo conceptual NoSQL donde la estructura y patrones de acceso están integradas en un mismo modelo conceptual.


Cabe destacar que el GDM es un modelo conceptual independiente del paradigma NoSQL; es decir, puede representar una base de datos NoSQL de clave-valor, orientado a documentos, orientado a columnas u orientada a grafos.


Tal como está en la figura ~\ref{img:mortadelo-process}, Mortadelo empieza con el GDM bien construido, genera una tranformación ``modelo a modelo" (M2M) al modelo lógico deseado, ya sea de clave-valor, orientado a documentos, a columnas o a grafos, para posteriormente realizar la generación del modelo físico de la base de datos NoSQL.


\begin{figure}[h!t] 
    \centering
    \includegraphics[width=0.75\textwidth]{mortadelo/01.png}
    \caption{Mortadelo}
    \label{img:mortadelo-process}
\end{figure}


El modelo lógico en el primer paso es una representación intermedia que contiene información específica del paradigma que describen.


Finalmente, para el segundo paso de la transformacion el modelo lógico pasa por una transformación ``modelo a texto"  (M2T) que genera el modelo físico del paradigma escogido.

\subsubsection*{Generic Data Metamodel}

Como ya se mencionó y como se puede ver en la figura ~\ref{img:mortadelo-gdm}, el modelo conceptual que propone de la Vega contiene en un mismo modelo los elementos de la estructura y el cómo se realizarán las consultas (\textit{structure model} y \textit{access queries}).


\begin{figure}[h!t] 
    \centering
    \includegraphics[width=0.75\textwidth]{mortadelo/GDM.png}
    \caption{Mortadelo}
    \label{img:mortadelo-gdm}
\end{figure}

El modelo de estructura está definido de manera semejante a la notación UML y la estructura de datos está definida por la especificación de entidades. Estas entidades contienen dos tipos de elementos:

\begin{enumerate}
    \item atributos primitivos: almacenan valores de algún tipo
    \item referencias a otras entidades: las referencias de una entidad pueden tener una cardinalidad variable (1, 2, 4 o ilimitada).
\end{enumerate}
