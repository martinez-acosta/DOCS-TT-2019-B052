\section{HacKolade}
De acuerdo con sitio web de la herramienta\cite{hackolade_hackolade_2020}, Hackolade es un \textit{software} con capacidad de representar objetos JSON anidados; la aplicación combina la representación gráfica de colecciones (término usado en las bases de datos orientados a documentos) y vistas en un diagrama.


La aplicación está basada en la denormalización, el polimorfismo y las matrices anidadas JSON; la representación gráfica de la definición del esquema JSON de cada colección está en una vista de árbol jerárquica. 


Lamentablemente, en el sitio web de HacKolade no hay información sobre el modelo conceptual en el que está basado su aplicación.


Es una herramienta de modelado de datos para MongoDB, Neo4j, Cassandra, Couchbase, Cosmos DB, DynamoDB, Elasticsearch, HBase, Hive, Google BigQuery, Firebase/Firestore, MarkLogic, entre otros.