\section{HacKolade}
De acuerdo al sitio web de Hackolade\cite{hackolade_hackolade_2020}, es un \textit{software} con capacidad de representar objetos JSON anidados; la aplicación combina la representación gráfica de colecciones (término usado en las bases de datos orientados a documentos) y vistas en un diagrama.


La aplicación está basada en la desnormalización, el polimorfismo y matrices anidadas JSON; la representación gráfica de la definición del esquema JSON de cada colección está en una vista de árbol jerárquica. 


Hackolade genera dinámicamente \textit{scripts} MongoDB a medida que construye un modelo de datos, deriva modelos de datos por medio de ingeniería inversa de instancias MongoDB, dando facilidad para obtener descripciones, propiedades y restricciones.


Lamentablemente, en el sitio web de HacKolade no hay información sobre qué modelo conceptual está basado su aplicación, así que se asume que usan una propuesta propia para el modelo conceptual del que hacen uso.


Es una herramienta de modelado de datos para MongoDB, Neo4j, Cassandra, Couchbase, Cosmos DB, DynamoDB, Elasticsearch, HBase, Hive, Google BigQuery, Firebase/Firestore, MarkLogic, entre otros.