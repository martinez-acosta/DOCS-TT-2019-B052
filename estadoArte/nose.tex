\section{NoSE: Schema Design for NoSQL Applications}
De acuerdo con Michael J. Mior\cite{mior_nose_2017}, NoSE usa un modelo conceptual de los datos y una descripción de cómo se accederán a ellos, que es conocido como \textit{workload} y es usado para generar un modelo físico de una base de datos NoSQL orientado a columnas.


Para el esquema generado, NoSE debe tener un modelo conceptual que describa la información que se almacenará, por ello NoSE espera este modelo conceptual en forma de un grafo de entidad.


Los grafos de entidad son un tipo restringido del modelo de entidad-relación. Cada cuadro representa un tipo de entidad, cada borde es una relación entre entidades y la cardinalidad asociada de la relación (uno a muchos, uno a uno o muchos a muchos). Las entidades tienen atributos, uno o más de los cuales sirven como clave. 


Respecto al \textit{workload}, se describe como un conjunto de consultas parametrizadas y declaraciones de ``actualización''. Cada consulta y actualización está asociada con un peso que indica su frecuencia relativa en la carga de trabajo.