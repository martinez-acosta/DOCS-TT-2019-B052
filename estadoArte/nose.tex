\section{NoSE: Schema Design for NoSQL Applications}
De acuerdo con Michael J. Mior\cite{mior_nose_2017}, NoSE usa un modelo conceptual junto con el \textit{workload} para describir cómo se accederán a los datos y así genera un modelo físico de una base de datos NoSQL orientado a columnas.


NoSE debe tener un modelo conceptual que describa la información que se almacenará, por ello NoSE espera este modelo conceptual en forma de un grafo de entidad.


Los grafos de entidad son un tipo restringido del modelo de entidad-relación; cada cuadro representa un tipo de entidad, tienen atributos en los que uno más sirven como clave para identificarla, cada borde es una relación entre entidades y la cardinalidad asociada de la relación (uno a muchos, uno a uno o muchos a muchos). 


Respecto al \textit{workload}, se describe como un conjunto de consultas parametrizadas y declaraciones de ``actualización''; cada consulta y actualización está asociada con un peso que indica su frecuencia relativa en la carga de trabajo.