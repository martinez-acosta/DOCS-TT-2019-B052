\section{Kashliev Data Modeler}
De acuerdo con el sitio web de KDM (Kashliev Data Modeler)\cite{datafluent_kashliev_2020}, esta es una herramienta de modelado desarrollada el profesor asistente del Departamento de informática de la Universidad del Oeste Michigan, Andrii Kashliev, que automatiza el diseño de esquemas para una base de datos NoSQL orientada a columnas en Apache Cassandra. 


Por medio de una aplicación web, KDM ayuda al usuario comenzando con un modelo conceptual ER más consultas asociadas para generar un modelo lógico y físico de datos o \textit{scripts} CQL (Cassandra Query Language). 


KDM automatiza: 

\begin{enumerate}
    \item El mapeo conceptual a lógico.
    \item El mapeo lógico a físico.
    \item La generación de \textit{scripts} CQL.
\end{enumerate}

El modelo conceptual usado en KDM es el propuesto por Chebotko\cite{chebotko_big_2015} y automatiza el proceso de transformación entre los tres niveles de modelos (conceptual, lógico y físico).


Además, cuenta con una versión de prueba de tiempo indefinido con características limitadas que permite la generación de un modelo lógico y guardar los proyectos.
