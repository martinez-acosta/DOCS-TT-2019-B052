\documentclass[12pt, a4paper]{report}
\usepackage[utf8]{inputenc}
\usepackage{csquotes}
\usepackage{graphicx}
\usepackage{longtable}
\usepackage{booktabs}
\usepackage{caption}
\usepackage{subcaption}
\usepackage{amsmath}
\usepackage{amsthm}
\usepackage[linesnumbered,ruled]{algorithm2e}
\usepackage{float}
\usepackage{longtable}
\usepackage{multirow}
\usepackage{tabularx}
\usepackage[table,xcdraw]{xcolor}
\usepackage[a4paper,width=150mm,top=25mm,bottom=25mm,bindingoffset=6mm]{geometry}
\usepackage{fancyhdr}
%\usepackage[spanish=nohyphenation,mexico=nohyphenation]{hyphsubst}
\usepackage[spanish,mexico]{babel}
\usepackage{color}   %May be necessary if you want to color links
\usepackage{hyperref}
\usepackage{lscape}
\usepackage{graphicx}
\usepackage{array}% http://ctan.org/pkg/array
\usepackage[bottom]{footmisc}
\hyphenation{clave-valor}
\tolerance=1
\emergencystretch=\maxdimen
\hyphenpenalty=10000
\hbadness=10000
\graphicspath{ {images/} }
\providecommand{\keywords}[1]{\textbf{\textit{keywords---}} #1}


\hypersetup{
    colorlinks=true, %set true if you want colored links
    linktoc=all,     %set to all if you want both sections and subsections linked
    linkcolor=blue,  %choose some color if you want links to stand out
}
\setcounter{tocdepth}{5}
\setcounter{secnumdepth}{5}
\pagestyle{fancy}
\fancyhf{}
\fancyhead{}
\fancyfoot{}
\fancyfoot[LE,RO]{\thepage}
\setlength{\parindent}{2em}
\setlength{\parskip}{0.5em}
\renewcommand{\headrulewidth}{0.4pt}
\renewcommand{\footrulewidth}{0pt}

\fancypagestyle{plain}{%
\fancyhf{}
\fancyhead{}
\fancyfoot{}
\fancyfoot[LE,RO]{\thepage}
\setlength{\parindent}{2em}
\setlength{\parskip}{0.5em}
\renewcommand{\headrulewidth}{0.4pt}
\renewcommand{\footrulewidth}{0pt}
}

\theoremstyle{definition}
\newtheorem{definition}{Definition}[section]

\theoremstyle{remark}
\newtheorem*{remark}{Remark}

\usepackage[style=ieee,sorting=none]{biblatex}

\addbibresource{references.bib}

\title{Herramienta para transformar un modelo entidad-relación a un modelo no relacional}
\author{Martínez Acosta Eduardo, Aparicio Quiroz Omar}

\begin{document}
%%Portada
\begin{titlepage}
    \begin{center}
        \vspace*{1cm}
        
        \Large
        \textbf{Instituto Politécnico Nacional}\\
        \textbf{Escuela Superior de Cómputo}\\
        
        \vspace{0.5cm}
        \LARGE
        Trabajo Terminal No. 2019-B052\\
        \vspace{1cm}
        % \Huge
        Herramienta para transformar un modelo entidad-relación a un modelo no relacional

        \vspace{1.5cm}
        \begin{center}
            
            Presentan: \\
            Aparicio Quiroz Omar\\
            Martínez Acosta Eduardo\\
            
            \vfill
            
            Directoras:\\
            M. en C. Ocotitla Rojas Nancy\\
            D. en C. Chavarria Baez Lorena
            
            \vspace{0.8cm}
            
        \end{center}

    \end{center}
\end{titlepage}

\tableofcontents
\listoffigures
\listoftables
\begin{abstract}
En el presente trabajo se propone una aplicación web que, de acuerdo con reglas de validación establecidas para el modelo entidad-relación básico, permita el mapeo del modelo entidad-relación básico con consultas al modelo relacional o a un modelo conceptual NoSQL y permita generar sus esquemas de bases de datos, respectivamente.


\keywords{IPN, ESCOM, modelo conceptual, NoSQL, entidad-relación, modelo relacional, validación estructural, GDM, Generic Data Metamodel, orientado a documentos, bases de datos, Scrum, modelo lógico, modelo físico, SQL, MongoDB, Nuxt, Python.}
\end{abstract}



\chapter{Introducción}
En los últimos años, los sistemas \textit{Not only SQL} o NoSQL han surgido como alternativa a los sistemas de bases de datos relacionales y se enfocan, principalmente, en almacenar datos que son probables que se consultarán juntos.


Los sistemas NoSQL admiten diferentes modelos de datos como los de clave-valor, documentos, columnas y grafos; con ello se enfocan en guardar datos de una manera apropiada de acuerdo con los requisitos actuales para la gestión de datos en la web o en la nube; enfatizando la escalabilidad, la tolerancia a fallos y la disponibilidad a costa de la consistencia.



De acuerdo a Google Trends\cite{google_google_2020}, en 2009 aumentó el interés en los sistemas NoSQL y, desde entonces, también empezaron a resaltar algunas problemáticas propias de estos sistemas, porque si bien resuelven algunos dilemas de las bases de datos relacionales, por enfocarse en la ya mencionada escalabilidad, la tolerancia a fallos o la flexibilidad para realizar cambios en el esquema de la base de datos, la heterogeneidad de los sistemas NoSQL ha llevado a una amplia diversificación de las interfaces de almacenamiento de datos, provocando la pérdida de un paradigma de modelado común o de un lenguaje de consultas estándar como SQL.


Respecto al modelado de datos, en el diseño de las bases de datos tradicionales, el modelo entidad-relación\cite{codd_relational_nodate} es el modelo conceptual más usado y tiene procedimientos bien establecidos como resultado de décadas de investigación; sin embargo, para los sistemas NoSQL los enfoques tradicionales de diseño de bases de datos no proporcionan un soporte adecuado para satisfacer el modelado de sus diferentes modelos de datos y para abordar esta problemática se han creado varias metodologías de diseño para sistemas NoSQL en los últimos años, porque los diseñadores de bases de datos deben tener en cuenta no solo qué datos se almacenarán en la base de datos, sino también cómo se accederán a ellos para modelar estos sistemas \cite{li_transforming_2010,chebotko_big_2015,mior_nose_2017}. 


Para tener en cuenta cómo se accederán a los datos se debe conocer cómo se realizarán las consultas de los mismos y de acuerdo al trabajo de Mosquera\cite{martinez-mosquera_modeling_2020}, que es una investigación de 1376 \textit{papers} sobre el modelado de sistemas NoSQL, se muestra que de las metodologías propuestas la mayoría usa el modelo entidad-relación para modelado conceptual, mientras que otros proponen su propio modelo conceptual y en cada propuesta se presenta una manera distinta de representar las consultas de los datos; en resumen, no hay una tendencia en el modelado de datos NoSQL y en la literatura sobre el tema solo existen unas cinco herramientas implementadas de las diversas propuestas para el modelado de las bases de datos no relacionales.



Lo que resta del capítulo está organizado de la siguiente manera: primero se muestra la problemática a resolver, después la propuesta de solución, la justificación, los objetivos del proyecto y por último se menciona el alcance con las limitaciones del mismo.
\section{Descripción del problema}
Como se ha visto en la introducción, son pocas las herramientas propuestas en la literatura para el modelado de sistemas NoSQL en sus tres niveles de abstracción (nivel conceptual, lógico y físico).


Solo en el modelado conceptual, el modelo entidad-relación puede considerarse como una tendencia, sin embargo, el modelo entidad-relación por sí mismo no es suficiente para representar cómo se consultarán los datos ni tampoco con qué frecuencia se accederán a ellos; por eso, para modelar sistemas NoSQL es necesario conocer enfoques de desarrollo como el \textit{query-driven design}, el \textit{domain-driven design}, el \textit{data-driven design} o el \textit{workload-driven design}.


Ahora bien, respecto a los modelos de datos NoSQL, no hay un modelo estándar ni tampoco existe un acuerdo sobre la mejor definición de reglas de transformación entre modelos en los tres niveles de abstracción. Además, en la literatura sobre el tema se pueden encontrar unos 36 estudios que proponen diferentes enfoques para las transformaciones.


Asimismo, por los pocos años que han pasado desde que el interés sobre el tema aumentó y que son usados estos sistemas de bases de datos, aún no se ven reflejados en los programas de estudio o solo se da una introducción del tema a los estudiantes de licenciatura en ingeniería en sistemas o carreras afines. 


En consecuencia, para el estudiante es un problema tener que enfrentarse a diseñar un sistema NoSQL, porque no se le enseña ninguna metodología de desarrollo para sistemas NoSQL, tampoco conoce los distintos modelos de datos NoSQL que hay, mucho menos sus ventajas o desventajas para poder elegir el modelo más apropiado para su aplicación y, como resutlado, no pueden visualizar el diseño de una base de datos no relacional a partir de los conocimientos adquiridos en la asignatura de Bases de Datos.


Además, la escasez de herramientas CASE para el diagramado de esquemas no relacionales o herramientas que apoyen la migración de un modelo entidad-relación o relacional a uno no relacional aumenta la complejidad para que el estudiante haga uso de una base de datos no relacional.

\section{Propuesta de solución}

De acuerdo con la problemática antes mencionada, se ha desarrollado una aplicación web que permite:

\begin{itemize}
    \item La edición de un diagrama entidad-relación básico (no extendido).
    \item La validación estructural del diagrama entidad-relación básico.
    \item Transformar el diagrama entidad-relación básico al modelo relacional.
    \item Obtener la definición del esquema relacional de la base de datos.
    \item Obtener entidades del modelo conceptual de un modelo de datos NoSQL a partir de un diagrama entidad-relación básico.
    \item Obtener el modelo lógico de un modelo de datos NoSQL desde su modelo conceptual por medio de consultas a las entidades del modelo conceptual NoSQL.
    \item Obtener la definición del esquema NoSQL de la base de datos para probarlo en un NoSQL DBMS (\textit{NoSQL Database Management System}).
\end{itemize}
\section{Justificación}

Las pocas herramientas para modelado conceptual NoSQL no ofrecen un modelado desde un modelo conceptual relacional, porque se enfocan en un modelo conceptual específico para cada tipo sistemas NoSQL (clave-valor, orientado a documentos, columnas o grafos). Tampoco ofrecen validaciones para sus diagramas ni dan la posiblidad de obtener el esquema SQL, ya que no están enfocados en sistemas relacionales.


Asimismo, como se mencionó en la introducción, por los pocos años que han pasado desde que el interés y el uso de los sistemas NoSQL aumentó, estos sistemas todavía no están reflejados en los programas de estudio.


En consecuencia, para abordar esta problemática en la que el estudiante diseñe un sistema NoSQL sin haber aprendido ninguna metodología de desarrollo para dichos sistemas y sin que logre visualizar en primera instancia el diseño de una base de datos no relacional a partir de lo que ve en la asignatura de Bases de Datos, se propone una herramienta CASE que apoye al estudiante a diseñar una base de datos NoSQL a partir de los conocimientos adquiridos del estudiante.


Se usará como modelo conceptual el modelo entidad-relación por ser un modelo bien conocido para el estudiante de licenciatura en sistemas, que está probado en el área de las bases de datos y de acuerdo con Mosquera\cite{martinez-mosquera_modeling_2020}, es también el más usado en las investigaciones sobre el modelado conceptual de sistemas NoSQL.


%Respecto al modelo físico que obtenga la herramienta a desarrollar, se probará en el esquema orientado a documentos en una base de datos NoSQL orientada a documentos.


\section{Objetivo}
Desarrollar una herramienta web que permita la edición de un diagrama de bases de datos bajo el modelo entidad-relación y realice la validación del mismo.


El diagrama podrá ser transformado al modelo relacional con la posibilidad de obtener el esquema de la base de datos en sentencias SQL, o bien, obtendrá el esquema de base de datos en un modelo de datos no relacional y tendrá la posibilidad de implementar el modelo no relacional a un SGBDNR (Sistema Gestor de Base de Datos No Relacional).


\section{Alcance y limitaciones}

De acuerdo con Dullea\cite{dullea_analysis_2003}, un diagrama ER es válido solo si es estructural y semánticamente válido; sin embargo, hasta donde se sabe, en los últimos 17 años no hay estudios sobre la validez semántica de un diagrama ER, porque como expresa Dullea en su trabajo, la validez semántica depende del minimundo que se quiere representar en el diagrama y es imposible definir una métrica generalizada; por ello la validez de un diagrama ER será solo estructuralmente.


Se usará la notación oficial en los modelos de datos conceptuales entidad-relación y relacional, es decir, la notación de Chen y la notación de Codd, respectivamente.


Para la generación de \textit{scripts} del esquema de datos relacional se usará el lenguaje de consultas SQL.


Asimismo, los \textit{scripts} generados se probarán con MySQL porque es uno de los DBMS que se usa en la asignatura de Base de Datos y están familiarizados los estudiantes de ESCOM.


Para la transformación del modelo ER al esquema del modelo relacional no se realizará ninguna normalización, porque está fuera del alcance para el equipo programar la lógica necesaria y se necesita un diagrama no normalizado para la transformación al modelo conceptual NoSQL.


Dado que hay varias propuestas en la literatura sobre modelos conceptuales NoSQL, se optará por usar la propuesta de Alfonso de la Vega\cite{de_la_vega_mortadelo_2020}, ya que el modelo que propone, conocido como Generic Data Metamodel, es un metamodelo conceptual que describe un modelo de datos NoSQL independientemente de si es de clave-valor, orientado a documentos, a columnas o grafos.


Los \textit{scripts} NoSQL generados por la aplicación solo serán probados y ejecutados en un único NoSQL DBMS.


Se eligirá el modelo de datos orientado a documentos y se usará MongoDB para probar los \textit{scripts} NoSQL generados por la aplicación, porque de acuerdo a Mosquera\cite{martinez-mosquera_modeling_2020}, las de bases de datos orientadas a documentos son las más estudiadas y MongoDB es el NoSQL DBMS más probado para modelado físico en la literatura del tema.


Respecto al registro de los usuarios, se podrá acceder a la aplicación con un correo válido o uno que no lo sea, teniendo en cuenta de que si no crea su cuenta con un correo válido, no le llegará notificación de correo que le indica que se ha registrado exitosamente junto con su contraseña.


El diagramador ER solo podrá editar elementos del modelo entidad-relación, no se podrá diagramar elementos del modelo entidad-relación extendido ni de ningún otro modelo.


No se podrá editar el diagrama generado para el esquema del modelo relacional; tampoco se podrá editar el diagrama conceptual del modelo NoSQL.


La propuesta de solución se podrá ejecutar en el navegador web Google Chrome versión 81 en adelante.
%El grado de partición en el modelo entidad-relación será máximo de dos entidades.
    
\chapter{Estado del arte}
Para la elaboración de la presente propuesta de Trabajo Terminal, ha sido necesaria una investigación de herramientas similares en el ámbito académico y empresarial.


Lo que resta de este capítulo está organizado de la siguiente manera: se presenta cada oferta con una breve descripción y se finaliza con una tabla comparativa.


\section{Kashliev Data Modeler}
Es una herramienta de modelado desarrollada a partir del 2015 por Andrii Kashliev, profesor asistente del Departamento de informática de la Universidad del Oeste Michigan, que automatiza el diseño de esquemas para Apache Cassandra, una base de datos NoSQL orientada a columnas. 


Por medio de una aplicación web, KDM ayuda al usuario en el modelado de datos, comenzando con un modelo conceptual de datos (de una notación familiar al modelo entidad-relación) y generando un modelo físico de datos o \textit{script} CQL (Cassandra Query Language). 


KDM automatiza: 

\begin{enumerate}
    \item El mapeo conceptual a lógico
    \item El mapeo lógico a físico
    \item La generación de \textit{script} CQL 
\end{enumerate}

El modelo de datos usado en KDM es el propuesto por Chebotko\cite{chebotko_big_2015} y por medio de un algoritmo, KDM automatiza el proceso de pasar del modelo de datos a un modelo de datos físicos. 


Además, KDM cuenta con una versión de prueba de tiempo indefinido con características limitadas que permite la generación de un modelo lógico y guardar los proyectos de KDM.

\section{NoSQL Workbench for Amazon DynamoDB}

De acuerdo al sitio web de Amazon\cite{amazon_nosql_2020}, NoSQL Workbench for Amazon DynamoDB es una herramienta desarrollada por Amazon que proporciona funciones de desarrollo de consultas, modelado y visualización de datos para diseñar, crear, consultar y administrar bases de datos NoSQL orientadas a columnas.


NoSQL Workbench for Amazon DynamoDB usa el modelo conceptual propuesto por Chebotko\cite{chebotko_big_2015}.


El visualizador de modelo de datos proporciona un lienzo donde se asignan consultas y visualizan las facetas (parte de la base de datos) de la aplicación sin tener que escribir código.


Cada faceta corresponde a un patrón de acceso diferente en DynamoDB, donde cada patrón se agrega manualmente; cuenta con un generador de operaciones para ver, explorar y consultar conjuntos de datos.


Por último, admite la proyección, la declaración de expresiones condicionales y permite generar código de muestra en varios idiomas.


\section{HacKolade}
De acuerdo al sitio web de Hackolade\cite{hackolade_hackolade_2020}, es un \textit{software} con capacidad de representar objetos JSON anidados; la aplicación combina la representación gráfica de colecciones (término usado en las bases de datos orientados a documentos) y vistas en un diagrama.


La aplicación está basada en la desnormalización, el polimorfismo y matrices anidadas JSON; la representación gráfica de la definición del esquema JSON de cada colección está en una vista de árbol jerárquica. 


Hackolade genera dinámicamente \textit{scripts} MongoDB a medida que construye un modelo de datos, deriva modelos de datos por medio de ingeniería inversa de instancias MongoDB, dando facilidad para obtener descripciones, propiedades y restricciones.


Lamentablemente, en el sitio web de HacKolade no hay información sobre qué modelo conceptual está basado su aplicación, así que se asume que usan una propuesta propia para el modelo conceptual del que hacen uso.


Es una herramienta de modelado de datos para MongoDB, Neo4j, Cassandra, Couchbase, Cosmos DB, DynamoDB, Elasticsearch, HBase, Hive, Google BigQuery, Firebase/Firestore, MarkLogic, entre otros.
\section{Mortadelo}
De acuerdo con de la Vega\cite{de_la_vega_mortadelo_2020}, Mortadelo está basado en el \textit{model driven}; es decir que para diseñar bases de datos NoSQL necesita de modelos conceptuales definidos.


Lo anterior implica usar dos modelos diferentes pero interrelacionados que añaden complejidad al modelado de la base datos; por esta razón de la Vega propone el GDM (\textit{Generic Data Metamodel}), que es un modelo conceptual NoSQL donde la estructura (entidades, atributos, relaciones entre entidades) y patrones de acceso (cómo se consultarán los datos) están integradas en un mismo modelo conceptual.


Cabe destacar que el GDM es un modelo conceptual independiente del paradigma NoSQL, por lo que puede representar una base de datos NoSQL de clave-valor, orientado a documentos, orientado a columnas u orientado a grafos.

La figura~\ref{img:mortadelo-process} muestra los tres pasos principales de Mortadelo, empezando desde la izquierda con su propuesta de modelo conceptual NoSQL, el \textit{Generic Data Metamodel}, siguiendo con una representación de los modelos lógicos orientado a columnas o documentos en los que realiza ``modelo a modelo'' (M2M) para generar el modelo lógico, para finalizar una transformación ``modelo a texto'' (M2T) con el modelo físico de cada modelo lógico, respectivamente.


\begin{figure}[H] 
    \centering
    \includegraphics[width=0.75\textwidth]{mortadelo/01.png}
    \caption{Mortadelo}
    \label{img:mortadelo-process}
\end{figure}
\subsubsection*{Mortadelo: modelo conceptual (Generic Data Metamodel)}

La figura~\ref{img:mortadelo-gdm} muestra el \textit{Generic Data Metamodel}, que está compuesto por clases interrelacionadas entre sí con notación UML\footnote{La simbología usada en los diagramas de clase UML está en el apéndice.} y consta de dos secciones principales: los elementos de la estructura (\textit{structure model elements}) y el cómo se realizarán las consultas (\textit{access queries elements}).


%Como se aprecia en la figura~\ref{img:mortadelo-gdm}, el modelo conceptual que propone de la Vega contiene en un mismo modelo con notación UML los elementos de la estructura (\textit{structure model elements}) y el cómo se realizarán las consultas (\textit{access queries elements}).


\begin{figure}[H] 
    \centering
    \includegraphics[width=0.75\textwidth]{mortadelo/GDM.png}
    \caption{Generic Data Metamodel}
    \label{img:mortadelo-gdm}
\end{figure}


A continuación se da una explicación de cada elemento de la figura~\ref{img:mortadelo-gdm} donde la clase \textit{model} es para indicar que un modelo GDM tiene $n$ entidades y $n$ consultas donde $n=0,1,...,n$.


\paragraph*{Structure model elements}


\begin{itemize}    
    
    \item Clase \textit{entity}: contiene \textit{features} y solo es referenciada directamente desde las clases \textit{from} y \textit{reference}.
    \item Clase \textit{feature}: es una clase abstracta, una \textit{reference}, un \textit{attribute}, o hereda de la clase \textit{annotatable element} para que la instancia de la clase sea comentada con indicadores de texto que proveen información extra para generar el modelo lógico. 
    \item Clase \textit{reference}: empieza con la palabra clave ``ref'', un nombre de tipo de entidad, una cardinalidad y un nombre de referencia; por ejemplo: ``ref Category[*] categories'' define una referencia llamada \textit{categories}, del tipo de entidad \textit{category} con una cardinalidad de cero a varios.
    \item Clase \textit{attribute}: contiene un tipo y un identificador de unicidad.
    \item Clase \textit{annotatable element}: es una clase abstracta para permitir que una clase contenga anotaciones.
    \item Clase \textit{annotation}: es un indicador de texto que provee información extra para generar el modelo lógico.
    
\end{itemize}

\paragraph*{Access queries elements}


\begin{itemize}
    
    \item Clase \textit{query}: tiene solo un elemento de la clase \textit{from}, $n$ elementos de la clase \textit{attribute selection}, $n$ elementos de la clase \textit{inclusion} y tiene o no un único elemento de la clase \textit{boolean expression}.
    \item Clase \textit{from}: asocia una clase \textit{query} con la clase \textit{entity}; es la clase que permite referenciar un tipo de entidad.
    \item Clase \textit{attribute selection}: accede a los atributos de la \textit{entity} referenciada por la clase \textit{from} o la clase \textit{inclusion}.
    \item Clase \textit{boolean expression}: expresa una expresión booleana para declarar alguna restricción.
    \item Clase \textit{inclusion}: permite acceder en una \textit{query} a los atributos de otros tipos de entidad.
    
\end{itemize}

La figura~\ref{img:mortadelo-gdm.textual.notation} es una instancia del modelo conceptual GDM en su notación textual en la que se nota que, por ejemplo, en la tercera consulta se accede a los elementos de la entidad \textit{category} a través del elemento ref de la entidad \textit{product}.


\begin{figure}[H] 
    \centering
    \includegraphics[width=0.65\textwidth]{mortadelo/GDM-textual-notation.png}
    \caption{Notación textual del GDM}
    \label{img:mortadelo-gdm.textual.notation}
\end{figure}


Para más detalles sobre su modelo lógico orientado a documentos y algoritmos asociados, visite la sección~\ref{alg:gdm-to-logic}.
%Como se ha mostrado, Mortadelo es quizá la herramienta más completa de las estudiadas y Alfonso de la Vega expone en su \textit{paper} no solo el modelado conceptual, lógico y físico de una base de datos NoSQL, sino también explica a detalle los algoritmos usados con casos de uso.
\section{NoSE: Schema Design for NoSQL Applications}
De acuerdo con Michael J. Mior\cite{mior_nose_2017}, NoSE usa un modelo conceptual junto con el \textit{workload} para describir cómo se accederán a los datos y así genera un modelo físico de una base de datos NoSQL orientado a columnas.


NoSE debe tener un modelo conceptual que describa la información que se almacenará, por ello NoSE espera este modelo conceptual en forma de un grafo de entidad.


Los grafos de entidad son un tipo restringido del modelo de entidad-relación; cada cuadro representa un tipo de entidad, tienen atributos en los que uno más sirven como clave para identificarla, cada borde es una relación entre entidades y la cardinalidad asociada de la relación (uno a muchos, uno a uno o muchos a muchos). 


Respecto al \textit{workload}, se describe como un conjunto de consultas parametrizadas y declaraciones de ``actualización''; cada consulta y actualización está asociada con un peso que indica su frecuencia relativa en la carga de trabajo.
%\subsection{Hackolade}
Hackolade aplica parte de la teoría de la relación de entidad a bases de datos no  relacionales para representar datos desnormalizados de una manera fácil de usar.


Diseño de esquema para:

\begin{itemize}
	\item Documento: MongoDB, Couchbase, Elasticsearch, Google Firebase y Firestore.
	\item Grafos: Neo4j, TinkerPop.
	\item Orientado a columnas: HBase, Cassandra y Datastax.
	\item Clave-Valor: DynamoDB.
	\item Multimodelo: Cosmos DB, MarkLogic.
	\item Análisis de Big Data: Hive, HBase, AWS Glue Data Catalog.
	\item Formatos de datos: esquema JSON, Avro.
	\item API REST: Swagger 2, OpenAPI 3, registro de esquema AWS EventBridge.
	
\end{itemize}
Ventajas:
\begin{itemize}
	\item Interfaz de usuario amigable.
	\item Navegación visual del modelo.
	\item Edición de vista de árbol de colección gráfica.
	\item Diseñado específicamente para JSON y el anidamiento de subobjetos.
	\item Ingeniería inversa y directa de modelos de datos.
	\item Generar esquemas Mongoose, validador MongoDB, scripts DynamoDB o esquemas otomanos Couchbase.
	\item Producir documentación legible para humanos en formato HTML o PDF.
	\item Disponible para Windows, Mac, Linux.
\end{itemize}
\subsection{KDM}
Es una herramienta para automatizar el diseño de esquemas para Cassandra. Simplifica el proceso de modelado de datos, empezando desde el modelo de datos conceptual en una notación familiar de Chen y termina con un  script CQL(Cassandra Query Language).

\subsection{NoSQL Workbench for Amazon DynamoDB}


NoSQL Workbench para Amazon DynamoDB es una aplicación multiplataforma para el desarrollo y operaciones de bases de datos modernas y está disponible para Windows y macOS.


También es una herramienta visual unificada que proporciona funciones de modelado de datos, visualización de datos y desarrollo de consultas para ayudar a diseñar, crear, consultar y administrar tablas de DynamoDB.


Características: 

\begin{enumerate}
	\item Modelado de datos
	Puede crear nuevos modelos de datos o diseñar modelos basados en modelos de datos existentes que satisfagan los patrones de acceso a datos de su aplicación. 
	
	
	También puede importar y exportar el modelo de datos diseñado al final del proceso.
	\item Visualización de datos
	El visualizador de modelo de datos proporciona un lienzo donde puede asignar consultas y visualizar los patrones de acceso de la aplicación sin tener que escribir código. 
	
	
	Puede agregar datos manualmente a su modelo de datos o importar datos desde MySQL. 
	\item Gran interfaz gráfica
	NoSQL Workbench proporciona una rica interfaz gráfica de usuario para que pueda desarrollar y probar consultas. 
	
	
	Puede usar el generador de operaciones para ver, explorar y consultar conjuntos de datos. También admite la proyección, la expresión de condiciones y le permite generar código de muestra en varios idiomas.
	
\end{enumerate}


\begin{longtable}{  p{3cm}| p{4cm}| p{4cm}| p{4cm} }
		
		\caption{Software disponible.\label{long}}\\
		
		\hline
		\multicolumn{4}{ c }{Oferta de Software}\\
		\hline
		\textbf{Software} & \textbf{KDM} & \textbf{NoSQL Workbench} & \textbf{Hackolade} \\
		\hline
		\endfirsthead
		
		\hline
		\multicolumn{2}{|l|}{Continuación de Tabla \ref{long}}\\
		\hline
		Continuación de tabla\\
		\hline
		\endhead
		
		\hline
		\endfoot
		
		\hline
		\multicolumn{4}{ c }{Fin de Tabla}\\
		\hline%\hline
		\endlastfoot
		
		\textbf{Caracteristica} & Herramienta para automatizar el diseño de esquemas para Cassandra. Simplifica un proceso de modelado de datos de extremo a extremo que comienza con un modelo de datos conceptual visualizado en una notación familiar de Chen y termina con un  script CQL  & Herramienta visual unificada para diseñadores de bases de datos, desarrolladores y           administradores; proporciona modelado de datos relacional y orientado a objetos,          desarrollo de sentencias de SQL (Structured Query Language en inglés [6]) y            herramientas de administración integrales para la configuración del servidor, la          administración de usuarios, la copia de seguridad. & Herramienta que aplica parte de la teoría de la relación de entidad a bases de datos no                relacionales para representar datos desnormalizados de una manera fácil de usar \\ 
		
		\textbf{Inicio de desarrollo} & 2015 & 2019 & 2017 \\ 
		
		textbf{Estatus } & Mejora intermitente & Versión Demo & Finalizado. Desarrolllo constante \\
		
		\textbf{Precio} & Gratuito (pendiente 2020) & Gratuito & Desde 75 euros/Mes \\ 
		
		\textbf{Desarrollador } & Andrii Kashliev. Profesor Asistente del Departamento de Informática de la Universidad de	Eastern Michigan
		 & Oracle Corporation & IntegrIT SA / NV, firma consultora de estrategia de TI y proveedor de software independiente \\
		
		\textbf{Plataforma } & Web & Windows y macOS & Windows, Mac, Linux \\
		
		\textbf{Manejador de BD } & Columna & Clave valor y documentos & Documento, grafos, columna, llave-valor.\\
		
		\textbf{Orientación } & Apache Cassandra & DynamoDB & MongoDB, Couchbase, Elasticsearch, Google Firebase and Firestore, Neo4j, TinkerPop, HBase, Cassandra, Datastax, DynamoDB
		 \\
		
%		\textbf{ } & 2 & 3 & 4 \\
	\end{longtable}

\input{estadoArte/conclusion}
\subsection{Hackolade}
Hackolade aplica parte de la teoría de la relación de entidad a bases de datos no  relacionales para representar datos desnormalizados de una manera fácil de usar.


Diseño de esquema para:

\begin{itemize}
	\item Documento: MongoDB, Couchbase, Elasticsearch, Google Firebase y Firestore.
	\item Grafos: Neo4j, TinkerPop.
	\item Orientado a columnas: HBase, Cassandra y Datastax.
	\item Clave-Valor: DynamoDB.
	\item Multimodelo: Cosmos DB, MarkLogic.
	\item Análisis de Big Data: Hive, HBase, AWS Glue Data Catalog.
	\item Formatos de datos: esquema JSON, Avro.
	\item API REST: Swagger 2, OpenAPI 3, registro de esquema AWS EventBridge.
	
\end{itemize}
Ventajas:
\begin{itemize}
	\item Interfaz de usuario amigable.
	\item Navegación visual del modelo.
	\item Edición de vista de árbol de colección gráfica.
	\item Diseñado específicamente para JSON y el anidamiento de subobjetos.
	\item Ingeniería inversa y directa de modelos de datos.
	\item Generar esquemas Mongoose, validador MongoDB, scripts DynamoDB o esquemas otomanos Couchbase.
	\item Producir documentación legible para humanos en formato HTML o PDF.
	\item Disponible para Windows, Mac, Linux.
\end{itemize}
\subsection{KDM}
Es una herramienta para automatizar el diseño de esquemas para Cassandra. Simplifica el proceso de modelado de datos, empezando desde el modelo de datos conceptual en una notación familiar de Chen y termina con un  script CQL(Cassandra Query Language).

\subsection{NoSQL Workbench for Amazon DynamoDB}


NoSQL Workbench para Amazon DynamoDB es una aplicación multiplataforma para el desarrollo y operaciones de bases de datos modernas y está disponible para Windows y macOS.


También es una herramienta visual unificada que proporciona funciones de modelado de datos, visualización de datos y desarrollo de consultas para ayudar a diseñar, crear, consultar y administrar tablas de DynamoDB.


Características: 

\begin{enumerate}
	\item Modelado de datos
	Puede crear nuevos modelos de datos o diseñar modelos basados en modelos de datos existentes que satisfagan los patrones de acceso a datos de su aplicación. 
	
	
	También puede importar y exportar el modelo de datos diseñado al final del proceso.
	\item Visualización de datos
	El visualizador de modelo de datos proporciona un lienzo donde puede asignar consultas y visualizar los patrones de acceso de la aplicación sin tener que escribir código. 
	
	
	Puede agregar datos manualmente a su modelo de datos o importar datos desde MySQL. 
	\item Gran interfaz gráfica
	NoSQL Workbench proporciona una rica interfaz gráfica de usuario para que pueda desarrollar y probar consultas. 
	
	
	Puede usar el generador de operaciones para ver, explorar y consultar conjuntos de datos. También admite la proyección, la expresión de condiciones y le permite generar código de muestra en varios idiomas.
	
\end{enumerate}


\begin{longtable}{  p{3cm}| p{4cm}| p{4cm}| p{4cm} }
		
		\caption{Software disponible.\label{long}}\\
		
		\hline
		\multicolumn{4}{ c }{Oferta de Software}\\
		\hline
		\textbf{Software} & \textbf{KDM} & \textbf{NoSQL Workbench} & \textbf{Hackolade} \\
		\hline
		\endfirsthead
		
		\hline
		\multicolumn{2}{|l|}{Continuación de Tabla \ref{long}}\\
		\hline
		Continuación de tabla\\
		\hline
		\endhead
		
		\hline
		\endfoot
		
		\hline
		\multicolumn{4}{ c }{Fin de Tabla}\\
		\hline%\hline
		\endlastfoot
		
		\textbf{Caracteristica} & Herramienta para automatizar el diseño de esquemas para Cassandra. Simplifica un proceso de modelado de datos de extremo a extremo que comienza con un modelo de datos conceptual visualizado en una notación familiar de Chen y termina con un  script CQL  & Herramienta visual unificada para diseñadores de bases de datos, desarrolladores y           administradores; proporciona modelado de datos relacional y orientado a objetos,          desarrollo de sentencias de SQL (Structured Query Language en inglés [6]) y            herramientas de administración integrales para la configuración del servidor, la          administración de usuarios, la copia de seguridad. & Herramienta que aplica parte de la teoría de la relación de entidad a bases de datos no                relacionales para representar datos desnormalizados de una manera fácil de usar \\ 
		
		\textbf{Inicio de desarrollo} & 2015 & 2019 & 2017 \\ 
		
		textbf{Estatus } & Mejora intermitente & Versión Demo & Finalizado. Desarrolllo constante \\
		
		\textbf{Precio} & Gratuito (pendiente 2020) & Gratuito & Desde 75 euros/Mes \\ 
		
		\textbf{Desarrollador } & Andrii Kashliev. Profesor Asistente del Departamento de Informática de la Universidad de	Eastern Michigan
		 & Oracle Corporation & IntegrIT SA / NV, firma consultora de estrategia de TI y proveedor de software independiente \\
		
		\textbf{Plataforma } & Web & Windows y macOS & Windows, Mac, Linux \\
		
		\textbf{Manejador de BD } & Columna & Clave valor y documentos & Documento, grafos, columna, llave-valor.\\
		
		\textbf{Orientación } & Apache Cassandra & DynamoDB & MongoDB, Couchbase, Elasticsearch, Google Firebase and Firestore, Neo4j, TinkerPop, HBase, Cassandra, Datastax, DynamoDB
		 \\
		
%		\textbf{ } & 2 & 3 & 4 \\
	\end{longtable}


\chapter{Marco teórico}
El capítulo está organizado de la siguiente manera: primero se muestra cada tema que el lector debe conocer para entender el desarrollo de la propuesta de solución, después se mostrará cada tecnología a usar y finalmente se tendrá un apartado de conclusión en que estará el porqué se ha elegido cada tecnología.



\section{Modelos de datos para bases de datos}
De acuerdo a la bibliografía de Elmasri \cite{ramez_elmasri_fundamentos_nodate}, un modelo de datos es una colección de conceptos que describen una estructura de una base de datos. 


Los modelos de datos de alto nivel o conceptuales ofrecen conceptos visuales simples que representan un modelo de datos, mientras que los modelos de datos de bajo nivel o físicos ofrecen conceptos que describen los detalles de cómo se implementa el almacenamiento de los datos en el sistema de la base de datos.


Los modelos de datos conceptuales utilizan conceptos como entidades, atributos y relaciones o, en el caso de ser bases de datos no relacionales, no hay un estándar definido para modelar conceptualmente este tipo de bases de datos.


Una entidad representa un objeto o concepto del mundo real, un atributo representa alguna propiedad que describe a una entidad y una relación es una asociación entre entidades.


A continuación se presentan varios de los modelos de datos existentes.

\subsection{Modelo entidad-relación}

De acuerdo a Elmasri \cite{ramez_elmasri_fundamentos_nodate}, el modelo entidad-relación, que fue creado y formalizado por Peter Chen en 1976\cite{chen_entity-relationship_nodate}, se utiliza con frecuencia para el diseño conceptual de las aplicaciones de base de datos. 


En esta sección se describen los conceptos básicos y las restricciones del modelo ER. 

\subsubsection{Entidades}
El objeto básico representado por el modelo ER es una entidad, que es una cosa del mundo real con una existencia independiente.


Una entidad puede ser un objeto con una existencia física (por ejemplo, una persona en particular, un coche, una casa o un empleado) o puede ser un objeto con una existencia conceptual (por ejemplo, una empresa, un trabajo o un curso universitario).

\subsubsection*{Tipo de entidad}

Un tipo de entidad define una colección (o conjunto) de entidades que tienen los mismos atributos.


La colección de todas las entidades de un tipo de entidad en particular de la base de datos se denomina conjunto de entidades; se usa el mismo nombre del tipo de entidad para hacer referencia al conjunto de entidades. 

\subsubsection*{Tipos de entidades débiles}

Los tipos de entidad que no tienen atributos clave propios se denominan tipos de entidad débiles. En contraposición, los tipos de entidad regulares que tienen un atributo clave se denominan tipos de entidad fuertes.


Las entidades que pertenecen a un tipo de entidad débil, se identifican como relacionadas con entidades específicas de otro tipo de entidad ( llamada entidad identificado o propietario) en combinación con uno de sus valores de atributo.


Este tipo de relación que relaciona un tipo de entidad débil con su propietario lo podemos llamar relación identificativa del tipo de entidad débil. 


Un tipo de entidad débil siempre tiene una restricción de participación total (dependencia de existencia) respecto a su relación identificativa, porque una entidad débil no puede identificarse sin una entidad propietaria. No obstante, no toda dependencia de existencia produce un tipo de entidad débil.


Un tipo de entidad débil normalmente tiene una clave parcial, que es el conjunto de atributos que pueden identificar sin lugar a dudas las entidades débiles que están relacionadas con la misma entidad propietaria. 


En los diagramas ER, tanto el tipo de la entidad débil como la relación identificativa, se distinguen rodeando sus cuadros y rombos mediante unas líneas dobles.


El atributo de clave parcial aparece subrayado con una línea discontinua o punteada. Los tipos de entidades débiles se puede representar a veces como atributos complejos (compuestos, multivalor). 


En general, se puede definir cualquier cantidad de niveles de tipos de entidad débil; un tipo de entidad propietaria puede ser ella misma un tipo de entidad débil. 


Además, un tipo de entidad débil puede tener más de un tipo de entidad identificativa y un tipo de relación identificativa de grado superior a dos.

\subsubsection{Atributos}
Cada entidad tiene atributos, que son propiedades particulares que la describen y en el modelo ER hay varios tipos: simple frente a compuesto, monovalor frente a multivalor, almacenado frente a derivado y nulo.


\paragraph*{Atributos compuestos frente a atributos simples} Los atributos compuestos se pueden dividir en subpartes más pequeñas que representan atributos más básicos con significados independientes.


Los atributos que no son divisibles se denominan atributos simples o atómicos, mientras que los atributos compuestos pueden forma una jerarquía. 


El valor de un atributo compuesto es la concatenación de los valores de sus atributos simples.

\paragraph*{Atributos monovalor y multivalor}  La mayoría de los atributos tienen un solo valor para una entidad en particular; dichos atributos reciben el nombre de monovalor o de un solo valor. 


En algunos casos, un atributo puede tener un conjunto de valores para la misma entidad y se denominan multivalor.


Un atributo multivalor puede tener límites superior e inferior para restringir el número de valores permitidos para cada entidad individual.


\paragraph*{Atributos almacenados y derivados}
El atributo derivado puede calcularse u obtenerse a partir de otro atributo, que se denomina almacenado.


\paragraph*{Atributos complejos}

Los atributos complejos son los atributos compuestos y multivalor que se anidan arbitrariamente.

Podemos representar el anidamiento arbitrario agrupando componentes de un atributo compuesto entre paréntesis () separando los componentes con comas, y mostrando los atributos multivalor entre llaves \{\}. 

Por ejemplo, si una persona puede tener más de una residencia y cada
residencia puede tener una sola dirección y varios teléfonos, el atributo TlfDir de una persona se puede especificarse Como


\{TlfDir(\{Tlf(CodÁrea,NumTlf)\},\\
Dir(DirCalle(Número,Calle,NumApto),\\
Ciudad,Provincia,CP))\}

Los atributos Tlf y Dir son compuestos.
\paragraph*{Atributos clave de un tipo de entidad}
Una restricción importante de las entidades de un tipo de entidad es la clave o restricción de unicidad de los atributos.


Un tipo de entidad normalmente tiene un atributo cuyos valores son distintos para cada entidad del conjunto de entidades.


Los valores de un atributo en que se pueden utilizar para identificar cada entidad inequívocamente se denomina atributo clave.


En la notación diagramática ER cada atributo clave tiene su nombre subrayado dentro del óvalo y algunos tipos de entidad tienen más de un atributo clave. 


Un tipo de entidad que carece de clave se le denomina tipo de entidad débil (que se explicará más adelante).


\paragraph*{Conjuntos de valores (dominios) de atributos} Cada atributo simple de un tipo de entidad está asociado con un conjunto de valor (o dominio de valores) que especifica el conjunto de los valores que se pueden asignar a ese atributo por cada entidad individual. 


Los conjuntos de valores no se muestran en los diagramas ER; normalmente se especifican mediante los tipos de datos básicos disponibles en la mayoría de los lenguajes de programación, como entero, cadena, booleano, flotante, tipo enumerado, subrango, etcétera. 


También se emplean otros tipos de datos adicionales para representar la fecha, la hora y otros.

\paragraph*{Atributos de los tipos de relación}


Los tipos de relación también pueden tener atributos; los atributos de los tipos de relación 1:1 o 1:N se pueden trasladar a uno de los tipos de entidad participantes.


En el caso de un tipo de relación 1:N, un atributo de relación solo se puede migrar al tipo de entidad que se encuentra en el lado N de la relación. 


Para los tipos de relación M:N, algunos atributos pueden determinarse mediante la combinación de entidades participantes en una instancia de relación, no mediante una sola relación. Dichos atributos deben especificarse como atributos de relación.

\subsubsection{Relaciones}

Un tipo de relación R entre n tipos de entidades $E_1, E_2, ..., E_n$ define un conjunto de asociaciones (o un conjunto de relaciones) entre las entidades de esos tipos de entidades. 


Como en el caso de los tipos de entidades y los conjuntos de entidades, normalmente se hace referencia a un tipo de relación y su correspondiente conjunto de relaciones con el mismo nombre $R$.


Matemáticamente, el conjunto de relaciones $R$ es un conjunto de instancias de relación $r_i$, donde cada $r_i$ asocia $n$ entidades individuales ($e_1, e_2,..., e_n$) y cada entidad $e_j$ de $r_i$ es un miembro del tipo de entidad $E_j, 1 \leq j \leq n$. Por tanto, un tipo de relación es una relación matemática en $E_1, E_2,..., E_n$.


De forma alternativa, se puede definir como un subconjunto del producto cartesiano $E_1 \times E_2,\times ... 	\times E_n$. Se dice que cada uno de los tipos de entidad $E_1, E_2,..., E_n$ participa en el tipo de relación $R$; de forma parecida, cada una de las entidades individuales $e_1, e_2,..., e_n$ se dice que participa en la instancia de relación $r_i = (e_1, e_2,..., e_n)$.


En los diagramas ER, los tipos de relaciones se muestran mediante rombos, conectados a su vez mediante líneas a los rectángulos que representan los tipos de entidad participantes y el nombre de la relación se muestra dentro del rombo.


\paragraph*{Grado de relación, nombres de rol y relaciones recursivas}
\paragraph*{Grado de un tipo de relación}
El grado de un tipo de relación es el número de tipos de entidades participantes. Un tipo de relación de grado dos se denomina binario y uno de grado tres ternario. Las relaciones pueden ser generalmente de cualquier grado, pero las más comunes son las relaciones binarias.

\paragraph*{Nombres de rol y relaciones recursivas}
Cada tipo de entidad que participa en un tipo de relación juega un papel o rol particular en la relación.


El nombre de rol hace referencia al papel que una entidad participante del tipo de entidad juega en cada instancia de relación y ayuda a explicar el significado de la relación.


Los nombres de rol no son técnicamente necesarios en los tipos de relación donde todos los tipos de entidad participantes son distintos, puesto que cada nombre de tipo de entidad participante se puede utilizar como participación.

Cuando un tipo de entidad se relaciona consigo misma, se tiene una relación recursiva y es necesario indicar los roles que juegan los miembros en la relación.


\paragraph*{Restricciones en los tipos de relaciones}


Los tipos de relaciones normalmente tienen ciertas restricciones que limitan las posibles combinaciones entre las entidades que pueden participar en el conjunto de relaciones correspondiente.


Estas restricciones están determinadas por la situación del minimundo representado por las relaciones. 


Podemos distinguir dos tipos principales de restricciones de relación: razón de cardinalidad y participación.


\paragraph*{Razones de cardinalidad para las relaciones binarias}
La razón de cardinalidad de una relación binaria especifica el número máximo de instancias de relación en las que una entidad puede participar.


Las posibles razones de cardinalidad para los tipos de relación binaria son 1:1, 1:N, N:1 y M:N.
\begin{enumerate}
    \item uno a uno: una relación R de X a Y es uno a uno si cada entidad en X se asocia con cuando mucho una entidad en Y e, inversamente, cada entidad en Y se asocia con cuando mucho una entidad en X.
    \item uno a muchos: una relación R de X a Y es uno a muchos si cada entidad en X se puede asociar con muchas entidades en Y, pero cada entidad en Y se asocia con cuando mucho una entidad en X. 
    \item muchos a uno: una relación R de X a Y es muchos a uno si cada entidad en X se asocia con cuando mucho una entidad en Y, pero cada entidad en Y se puede asociar con muchas entidades en X. 
    \item muchos a muchos: una relación R de X a Y es muchos a muchos si cada entidad en X se puede asociar con muchas entidades en Y y cada entidad en Y se puede asociar con muchas entidades en X. 
\end{enumerate}

\paragraph*{Restricciones de participación y dependencias de existencia}
La restricción de participación especifica si la existencia de una entidad depende de si está relacionada con otra entidad a través de un tipo de relación.


Esta restricción especifica el número mínimo de instancias de relación en las que puede participar cada entidad y en ocasiones recibe el nombre de restricción de cardinalidad mínima.


Hay dos tipos de restricciones de participación, total y parcial; la participación total también se conoce como dependencia de existencia.

\begin{enumerate}
    \item Participación total: si todo miembro de un conjunto de entidades debe participar en una relación, es una participación total del conjunto de entidades en la relación. Esto se denota al dibujar una línea doble desde el rectángulo de entidades hasta el rombo de relación.
    \item Participación parcial: una línea sencilla indica que algunos miembros del conjunto de entidades no deben participar en la relación.
    \end{enumerate}

Nos referiremos a la razón de cardinalidad y a las restricciones de participación, en conjunto, como restricciones estructurales de un tipo de relación.





\subsubsection{Resumen de la notación para los diagramas ER}

\begin{longtable}[l]{ c p{7cm} }

    \caption{Notación para los diagramas ER\label{long}}\\
    
    \hline
    \multicolumn{2}{ c }{Notación para los diagramas ER}\\
    \hline
    \multicolumn{1}{c}{Símbolo} & \multicolumn{1}{c}{Significado}\\
    \hline
    \endfirsthead
    
    \hline
    \multicolumn{2}{|l|}{Continuación de Tabla \ref{long}}\\
    \hline
    Continuación la notación para los diagramas ER\\
    \hline
    \endhead
    
    \hline
    \endfoot
    
    \hline
    \multicolumn{2}{ c }{Fin de Tabla}\\
    \hline%\ 
    \endlastfoot
    
    \parbox[c]{7em}{\includegraphics[width=\linewidth]{modeloEntidadRelacion/entidad.png}} & \multicolumn{1}{c}{Entidad}\\
    \parbox[c]{7em}{\includegraphics[width=\linewidth]{modeloEntidadRelacion/entidadDebil.png}} & \multicolumn{1}{c}{Entidad débil}\\
    \multicolumn{1}{c}{\parbox[c]{7em}{\includegraphics[width=0.5\linewidth]{modeloEntidadRelacion/relacion.png}}} & \multicolumn{1}{c}{Relación}\\
    \parbox[c]{7em}{\includegraphics[width=0.5\linewidth]{modeloEntidadRelacion/relacionDeIdentificacion.png}} & \multicolumn{1}{c}{Relación de identificación}\\
    \parbox[c]{7em}{\includegraphics[width=\linewidth]{modeloEntidadRelacion/atributo.png}} & \multicolumn{1}{c}{Atributo}\\
    \parbox[c]{7em}{\includegraphics[width=\linewidth]{modeloEntidadRelacion/atributoClave.png}} & \multicolumn{1}{c}{Atributo clave}\\
    \parbox[c]{7em}{\includegraphics[width=\linewidth]{modeloEntidadRelacion/atributoMultivalor.png}} & \multicolumn{1}{c}{Atributo multivalor}\\
    \parbox[c]{7em}{\includegraphics[width=\linewidth]{modeloEntidadRelacion/atributoCompuesto.png}} & \multicolumn{1}{c}{Atributo compuesto}\\
    \parbox[c]{7em}{\includegraphics[width=\linewidth]{modeloEntidadRelacion/atributoDerivado.png}} & \multicolumn{1}{c}{Atributo derivado}\\
    \parbox[c]{7em}{\includegraphics[width=\linewidth]{modeloEntidadRelacion/participacionTotal.png}} & \multicolumn{1}{c}{Participación total}\\
    \parbox[c]{7em}{\includegraphics[width=\linewidth]{modeloEntidadRelacion/razonCardinalidad.png}} & \multicolumn{1}{c}{Razón de cardinalidad}\\
    \parbox[c]{7em}{\includegraphics[width=\linewidth]{modeloEntidadRelacion/restriccionEstructural.png}} & \multicolumn{1}{c}{Restricción estructural}\\
\end{longtable}  
\subsection{El modelo entidad-relación extendido}

De acuerdo a la bibliografía de Catherine\cite{catherine_m_ricardo_bases_nodate}, el modelo entidad-relación extendido (EE-R) extiende el modelo ER para permitir la inclusión de varios tipos de abstracción, y para expresar restricciones más claramente. A los diagramas ER estándar se agregan símbolos adicionales para crear diagramas EE-R que expresen estos conceptos.

\subsubsection{Especialización}
Con frecuencia, un conjunto de entidades contiene uno o más subconjuntos que tienen atributos especiales o que participan en relaciones que otros miembros del mismo conjunto  de entidades no tiene.


El método de identificar subconjuntos de conjuntos de entidades existentes, llamado especialización, corresponde a la noción de herencia de subclase y clase en el diseño orientado a objetos, donde se representa mediante jerarquías de clase.


El circulo que conecta a la superclase con las subclases se llama se llama círculo de especialización.  Cada subclase se conecta al círculo mediante una línea que tiene un símbolo de herencia, un símbolo de subconjunto o copa, con el lado abierto de frente a la superclase. Las subclases heredan los atributos de la superclase y opcionalmente pueden tener atributos locales distintos.


Dado que cada miembro de una subclase es miembro de la superclase, al círculo de especialización a veces se le conoce como relación isa.


En ocasiones una entidad tiene solo un subconjunto con propiedades o relaciones especiales de las que quiere tener información. Solo contiene una subclase para una especialización. En este caso, en el diagrama EE-R se omite el círculo y simplemente se muestra la subclase conectada mediante una línea de subconjunto a la superclase.


Las subclases también pueden participar en relaciones locales que no se apliquen a la superclase o a otras subclases en la misma jerarquía.

\subsubsection{Generalización}
Además de la especialización, también se pueden crear jerarquías de clase al reconocer que dos o más clases tienen propiedades comunes e identificar una superclase común para ellas, un proceso llamado generalización. Estos dos procesos son inversos uno de otro, pero ambos resultan en el mismo tipo de diagrama jerárquico.


\subsubsection{Restricciones}

Las subclases pueden ser traslapantes (\textit{overlapping}), lo que significa que la misma instancia de entidad puede pertenecer a más de una de las subclases, o desarticuladas (\textit{disjoint}), lo que significa que no tienen miembros en común. A esto se le refiere como restricción de desarticulación (\textit{disjointness}) y se expresa al colocar una letra adecuada, $d$ u $o$, en el círculo de especialización. Una $d$ indica subclases de desarticulación y una $o$ indica subclases de traslapamiento. 


Una especialización también tiene una restricción de completud (\textit{completeness}), que muestra si todo miembro del conjunto de entidades debe participar en ella.


Si todo miembro de la superclase debe pertenecer a alguna subclase, se tiene una especialización total. Si a algunos miembros de la superclase no se les puede permitir pertenecer a alguna subclase, la especialización es parcial.


En algunas jerarquías de especialización es posible identificar la subclase a la que pertenece una entidad al examinar una condición o predicado específico para cada subclase, es decir, es una especialización definida por predicado, pues la membresía a la subclase está determinada por un predicado.


Algunas especializaciones definidas por predicado usan el valor del mismo atributo en el predicado definitorio para todas las subclases. Estas se llaman especializaciones definidas por atributo. 


Las especializaciones que no están definidas por predicado se dice que son definidas por el usuario, pues el usuario es el responsable de colocar la instancia de entidad en la subclase correcta.


\subsubsection{Jerarquías múltiples y herencia}

Cuando el mismo conjunto de entidades puede ser una subclase de dos o más superclases, se dice que tal clase es una subclase compartida y tiene herencia múltiple de sus superclases.


\subsubsection{Unión}

Mientras que una subclase compartida representa un miembro de todas sus superclases y hereda atributos de todas ellas, una subclase se puede relacionar con la de una colección, llamada unión o categoría de superclases, en vez de pertenecer a todas ellas. En este caso, una instancia de la subclase hereda solo los atributos de una de las superclases, dependiendo de a cuál miembro de la unión pertenece.


Las categorías pueden ser parciales o totales, dependiendo de si cada miembro de los conjuntos que constituyen la unión participan en ella. 


 
\subsection{Modelo relacional}

De acuerdo a la bibliografia de Elmasri\cite{ramez_elmasri_fundamentos_nodate}, el modelo relacional introducido por Ted Codd en 1970\cite{codd_relational_nodate} utiliza el concepto de una relación matemática como bloque de construcción básico y tiene su base teórica en la teoría de conjuntos y la lógica del predicado.

La lógica de predicado, utilizada ampliamente en matemáticas, proporciona un marco en el que una afirmación (declaración de hecho) se verifica como verdadera o falsa.


La teoría de conjuntos es una ciencia matemática que trata con conjuntos o grupos de cosas y se utiliza como base para la manipulación de datos en el modelo relacional.


El modelo relacional representa la base de datos como una colección de relaciones. Cuando una relación está pensada como una tabla de valores, cada fila representa una colección de valores relacionados.


Asimismo, cada fila de la tabla representa un hecho que, por lo general, corresponde con una relación o entidad real. El nombre de la tabla y de las columnas se utiliza para ayudar a interpretar el significado de cada uno de los valores de las filas.


En terminología formal, una fila recibe el nombre de tupla, una cabecera de columna es un atributo y el nombre de la tabla una relación. El tipo de dato que describe los valores en cada columna está representado por un dominio de posibles valores. 


Basado en estos conceptos, el modelo relacional tiene tres componentes bien definidos:
\begin{enumerate}
    \item Una estructura de datos lógica representada por relaciones.
    \item Un conjunto de reglas de integridad para garantizar que los datos sean consistentes.
    \item Un conjunto de operaciones que define cómo se manipulan los datos.
\end{enumerate}
\subsubsection{Relaciones}
\paragraph*{Estructuras de datos relacionales}
Su usan tablas con relación entre ellas.
\paragraph*{Tablas}
En este modelo, las tablas se usan para contener información acerca de los objetos a representar en la base de datos. Al usar los términos del modelo entidad-relación, los conjuntos de entidades y de relaciones se muestran usando tablas.


Una relación o esquema de relación se representa como una tabla bidimensional en la que las filas de la tabla corresponden a registros individuales y las columnas corresponden a atributos.

Formalmente, un esquema de relación $R$, denotado por $R(A_1, A_2,..., A_n)$ está constituido por un nombre de relación $R$ y una lista de atributos $A_1, A_2,..., A_n$. 


Cada atributo $A_i$ es el nombre de un papel jugado por algún dominio $D$ en el esquema de relación $R$. Se dice que $D$ es el dominio de $A_i$ y se especifica como $dom(A_i)$. 


Un esquema de relación se utiliza para describir una relación; se dice que $R$ es el nombre de la misma. El grado de una relación es el número de atributos $n$ de la misma.


La figura~\ref{img:modeloRelacional-Tabla} muestra de manera visual el modelo relacional; una tabla en el modelo relacional está compuesto de atributos, tiene un nombre de relación y tiene $n$ tuplas.


\begin{figure}[H]
    \centering
    \includegraphics[width=\textwidth]{modeloRelacional/tabla.png}
    \caption{Tabla en el modelo relacional}
    \label{img:modeloRelacional-Tabla}
\end{figure} 
Cada fila de la tabla corresponde a un registro individual o instancia de entidad. En el modelo relacional cada fila se llama tupla y la tabla que representa una relación tiene las siguientes características:
\begin{itemize}
    \item Cada celda de la tabla contiene solo un valor.
    \item Cada columna tiene un nombre distinto, que es el nombre del atributo que representa.
    \item Todos los valores en una columna provienen del mismo dominio, pues todos son valores del atributo correspondiente.
    \item Cada tupla o fila es distinta; no hay tuplas duplicadas.
    \item El orden de las tuplas o filas es irrelevante.
\end{itemize}
\paragraph*{Relaciones y tablas de bases de datos}   
Una relación (o estado de relación) $r$ del esquema $R(A_1, A_2,..., A_n)$, también especificado como $r(R)$, es un conjunto de n-tuplas $r={t_1, t_2,..., t_m}$.


Cada tupla $t$ es una lista ordenada de $n$ valores $t=<v_1, v_2,...,v_n>$, donde $v_i$, $1 \leq i \leq n$, es un elemento de $dom(A_i)$ o un valor especial NULL.


El i-ésimo valor de la tupla $t$, que se corresponde con el atributo $A_i$ , se referencia como $t[A_i]$ o $t[i]$ si utilizamos una notación posicional.

\subsubsection{Claves}

En el modelo relacional, las claves son importantes porque aseguran que cada fila en una tabla sea unívocamente identificable. 

También son usadas para establecer relaciones entre tablas y asegurar la integridad de los datos.

Una clave es un atributo o grupo de atributos que identifican los valores de otros atributos. 

\paragraph*{Clave compuesta}
Una clave compuesta es una clave que se compone de más de un atributo. Un atributo que forma parte de una clave se denomina atributo clave.

\paragraph*{Superclave}
Un atributo o atributos que identifican de manera única cualquier fila de una tabla

\subsubsection{Restricciones de integridad}
\paragraph*{Integridad de dominio}
La integridad de dominio es la validez de las restricciones que debe cumplir una determinada columna de la tabla.
\paragraph*{Integridad de entidad}
Todas las claves principales son únicas, y ninguna clave primaria debe ser nula.
\paragraph*{Integridad referencial}
Una clave externa es ser nula siempre que no sea parte de la clave principal de su tabla, o tiene el valor que coincida con el valor de la clave primaria en una tabla con la que está relacionada (cada valor de clave externa no nula debe hacer referencia a un valor de clave primaria existente).

\subsubsection{Propiedades de las relaciones}
\paragraph*{Grado}
El número de columnas en una tabla se llama grado de la relación. Una relación con una sola columna es de grado uno y se llama relación unaria. Una relación con dos columnas se llama binaria, una con tres columnas se llama ternaria y, después de ella, por lo general se usa el término n-aria. El grado de una relación es parte de la intensión de la relación y nunca cambia.


\paragraph*{Cardinalidad}
La cardinalidad de una relación es el número de entidades a las que otra entidad mapea dicha relación.

\subsubsection{ACID}
El modelo relacional en las transacciones cumple con las propiedades de ACID, que es el acrónimo de \textit{Atomicity} (atomicidad), \textit{Consistency} (consistencia), \textit{Isolation} (aislamiento) y \textit{Durability} (durabilidad). 

\paragraph*{Atomicidad}
Requiere que se completen todas las operaciones (solicitudes SQL) de una transacción;
si no, la transacción se cancela. 


Si una transacción $T_{1}$ tiene cuatro solicitudes SQL, las cuatro solicitudes deben completarse con éxito; de lo contrario, se anula toda la transacción.


En otras palabras, una transacción se trata como una unidad de trabajo única, indivisible y lógica.
\paragraph*{Consistencia}
Indica la permanencia del estado consistente de la base de datos. Una transacción lleva una base de datos de un estado consistente a otro. 


Cuando se completa una transacción, la base de datos debe estar en un estado coherente. Si alguna de las partes de la transacción viola una restricción de integridad, se anula la transacción completa.
\paragraph*{Aislamiento}
Significa que los datos utilizados durante la ejecución de una transacción no es utilizada por una segunda transacción hasta que se complete la primera. 


En otras palabras, si la transacción $T_{1}$ se está ejecutando y está utilizando el elemento de datos $X$, ninguna otra transacción accede a ese elemento de datos ($T_{2}...T_{n}$) hasta que finalice $T_{1}$.


Esta propiedad es particularmente útil en entornos de bases de datos multiusuario porque varios usuarios acceden y actualizan la base de datos al mismo tiempo.
\paragraph*{Durabilidad}

Garantiza que una vez que se realizan y confirman los cambios en la transacción, no se deshacen ni pierden, incluso en el caso de una falla del sistema.
\subsubsection{Structured Query Language}
\textit{Structured Query Language} o SQL está basado en el álgebra relacional, en el cálculo relacional y es un lenguaje de manipulación de datos, un lenguaje de definición de datos, un lenguaje de control de transacciones y un lenguaje de control de datos.

\paragraph*{Lenguaje de manipulación de datos (DML)}
Un \textit{Data Manipulation Language} o DML incluye comandos para insertar, actualizar, eliminar y recuperar datos dentro de las tablas de la base de datos. 

\paragraph*{Lenguaje de definición de datos (DDL)}
 Un \textit{Data Definition Language} o DDL incluye comandos para crear objetos de base de datos como tablas, índices y vistas, así como comandos para definir accesos a objetos de la base de datos. 

\paragraph*{Lenguaje de control de transacciones (TCL)}
Los comandos de un \textit{Transaction Control Language} o TCL se ejecutan dentro del contexto de una transacción, que es una unidad lógica de trabajo compuesta por una o más instrucciones SQL. 


SQL proporciona comandos para controlar el procesamiento de estas transacciones atómicas.

\paragraph*{Lenguaje de control de datos (DCL)}
Los comandos de un \textit{Data Control Language} o DCL se utilizan para controlar el acceso a los objetos de datos, como otorgar a un usuario permiso para ver solo una tabla y otorgar a otro usuario permiso para cambiar los datos de la mista tabla.


%\renewcommand*{\arraystretch}{1.4}
\begin{longtable}[l]{ l p{7cm} }

    \caption{Comandos de SQL.\label{long}}\\
    
    \hline
    \multicolumn{2}{ c }{Comandos de manipulación de datos}\\
    \hline
    \multicolumn{1}{c}{Comando} & \multicolumn{1}{c}{Descripción}\\
    \hline
    \endfirsthead
    
    \hline
    \multicolumn{2}{|l|}{Continuación de Tabla \ref{long}}\\
    \hline
    Continuación de comandos SQL\\
    \hline
    \endhead
    
    \hline
    \endfoot
    
    \hline
    \multicolumn{2}{ c }{Fin de Tabla}\\
    \hline%\hline
    \endlastfoot
    
    \textbf{SELECT} & \textbf{Selecciona atributos de filas en una o más tablas o vistas}\\
    \qquad FROM  & Especifica las tablas de las que se deben recuperar los datos\\
    \qquad WHERE  & Restringe la selección de filas en función de una expresión condicional\\
    \qquad GROUP BY & Agrupa las filas seleccionadas en función de uno o más atributos\\
    \qquad HAVING & Restringe la selección de filas agrupadas en función de una condición\\
    \qquad ORDER BY & Ordena las filas seleccionadas en función de uno o más atributos\\
     \textbf{INSERT} & \textbf{Inserta filas en una tabla}\\
     \textbf{UPDATE} & \textbf{Modifica los valores de un atributo en una o más filas de la tabla}\\
    \textbf{DELETE} & \textbf{Elimina una o más filas de una tabla}\\
    \textbf{Operadores de comparación} \\
    \qquad $=,<,>,\leq,\geq,<>,!=$ & Usados en expresiones condicionales\\
    \textbf{Operadores lógicos} \\
    \qquad AND, OR, NOT & Usados en expresiones condicionales\\
    \textbf{Operadores especiales} \\
    \qquad BETWEEN & Comprueba si un valor de atributo está dentro de un rango\\
    \qquad IN & Comprueba si un valor de atributo coincide con algún valor dentro de una lista de valores\\
    \qquad LIKE & Comprueba si un valor de atributo coincide con un patrón de cadena dado\\
    \qquad IS NULL & Comprueba si un valor de atributo es nulo\\
    \qquad EXIST & Comprueba si una subconsulta devuelve alguna fila\\
    \qquad DISTINCT & Limita que los valores sean únicos\\
    \textbf{Comandos de definición de datos}\\
    \textbf{CREATE SCHEMA} & \textbf{Crea el esquema de la base de datos}\\
    \textbf{CREATE TABLE} & Crea una nueva tabla en el esquema de la base de datos del usuario\\
    \qquad NOT NULL & Asegura que una columna no tendrá valores nulos\\
    \qquad UNIQUE & Asegura que una columna no tendrá valores duplicados\\
    \qquad PRIMARY KEY & Define una clave primaria para una tabla\\
    \qquad FOREIGN KEY & Define una clave externa para una tabla\\
    \qquad DEFAULT & Define un valor predeterminado para una columna (cuando no se proporciona ningún valor)\\
    \qquad CHECK & Valida datos en un atributo\\
    \textbf{CREATE INDEX} & \textbf{Crea un índice para una tabla}\\
    \textbf{CREATE VIEW} & \textbf{Crea un subconjunto dinámico de filas y columnas a partir de una o más tablas}\\
    \textbf{ALTER TABLE} & \textbf{Modifica la definición de una tabla (agrega, modifica o elimina atributos o restricciones)}\\
    \textbf{CREATE TABLE AS} & \textbf{Crea una nueva tabla basada en una consulta en el esquema de la base de datos del usuario}\\
    \textbf{DROP TABLE} & \textbf{Elimina permanentemente una tabla (y sus datos)}\\
    \textbf{DROP INDEX} & \textbf{Elimina permanentemente un índice}\\
    \textbf{DROP VIEW} & \textbf{Elimina permanentemente una vista}\\
    \textbf{Lenguaje de control de transacciones}\\
    \qquad COMMIT & Guarda de manera permanente los cambios en los datos\\
    \qquad ROLLBACK & Restaura los datos a sus valores anteriores\\
    \textbf{Lenguaje de control de datos}\\
    \qquad GRANT & Le da un usuario permiso de hacer una acción de sistema o de acceder a datos de un objeto\\
    \qquad REVOKE & Le quita el privilegio a un usario de hacer algunas operaciones\\
\end{longtable} 
%\subsection{Álgebra relacional}
El álgebra relacional es muy importante por varias razones. La primera, porque proporciona un fundamento formal para las operaciones del modelo relacional. La segunda razón, y quizá la más importante, es que se utiliza como base para la implementación y optimización de consultas en los RDBMS. Tercera, porque algunos de sus conceptos se han incorporado al lenguaje estándar de consultas SQL para los RDBMS.


\subsubsection{Operaciones relacionales unarias}
\paragraph{Selección}

SELECCIÓN se emplea para seleccionar un subconjunto de las tuplas de una relación que satisfacen una condición de selección. Se puede considerar esta operación como un filtro que mantiene sólo las tuplas que satisfacen una determinada condición. 


SELECCIÓN puede visualizarse también como una partición horizontal de la relación en dos conjuntos de tuplas: las que satisfacen la condición son seleccionadas y las que no, descartadas. 


En general, SELECCIÓN está designada como:


$\sigma<$condición de selección$>(R)$


donde el símbolo $\sigma$ (sigma) se utiliza para especificar el operador de SELECCIÓN, mientras que la condición de selección es una expresión lógica (o booleana) especificada sobre los atributos de la relación R. 


Observe que R es, generalmente, una expresión de álgebra relacional cuyo resultado es una relación: la más sencilla de estas expresiones es sólo el nombre de una relación de base de datos. El resultado de SELECCIÓN tiene los mismos atributos que R.


SELECCIÓN es unaria, es decir, se aplica a una sola relación. Además, esta operación se aplica a cada tupla individualmente; por consiguiente, las condiciones de selección no pueden implicar a más de una tupla. 


El grado de la relación resultante de una operación SELECCIÓN (su número de atributos) es el mismo que el de R. El número de tuplas en la relación resultante es siempre menor que o igual que el número de tuplas en R.

\paragraph{Proyección}


PROYECCIÓN selecciona ciertas columnas de la tabla y descarta otras. Si sólo estamos interesados en algunos atributos de una relación, usamos la operación PROYECCIÓN para planear la relación sólo sobre esos atributos. 


Por consiguiente, el resultado de esta operación puede visualizarse como una partición vertical de la relación en otras dos: una contiene las columnas (atributos) necesarias y otra las descartadas. 

La forma general de la operación PROYECCIÓN es:


$\Pi<$lista de atributos>(R)


donde $\Pi$ (pi) es el símbolo usado para representar la operación PROYECCIÓN, mientras que <lista de atributos> contiene la lista de campos de la relación R que queremos. 


De nuevo, observe que R es, en general, una expresión de álgebra relacional cuyo resultado es una relación, cuyo caso más simple es obtener sólo el nombre de una relación de base de datos.


El resultado de la operación PROYECCIÓN sólo tiene los atributos especificados en <lista de atributos> en el mismo orden a como aparecen en la lista. Por tanto, su grado es igual al número de atributos contenidos en <lista de atributos>.


Si la lista de atributos sólo incluye atributos no clave de R, es posible que se dupliquen tuplas. La operación PROYECCIÓN elimina cualquier tupla duplicada, por lo que el resultado de la misma es un conjunto de tuplas y, por consiguiente, una relación válida. Esto se conoce como eliminación de duplicados.

\paragraph{Renombrar}

Podemos definir una operación RENOMBRAR como un operador unario. Una operación RENOMBRAR
aplicada a una relación R de grado n aparece denotada de cualquiera de estas tres formas


$\rho_S (B_1,B_2,...,B_n)$ o $\rho_S (R)$ o $\rho_{(B_1,B_2,...,B_n)}(R)$


donde el símbolo $\rho$ (rho) se utiliza para especificar el operador RENOMBRAR, S es el nombre de la nueva relación y $B_{1} , B_{2} ,..., B_{n}$ son los de los nuevos atributos.


La primera expresión renombra tanto la relación como sus atributos, la segunda solo lo hace con la relación y la tercera solo con los atributos. Si los atributos de R son $(A_1, A_2,...,A_n)$ por este orden, entonces cada $A_i$ es renombrado como $B_i$. 
\subsubsection{Operaciones de álgebra relacional de la teoría de conjuntos}

Para combinar los elementos de dos conjuntos se utilizan varias operaciones de la teoría de conjuntos, como la UNIÓN, la INTERSECCIÓN y la DIFERENCIA DE CONJUNTOS (llamada también a veces MENOS, o MINUS). Todas ellas son operaciones binarias, es decir, se aplican a dos conjuntos de tuplas.


Cuando se refieren a las bases de datos relacionales, las relaciones sobre las que se aplican estas tres operaciones deben tener el mismo tipo de tuplas; esta condición recibe el nombre de compatibilidad de unión. 


Dos relaciones $R(A_1, A_2,...,A_n)$ y $S(B_1,B_2,...,B_n)$ se dice que son de unión compatible si tienen el mismo grado n y si el dom($A_i$) = dom($B_i$ ) para $1 \leq i \leq n$. 


Esto significa que ambas relaciones tienen el mismo número de atributos y que cada par correspondiente cuenta con el mismo dominio.


Podemos definir las tres operaciones UNIÓN, INTERSECCIÓN y DIFERENCIA DE CONJUNTO en dos relaciones de unión compatible R y S del siguiente modo:

\begin{itemize}
    \item UNIÓN El resultado de esta operación, especificada como $R \cup S$, es una relación que incluye todas las tuplas que están tanto en R como en S o en ambas, R y S. Las tuplas duplicadas se eliminan.

    \item INTERSECCIÓN El resultado de esta operación, especificada como $R \cap S$, es una relación que incluye todas las tuplas que están en R y en S.
    
    \item  DIFERENCIA DE CONJUNTO (o MENOS ). El resultado de esta operación, especificada como $R - S$, es una relación que incluye todas las tuplas que están en R pero no en S.
\end{itemize}

\paragraph{Producto cartesiano}

Se trata también de una operación de conjuntos binarios, aunque no es necesario que las relaciones en las que se aplica sean una unión compatible. En su forma binaria produce un nuevo elemento combinando cada miembro (tupla) de una relación (conjunto) con los de la otra.


En general, el resultado de $R(A_1, A_2,...,A_n) \times S(B_1, B_2,..., B_m)$ es una relación Q con un grado de n + m atributos $Q(A_1, A_2,..., A_n, B_1, B_2,..., B_m)$, en este orden.


La relación resultante Q tiene una tupla por cada combinación de éstas (una para R y otra para S). Por tanto, si R tiene n R tuplas (indicado como $|R| = n_R$ ), y S cuenta con $n_S$ tuplas, $R \times S$ tendrá $n_R \ast	 n_S$ tuplas.


La operación PRODUCTO CARTESIANO n-ario es una extensión del concepto indicado más arriba que produce nuevas tuplas concatenando todas las posibles combinaciones de tuplas desde n relaciones subyacentes.


Es útil cuando va seguida por una selección que correlacione los valores de los atributos procedentes de las relaciones componentes.

\subsubsection{Operaciones relacionales binarias}
\paragraph{Concatenación}
CONCATENACIÓN, especificada mediante $\bowtie $, se emplea para combinar tuplas relacionadas de dos relaciones en una sola. Esta operación es muy importante para cualquier base de datos relacional que cuente con más de una relación, ya que nos permite procesar relaciones entre relaciones.


La forma general de una CONCATENACIÓN en dos relaciones $R(A_1, A_2,...,A_n)$ y $S(B_1, B_2,..., B_m)$ es:


$R\bowtie<$condición de conexión$> S$

El resultado de la CONCATENACIÓN es una relación Q de n + m atributos $Q(A_1, A_2,..., A_n, B_1, B_2,..., B_m)$ por este orden; Q tiene una tupla por cada combinación de éstas (una para R y otra para S) siempre que dicha combinación satisfaga la condición de conexión.


Ésta es la principal diferencia existente entre el PRODUCTO CARTESIANO y la CONCATENACIÓN. En la CONCATENACIÓN sólo aparecen en el resultado las combinaciones de tuplas que satisfacen la condición de conexión, mientras que en el PRODUCTO CARTESIANO
se incluyen todas las combinaciones de tuplas.


La condición de conexión está especificada sobre los atributos de las dos relaciones R y S y es evaluada para cada combinación de tuplas, incluyéndose en la relación Q resultante en forma de una única tupla combinada sólo aquéllas cuya condición de conexión se evalúe como VERDADERO.

\paragraph{División}

La DIVISIÓN, especificada mediante $\div$, es útil para cierto tipo de consultas que a veces se realizan en aplicaciones de bases de datos. 


Para que una tupla t aparezca en el resultado T de la DIVISIÓN, los valores de aquélla deben aparecer en R en combinación con cada tupla en S.


Observe que en la formulación de la operación DIVISIÓN, las tuplas de la relación denominador restringen la relación numerador seleccionando aquellas tuplas del resultado que sean iguales a todos los valores presentes en el denominador. 

La operación DIVISIÓN está definida por conveniencia para gestionar las consultas que implican una cuantificación universal.
\subsection{Modelos NoSQL}
De acuerdo a la bibliografia de Catherine \cite{cristina_marta_bender_topicos_nodate}, el término NoSQL significa \textit{not only SQL} y se usa para agrupar sistemas de bases de datos diferentes a los relacionales.


Por los nuevos requerimientos en la época actual como disponibilidad total, tolerancia a fallos, almacenamiento de penta bytes de información distribuida en miles de servidores, la necesidad de nodos con escalabilidad horizontal, entre otros, surge la necesidad de sistemas de bases de datos no relacionales.


Estos tipos de sistemas no requieren esquemas fijos, son fáciles y rápidos en la instalación, usan lenguajes no declarativos, ofrecen alto rendimiento y disponibilidad, evitan operaciones de junturas, soportan paralelismo y escalan principalmente en forma horizontal soportando estructuras distribuidas que no necesariamente cumplen las propiedades ACID\cite{cristina_marta_bender_topicos_nodate}, sino que se enfocan en el modelo de consistencia de datos BASE.

\subsubsection{Teorema CAP}
En el  Simposio de Principios de Computación Distribuida (PODC, en inglés) en el año 2000\cite{brewer_towards_2000}, el Dr. Eric Brewer declaró en su presentación que ``en cualquier sistema de datos altamente distribuido hay tres propiedades comúnmente deseables: \textit{Consistency} (consistencia), \textit{Availability} (disponibilidad) y \textit{Partition tolerance} (tolerancia al particionado). Sin embargo, es imposible que un sistema proporcione las tres propiedades al mismo tiempo".


El acrónimo CAP representa las tres propiedades deseables:

\paragraph*{Consistencia}

En una base de datos distribuida, la consistencia tiene el papel más importante. Todos los nodos deben ver los mismos datos al mismo tiempo, lo que significa que las réplicas deben actualizarse inmediatamente. Sin embargo, esto implica lidiar con la latencia y los atrasos de la red.

No hay que confundir la consistencia en la gestión de transacciones con la consistencia del teorema CAP. La consistencia de la gestión de transacciones se refiere al resultado cuando la ejecución de una transacción en una base de datos cumple con todas las restricciones de integridad.


La consistencia en CAP se basa en la suposición de que todas las transacciones tienen lugar al mismo tiempo en todos los nodos, como si se estuvieran ejecutando en una base de datos de un solo nodo (todos los nodos ven los mismos datos al mismo tiempo).

\paragraph*{Disponibilidad}
En términos simples, el sistema siempre cumple una solicitud. Ninguna solicitud recibida se pierde y este es un requisito fundamental para todas las organizaciones centradas en la web.

\paragraph*{Tolerancia al particionado}
El sistema continúa funcionando incluso en caso de falla de un nodo. Esto es equivalente a la transparencia de fallas en bases de datos distribuidas. El sistema fallará solo si fallan todos los nodos.

Aunque el teorema CAP se centra en sistemas basados ​​en la web altamente distribuidos, sus implicaciones están muy extendidas para todos los sistemas distribuidos, incluidas las bases de datos.


En los sistemas de bases de datos, las propiedades ACID aseguran que todas las transacciones exitosas den como resultado un estado de base de datos consistente, uno en el que todas las operaciones de datos siempre devuelven los mismos resultados. 


Para bases de datos distribuidas centralizadas y pequeñas, la latencia no es un problema, pero para una base de datos altamente distribuida el garantizar transacciones ACID sin pagar un alto precio en latencia de red o en conflictos de datos.


La relación entre consistencia y disponibilidad ha generado un nuevo tipo de sistemas de datos distribuidos, diferente al ACID, denominados BASE, \textit{Basically Available} (básicamente disponibles), \textit{Soft state} (estado suave), \textit{Eventually consistent} (eventualmente consistente).

\paragraph*{BASE}

BASE se refiere a un modelo de consistencia de datos en el que los cambios de datos no son inmediatos, sino que se propagan lentamente a través del sistema hasta que todas las réplicas sean consistentes. 


En la práctica, la aparición de bases de datos distribuidas NoSQL proporciona un espectro de consistencia que va desde lo altamente consistente (ACID) hasta lo eventualmente consistente (BASE).

\subsubsection{Clave-Valor}
De acuerdo con Coronel\cite{coronel_database_nodate}, una base de datos de clave-valor es un paradigma de modelo de datos diseñado para almacenar, recuperar y administrar arreglos asociativos.


Comúnmente se usa un diccionario o tabla \textit{hash} que contiene una colección de registros anidados, secuencias de bits que se almacenan y se recuperan utilizando una clave que identifica de manera única el registro y se utiliza para encontrar rápidamente los datos dentro de la base de datos.


No obstante, es responsabilidad de las aplicaciones que hagan uso de este tipo de base de datos interpretar el significado de los datos; no hay claves foráneas y las relaciones no son rastreables entre claves, lo que permite que el DBMS sea rápido y escalable.


La figura~\ref{img:claveValor-bucket} es una representación visual de un \textit{bucket} usado en las bases de datos NoSQL de clave-valor.

\begin{figure}[H]
    \centering
    \includegraphics[width=0.75\textwidth]{noSQL/bucket.png}
    \caption{\textit{Bucket} en el almacenamiento clave-valor}
    \label{img:claveValor-bucket}
\end{figure} 


Los pares de clave-valor generalmente se organizan en \textit{buckets}; todas las claves dentro un \textit{bucket} deben ser únicas, pero está permitido que se repitan en otros \textit{buckets} y todas las operaciones se basan en el \textit{bucket} + la clave.


En este tipo de bases de datos se usan las operaciones de \textit{get}, \textit{store} y \textit{delete}; la operación \textit{get} o \textit{fetch} es usada para obteber el valor de un par; el operador de \textit{store} almacena datos en una clave. 


Si la combinación de \textit{bucket} + clave no existe, se añade como un nuevo par de clave-valor; en cambio, si existe la combinación de \textit{bucket} + clave, el valor es reemplazado por el nuevo; el operador de \textit{delete} es para eliminar un par de clave-valor.


De acuerdo con Sadalge\cite{sadalage_nosql_nodate}, algunas de las bases de datos de clave-valor populares son Riak, Redis, Memcached DB, Berkeley DB, HamsterDB, Amazon DynamoDB y Project Voldemort (una implementación de código abierto de Amazon DynamoDB).



\subsubsection{Orientado a documentos}
Una base de datos orientada a documentos es una base de datos NoSQL que almacena datos en documentos etiquetados en pares clave-valor.


A diferencia de una base de datos clave-valor donde el componente de valor contiene cualquier tipo de datos, una base de datos de documentos siempre almacena un documento en el componente de valor y puede estar en cualquier formato codificado, como XML, JSON o BSON.


La figura~\ref{img:documentos-documento} es la representación visual de un \textit{bucket}, donde tiene una clave como identificador único y documentos anidados.

\begin{figure}[H]
    \centering
    \includegraphics[width=0.75\textwidth]{noSQL/documento.png}
    \caption{Documento en el almacenamiento de documentos}
    \label{img:documentos-documento}
\end{figure}
Otra diferencia importante es que si bien las bases de datos clave-valor no intentan comprender el contenido del componente de valor, las bases de datos de documentos sí lo hacen.% Las etiquetas representan porciones de un documento.


Por ejemplo, hay documentos con etiquetas para identificar qué texto en el documento representa el título, el autor y el cuerpo del documento.


Dentro del cuerpo del documento, existen etiquetas adicionales para indicar capítulos y secciones. A pesar del uso de etiquetas en los documentos, las bases de datos de documentos se consideran sin esquema, es decir, no imponen una estructura predefinida en los datos almacenados.


Para una base de datos de documentos, no tener esquemas significa que aunque todos los documentos tienen etiquetas, no todos tienen las mismas etiquetas, por lo es posible que cada documento tenga su propia estructura.


Las etiquetas en una base de datos de documentos son extremadamente importantes porque son la base de la mayoría de las capacidades adicionales que tienen las bases de datos de documentos sobre las bases de datos clave-valor.


Las etiquetas dentro del documento son accesibles para el DBMS, lo que hace posible consultas complejas. Al igual que las bases de datos clave-valor agrupan pares clave-valor en grupos lógicos llamados \textit{buckets}, las bases de datos de documentos agrupan documentos en grupos lógicos llamados colecciones.


Si bien es posible recuperar un documento especificando la colección y la clave, también es posible realizar consultas en función del contenido de las etiquetas.


Las bases de datos de documentos tienden a funcionar bajo el supuesto de que un documento es independiente, o sea que no está en diferentes tablas como en una base de datos relacional.


Una base de datos de documentos asume que todos los datos relacionados de una orden estén en un solo documento; por lo tanto, cada orden en una colección contendría datos sobre el cliente, el pedido en sí y los productos comprados en esa orden.


Las bases de datos de documentos no almacenan relaciones como se hace en el modelo relacional y generalmente no tienen soporte para operaciones como la unión.
\subsubsection{Orientado a columnas}
De acuerdo con Coronel\cite{coronel_database_nodate}, el modelo de base de datos NoSQL orientado a columnas se originó con el BigTable de Google. 

La figura~\ref{img:familia-columna} representa una familia de columnas; la imagen de arriba es la partición de las diferentes familias de columnas y en la imagen de abajo se nota cada partición individual.

\begin{figure}[H] 
    \centering
    \includegraphics[width=0.75\textwidth]{noSQL/columna.png}
    \caption{Familia de columnas}
    \label{img:familia-columna}
\end{figure}


Las bases de datos de la familia de columnas son parecidas a las relaciones del modelo relacional y organizan datos en pares nombre-valor donde el nombre actúa también como la clave; como se nota en la figura~\ref{img:familia-columna}, un par de clave-valor representa una columna y siempre contiene una fecha que sirve para resolver conflictos de escritura o datos expirados.

\subsubsection{Orientado a grafos}
De acuerdo con Coronel\cite{coronel_database_nodate}, una base de datos NoSQL orientada a grafos está basada en la teoría de grafos para almacenar datos con muchas relaciones.



La figura~\ref{img:nosql-grafo} representa un grafo de una bilioteca, donde cada rectángulo es un nodo y están asociados entre sí por relaciones.
\begin{figure}[H] 
    \centering
    \includegraphics[width=0.75\textwidth]{noSQL/grafo.png}
    \caption{modelo conceptual orientado a grafos}
    \label{img:nosql-grafo}
\end{figure}


Como se muestra en la figura~\ref{img:nosql-grafo}, los componentes principales de las bases de datos de grafos son nodos, aristas y propiedades; el nodo es una instancia específica de algo sobre lo que queremos mantener datos.


Las propiedades son como atributos; son los datos que necesitamos almacenar sobre el nodo; todos los nodos tienen propiedades como nombre y apellido, pero no todos los nodos deben tener las mismas propiedades.


Un borde es una relación entre nodos, está representada por una flecha en la figura~\ref{img:nosql-grafo} y es posible que estén en una dirección o ser bidireccionales.


Para hacer una consulta se atraviesa el grafo y los recorridos se enfocan en las relaciones entre nodos, como la ruta más corta y el grado de conexión.


\section{Unified Modeling Language}
De acuerdo con Wikipedia\cite{wikipedia_diagrama_2020}, un diagrama de clases en Lenguaje Unificado de Modelado (UML) es un tipo de diagrama de estructura estática que describe la estructura de un sistema mostrando las clases del sistema, sus atributos, operaciones (o métodos), y las relaciones entre los objetos.


\subsection*{Miembros}
UML proporciona mecanismos para representar los miembros de la clase, como atributos y métodos, así como información adicional sobre ellos.


\subsubsection*{Visibilidad}
Para especificar la visibilidad de un miembro de la clase (es decir, cualquier atributo o método), se coloca uno de los siguientes signos delante de ese miembro:

\begin{enumerate}
    \item +	Público
    \item -	Privado
    \item \# Protegido
    \item /	Derivado (se puede combinar con otro)
    \item ~	Paquete
\end{enumerate}

\subsubsection*{Ámbitos}

UML especifica dos tipos de ámbitos para los miembros: instancias y clasificadores y estos últimos se representan con nombres subrayados.

\begin{enumerate}
    \item Los miembros clasificadores se denotan comúnmente como “estáticos” en muchos lenguajes de programación. Su ámbito es la propia clase.
    \begin{enumerate}
        \item Los valores de los atributos son los mismos en todas las instancias.
        \item La invocación de métodos no afecta al estado de las instancias.
    \end{enumerate}
    \item Los miembros instancias tienen como ámbito una instancia específica.
\begin{enumerate}
    \item Los valores de los atributos pueden variar entre instancias.
    \item La invocación de métodos puede afectar al estado de las instancias(es decir, cambiar el valor de sus atributos).
\end{enumerate}
\end{enumerate}
Para indicar que un miembro posee un ámbito de clasificador, hay que subrayar su nombre. De lo contrario, se asume por defecto que tendrá ámbito de instancia.

\subsection*{Relaciones}
Una relación es un término general que abarca los tipos específicos de conexiones lógicas que se pueden encontrar en los diagramas de clases y objetos. UML presenta las siguientes relaciones:

\subsubsection*{Enlace}
Un enlace es la relación más básica entre objetos.

\subsubsection*{Asociación}

\begin{figure}[h!t] 
    \centering
    \includegraphics[width=0.65\textwidth]{uml/01.png}
    \caption{Asociación}
    \label{img:uml-asociacion}
\end{figure}

Una asociación representa a una familia de enlaces. Una asociación binaria (entre dos clases) normalmente se representa con una línea continua. Una misma asociación puede relacionar cualquier número de clases. Una asociación que relacione tres clases se llama asociación ternaria.

A una asociación se le puede asignar un nombre, y en sus extremos se puede hacer indicaciones, como el rol que desempeña la asociación, los nombres de las clases relacionadas, su multiplicidad, su visibilidad, y otras propiedades.

Hay cuatro tipos diferentes de asociación: bidireccional, unidireccional, agregación (en la que se incluye la composición) y reflexiva. Las asociaciones unidireccional y bidireccional son las más comunes.

Por ejemplo, una clase vuelo se asocia con una clase avión de forma bidireccional. La asociación representa la relación estática que comparten los objetos de ambas clases.

\subsubsection*{Agregación}


\begin{figure}[h!t] 
    \centering
    \includegraphics[width=0.65\textwidth]{uml/02.png}
    \caption{Agregación}
    \label{img:uml-agregacion}
\end{figure}


La agregación o agrupación es una variante de la relación de asociación “tiene un”: la agregación es más específica que la asociación. Se trata de una asociación que representa una relación de tipo parte-todo o parte-de.


Como se puede ver en la imagen del ejemplo (en inglés), un Profesor 'tiene una' clase a la que enseña.


Al ser un tipo de asociación, una agregación puede tener un nombre y las mismas indicaciones en los extremos de la línea. Sin embargo, una agregación no puede incluir más de dos clases; debe ser una asociación binaria.


Una agregación se puede dar cuando una clase es una colección o un contenedor de otras clases, pero a su vez, el tiempo de vida de las clases contenidas no tienen una dependencia fuerte del tiempo de vida de la clase contenedora (de el todo). Es decir, el contenido de la clase contenedora no se destruye automáticamente cuando desaparece dicha clase.


En UML, se representa gráficamente con un rombo hueco junto a la clase contenedora con una línea que lo conecta a la clase contenida. Todo este conjunto es, semánticamente, un objeto extendido que es tratado como una única unidad en muchas operaciones, aunque físicamente está hecho de varios objetos más pequeños.

\begin{figure}[h!t] 
    \centering
    \includegraphics[width=0.65\textwidth]{uml/03.png}
    \caption{Asociación rombo sin rellenar y composición rombo negro}
    \label{img:uml-composicion}
\end{figure}

\subsubsection*{Composición}

La representación en UML de una relación de composición es mostrada con una figura de diamante rellenado del lado del la clase contenedora, es decir al final de la línea que conecta la clase contenido con la clase contenedor.


\paragraph*{Diferencias entre Composición y Agregación}

\subparagraph*{Relación de Composición}

\begin{enumerate}
    \item Cuando intentamos representar un todo y sus partes. Ejemplo, un motor es una parte de un coche.
    \item Cuando se elimina el contenedor, el contenido también es eliminado. Ejemplo, si eliminamos una universidad eliminamos igualmente sus departamentos.
\end{enumerate}


\subparagraph*{Relación de Agrupación}

\begin{enumerate}
    \item Cuando representamos las relaciones en un software o base de datos. Ejemplo, el modelo de motor MTR01 es parte del coche MC01. Como tal, el motor MTR01 puede hacer parte de cualquier otro modelo de coche, es decir si eliminamos el coche MC01 no es necesario eliminar el motor pues podemos usarlo en otro modelo.
    \item Cuando el contenedor es eliminado, el contenido usualmente no es destruido. Ejemplo, un profesor tiene estudiantes, cuando el profesor muere los estudiantes no mueren con él o ella.
\end{enumerate} 
\section{Tecnologías a usar}
Para seleccionar las tecnologías a usar para el desarrollo de la aplicación web, se ha optado por hacer un estudio y comparación de tecnologías similares.


Lo que resta de esta sección está organizado de la siguiente manera: primero se muestra cada tecnología usar. En caso de que sea la única opción se describirá qué es la herramienta y en caso de que haya varias opciones, se explicará cada opción y se tendrá un apartado al final de cada comparación sobre la herramienta que se ha elegido.


Finalmente, en la subsección de conlusiones se pondrá un resumen de cada tecnología escogida y el porqué.
\subsection{Hypertext Transfer Protocol}
De acuerdo con la W3C\cite{noauthor_http_nodate}, Hypertext Transfer Protocol (HTTP, o protocolo de transferencia de hipertexto), es el protocolo de comunicación que permite transferir información en la World Wide Web.


HTTP fue desarrollado por el World Wide Web Consortium y la Internet Engineering Task Force, colaboración que culminó en 1999 con la publicación de varios RFC, siendo el más importante el RFC 2616 que especifica la versión 1.1 del protocolo.


HTTP es un protocolo sin estado, es decir, no guarda ninguna información sobre conexiones anteriores; sin embargo, el desarrollo de aplicaciones web necesita frecuentemente mantener un estado, por lo que se usan \textit{cookies}, que son archivos generados en un servidor que son almacenados en el sistema cliente.


Es un protocolo orientado a transacciones y sigue el esquema petición-respuesta entre un cliente y un servidor; al cliente se le suele llamar ``agente de usuario'' (o \textit{user agent}), que realiza una petición enviando un mensaje con cierto formato al servidor, mientras que al servidor se le suele llamar servidor web y envía un mensaje de respuesta. 


HTTP tiene métodos de petición flexibles que permiten añadir nuevos métodos o funcionalidades; el número de métodos ha ido en aumento según se avanza en las versiones del protocolo donde los más importantes son:


\begin{enumerate}
    \item Método \textit{get}: solicita una representación del recurso especificado; solo deben recuperar datos. 
    \item Método \textit{head}: pide una respuesta idéntica a la que correspondería a una petición \textit{get}, pero en la respuesta no se devuelve el cuerpo; esto es útil para poder recuperar los metadatos de los encabezados de respuesta, sin tener que transportar todo el contenido.
    \item Método \textit{post}: envía datos para que sean procesados por un recurso identificado que se incluirán en el cuerpo de la petición. 
    \item Método \textit{put}: sube o carga un recurso especificado (archivo o fichero) y es más eficiente que el método \textit{post}, porque permite escribir un archivo en una conexión socket establecida con el servidor.
\end{enumerate}
\subsection{HTML 5 }

De acuerdo a la documentación de Mozilla\cite{noauthor_html_nodate}, HyperText Markup Language, abreviado HTML, o lenguaje de marcado de hipertextos, es la pieza más básica para la construcción de la web y se usa para definir el sentido y estructura del contenido en una página web. 


Es un estándar a cargo del World Wide Web Consortium (W3C) o Consorcio WWW, organización dedicada a la estandarización de casi todas las tecnologías ligadas a la web. 


HTML hace uso de enlaces que conectan las páginas web entre sí, ya sea dentro de un mismo sitio web o entre diferentes sitios web.


Un elemento HTML se separa de otro texto en un documento por medio de ``etiquetas", las cuales consisten en elementos rodeados por ``<,>".


Ejemplos de estas etiquetas son <head>, <title>, <body>, <header>, <p>, <ul>, <ol>, <li>, y otros más.


HTML 5 (HyperText Markup Language, versión 5) es la quinta revisión importante del lenguaje básico de la World Wide Web, HTML.


HTML5 establece elementos y atributos que reflejan el uso típico de los sitios web modernos. Algunos de ellos son técnicamente similares a las etiquetas <div> y <span>, pero tienen un significado semántico, como por ejemplo <nav> (bloque de navegación del sitio web) y <footer>.


Características:

\begin{enumerate}
    \item Incorpora etiquetas: \textit{canvas} 2D y 3D, audio, vídeo con codecs para mostrar los contenidos multimedia. Actualmente hay una lucha entre imponer codecs libres (WebM + VP8) o privados (H.264/MPEG-4 AVC).
    \item Etiquetas para manejar grandes conjuntos de datos: Datagrid, Details, Menu y Command. Permiten generar tablas dinámicas que pueden filtrar, ordenar y ocultar contenido en cliente.
    \item Mejoras en los formularios: Nuevos tipos de datos (\textit{email, number, url, datetime})  y facilidades para validar el contenido sin JavaScript.
    \item Visores: MathML (fórmulas matemáticas) y SVG (gráficos vectoriales); en general se deja abierto a poder interpretar otros lenguajes XML.
    \item Drag \& Drop: nueva funcionalidad para arrastrar objetos como imágenes.
\end{enumerate}

Respecto a la compatibilidad con los navegadores, la mayoría de elementos de HTML5 son compatibles con Firefox 19, Chrome 25, Safari 6 y Opera 12 en adelante.


\subsection{CSS3 vs SCSS}

De acuerdo a la documentación de la W3C\cite{noauthor_what_nodate}, Cascading Style Sheets, CSS, o hojas de estilo en cascada, es un lenguaje de diseño gráfico para definir y crear la presentación de un documento estructurado escrito en un lenguaje de marcado.


CSS está diseñado principalmente para marcar la separación del contenido del documento y la presentación del mismo, con características como las capas o layouts, los colores y las fuentes.


La separación entre el contenido del documento y la presentación busca mejorar la accesibilidad, proveer más flexibilidad y control, permitir que varios documentos HTML compartan un mismo estilo usando una sola hoja de estilos separada en un archivo .css y reducir la complejidad y la repetición de código en la estructura del documento.


La especificación CSS describe un esquema prioritario para determinar qué reglas de estilo se aplican si más de una regla coincide para un elemento en particular. Estas reglas son aplicadas con un sistema llamado de cascada, de modo que las prioridades son calculadas y asignadas a las reglas, así que los resultados son predecibles.


La especificación CSS es mantenida por el World Wide Web Consortium (W3C). El MIME type text/css está registrado para su uso por CSS descrito en el RFC 23185​. El W3C proporciona una herramienta de validación de CSS gratuita para los documentos CSS.



CSS se ha creado en varios niveles y perfiles. Cada nivel de CSS se construye sobre el anterior, generalmente añadiendo funciones al nivel previo.


Los perfiles son, generalmente, parte de uno o varios niveles de CSS definidos para un dispositivo o interfaz particular. Actualmente, pueden usarse perfiles para dispositivos móviles, impresoras o televisiones.


La última versión del estándar, CSS3.1, está dividida en varios documentos separados, llamados ``módulos".


Cada módulo añade nuevas funcionalidades a las definidas en CSS2, de manera que se preservan las anteriores para mantener la compatibilidad.


Los trabajos en el CSS3.1 comenzaron a la vez que se publicó la recomendación oficial de CSS2 y los primeros borradores de CSS3.1 fueron liberados en junio de 1999.


Debido a la modularización del CSS3.1, diferentes módulos pueden encontrarse en diferentes estados de su desarrollo,​ de forma que hay alrededor de cincuenta módulos publicados,​ tres de ellos se convirtieron en recomendaciones oficiales de la W3C en 2011: ``Selectores", ``Espacios de nombres", y ``Color".


Respecto al soporte de los navegadores web, cada navegador web usa un motor de renderizado para renderizar páginas web, y el soporte de CSS no es exactamente igual en ninguno de los motores de renderizado. Ya que los navegadores no aplican el CSS correctamente, muchas técnicas de programación han sido desarrolladas para ser aplicadas por un navegador específico (comúnmente conocida esta práctica como \textit{CSS hacks} o \textit{CSS filters}).


Además, CSS3 definen muchas queries, entre las cuales se provee la directiva @supports que permite a los desarrolladores especificar navegadores con soporte para alguna función en específico directamente en el CSS3.1​. 


\subsection{JavaScript vs TypeScript}

\subsubsection*{JavaScript}
De acuerdo con la documentación de Mozilla\cite{noauthor_javascript_nodate}, JavaScript es una marca registrada con licencia de Sun Microsystems (ahora Oracle) que se usa para describir la implementación del lenguaje de programación JavaScript.


Debido a problemas de registro de marcas en la Asociación Europea de Fabricantes de Computadoras, la versión estandarizada del lenguaje tiene el nombre de ECMAScript, sin embargo, en la práctica se conoce como lenguaje JavaScript. 


Abreviado como JS, es un lenguaje ligero e interpretado, orientado a objetos, basado en prototipos, imperativo, débilmente tipado y dinámico; es usado en node.js, Apache CouchDB y Adobe Acrobat.


EL núcleo del lenguaje JavaScript está estandarizado por el Comité ECMA TC39 como un lenguaje llamado ECMAScript. La última versión de la especificación es ECMAScript 6.0.


El estándar ECMAScript define 

\begin{enumerate}
    \item Sintaxis: reglas de análisis, palabras clave, flujos de control, inicialización literal de objetos.
    \item Mecanismos de control de errores: \textit{throw, try/catch}, habilidad para crear tipos de Errores definidos por el usuario.
    \item Tipos: \textit{boolean, number, string, function, object}.
    \item  Objetos globales: en un navegador, los objetos globales son los objetos de la ventana, pero ECMAScript solo define API no especificas para navegadores, como \textit{parseInt, parseFloat, decodeURI, encodeURI}.
    \item Mecanismo de herencia basada en prototipos.
    \item Objetos y funciones incorporadas.
    \item Modo estricto.
\end{enumerate}

La sintaxis básica es similar a Java y C++ con la intención de reducir el número de nuevos conceptos necesarios para aprender el lenguaje. Las construcciones del lenguaje, tales como sentencias \textit{if}, bucles \textit{for} y \textit{while}, bloques \textit{switch} y \textit{try catch} funcionan de la misma manera que en estos lenguajes (o casi).


JavaScript puede funcionar como lenguaje procedimental y como lenguaje orientado a objetos. Los objetos se crean programáticamente añadiendo métodos y propiedades a lo que de otra forma serían objetos vacíos en tiempo de ejecución, en contraposición a las definiciones sintácticas de clases comunes en los lenguajes compilados como C++ y Java. Una vez se ha construido un objeto, puede usarse como modelo (o prototipo) para crear objetos similares.


Las capacidades dinámicas de JavaScript incluyen construcción de objetos en tiempo de ejecución, listas variables de parámetros, variables que pueden contener funciones, creación de scripts dinámicos (mediante eval), introspección de objetos (mediante \textit{for ... in}), y recuperación de código fuente (los programas de JavaScript pueden decompilar el cuerpo de funciones a su código fuente original).


Desde el 2012, todos los navegadores modernos soportan completamente ECMAScript 5.1. Los navegadores más antiguos soportan por lo menos ECMAScript 3. 


El 17 de Julio de 2015, ECMA International publicó la sexta versión de ECMAScript, la cual es oficialmente llamada ECMAScript 2015 y fue inicialmente nombrada como ECMAScript 6 o ES6. Desde entonces, los estándares ECMAScript están en ciclos de lanzamiento anuales.


\subsubsection*{TypeScript}
De acuerdo con TypeScript Publishing\cite{typescript_publishing_typescript_2019}, TypeScript es por definición es JavaScript para el desarrollo de aplicaciones, siendo también un superconjunto del mismo.


TypeScript es un lenguaje compilado orientado a objetos. Fue diseñado por Anders Hejlsberg (diseñador de C\#) en Microsoft. TypeScript es tanto un lenguaje como un conjunto de herramientas. TypeScript es un superconjunto de JavaScript que genera código JavaScript.


\paragraph*{Características}
\begin{itemize}
    \item Compilación: cuenta con un transpilador para la verificación de errores si hay errores de compilación, cosa que no es posible con JavaScript.
    \item Tipeo estático fuerte: provee un sistema opcional de tipeo estático y de inferencia de tipos a traves del TypeScript Language Service, lo que permite inferir el tipo de una variabla declara sin tipo en función de su valor.
    \item Definiciones de tipo: permite la extensión del lenguaje con bibliotecas externas JavaScript.
    \item Programación orientaba a objetos: admite conceptos como clases, interfaces, herencia, etc.
\end{itemize}



\subsubsection*{Elección}

De acuerdo con el sitio stack overflow\cite{noauthor_stack_nodate}, JavaScript es el lenguaje más popular de 2019 y aunque TypeScript es de los lenguajes que tienen un mayor nivel de aceptación, se usará JavaScript no solo por ser el lenguaje más popular y, en consecuencia, con más compatibilidad y material de ayuda, sino también porque el equipo está acostumbrado a este lenguaje.


\subsection{JavaScript Web Frameworks: Vue/Nuxt vs React vs Angular}

De acuerdo con Wikipedia\cite{wikipedia_contributors_web_2020}, un web \textit{framework} es un conjunto de \textit{software} que permite el desarrollo de una aplicación web y en el lenguaje JavaScript hay varias opciones, incluidas los más populares Vue, React y Angular.

\subsubsection*{React}
De acuerdo con Wikipedia\cite{wikipedia_react_2020}, React es una biblioteca Javascript de código abierto diseñada para crear interfaces de usuario con el objetivo de facilitar el desarrollo de \textit{single page applications}.

\paragraph*{Características}
\begin{enumerate}
    \item Virtual DOM: React usa un virtual DOM propio en lugar del navegador.
    \item Props: son definidos como los atributos de configuración para dicho componente.
    \item Estado de cada componente: lleva un registro de las propiedades y atributos del componente.
    \item Ciclos de vida: son la serie de estados por los cuales pasan los componentes statefull a lo largo de su existencia. 
\end{enumerate}

\subsubsection*{Angular}
De acuerdo con Wikipedia\cite{wikipedia_angular_2020}, Angular es un web \textit{framework} desarrollado en TypeScript de código abierto mantenido por Google que se utiliza para crear y mantener \textit{single page aplications}. 

\begin{enumerate}
    \item Generación de código
    \item Componentes
    \item Ciclos de vida de componentes
    
\end{enumerate}

\subsubsection*{Vue/Nuxt}

De acuerdo con la documentación de Vue.js\cite{noauthor_que_nodate}, Vue es un \textit{framework} progresivo para desarrollar interfaces de usuario. A diferencia de otros \textit{frameworks} monolíticos, Vue está diseñado desde cero para ser utilizado incrementalmente.


La librería central está enfocada solo en la capa de visualización y es fácil de utilizar e integrar con otras librerias o proyectos existentes. Por otro lado, Vue también es perfectamente capaz de impulsar sofisticadas \textit{single-page applications} cuando se utiliza en combinación con librerías de apoyo.


\paragraph*{Comparación con React}

React y Vue comparten muchas similitudes; ambos utilizan un DOM virtual, proporcionan componentes de vista reactivos y componentes, mantienen el enfoque en la librería central, con temas como el enrutamiento y la gestión global del estado manejadas por librerías asociadas.


Tanto React como Vue ofrecen un rendimiento comparable en los casos de uso más comunes, con Vue normalmente un poco por delante debido a su implementación más ligera del DOM virtual.


En Vue, las dependencias de un componente se rastrean automáticamente durante su renderizado, por lo que el sistema sabe con precisión qué componentes deben volver a renderizarse cuando cambia el estado. Se puede considerar que cada componente tiene un \textit{shouldComponentUpdate} automáticamente implementado.


En general, esto elimina la necesidad de toda una clase de optimizaciones de rendimiento a los desarrolladores y les permite centrarse más en la construcción de la aplicación en sí misma a medida que va escalando.

\paragraph*{Comparación con Angular}
En términos de rendimiento, ambos \textit{frameworks} son excepcionalmente rápidos y no hay suficientes datos de casos de uso en el mundo real para hacer un veredicto.


Vue es mucho menos intrusivo en las decisiones del desarrollador que Angular, ofreciendo soporte oficial para una variedad de sistemas de desarrollo, sin restricciones sobre cómo estructurar su aplicación. Muchos desarrolladores disfrutan de esta libertad, mientras que algunos prefieren tener solo una forma correcta de desarrollar cualquier aplicación.


Para empezar con Vue, todo lo que se necesita es familiarizarse con HTML y ES5 JavaScript; con estas habilidades básicas, se puede empezar a desarrollar aplicaciones no triviales.


La curva de aprendizaje de Angular es mucho más pronunciada. La superficie de la API del framework es enorme y como usuario necesitará familiarizarse con muchos más conceptos antes de ser productivo. La complejidad de Angular se debe en gran medida a su objetivo de diseño de apuntar solo a aplicaciones grandes y complejas - pero eso hace que el \textit{framework} sea mucho más difícil de entender para los desarrolladores con menos experiencia.

\subsubsection*{Nuxt.js}
De acuerdo a la documentación de Nuxt\cite{noauthor_what_nodate-1}, Nuxt.js es un \textit{framework} progresivo basado en Vue.js para crear aplicaciones web. Se basa en Vue.js y herramientas de desarrollo como webpack, Babel y PostCSS. El objetivo de Nuxt es hacer que el desarrollo web en Vue.js sea eficaz.

\paragraph*{Características}
\begin{enumerate}
    \item Manejo de archivos Vue (*.vue)
    \item División automática de código
    \item Representación del lado del servidor
    \item Potente sistema de enrutamiento con datos asincrónicos
    \item Servicio de archivos estáticos
    \item Soporte sintaxis ES2015+ (Javascript ES6)
    \item Gestión del elemento <head> (<title>, <meta>, etc.)
    \item Preprocesador: Sass, Less, Stylus, etc.
\end{enumerate}



\subsubsection*{Elección}

Se usará Vue/Nuxt por ser el \textit{framework} con el que el equipo está más acostumbrado, además de ser el máx flexible de las opciones expuestas.
\subsection*{Material Design}
De acuerdo a la documentación de Google\cite{noauthor_introduction_nodate}, Material Design es es un lenguaje visual que sintetiza los principios clásicos del buen diseño respecto a las ideas de Google.

Material design es una normativa de diseño enfocado en la visualización del sistema operativo Android, además en la web y en cualquier plataforma. Fue desarrollado por Google y anunciado en la conferencia Google I/O celebrada el 25 de junio de 2014. 


Sus objetivos son crear un lenguaje visual que sintetice los principios clásicos del buen diseño, unificar el desarrollo de un único sistema subyacente que unifique la experiencia del usuario en plataformas y dispositivos, así como personalizar el lenguaje visual de Material Design.


Material Design está inspirado en el mundo físico y sus texturas, incluida la forma en que reflejan la luz y proyectan sombras. El objetivo de Material es hacer imaginar los medios de papel y tinta.


El diseño de materiales se guía por métodos de diseño de impresión (tipografía, cuadrículas, espacio, escala, color e imágenes) para crear jerarquía, significado y enfoque que sumergen a los espectadores en la experiencia.


Material Design mantiene la misma interfaz de usuario en todas las plataformas, utilizando componentes compartidos en Android, iOS, Flutter y la web.

\subsection{MongoDB vs Apache CouchDB}
\subsubsection*{MongoDB}
De acuerdo a la documentación de Wikipedia\cite{wikipedia_mongodb_2020}, MongoDB (del inglés humongous, ``enorme") es un sistema de base de datos NoSQL, orientado a documentos y de código abierto.


En lugar de guardar los datos en tablas, tal y como se hace en las bases de datos relacionales, MongoDB guarda estructuras de datos BSON (una especificación similar a JSON) con un esquema dinámico, haciendo que la integración de los datos en ciertas aplicaciones sea más fácil y rápida.

\paragraph*{Características}
\begin{enumerate}
    \item Consultas ad hoc: MongoDB soporta la búsqueda por campos, consultas de rangos y expresiones regulares.
    \item Indexación: cualquier campo en un documento de MongoDB puede ser indexado, al igual que es posible hacer índices secundarios. 
    \item Replicación: MongoDB soporta el tipo de replicación primario-secundario. 
    \item Balanceo de carga; MongoDB puede escalar de forma horizontal usando el concepto de \textit{sharding}.
    \item Almacenamiento de archivos: MongoDB puede ser utilizado como un sistema de archivos, aprovechando la capacidad de MongoDB para el balanceo de carga y la replicación de datos en múltiples servidores. 
    \item Agregación: MongoDB proporciona un framework de agregación que permite realizar operaciones similares al GROUP BY de SQL.
\end{enumerate}


\subsubsection*{Apache CouchDB}

\subsection{GoJS vs mxGraph vs D3.js}
De acuerdo a la documentación de GoJS\cite{noauthor_gojs_nodate}, GoJS es una biblioteca de JavaScript y TypeScript para crear diagramas y gráficos interactivos.


GoJS le permite crear todo tipo de diagramas y gráficos para sus usuarios, desde simples diagramas de flujo y organigramas hasta diagramas industriales altamente específicos, diagramas SCADA y BPMN, diagramas médicos como genogramas y diagramas de modelos de brotes, y más. 


GoJS facilita la construcción de diagramas JavaScript de nodos complejos, enlaces y grupos con plantillas y diseños personalizables.


GoJS ofrece muchas funciones avanzadas para la interactividad del usuario, como arrastrar y soltar, copiar y pegar, edición de texto en el lugar, información sobre herramientas, menús contextuales, diseños automáticos, plantillas, enlace de datos y modelos, gestión de estado y deshacer transaccional, paletas , descripciones generales, controladores de eventos, comandos, herramientas extensibles para operaciones personalizadas y animaciones personalizables.


GoJS se implementa en TypeScript y puede usarse como una biblioteca de JavaScript o incorporarse a su proyecto desde fuentes de TypeScript. 


GoJS normalmente se ejecuta completamente en el navegador, renderizando a un elemento HTML Canvas o SVG sin ningún requisito del lado del servidor.



\subsection{Python 3 vs Java}
De acuerdo con Mark Lutz\cite{lutz_learning_2013}, Python es un lenguaje de programación interpretado, interactivo y orientado a objetos. Incorpora módulos, excepciones, tipeo dinámico, tipos de datos dinámicos y clases.


Tiene sintaxis clara, interfaces para muchas llamadas de sistema y bibliotecas, así como para varios sistemas de ventanas. Además, es extensible en C o C++. 


También se puede usar como un lenguaje de extensión para aplicaciones que necesitan una interfaz programable y es portátil: se ejecuta en muchas variantes de Unix, en Mac y en Windows 2000 y versiones posteriores.


La biblioteca estándar del lenguaje, cubre áreas como el procesamiento de cadenas (expresiones regulares, Unicode, cálculo de diferencias entre archivos), protocolos de Internet (HTTP, FTP, SMTP, XML-RPC, POP, IMAP, programación CGI), ingeniería de software (pruebas unitarias, registro, creación de perfiles, análisis del código Python) e interfaces del sistema operativo (llamadas al sistema, sistemas de archivos, sockets TCP/IP).

\subsubsection*{Fortalezas}

\paragraph*{Es orientado a objetos y funcional}
Python es un lenguaje orientado a objetos; su modelo de clase admite nociones avanzadas como el polimorfismo, la sobrecarga del operador y la herencia múltiple; sin embargo, en el contexto de la simple sintaxis y escritura de Python, la programación orientada a objetos es fácil de aplicar.


Además de servir para la estructuración y reutilización de código, la naturaleza orientada a objetos de Python lo hace ideal como herramienta de secuencias de comandos para otros lenguajes de sistemas orientados a objetos. Por ejemplo, con el código apropiado, los programas Python pueden especializar clases implementadas en C++, Java y C\#.


No obstante, la programación orientada a objetos es una opción en Python. Al igual que C++, Python admite modos de programación tanto procedimentales como orientados a objetos. Las herramientas orientadas a objetos se pueden aplicar siempre que las restricciones lo permitan. 


Además de sus paradigmas originales de procedimientos (basados ​​en declaraciones) y orientados a objetos (basados ​​en clases), Python en los últimos años ha adquirido soporte incorporado para la programación funcional, un conjunto que incluye generadores, comprensiones, cerraduras, mapas, decoradores , funciones anónimas lambdas.

\paragraph*{Es extensible}
Su conjunto de herramientas lo ubica entre los lenguajes de \textit{scripting} tradicionales como Tcl, Scheme y Perl; y los lenguajes de desarrollo de sistemas como C, C ++ y Java.


Python proporciona toda la simplicidad y facilidad de uso de un lenguaje de programación, junto con herramientas de ingeniería de software más avanzadas que normalmente se encuentran en lenguajes compilados.

A diferencia de algunos lenguajes de secuencias de comandos, esta combinación hace que Python sea útil para proyectos de desarrollo a gran escala. Algunas de las herramientas de Python son:
\paragraph*{Escritura dinámica}
Python realiza un seguimiento de los tipos de objetos que utiliza su programa cuando se ejecuta; eso no requiere declaraciones complicadas de tipo y tamaño en su código. De hecho, no existe una declaración de tipo o variable en Python. 


Debido a que el código Python no restringe los tipos de datos, también se aplica automáticamente a toda una gama de objetos.

\paragraph*{Gestión automática de la memoria}
Python asigna automáticamente objetos y los reclama el recolector de basura cuando ya no se usan y la mayoría puede crecer y reducirse según la demanda. Es decir, Python realiza un seguimiento de los detalles de la memoria de bajo nivel.


\paragraph*{Tipos de objetos incorporados}
Python proporciona estructuras de datos de uso común como listas, diccionarios y cadenas como partes intrínsecas del lenguaje. Son flexibles y fáciles de usar. Por ejemplo, los objetos integrados pueden crecer y reducirse según demanda, pueden anidarse arbitrariamente para representar información compleja, y más.
\paragraph*{Herramientas incorporadas}
Para procesar todos esos tipos de objetos, Python viene con operadores potentes y estándar, que incluyen concatenación (unir colecciones), segmentar (extraer secciones), ordenar, mapear y más.
\paragraph*{Utilidades de biblioteca}
Para tareas más específicas, Python también viene con una gran colección de herramientas de biblioteca precodificadas que admiten todo, desde la coincidencia de expresiones regulares hasta la creación de redes. Una vez que aprende el lenguaje en sí, las herramientas de la biblioteca de Python son donde ocurre gran parte de la acción a nivel de aplicación.
\paragraph*{Utilidades de terceros}
Debido a que Python es de código abierto, los desarrolladores pueden contribuir con herramientas precodificadas que admitan tareas que aún no son herramientas estándar; en la Web, encontrará soporte gratuito para COM, imágenes, programación numérica, XML y acceso a bases de datos.
\subsection{Django vs Flask}
Para elegir el lenguaje a usar para el back end se realizó un estudio de lenguajes apropiados para usar con un JavaScript web \textit{framework}.

\subsubsection*{Flask}
Flask es un micro \textit{web framework} escrito en Python. Se clasifica como micro porque no requiere herramientas o bibliotecas particulares.


No tiene capa de abstracción de base de datos, validación de formularios ni ningún otro componente donde las bibliotecas de terceros preexistentes brinden funciones comunes. Sin embargo, Flask admite extensiones que pueden agregar características de la aplicación como si se implementaran en el propio Flask.


Existen extensiones para mapeadores relacionales de objetos, validación de formularios, manejo de carga, varias tecnologías de autenticación abiertas y varias herramientas relacionadas con el marco común. Las extensiones se actualizan con mucha más frecuencia que el programa central Flask.


\paragraph*{Ventajas y desventajas de Flask}
Está basado en la especificación WSGI de Werkzeug y el motor de templates Jinja2; además, tiene una licencia BSD.


Entre las ventajas y desventajas, destacamos:

\subparagraph*{Ventajas}
\begin{enumerate}
    \item Es un framework que se destaca en instalar extensiones o complementos de acuerdo al tipo de proyecto que se va a desarrollar, es decir, es perfecto para el prototipado rápido de proyectos.
    \item Incluye un servidor web, así podemos evitamos instalar uno como Apache o Nginx. Además, nos ofrece soporte para pruebas unitarias y para Cookies de seguridad (sesiones del lado del cliente), apoyándose en el motor de plantillas ​Jinja2​.
    \item Su velocidad es mejor a comparación de Django. Generalmente el desempeño que tiene Flask es superior debido a su diseño minimalista que tiene en su estructura.
    \item  Flask permite combinarse con herramientas para potenciar su funcionamiento, por ejemplo: Jinja2, SQLAlchemy, Mako y Peewee entre otras.
\end{enumerate}
\subparagraph*{Desventajas}

\begin{enumerate}
    \item  Su sistema de autenticación de usuarios es muy básico, a comparación del potente sistema de autenticación que utiliza Django, este puede crear un sistema de Login API sencillo para aplicaciones más pequeñas.
    \item  Su representación de Plugins no es tan extensa como la tiene Django.
    \item  Es complicado en las pruebas unitarias o migraciones.
    \item El ORM (Mapeo objeto relacional) para conectar con las bases de datos, SQLAlchemy es externo.
\end{enumerate}


\subsubsection*{Django}

De acuerdo con Wikipedia\cite{wikipedia_django_2020}, Django es un framework de desarrollo web de código abierto, escrito en Python, que respeta el patrón de diseño conocido como MVC (Modelo–Vista–Controlador).

\paragraph*{Características}
\begin{enumerate}
    \item Aplicaciones ``enchufables" que pueden instalarse en cualquier página gestionada con Django.
    \item Una API de base de datos robusta.
    \item Un sistema incorporado de ``vistas genéricas" que ahorra tener que escribir la lógica de ciertas tareas comunes.
    \item Un sistema extensible de plantillas basado en etiquetas, con herencia de plantillas.
    \item Un despachador de URL basado en expresiones regulares.
    \item Un sistema ``middleware" para desarrollar características adicionales.
    \item Documentación incorporada accesible a través de la aplicación administrativa.
\end{enumerate}

\subsubsection*{Elección}

Se ha elegido Flask por ser un \textit{framework} web conocido por el equipo y que ha dado resultados.
\section{Conclusiones}

\begin{enumerate}
    \item Tomando en cuenta la experiencia del equipo con CSS, además de que el proyecto no se enfocará en hacer muchas hojas de estilo para cada componente, página o vista de la aplicación web, se usará CSS en lugar de Sass.
    \item De acuerdo con el sitio stack overflow\cite{noauthor_stack_nodate}, JavaScript es el lenguaje más popular de 2019 y aunque TypeScript es de los lenguajes que tienen un mayor nivel de aceptación, se usará JavaScript no solo por ser el lenguaje más popular y, en consecuencia, con más compatibilidad y material de ayuda, sino también porque el equipo está acostumbrado a este lenguaje y tiene experiencia con web \textit{frameworks} escritos en JavaScript.
    \item Se usará Vue/Nuxt por ser el \textit{framework} con el que el equipo está más acostumbrado, además de ser el máx flexible de las opciones expuestas.
    \item Se ha elegido usar en una primera instancia Vuetify porque es un CSS \textit{framework} que está integrado en las tecnologías asociadas de Vue, como Vue Router, Vue Meta. Asimismo, sus componentes son simples de entender y de implementar.
    \item De acuerdo con Mosquera\cite{martinez-mosquera_modeling_2020}, MongoDB es la base de datos NoSQL orientada a documentos más popular y usada en los \textit{papers} de investigación. Por ello se usará MongoDB como base de datos.
    \item Se llevó a la práctica en el prototipo funcional las tres opciones antes expuestas junto con algunas otras y se decidió usar GoJS para el proyecto por ser la biblioteca JavaScript más completa para diagramado y que genere los diagramas con un JSON simple para parsear esos datos y realizar la conversión.
    \item Se ha elegido Python para desarrollar los algoritmos del proyecto dado que es multiplataforma y es de fácil integración con el \textit{framework} web elegido para el back end.
    \item Se ha elegido Flask por ser un \textit{framework} web conocido por el equipo y que ha dado resultados.

\end{enumerate}

\chapter{Análisis del diseño}
% Análisis del diseño
% 4.1 Metodología
% 4.1.1 Scrum qué es scrum, para qué sirve y por qué lo elegimos, cómo lo vamos a aplicar
% 4.2 factibilidad si es viable o no 
% 4.2.4 conclusion si es factible ténicamente por ser un trabajo escolar y tenemos el personal capacitado 
% 4.3 analisis del sistema 
% 4.3.1 historias de usuario
% 4.3.2 lista de producto
% 4.3.3 diagrama de actividades, similar al de Omar (el de omar es de funcional) en la presentación
% 4.3.4 Algoritmos para el desarrollo del sistema (Capitulo cinco adjuntar en cap 4 en la parte de análisis)
% 4.4 conclusiones generales

% Capítulo 5: Diseño del sistema
% 5.1 diagrama de clases 
% 5.2 diagrama de la base de datos
% 5.3 arquitectura del sistema 
% 5.4 diseño de la UI 
% 5.5 conclusiones generales

% agregar anexo para ver prototipo funcional, se menciona lo del calendario, cómo acceder, agregar cómo se añade la seguridad inherente por las herramientas que usamos (aspectos del modelado relacional como la normalización). 

% algoritmos 



Aquí falta poner la introducción y el capítulo está organizado de la siguiente manera blablabla y tiene tal, tal y tal.


\section{Metodología}
De acuerdo con la Universidad Católica los Ángeles\cite{universidad_catolica_los_angeles_metodologidesarrollo_2020}, en el campo del desarrollo de \textit{software} hay dos grupos de metodologías: las tradicionales y las ágiles.


Las tradicionales se centran en cumplir con un plan rígido de trabajo establecido en la etapa inicial del proyecto, mientras que las ágiles permiten realizar cambios en los requerimientos conforme avance el mismo.


Dado que cualquier cambio en el proceso de una metodologia tradicional genera la necesidad de una reconstrucción del plan de trabajo (invirtiendo tiempo que se podría usar para desarrollar), surgieron las metodologías ágiles, que permiten realizar cambios en los requerimientos conforme avance el proyecto. 


Tomando en cuenta la experiencia del equipo, esta forma de trabajo permite mostrar avances funcionales en el producto en un periodo de tiempo corto para realizar una evaluación y en caso de ser requerido se sugieran cambios.


Se han propuesto muchos modelos ágiles de proceso y están en uso en toda la industria; entre ellos se encuentran los siguientes:


\begin{itemize}
	\item DAS (Desarrollo Adaptativo de Software): ASD tiene como fundamento la teoría de sistemas adaptativos complejos; por ello, interpreta los proyectos de \textit{software} como sistemas adaptativos complejos compuestos
    por agentes (los interesados), entornos (organizacional o tecnológico) y salidas (el producto desarrollado)\cite{cadavid_revision_2013}.
	\item Scrum: La metodología Scrum para el \textit{desarrollo} ágil de software es un marco de trabajo diseñado para lograr la colaboración eficaz de equipos en proyectos, que emplea un conjunto de reglas y artefactos y define roles que generan la estructura necesaria para su correcto funcionamiento\cite{cadavid_revision_2013}.
	\item MDSD (Método de Desarrollo de Sistemas Dinámicos): DSDM es un marco de trabajo creado para entregar la solución correcta en el momento correcto; utiliza un ciclo de vida iterativo, fragmenta el proyecto en periodos cortos de tiempo y define entregables para cada uno de estos periodos; tiene roles claramente definidos y especifica su trabajo dentro de periodos de tiempo\cite{cadavid_revision_2013}.
	\item Crystal: la filosofía de Crystal define el desarrollo como un juego cooperativo de invención y comunicación cuya meta principal es entregar \textit{software} útil, que funcione y su objetivo secundario es preparar el próximo juego\cite{cadavid_revision_2013}.
\end{itemize}

\subsection{Scrum}

De acuerdo con Ken Schwaber\cite{the_scrum_guide_definitive_2020}, Scrum es un marco de trabajo para la entrega de productos incrementales y de máximo valor productivo.

Un artefacto es un elemento que garantiza la transparencia, es el registro de la información fundamental del proceso Scrum y a continuación se describen sus cuatro artefactos principales:

\begin{itemize}
	\item Lista de producto (\textit{product backlog}):	es el listado de todas las tareas que necesita el proyecto para alcanzar su realización; al iniciar el desarrollo del proyecto, esta lista no se encuentra completa y conforme avanzan los \textit{sprints} se le añaden tareas para solventar las necesidades que van surgiendo gracias a la retroalimentación del cliente.
	\item Lista de pendientes del \textit{sprint} (\textit{sprint backlog}): es la lista de tareas seleccionadas del \textit{product backlog} que se planifica realizar durante el periodo del \textit{sprint} y define a los responsables de cada tarea.
	\item \textit{Sprint}: es el corazón de Scrum, tiene un periodo de tiempo determinado de un mes o incluso menos donde el equipo completa conjuntos de tareas incluidas en el \textit{backlog} para crear un incremento del producto utilizable.
	\item Incremento: es la suma de todos los elementos de la lista de productos completados durante un \textit{sprint} unido con los incrementos de los \textit{sprints} anteriores; al finalizar el \textit{sprint}, el nuevo incremento debe estar en condiciones de ser utilizado.
\end{itemize}

El equipo Scrum (\textit{scrum team}) consiste en los siguientes roles:
\begin{itemize}
	\item El dueño de producto (\textit{product owner}): es la persona responsable de maximizar el valor del producto y el trabajo del equipo de desarrollo; es el único responsable de gestionar la lista de producto y cualquier cambio a esa lista debe ser revisada y aprobada por él.
	\item El equipo de desarrollo (\textit{development team}): son los profesionales que realizan el trabajo para la entrega de un incremento en el producto en cada \textit{sprint}; es un grupo autoorganizado y multifuncional donde cada miembro del equipo tiene habilidades especializadas, pero que la responsabilidad de las tareas completadas, incrementos del producto o retrasos recaen en el equipo como un todo.
	\item  El Scrum master: es la persona responsable de asegurar que Scrum es entendido y adoptado por todos los involucrados en el proyecto, asegurándose de ayudar a las personas externas al equipo a entender qué interacciones son útiles con el equipo de desarrollo.
\end{itemize}


Scrum tiene cuatro eventos principales en  un \textit{sprint}, que sirven para la inspección y adaptación del producto que se describen a continuación:
\begin{itemize}
	\item Planificación del \textit{sprint} (\textit{sprint planning}): es una reunión con el equipo de desarrollo que tiene una duración máxima de 8 horas para el \textit{sprint} de un mes; el Scrum Master es el encargado de que los asistentes entiendan el propósito de dicha reunión.
	\item Scrum diario (\textit{daily scrum}): es una reunión de un máximo de 15 minutos en la cual el equipo expone sus actividades y planifica las tareas de las próximas 24 horas.
	\item Revisión del \textit{sprint} (\textit{sprint review}): al finalizar cada \textit{sprint} se lleva a cabo una reunión para la revisión del incremento del producto y en caso de ser necesario realizar ajustes a la lista de producto.
	\item Retrospectiva del \textit{sprint} (\textit{sprint retrospective}): es cuando el equipo de desarrollo tiene la oportunidad de pensar en mejoras para el próximo \textit{sprint}.
\end{itemize}

por qué Scrum?

cómo vamos a aplicar Scrum?

Como la metodología permite definir un periodo de hasta un mes para cada \textit{sprint}, se ha optado por un periodo de 30 días, contemplándose un total de ocho \textit{sprints}, donde al término de cada uno se tendrá un avance del sistema.
\section{Factibilidad}

De acuerdo con Sommerville\cite{sommerville_software_2011}, un estudio de factibilidad es un estudio breve que responde tres preguntas clave:

\begin{enumerate}
    \item ¿El sistema contribuye a los objetivos generales de la organización? 
    \item ¿Se puede implementar el sistema dentro del cronograma y el presupuesto utilizando las tecnologías actuales? 
    \item ¿Se puede integrar el sistema con otros sistemas que se utilizan? 
\end{enumerate}


Para responder estas preguntas se expone la factibilidad técnica, factibilidad económica, costos de desarrollo y conclusiones.

\subsection{Factibilidad técnica}\label{ref:factibilidad-tecnica}

De acuerdo con Pressman\cite{pressman_software_2005}, este estudio determina si el equipo de desarrollo cuenta con los recursos técnicos necesarios para la realización del sistema propuesto; esto se realiza considerando la disponibilidad de los recursos tanto de \textit{hardware}, \textit{software} y recurso humano.

\subsubsection*{Sistema operativo}

De acuerdo con Stallings\cite{stallings_operating_2012}, un sistema operativo es el \textit{software} principal o conjunto de programas de un sistema informático que gestiona los recursos de \textit{hardware} y provee servicios a los programas de aplicación de \textit{software}, ejecutándose en modo privilegiado respecto de los restantes.


Este es un elemento importante, ya que debe cumplir con las características de estabilidad, velocidad, seguridad y escalabilidad para soportar la instalación del sistema.


A continuación se presentan diferentes sistemas operativos que cumplen con las características mencionadas y que son suficientes para albergar el sistema:

\begin{itemize}
    \item Windows: es un grupo de varias familias de sistemas operativos gráficos patentados, son desarrollados y comercializados por Microsoft; cada familia atiende a un determinado sector de la industria informática\cite{wilson_about_2015}.
    \item GNU/Linux: es una familia de sistemas operativos tipo Unix de código abierto basados en el núcleo de Linux creado por Linus Torvalds\cite{love_linux_2010}.
\end{itemize}

El sistema operativo elegido para el desarrollo de la propuesta de solución es GNU/Linux porque es de libre acceso, es gratuito y los integrantes del equipo tienen experencia con este sistema.


\subsubsection*{Lenguaje de desarrollo}

De acuerdo con Aaby\cite{aaby_introduction_1996}, un lenguaje de programación es un lenguaje formal (es decir, un lenguaje con reglas gramaticales definidas) que le proporciona a una persona, en este caso el programador, la capacidad de escribir una serie de instrucciones o secuencias de órdenes en forma de algoritmos con el fin de controlar el comportamiento físico o lógico de una computadora, de manera que es posible obtener diversas clases de datos o ejecutar determinadas tareas.


Se valora que el lenguaje de programación para el desarrollo de la propuesta de solución debe tener soporte para conexión a base de datos, sea posible usarlo para el desarrollo, sea vigente (que esté en continua mejora), sea fácil de administrar y se integre con algún \textit{framework} web.


A continuación se presenta una lista de lenguajes de desarrollo que cumplen dichas características:

\begin{itemize}
    \item Java: es un lenguaje de programación de propósito general que está basado en clases, orientado a objetos y diseñado para tener la menor cantidad posible de dependencias de implementación; su objetivo es permitir seguir el  \textit{write once, run anywhere}\cite{joy_java_2000}.
    \item Python: es un lenguaje de programación interpretado, de alto nivel y de propósito general; la filosofía de diseño de Python enfatiza la legibilidad del código con su uso notable de espacios en blanco significativos; Sus construcciones de lenguaje y su enfoque orientado a objetos tienen como objetivo ayudar a los programadores a escribir código claro y lógico para proyectos de pequeña y gran escala\cite{van_rossum_python_2007}.
    \item C\#: es un lenguaje de programación multiparadigma de propósito general como imperativo, declarativo, funcional, genérico, orientado a objetos y orientado a componentes\cite{hejlsberg_c_2003}.
    \item JavaScript: es un lenguaje de programación interpretado, de alto nivel y de propósito general; se escribe dinámicamente; es compatible con múltiples paradigmas de programación, incluida la programación orientada a objetos\cite{noauthor_javascript_nodate}.
\end{itemize}

Los lenguajes de programación Python y JavaScript se han elegido para desarrollar la propuesta de solución; para más detalles revise la sección~\ref{ref:conclusiones-cap3}

\subsubsection*{Sistema gestor de base de datos}

De acuerdo con Connolly\cite{connolly_database_2005}, es un sistema de \textit{software} que permite a los usuarios definir, crear, mantener y controlar el acceso a la base de datos.


Este es un factor muy importante, ya que determinará cómo se almacenará la información del sistema, por lo tanto debe ser escalable, seguro, contar con soporte para grandes cantidades de información y soporte para conexión con distintos lenguajes de programación.


A continuación se presenta una lista de sistemas gestores de bases de datos que cumplen dichas características:

\begin{itemize}
    \item MySQL: es un sistema de gestión de bases de datos relacionales de código abierto\cite{dubois_mysql_1999}.
    \item MongoDB: es un gestor de bases de datos orientado a documentos y utiliza documentos similares a JSON con esquemas opcionales\cite{banker_mongodb_2011}.
\end{itemize}

El gestor de base de datos que se ha elegido para desarrollar la propuesta de solución es MongoDB; para más detalles revise la sección~\ref{ref:databases}.


La tabla~\ref{tab:hw_devices} muestra las características de las computadoras con los que el equipo de desarrollo dispone; la primera columna hace referencia al número de computadora del que se dispone; la siguiente hace referencia a qué elementos contiene, la tercera columna sus especificaciones y la cuarta columna el costo de cada computadora.

\begin{table}[H]
    \centering
    \begin{tabular}{|c|c|c|c|}
        \hline
        Equipos & Elementos & Especificaciones & Costo \\ \hline
        \multirow{3}{*}{Laptop 1} & Memoria RAM & 8 GB & \\
        & Almacenamiento & 500 GB HDD & 22 500.00 mxn  \\
        & Procesador & Intel Core i5 6ta gen. & \\ \hline
        \multirow{3}{*}{Laptop 2} & Memoria RAM & 8 GB & \\
        & Almacenamiento & 256 GB SSD & 32 000.00 mxn \\
        & Procesador & Intel Core i5 8va gen. & \\ \hline
    \end{tabular}
    \caption{Computadoras con las que se cuenta}
    \label{tab:hw_devices}
\end{table}


Con los datos que están en la tabla~\ref{tab:hw_devices}, se concluye que la tecnología para el desarrollo del sistema existe y se cuenta con los recursos de \textit{hardware} suficientes para iniciar con su implementación.
\subsection{Factibilidad económica}

De acuerdo con Pressman\cite{pressman_software_2005}, el punto de función es una ``unidad de medida'' para expresar la cantidad de funcionalidad comercial que un sistema de información (como producto) proporciona a un usuario; los puntos de función se utilizan para calcular una medición de tamaño funcional (FSM) de \textit{software}.

Como señala Pressman en su libro, se utiliza una métrica por puntos de función para realizar una estimación del costo total del proyecto, incluyendo los salarios de los desarrolladores que harán la implementación del sistema, así como los gastos por pagos de servicios que sean necesarios como se muestra en la tabla \ref{tab:function_point_metrics}. 

\subsubsection*{Métricas orientadas a la función}


Para este proyecto, se considera que todas las funciones identificadas son de complejidad media con excepción de las entradas que tienen la complejidad más alta del sistema.


\begin{table}[H]
	\centering
	\begin{tabular}{|l|l|l|l|l|l|}
	\hline
	\multirow{2}{*}{Parámetro} & \multirow{2}{*}{Cuenta} & \multicolumn{3}{|l|}{Factores de ponderación} & \multirow{2}{*}{Total} \\ \cline{3-5}
														 &                         & baja       	& media       & alta		       &                        \\ \hline
	Entradas                   & 6                       & 3            & 4           & 6              & 36                     \\ \hline
	Salidas                     & 5                       & 4            & 5           & 7              & 25                     \\ \hline
	Tablas                     & 1                       & 3            & 4           & 6              & 4                      \\ \hline
	Interfaces                 & 4                       & 7            & 10          & 15             & 40                     \\ \hline
	Consultas                  & 4                       & 5            & 7           & 10             & 28                     \\ \hline
	Conteo total               &                         &              &             &		             & 133                    \\ \hline
	\end{tabular}
	\caption{Cálculo de las métricas por puntos de función}
	\label{tab:function_point_metrics}
	\end{table}


	\textbf{Fi (i = 1..14)} son factores de ajuste de valor basados en las respuetas de las preguntas de la tabla \ref{tab:questions_adjusment}. Los valores pueden ir de 0 (no importante o aplicable) a 5 (absolutamente esencial).

	\begin{table}
		\begin{tabular}{|p{9cm}|c|}
		\hline
		Pregunta                                                                                                                 & Ponderación \\ \hline
		¿Requiere el sistema métodos de seguridad y recuperación fiables?                                                       & 3           \\ \hline
		¿Se requiere comunicación especializada?                                                                              & 5           \\ \hline
		¿Existen funciones de procesamiento distribuido?                                                                         & 2           \\ \hline
		¿Es crítico el rendimiento?                                                                                              & 4           \\ \hline
		¿Se ejecutará el sistema en un entorno operativo existente y fuertemente utilizado?                                      & 4           \\ \hline
		¿Requiere el sistema una entrada de datos interactiva?                                                                    & 5           \\ \hline
		¿Requiere la entrada de datos interactiva que las transacciones de entrada se lleven a cabo sobre múltiples operaciones? & 5           \\ \hline
		¿Se actualizan los archivos maestros de forma interactiva?                                                               & 3           \\ \hline
		¿Son complejas las entradas, salidas, archivos o consultas?                                                               & 4           \\ \hline
		¿Es complejo el procesamiento interno?                                                                                   & 4           \\ \hline
		¿Se ha diseñado el código para ser reutilizable?                                                                         & 4           \\ \hline
		¿Están incluidas en el diseño la instalación y conversión?                                                               & 3           \\ \hline
		¿Se ha diseñado el sistema para soportar múltiples instalaciones en diferentes organizaciones?                           & 4           \\ \hline
		¿Se ha diseñado el sistema para facilitar los cambios y para ser fácilmente utilizable?                                  & 4           \\ \hline
		\centering $\sum Fi=$                                                                                                  	 & 54          \\ \hline
		\end{tabular}
		\caption{Factores de ajuste}
		\label{tab:questions_adjusment}
		\end{table}


\subsubsection*{Puntos de función}

La fórmula para obtener los puntos de función con los factores de ajuste es la siguiente:

\begin{equation} \label{eq:cap4-00.01}
	\mathrm{PF} = \mathrm{conteo\ total} *  (0.65 + (0.01 * \sum F_i)) 
\end{equation}
	
De \eqref{eq:cap4-00.01} se deben sustituir los valores del conteo total y los factores de ajuste:

\begin{equation} \label{eq:cap4-00.02}
	\mathrm{PF} =  133 *  (0.65 + (0.01 * 54))
\end{equation}
\begin{equation} \label{eq:cap4-00.03}
	\mathrm{PF} = 158.27 \approx 159
\end{equation}
	
		
De lo anterior, aproximadamente se obtienen \textbf{159} puntos de función; una vez obtenidos utilizando la llamada ``Ball-Park'' o ``Estimación Indicativa'', que es la técnica de macro-estimación que se utiliza en situaciones de falta de información sobre el proyecto.


De acuerdo con Carper\cite{abran_applied_2006}, la siguiente ecuación determina el esfuerzo de desarrollo de un proyecto:


\begin{equation} \label{eq:cap4-01}
	\mathrm{Esfuerzo} = (\frac{\mathrm{PF}}{150})*\mathrm{PF}^{0.4} 
\end{equation}
donde \eqref{eq:cap4-01} $\mathrm{PF}$ son puntos de función; al sustituir los valores en \eqref{eq:cap4-01}:

\begin{equation} \label{eq:cap4-02}
	\mathrm{Esfuerzo} = (\frac{159}{150})*159^{0.4}
\end{equation}
\begin{equation} \label{eq:cap4-03}
	\mathrm{Esfuerzo} = 8.05 \; \mathrm{meses}
\end{equation}
	



Los 8.05 meses de \eqref{eq:cap4-03}; considerando un total de 40 horas a la semana de trabajo y 4.34 semanas por mes, el total de horas para el desarrollo y conclusión del proyecto se obtiene de esta manera:

\begin{equation} \label{eq:cap4-04}
	\mathrm{Tiempo\; de\; desarrollo} =  40 * 4.34 * 8.05 
\end{equation}

\begin{equation} \label{eq:cap4-05}
	\mathrm{Tiempo\; de\; desarrollo} =  1397.48 \approx 1398\;  \mathrm{horas}
\end{equation}

	



Por ejemplo, una sola persona trabajando en el desarrollo del proyecto debería invertir 1398 horas con una jornada de 8 horas diarias de lunes a viernes hasta su finalización, por lo que si un equipo de desarrollo es de 2 personas con un horario de lunes a viernes de 4 horas diarias, el proyecto concluiría en 8.05 meses.


\subsection{Costos de desarrollo}

De acuerdo con \textit{Software Guru}\cite{pedro_galvan_estudio_2020}, en una publicación que recopila los datos de salarios en el área de desarrollo de \textit{software} para febrero de 2020, un desarrollador con 0 a 2 años de experiencia, como es el caso de un estudiante, tiene en un salario de \$ 15 000 mensuales en una jornada completa, considerando que este proyecto contempla jornadas de medio tiempo (4 horas) de lunes a viernes, se reduce la cifra antes mencionada. Teniendo en cuenta estos datos y un periodo de 9 meses, que es el tiempo aproximado de duración del proyecto, el costo total por salarioss para el equipo de desarrollo está desglosado en la tabla \ref{tab:devs_salary}.

\begin{table}[H]
    \centering
    \begin{tabular}{|c|c|c|c|}
    \hline
        Concepto & Costo Aprox. Semanal & Costo Aprox. Mensual & Monto total \\ \hline
        Salario & 1850 & 7500 & 67500 \\ \hline
    \end{tabular}
    \caption{Costos del personal}
    \label{tab:devs_salary}
\end{table}


Considerado a 2 personas en el equipo de desarrollo y un periodo de 9 meses(se agrego un mes mas para el caso de estudio) utilizando los salarios de la tabla \ref{tab:devs_salary} tenemos que los gastos totales se obtienen con la siguiente fórmula:

\begin{equation} \label{eq:cap4-07}
	\mathrm{salarios} =  \mathrm{No\; de\; integrantes} * \mathrm{salario/mes} * \mathrm{tiempo\; de\; desarrollo}  {pesos mexicanos}
\end{equation}

sustituyendo los calores en \eqref{eq:cap4-07}

\begin{equation} \label{eq:cap4-08}
	\mathrm{salarios} =  2 * 7500 * 9  {pesos mexicanos}
\end{equation}


Esto da como resultado un total final de \$ 135 000 por los salarios de los 2 integrantes del equipo de desarrollo. Se tomaron en cuenta 9 meses para todos los gastos, un mes extra a lo obtenido en estimación para utilizarse en el caso de estudio del sistema una vez concluido.

Los gastos por pagos de licencia de \textit{software} quedan excluidos, ya que las tecnologías seleccionadas son libres o gratuitas, lo cual no supone un costo para su uso. De igual manera, esto se encuentra simplificado en la tabla \ref{tab:sw_licences}.

\begin{table}
    \centering
    \begin{tabular}{|c|c|c|c|}
    \hline
        Software & Licencia & Costo \\ \hline
        Visual Studio Code  & MIT & 0  \\ \hline
        Gunicorn(flask Server) & MIT & 0 \\ \hline
        MongoDB Atlas & Apache v2 & 0 \\ \hline
    \end{tabular}
    \caption{Costos por licencias de software}
    \label{tab:sw_licences}
\end{table}

Otros gastos necesarios son los pagos por servicios requeridos listados en la tabla \ref{tab:services_costs}.

\begin{table}
    \centering
    \begin{tabular}{|c|c|c|c|}
    \hline
        Concepto & Costo Mensual & Monto total \\ \hline
        Luz & 250 & 2250  \\ \hline
        Internet & 349 & 3141 \\ \hline
        Heroku hosting & 0 (free plan) & 0 \\ \hline
        Netlify hosting & 0 (free plan) & 0 \\ \hline
    \end{tabular}
    \caption{Costos por servicios}
    \label{tab:services_costs}
\end{table}

Habiendo realizado la suma de todas las cantidades antes mencionadas, el total final se obtine de la siguiente manera:

\begin{equation} \label{eq:cap4-09}
    \mathrm{servicio_totales} = \mathrm{servicios_por_mes} * \mathrm{tiempo_de_desarrollo} {pesos mexicanos}
    \mathrm{servicio_totales} = 5391 * 9 {pesos mexicanos}
    \mathrm{servicio_totales} = 48 519.00 {pesos mexicanos}
\end{equation}

\begin{equation} \label{eq:cap4-10}
    \mathrm{Gastos\; totales} = \mathrm{salarios} + \mathrm{servicios\; totales} + \mathrm{costos_de_equipo_de_compumto}
    \mathrm{Gastos\; totales} = 135 000 + 48 519 + 54 500
    \mathrm{Gastos\; totales} = 238 019 {pesos mexicanos}
\end{equation}

\begin{center}

	salarios = 135 000.00
	servicios = (2250 + 3141) * 9 = 48 519.00

	Gastos totales = 135 000.00 + 48 519.00 = 183 519.00 pesos mexicanos.

\end{center}



%\subsection{Factibilidad operativa}
\subsection{Conclusiones}


La factibilidad operativa permite predecir si es posible poner en marcha el sistema propuesto, aprovechando todos los beneficios que se ofrecen a todos los usuarios involucrados en ello. La herramienta va dirigida a los estudiantes que se encuentren en un primer acercamiento a los modelos de bases de datos entidad-relación o relacional desde un enfoque conceptual, buscando principalmente mostrarles una aproximación a los modelos no relacionales de bases de datos. El sistema propuesto cuenta con una interfaz intuitiva para que el usuario final, los estudiantes, puedan visualizar, crear y editar un diagrama ER y las opciones que esta les brinde de manera comprensible.


Teniendo en cuenta los motivos anteriormente explicados, el sistema propuesto tiene una alta probabilidad de aceptación por parte de los usuarios finales al encontrarse en un entorno en el que se trabaja con \textit{software} continuamente, además del beneficio que aporta al plan de estudios actual al ofrecer una forma práctica de ver aplicado los conceptos adquiridos en el curso de base de datos, el cual solo contempla un alcance hasta la normalización de bases de datos relacionales y tener una introducción a los modelos no relacionales (noSQL). Un estudiante que ha cursado dicha asignatura se dará cuenta que el tiempo disponible durante el curso es limitado por la cantidad de módulos que pretende cubrir y en muchas ocasiones los docentes deben prescindir de ciertos temas para completar el temario.

Con la implantación de la aplicación web que se está proponiendo, los estudiantes que cursen la asignatura de de base de datos tendrán la oportunidad de conocer una opción más en cuanto a tecnologías de almacenamiento de datos para implementar en sus propios sistemas. De igual manera, puede impulsarlos a generar propuestas para la apertura de una asignatura optativa si el interes por estos modelos de datos resulta interesante para ellos.

Teniendo en cuenta los puntos mencionados anteriormente, se concluye que el sistema propuesto tendrá un uso en la institución y un potencial beneficio para los estudiantes y los involucrados en ello.

\begin{enumerate}
    \item ¿El sistema contribuye a los objetivos generales de la organización? la respuesta es si, ya que la organización centra su misión en formar profesionales en ingeniería, tecnologías y ciencias de la computación, para lograrlo de mantenerse a actualizado en las tecnologías emergentes y ofrecer a sus estudiantes un mayor abanico de opciones para su desarrollo profesional.
    \item ¿Se puede implementar el sistema dentro del cronograma y el presupuesto utilizando las tecnologías actuales? claro que es posible como se muestra en la factibilidad tecnica se cuenta con las tecnologías para el desarrollo del producto y el esfuerzo es ajjstado para el equipo de desarrollo pero queda en los limites del tiempo establecido en el cronograma, ademas de ser desarrollado con una metodología agíl lo que ofrece productos funcionales por cada iteración.
    \item ¿Se puede integrar el sistema con otros sistemas que se utilizan? esto es perfectamente viable, el sistema puede ser adaptado a otros sistemas por la facilidad de estar disponible en la web tiene una gran opoprtinidad para comunicarse con otros sistemas disponibles en la organización, por ejemplo puede integrarse directamente en la asignatura de base de datos como una opción mas para el modelado de diagramas ER con la ventaja de tener el acercamiento a los modelos no relacionales de manera practica.
\end{enumerate}
\section{Análisis del sistema}
De acuerdo con Pressman\cite{pressman_software_2005}, las condiciones del mercado cambian con rapidez, clientes y usuarios finales necesitan cambios constantes por nuevas amenazas competitivas; por ello los profesionales deben enfocar la ingeniería de \textit{software} en forma que les permita mantenerse ágiles para definir procesos maniobrables, adaptativos y esbeltos que satisfagan las necesidades de los negocios modernos.


Una filosofía ágil para la ingeniería de \textit{software} pone el énfasis en cuatro aspectos clave: la importancia de los equipos con organización propia que tienen el control sobre el trabajo que realizan, la comunicación y colaboración entre los miembros del equipo, profesionales y sus clientes, el reconocimiento de que el cambio representa una oportunidad y la insistencia en la entrega rápida de \textit{software} que satisfaga al consumidor.


\subsection{Historias de usuario}\label{sec:historias-usuario}

De acuerdo con Scrum México\cite{scrum_mexico_scrum_2020}, las historias de usuario conforman la técnica por la que el usuario especifica de manera general los requerimientos que el sistema debe cumplir.


Normalmente estas redacciones se llevan a cabo en tarjetas de papel donde se describen brevemente las funciones que el producto final debe poseer, ya sean requisitos funcionales o no.


El tratamiento de las historias de usuario es flexible y dinámico, cada una de ellas es lo suficientemente detallada y delimitada para que el equipo de desarrollo implemente durante la duración del \textit{sprint}.


Es habitual que se siga una plantilla para estas tarjetas, como la que se expone a continuación:

\begin{itemize}
	\item Como \textbf{<Usuario>}
	\item Quiero \textbf{<algún objetivo>}
	\item Para \textbf{<motivo>}
\end{itemize}


Una de sus grandes ventajas, dado el caso de que un usuario no sea lo suficientemente detallista con la historia, es que esta se puede partir en historias más pequeñas antes de que el equipo empiece a trabajar en ella.


Este es un ejemplo de historia de usuario para el desarrollo:

\begin{itemize}
	\item Como usuario,
	\item quiero ingresar al sistema con mi correo y contraseña
	\item para tener acceso a sus funciones.
\end{itemize}


Otra forma de darle detalle a las historias de usuario es mediante el añadido de un criterio de aceptación; un criterio de aceptación es una prueba que será cierta cuando el equipo de desarrollo complete la descripción de la tarjeta.


A continuación se listan las principales historias de usuario que se consideraron para el desarrollo de la propuesta de solución; tenga en cuenta que algunas de ellas tienen criterios de aceptación, pero en otras no se consideró necesarias porque son explícitamente claras.


\noindent\rule{\textwidth}{1pt}
\begin{itemize}
	\item N.° 1
	\item Como usuario,
	\item quiero ingresar al sistema con mi correo y contraseña,
	\item para tener acceso a sus funciones.
	\item Criterios de aceptación:
	\begin{itemize}
		\item El usuario recibirá un correo electrónico de confirmación de su alta en el sistema con el correo y contraseña que ingresó para tenerlos de respaldo.
	\end{itemize}
\end{itemize}
\noindent\rule{\textwidth}{1pt}
\begin{itemize}
	\item N.° 2
	\item Como usuario,
	\item quiero recuperar mi contraseña en caso de olvidarla,
	\item para no perder el trabajo realizado en el sistema.
	\item Criterios de aceptación:
	\begin{itemize}
		\item El usuario podrá ingresar una nueva contraseña siempre y cuando recuerde el correo electrónico con el que se dio de alta en el sistema.
		\item Al ingresar una nueva contraseña, recibirá un correo de confirmación del cambio de contraseña y sus datos permanecerán intactos.
	\end{itemize}
\end{itemize}
\noindent\rule{\textwidth}{1pt}
\begin{itemize}
	\item N.° 3
	\item Como usuario del sistema, quiero darme de alta con una contraseña fácil de recordar,
	\item pero que esté segura en la base de datos,
	\item para no tener comprometidos los diagramas que genere en el sistema.

	\item Criterios de aceptación:
	\begin{itemize}
		\item Asegurarse que el usuario ingrese una contraseña de al menos 8 caracteres.
		\item Se le solicitará al usuario que ingrese 2 veces la misma contraseña para asegurarse que le es fácil recordarla y que efectivamente es la misma.
		\item Antes de guardar la contraseña, esta deberá pasar por un método que la haga ilegible para el usuario (algún algoritmo de digestión o cifrado).
	\end{itemize}
\end{itemize}
\noindent\rule{\textwidth}{1pt}
\begin{itemize}
	\item N.° 4
	\item Como usuario quiero crear un diagrama ER arrastrando y soltando elementos de una ``paleta'',
	\item para hacerlo de manera más fácil e intuitiva.
	\item Criterios de aceptación:
	\begin{itemize}
		\item El usuario podrá empezar un nuevo diagrama al seleccionar la opción de diagramador ER.
		\item Tendrá a su disposición una paleta con los elementos permitidos en un diagrama ER básico.
		\item Podrá arrastrar y soltar los elementos de la paleta a un área delimitada para empezar con el diseño de su diagrama.
	\end{itemize}
\end{itemize}
\noindent\rule{\textwidth}{1pt}
\begin{itemize}
	\item N.° 5
	\item Como usuario quiero guardar mi último trabajo realizado en el diagramador ER,
	\item para poder consultarlo en otro momento.
	\item Criterios de aceptación:
	\begin{itemize}
		\item Dispondrá de un botón para poder guardar en la base de datos el diagrama que esté creando o editando.
		\item Antes de almacenar el diagrama en el \textit{canvas} o zona de diagramado, se le mostrará un mensaje de confirmación para guardar su diagrama actual.
	\end{itemize}
\end{itemize}
\noindent\rule{\textwidth}{1pt}
\begin{itemize}
	\item N.° 6
	\item Como usuario me gustaría poder ver el último trabajo que realice
	\item cuando seleccione la opción ``Entidad-Relación'',
	\item para poder modificar el diseño.
\end{itemize}
\noindent\rule{\textwidth}{1pt}
\begin{itemize}
	\item N.° 7
	\item Como usuario quiero tener la opción de cargar un diagrama a partir de un archivo,
	\item para hacer modificación de dicho diagrama y guardarlo de ser necesario.
	\item Criterios de aceptación:
	\begin{itemize}
		\item El usuario tendra un botón ``cargar'' en el menú del diagramador ER para poder importar un archivo con extensión .json.
		\item Al importar el archivo, este pasará por un proceso de validación para asegurarse que es un archivo .json válido.
		\item Durante el proceso de validación, se verificará que el contenido del archivo es un diagrama compatible con la estructura de los generados por el diagramador ER.
		\item Al contener información compatible, se mostrará en la zona de diagramado el contenido del archivo.
	\end{itemize}
\end{itemize}
\noindent\rule{\textwidth}{1pt}
\begin{itemize}
	\item N.° 8
	\item Como usuario quiero descargar el diagrama que esté visible en la página web,
	\item para poder distribuirlo como yo desee.
	\item Criterios de aceptación:
	\begin{itemize}
		\item El usuario dispondrá de un botón ``Descargar'' en el diagramador ER para obtener un archivo con el contenido del diagrama visible en la zona de diagramado.
		\item El archivo generado será de extensión .json con la información necesaria para que el diagramador lo cargue en otro momento.
	\end{itemize}
\end{itemize}
\noindent\rule{\textwidth}{1pt}
\begin{itemize}
	\item N.° 9
	\item Como usuario quiero tener una forma de validar mi diagrama ER,
	\item porque es importante saber si el diagrama que estoy creando es un diagrama válido del modelo ER.
	\item Criterios de aceptación:
	\begin{itemize}
		\item El usuario tendrá disponible un botón que al darle clic iniciará un proceso de validación del diagrama actual en la zona de diagramado.
		\item Al término del proceso de validación, se le mostrará un mensaje al usuario indicando si el diagrama cumple o no las reglas del modelo ER.
	\end{itemize}
\end{itemize}
\noindent\rule{\textwidth}{1pt}
\begin{itemize}
	\item N.° 10
	\item Como usuario, en caso de tener un diagrama ER válido,
	\item me gustaría poder tranformar el diagrama ER en su versión del modelo relacional,
	\item para poder ver el equivalente del diagrama ER en el modelo Relacional.
	\item Criterios de aceptación:
	\begin{itemize}
		\item El usuario dispondrá de la opción de tranformar al modelo relacional solamente después de haber validado que el diagrama ER cumple con las reglas.
		\item Posterior a la validación, se le mostrará al usuario un mensaje de confirmación y un botón para disparar el proceso de transformación a su equivalente relacional.
		\item Al terminar el proceso de transformación equivalente, se le redirijira al menú ``Relacional'' donde podrá visualizar el equivalente al modelo relacional.
	\end{itemize}
\end{itemize}
\noindent\rule{\textwidth}{1pt}
\begin{itemize}
	\item N.° 11
	\item Como usuario, después de observar el diagrama relacional,
	\item quiero obtener las sentencias SQL equivalentes,
	\item para poder crear el esquema de base de datos relacional en un DBMS.
	\item Criterios de aceptación:
	\begin{itemize}
		\item Las sentencias SQL solo podrán ser descargadas en el menú ``Relacional'' a un archivo con extension .sql dando clic a un botón con la leyenda ``Descargar SQL''.
		\item Solo se obtendrán las sentencias SQL de un diagrama ER creado y/o validado por el sistema.
	\end{itemize}
\end{itemize}
%\noindent\rule{\textwidth}{1pt}
%\begin{itemize}
	%\item N.° 12
	%\item Como usuario, una vez observado el equivalente relacional del diagrama ER,
	%\item quiero iniciar el proceso de transformación al modelo no relacional
	%\item para poder observar el cambio entre modelos.
	%\item Criterios de aceptación:
	%\begin{itemize}
		%\item al dar clic al botón “Transformar a NoSQl”, el usuario iniciará el proceso para obtener el equivalente del modelo relacional al modelo NR.
		%\item al término del proceso de transformación, se le redirigirá al menú ``No Relacional'' donde observará el modelo NoSQL equivalente a su diagrama ER.
%	\end{itemize}
%\end{itemize}
\noindent\rule{\textwidth}{1pt}
\begin{itemize}
	\item N.° 12
	\item Como usuario quiero transformar mi diagrama ER en su equivalente modelo conceptual NoSQL,
	\item para poder observar el cambio entre modelos.
	\item Criterios de aceptación:
	\begin{itemize}
		\item El usuario dispondrá de la opción de tranformar al modelo NoSQL solamente después de haber validado que el diagrama ER cumple con las reglas.
		\item Posterior a la validación, se le mostrará al usuario un mensaje de confirmación y un botón para disparar el proceso de transformación al modelo conceptual NoSQL.
		\item Una vez validado el diagrama ER e iniciado el proceso para la transformación al modelo conceptual NoSQL, se le indicará al usuario que el proceso tardará un tiempo.
		\item Al término del proceso de transformación, se le redirigirá al menú ``No Relacional'' donde observará el modelo conceptual NoSQL equivalente a su diagrama ER.
	\end{itemize}
\end{itemize}
\noindent\rule{\textwidth}{1pt}
\begin{itemize}
	\item N.° 13
	\item Como usuario quiero obtener el \textit{script} desde el modelo conceptual NoSQL,
	\item para poder generar la base de datos en un gestor de base de datos NoSQL orientado a documentos.
	\item Criterios de aceptación:
	\begin{itemize}
		\item El usuario dispondrá de la opción de obtener el \textit{script} solamente después de haber validado que el diagrama ER cumple con las reglas.
		\item Posterior a la validación, se le mostrará al usuario un mensaje de confirmación y un botón para disparar el proceso de generación de \textit{scripts} para el gestor de base de datos orientado a documentos.
		\item Una vez empezado el proceso de generación de \textit{scripts}, se le indicará al usuario que el proceso tardará un tiempo.
		\item Al término del proceso de transformación, se le redirigirá al menú ``No Relacional'' donde observará los \textit{scripts} NoSQL.
	\end{itemize}
\end{itemize}

\noindent\rule{\textwidth}{1pt}
\begin{itemize}
	\item N.° 14
	\item Como usuario me gustaría tener un reporte técnico y
	\item quiero que la redacción sea legible y referenciada,
	\item para compartirlo en el futuro con equipos de desarrollo y ver la posibilidad de agregar nuevas funciones al sistema.
\end{itemize}
\noindent\rule{\textwidth}{1pt}




Teniendo en cuenta que se está trabajando con una metodología ágil, estas historias de usuario pueden aumentar o en su defecto dividirse en historias más pequeñas dependiendo de los criterios del equipo de desarrollo durante el proceso de la implementación de cada historia.




\subsection{Lista de producto}

De acuerdo a Trigas Gallego\cite{manuel_trigas_gallego_metodologiscrum_2020}, la lista de producto es una lista ordenada de todo lo que sería necesario en el producto y es la fuente de requisitos para cualquier cambio a realizarse en el mismo; enumera las características, funcionalidades, requisitos, mejoras y correcciones que constituyen cambios a realizarse en el producto para entregas futuras.

La tabla~\ref{tab:lista-producto} muestra la lista de producto para el proyecto; muestra el número de historia, la tarea a realizar, así como su encargado.


\begin{longtable}{ p{2cm} | p{10cm} | p{2cm} }
	\hline
	N.° de Historia de Usuario & Requerimiento/Tarea & Responsable \\[0.5cm]
	\hline
	\hline

	\endfirsthead

	\multicolumn{3}{c}{Continuación de tabla de lista de producto }\\
	\hline
	\hline
	\endhead

	\hline
	\hline
	\caption{Lista de producto}
	\endlastfoot

	% template row table
	% \centering & & & \\[0.5cm]
	% \hline

	\centering 14 & Investigación de bases de datos relacionales. & Eduardo/Omar \\[0.5cm]
	\hline
	\centering 14 & Redacción y selección de las tecnologías a utilizar para el desarrollo de la plataforma.  & Eduardo \\[0.5cm]
	\hline
	\centering 14 & Investigación de bases de datos relacionales.  & Eduardo/Omar \\[0.5cm]
	\hline
	\centering 14 & Redacción de bases de datos relacionales en el documento técnico.  & Eduardo \\[0.5cm]
	\hline
	\centering 14 & Investigación de bases de datos no relacionales.  & Eduardo/Omar \\[0.5cm]
	\hline
	\centering 14 & Redacción de bases de datos no relacionales en el documento técnico.  & Eduardo \\[0.5cm]
	\hline
	\centering 14 & Investigación y selección del modelo de base de datos no relacional a utilizar junto a las tecnologías a utilizar.  & Eduardo/Omar \\[0.5cm]
	\hline
	\centering 14 & Análisis y diseño de la aquitectura web.  & Eduardo/Omar \\[0.5cm]
	\hline
	\centering 1 & Desarrollo de la estructura básica del \textit{backend}.  & Omar \\[0.5cm]
	\hline
	\centering 1 & Desarrollo de la estructura básica del \textit{frontend}.  & Eduardo \\[0.5cm]
	\hline
	\centering 1 & Agregar servicio \textit{backend} para registrar un usuario. & Omar \\[0.5cm]
	\hline
	\centering 1 & Agregar formulario para captura de datos de registro de un usuario en el \textit{frontend}. & Eduardo \\[0.5cm]
	\hline
	\centering 2 & Agregar servicio \textit{backend} para recuperar contraseña del usuario. & Omar \\[0.5cm]
	\hline
	\centering 2 & Agregar servicio \textit{backend} para envío de correo al usuario registrado y de recuperación de contraseña. & Omar \\[0.5cm]
	\hline
	\centering 2 & Agregar vista con formulario para recuperación de contraseña del usuario en el \textit{frontend}. & Eduardo \\[0.5cm]
	\hline
	\centering 2 & Integración de los servicios de registro y recuperación de contraseña en el \textit{frontend}. & Eduardo \\[0.5cm]
	\hline
	\centering 3 & Agregar servicio \textit{backend} para hacer ilegible la contraseña del usuario en la base de datos. & Omar \\[0.5cm]
	\hline
	\centering 4 & Planteamiento de escenarios de los esquemas entidad-relación.  & Eduardo \\[0.5cm]
	\hline
	\centering 4 & Agregar a la interfaz gráfica de la aplicación web el menú ``Entidad-Relación''. & Eduardo \\[0.5cm]
	\hline
	\centering 4 & Agregar íconos de los elementos basicos de un diagrama ER en el diagramador. & Eduardo \\[0.5cm]
	\hline
	\centering 5 & Agregar servicio \textit{backend} para guardar un diagrama ER en formato JSON en la base de datos e integrarlo al \textit{frontend}. & Omar \\[0.5cm]
	\hline
	\centering 5 & Agregar servicio \textit{backend} para recuperar el diagrama guardado del usuario de la base de datos y regresarlo en formato JSON.  & Omar \\[0.5cm]
	\hline
	\centering 6 & Recuperar el último diagrama del usuario del \textit{backend} y monstrarlo en el \textit{frontend}. & Eduardo \\[0.5cm]
	\hline
	\centering 6 & Manejar el estado de la intefaz web para no perder el diagrama ER que está editando el usuario. & Eduardo \\[0.5cm]
	\hline
	\centering 6 & Definición de las reglas del modelo entidad-relación. & Eduardo \\[0.5cm]
	\hline
	\centering 4 & Implementar la edición de diagramas ER en el \textit{frontend}.  & Eduardo \\[0.5cm]
	\hline
	\centering 7 & Habilitar la carga de un archivo en la aplicación web.  & Omar \\[0.5cm]
	\hline
	\centering 7 & Agregar la función para validar el contenido del archivo .json y pintarlo en la zona de diagramado. & Eduardo \\[0.5cm]
	\hline
	\centering 8 & Agregar la descarga del diagrama visible en la zona de diagramado a un archivo .json. & Eduardo \\[0.5cm]
	\hline
	\centering 9 & Agregar botón de validar al \textit{frontend} y mostrar el \textit{loader} mientras se procesa el diagrama ER. & Eduardo/Omar \\[0.5cm]
	\hline
	\centering 9 & Agregar servicio \textit{backend} para la validación del diagrama entidad-relación. & Eduardo/Omar \\[0.5cm]
	\hline
	\centering 9 & implementación de algoritmo para validación del diagrama ER en el \textit{backend}. & Eduardo/Omar \\[0.5cm]
	\hline
	\centering 9 & Pruebas de captura de distintos diagramas entidad-relación.  & Eduardo/Omar \\[0.5cm]
	\hline
	\centering 9 & Pruebas para validar el algoritmo de validación. & Eduardo/Omar \\[0.5cm]
	\hline
	\centering 10 & Agregar servicio al \textit{backend} para transformación del esquema entidad-relación al modelo relacional.  & Omar \\[0.5cm]
	\hline
	\centering 10 & Implementación del algoritmo de transformación ER -> relacional & Eduardo/Omar \\[0.5cm]
	\hline
	\centering 10 & Agregar menú relacional a la intefaz gráfica. & Eduardo/Omar \\[0.5cm]
	\hline
	\centering 10 & Prueba de transformación de distintos diagramas ER al modelo relacional. & Eduardo/Omar \\[0.5cm]
	\hline
	\centering 10 & Visualización de la transformación del modelo ER al modelo relacional. & Eduardo \\[0.5cm]
	\hline
	\centering 14 & Revision de la redacción del reporte técnico para presentación de TT1  & Eduardo \\[0.5cm]
	\hline
	\centering 11 & Agregar servicio \textit{backend} para la descarga del archivo .sql con las sentencias equivalentes. & Eduardo/Omar \\[0.5cm]
	\hline
	\centering 11 & Pruebas de coherencia de las sentencias equivalentes en el DBMS. & Eduardo/Omar \\[0.5cm]
	\hline
	\centering 12 & Definición de las reglas de transformación al modelo NoSQL.  & Eduardo/Omar \\[0.5cm]
	\hline
	\centering 12 & Pruebas de distintos escenarios del modelo relacional al modelo NoSQL.  & Eduardo/Omar \\[0.5cm]
	\hline
	\centering 12 & Agregar servicio al \textit{backend} para transformación del esquema relacional al modelo NoSQL.  & Omar \\[0.5cm]
	\hline
	\centering 12 & Agregar servicio al \textit{backend} para guardar el modelo NoSQL en la base de datos.  & Omar \\[0.5cm]
	\hline
	\centering 12 & Agregar menú no relacional a la intefaz gráfica. & Omar \\[0.5cm]
	\hline
	\centering 12 & Implementación del algortimo de transformación de modelo relacional al modelo conceptual NoSQL. & Eduardo \\[0.5cm]
	\hline
	\centering 12 & Comprobación de la coherencia de la transformación entre modelos relacional a no relacional.  & Eduardo/Omar \\[0.5cm]
	\hline
	\centering 13 & Agregar servicio \textit{backend} para transformación del modelo ER al modelo no relacional. & Eduardo/Omar \\[0.5cm]
	\hline
	\centering 13 & Ajustar la interfaz del menú ER para mostrar mensaje de transformación al modelo NoSQL. & Omar \\[0.5cm]
	\hline
	\centering 13 & Manejar el estado del diagrama ER y redireccionar al menú no relacional al terminar la tranformación. & Eduardo \\[0.5cm]
	\hline
	\centering 13 & Pruebas de caso de estudio para verificar la correcta transformación y coherencia de los datos.  & Eduardo/Omar \\[0.5cm]
	\hline
	\centering 14 & Revisión de la redacción del reporte técnico para presentación de TT2 & Eduardo \\[0.5cm]

    \label{tab:lista-producto}
\end{longtable}

Se considera la tabla~\ref{tab:lista-producto} como la lista de producto con las tareas necesarias para cumplir con todas las historias de usuario mencionadas en la sección anterior, considerando que es posible que cambien conforme avancen los \textit{sprints} y así añadir nuevas tareas.
\section{Conclusiones}

La factibilidad operativa permite predecir si es posible poner en marcha el sistema propuesto, aprovechando todos los beneficios que se ofrecen a todos los usuarios involucrados en ello.


La herramienta va dirigida a estudiantes de nivel medio o nivel superior que tengan un primer acercamiento a los modelos de bases de datos entidad-relación o relacional desde un enfoque conceptual, como es el caso en ESCOM, en la asignatura de Base de Datos; el sistema propuesto cuenta con una interfaz intuitiva para que el usuario final, los estudiantes, visualice, crear y editar un diagrama ER y las opciones que esta les brinde de manera comprensible.


Por lo explicado anteriormente, el sistema propuesto tiene una alta probabilidad de aceptación por parte de los usuarios finales al encontrarse en un entorno en el que se trabaja con \textit{software} continuamente, además del beneficio que aporta al plan de estudios actual al ofrecer una forma práctica de ver aplicado los conceptos adquiridos en la asignatura de Bases de Datos, el cual solo contempla un alcance hasta la normalización de bases de datos relacionales y tener una introducción a los modelos no relacionales NoSQL.


Un estudiante que ha cursado dicha asignatura se dará cuenta que el tiempo disponible durante el curso es limitado por la cantidad de módulos que pretende cubrir y en muchas ocasiones los docentes deben prescindir de ciertos temas para completar el temario.


Con la implementación de la propuesta de solución, los estudiantes que cursen la asignatura de Bases de Datos tendrán la oportunidad de conocer una opción más en cuanto a tecnologías de almacenamiento de datos para implementar en sus propios sistemas; de igual manera, los puede impulsar a solicitar la apertura de una asignatura optativa sobre los modelos de datos NoSQL si hay interés por estos temas.


Se concluye que el sistema propuesto tendrá un uso en la institución y un potencial beneficio para los estudiantes y los involucrados.


A continuación están las respuestas (con preguntas incluidas) de la sección~\ref{ref:sec-factibilidad}:


\begin{enumerate}
    \item ¿El sistema contribuye a los objetivos generales de la organización?\\ Sí, ya que la misión en ESCOM es formar profesionales líderes en saberes de ingeniería, tecnología y ciencias de la computación con una visión globalizada; así como contribuir con investigación y desarrollo tecnológico para el crecimiento del país; por lo que la propuesta de solución contribuye directamente a la visión de la ESCOM.
    \item ¿Se puede implementar el sistema dentro del cronograma y el presupuesto utilizando las tecnologías actuales? \\Es posible, tal como se muestra en la sección~\ref{ref:factibilidad-tecnica}, se cuenta con las tecnologías para el desarrollo del producto y el esfuerzo es ajustado para el equipo de desarrollo, pero queda en los límites del tiempo establecido en el cronograma; además de ser desarrollado con una metodología ágil que ofrece productos funcionales por cada iteración.
    \item ¿Se puede integrar el sistema con otros sistemas que se utilizan?\\ Sí, el sistema es integrable a otros sistemas por estar disponible en la web; por ejemplo, puede integrarse directamente en la asignatura de Bases de Datos como una opción más para el modelado de diagramas ER con la ventaja de tener el acercamiento a los modelos no relacionales de manera práctica.
\end{enumerate}


Como se ha mencionado en el documento, se ha considerado Scrum como metodología para desarrollar la propuesta de solución, porque el proyecto requiere de entregas regulares de su avance para realizar modificaciones con ayuda de la retroalimentación constante del cliente, en este caso las directoras del proyecto.


Esto otorga beneficios como poder responder con flexibilidad, adaptación a los requisitos de cliente, estrechar la relación con el mismo y mantener al equipo motivado con pequeñas entregas funcionales del producto; además, tomando en cuenta la experiencia del equipo, esta forma de trabajo permite mostrar avances funcionales en el producto en un periodo de tiempo corto para realizar una evaluación y en caso de ser requerido se sugieran cambios.


De la sección de algoritmos, para la validez estructural de un diagrama entidad-relación se hace uso del trabajo de Dullea~\cite{dullea_analysis_2003} para el análisis de las relaciones unarias y binarias; además, se proponen restricciones para las entidades, atributos y relaciones del modelo entidad-relación básico.


Del mapeo modelo entidad-relación básico a modelo relacional se ha optado por usar la propuesta de Elmasri~\cite{ramez_elmasri_fundamentos_nodate}; para la obtención del esquema SQL se ha decidido parsear el modelo lógico relacional obtenido por el algoritmo anterior y se indica en la sección~\ref{sec:esquema-sql} las reglas básicas para implementar el algoritmo en el DBMS MySQL.


Para el algoritmo de mapeo entre el modelo entidad-relación y el GDM se ha optado por proponer consultas similares al trabajo de Chebotko\cite{chebotko_big_2015} para generar las consultas del GDM; asimismo, las relaciones en el modelo entidad-relación son referencias en el GDM y una consulta válida en el modelo entidad-relación es toda consulta que permita llegar a un atributo de una entidad $Y$ desde una entidad $X$ recorriendo las rutas del diagrama entidad-relación.


Para generar el modelo lógico orientado a documentos a partir del GDM se hace uso del algoritmo propuesto por Alfonso de la Vega, en el que se generan árboles de acceso de las consultas en el GDM para generar los documentos anidados en el modelo lógico orientado a documentos.


Finalmente, para obtener el esquema de sentencias en MongoDB se da una serie de pasos en la sección~\ref{sec:logico-documentos-fisico} para implementar el algoritmo.

\chapter{Diseño del sistema}
De acuerdo con Pressman~\cite{pressman_software_2005}, el objetivo del diseño del \textit{software} es aplicar un conjunto de principios, conceptos y prácticas que llevan al desarrollo de un sistema o producto de alta calidad; la meta del diseño es crear un modelo de \textit{software} que implantará correctamente todos los requerimientos del usuario y causará placer a quienes lo utilicen; los diseñadores del \textit{software} deben elegir entre muchas alternativas de diseño y llegar a la solución que mejor se adapte a las necesidades de
los participantes en el proyecto.


En este capítulo se muestra el diseño del sistema; se ve el funcionamiento de los módulos del sistema, así como el diagrama de clases y el esquema de la base de datos; además, se muestra como el usuario interactúa con la interfaz gráfica.

\section{Diagrama de clases}

De acuerdo con Visual Paradigm~\cite{visual_paradigm_visual_2020}, el diagrama de clases es un modelo conceptual que permite visualizar la estructura de las clases de un sistema, sus atributos, métodos y las relaciones entre objetos.


El diagrama de clases de la propuesta de solución se muestra en la figura~\ref{img:classes_diagram}; en la cual se aprecia la clase dominante \textit{usuario} y las demás clases dependen de esta.


\begin{figure}[H]
  \centering
  \includegraphics[width=0.75\textwidth]{diseño/diagrama_de_clases.png}
  \caption{Diagrama de clases}
  \label{img:classes_diagram}
\end{figure}

Es posible interpretar el significado del diagrama de clases de la figura~\ref{img:classes_diagram} leyendo los puntos de la siguiente manera:

\begin{itemize}
  \item El usuario es la clase principal.
  \item Existe una agregación entre Usuario y DiagramaER.
  \item El ModeloGDM es parte del ModeloLogicoNoSQL; el ModeloGDM puede existir por sí mismo.
  \item El GDM depende del DiagramaER; sin embargo, el DiagramaER no depende del GDM.
  \item La ConsultaDeAccessoSimple depende del Diagrama ER.
  \item El ModeloLogicoRelacional depende del DiagramaRelacional.
\end{itemize}


\section{Diagrama de la base de datos}
Como MongoDB es una base de datos NoSQL orientada a documentos, los datos se almacenan en documentos JSON; teniendo en cuenta el diagrama de clases de la figura~\ref{img:classes_diagram}, se aprecia que las relaciones son por composición de 1 a 1 y la clase dominante es Usuario, es decir, las clases DiagramaER, DiagramaRelacional y DiagramaNoSQL no pueden existir si no existe una clase Usuario asociada a ellas.


Por esto la mejor manera de almacenar los datos es en documentos anidados, donde el documento principal es el usuario y este contendrá los documentos de las otras clases; en la figura \ref{img:database_schema} se representa la relación por composición de las clases de la figura~\ref{img:classes_diagram}.


\begin{figure}[H]
    \centering
    \includegraphics[width=0.75\textwidth]{diseño/diagrama_DB.png}
    \caption{Diagrama de la base de datos.}
    \label{img:database_schema}
\end{figure}


En la figura~\ref{img:database_schema} se pone en práctica el comportamiento del diagrama de clases de la figura~\ref{img:classes_diagram}, porque al eliminar un usuario internamente se eliminan los diagramas asociados a este; es por este motivo que el usuario debe estar 100 \% seguro al momento de eliminar su cuenta, ya que será imposible recuperar los diagramas que haya realizado un vez completada la acción.
\section{Arquitectura del sistema}

De acuerdo con Pressman\cite{pressman_software_2005} en su sección Diseño de la arquitectura , el diseño arquitectónico de un software es la representación de los componentes o módulos del sistema y cómo estos interactúan entre sí. En esta sección se muestra que diseño arquitectónico utilizará el sistema como parte de su implementación.

\subsection{Arquitectura cliente-servidor}

La arquitectura cliente servidor es un modelo de diseño del software compuesta por 2 partes, el clientes, es quien solicita información por medio de una petición y el servidor quien se encarga de proveer los datos que le son requeridos. Las aplicaciones web utilizan frecuentemente este modelo teniendo como clientes a un navegador web y el servidor es quien provee la información que solicitante, pudiendo atender a más de un cliente al mismo tiempo.

La figura \ref{img:arq_client_server} describe la arquitectura cliente servidor del sistema.

\begin{figure}[H]
    \centering
    \includegraphics[width=0.75\textwidth]{diseño/cliente_servidor.png}
    \caption{Arquitectura cliente-servidor.}
    \label{img:arq_client_server}
\end{figure}


En la imagen anterior se observa que para poder hacer uso del sistema el usuario debe contar con un dispositivo capaz de ejecutar una navegador web, como una computadora o teléfono inteligente( smartphone). El servidor será una computadora que se encuentra en un instancia de la nube la cual contiene los requisitos necesarios para el funcionamiento del sistema como son: el sistema operativo GNU/Linux, el servidor web gunicorn, las aplicaciones flask y nuxt, conexión a internet y una configuración de red   que permita la comunicación con la base de datos MongoDB.


\subsection{Patrón de diseñó MVC}

De acuerdo con Dragos-Paul Pop \cite{dragos_2014} en su trabajo \textit{`Designing an MVC Model for Rapid Web Application Development'}, el patrón MVC fue concevido por primera vez en la década de los 70 en la empresa Xerox Parc. y su principal propósito era  cerrar la brecha entre el modelo mental del usuario y el modelo digital de las aplicaciones en una computadora.


Más tarde el paradigma MVC fue descrito por  Krasner y Pope\cite{pope_1988} donde destacan que se pueden obtener enormes beneficios si se piensa en la modularidad de las aplicaciones. El patrón se divide en 3 partes principalmente: el modelo, la vista y el controlador los cuales se describen adelante.

\begin{itemize}
    \item El modelo: es el dominio principal de la aplicación, es quien se encarga del manejo de los datos  y toda la lógica del negocio, es decir, no tiene que tener conocimiento sobre la interfaz gráfica ya sea una aplicación web, de escritorio o móvil. Idealmente el modelo debe ser independiente de la plataforma en que se ejecuta.
    \item La vista : contiene todos los elementos visibles al usuario y en contra parte al modelo, esta no debe tener conocimiento de la lógica del negocio de la aplicación. Sus responsabilidades deben limitarse a definir la estructura y apariencia de los datos presentados en la interfaz gráfica.
    \item El controlador : es el intermediario entre la vista y el modelo, es quien gestiona el flujo de la aplicación por medio de la comunicación entre los otros 2 componentes.
\end{itemize}

La figura \ref{img:mvc_implementation} describe la implementación del patrón MVC en la propuesta de solución junto a las tecnologías utilizadas en cada componente del paradigma. Como se aprecia la vista queda a cargo del framework Nuxt quien es el encargado del renderizado de la interfaz gráfica para el usuario mostrando elementos como el formulario de login y el diagramador entidad-relación.

\begin{figure}[H]
  \centering
  \includegraphics[width=0.75\textwidth]{diseño/patron_mvc.png}
  \caption{Esquema MVC.}
  \label{img:mvc_implementation}
\end{figure}


De igual manera el framework flask será el encargado del rol de controlador, es decir, quién llevará a cabo al comunicación de los datos que el usuario proporcione en la vista para transformarlos en algo que el modelo pueda entender como el guardado del objeto json del diagrama entidad-relación para ser almacenado en la base de datos. Finalmente el modelo queda a cargo del lenguaje python para el manejo de los datos y el encargado de mantener la comunicación con la base de datos, pudiéndose apoyar en bibliotecas de terceros como el propio framework flask para transformar los datos de la lógica de negocio y convertirlos en algo que la vista pueda mostrar al usuario.

\section{Interfaz de usuario}

En esta sección se muestran las imágenes del prototipo funcional de la propuesta de solución junto a una breve descripción de cada una de las pantallas.

\subsection*{Pantalla de inicio}

La figura~\ref{img:app_home} muestra la pantalla inicial que el usuario observa apenas entra a la \href{https://serene-haibt-2239b4.netlify.app/}{dirección}; donde tiene la opción de registrarse o iniciar sesión en caso de no haberlo hecho anteriormente.

\begin{figure}[H]
    \centering
    \includegraphics[width=0.75\textwidth]{interfaz/home.png}
    \caption{Pantalla inicial.}
    \label{img:app_home}
\end{figure}

\subsection*{Pantalla de registro}

La figura \ref{img:app_register} cuenta con un formulario para el registro de un usuario nuevo, se deben llenar todos los datos cumpliendo una serie de reglas entre ellas:

\begin{itemize}
    \item La contraseña debe tener al menos 6 caracteres.
    \item Se solicita que repitan la contraseña en otro campo y deben coincidir.
    \item Se debe colocar un \textit{email} válido.
\end{itemize}

Una vez llenado los datos solicitados y al dar clic en el botón \textit{submit}, el usuario recibirá un correo de bienvenida en la dirección de \textit{submit} que proporcionó; al finalizar el proceso de registro del usuario en la base de datos, será redirigido automáticamente a la pantalla del diagramador entidad-relación.

\begin{figure}[H]
    \centering
    \includegraphics[width=0.75\textwidth]{interfaz/register.png}
    \caption{Pantalla de registro.}
    \label{img:app_register}
\end{figure}

\subsection*{Pantalla de login}

La figura~\ref{img:app_login} cuenta con un formulario para que el usuario ingrese el \textit{email} y contraseña que proporcionó en el momento de su registro; una vez que el sistema valide la existencia del usuario en la base de datos, será redirigido automáticamente a la pantalla del diagramador entidad-relación.

\begin{figure}[H]
    \centering
    \includegraphics[width=0.75\textwidth]{interfaz/login.png}
    \caption{Pantalla de login.}
    \label{img:app_login}
\end{figure}


\subsection*{Pantallas del diagramador entidad-relación}

La figura~\ref{img:app_diagrammerER} muestra lo que visualiza el usuario una vez que concluyó su registro o inició sesión; se puede apreciar las herramientas para crear/editar un diagrama entidad-relación, además de la zona de trabajo conocida como \textit{canvas}.

\begin{figure}[H]
    \centering
    \includegraphics[width=0.75\textwidth]{interfaz/er_diagramer.png}
    \caption{Pantalla del diagramador entidad-relación}
    \label{img:app_diagrammerER}
\end{figure}


La figura~\ref{img:app_errorDiagram} muestra los errores del diagrama entidad-relación después que el usuario hace click en el  botón ``validar diagrama'', en caso que el diagrama no cumpla con las reglas mencionadas en la sección \ref{cap:validationER}.

\begin{figure}[H]
    \centering
    \includegraphics[width=0.75\textwidth]{interfaz/invalid_diagramER.png}
    \caption{Pantalla de erroresal validar un diagrama.}
    \label{img:app_errorDiagram}
\end{figure}

La figura~\ref{img:app_validDiagram} muestra el modal que el usuario visualiza después de hacer click en el botón ``validar diagrama'' y este cumple con todas las reglas de validación estructural.

\begin{figure}[H]
    \centering
    \includegraphics[width=0.75\textwidth]{interfaz/valid_diagramER.png}
    \caption{Pantalla de erroresal validar un diagrama.}
    \label{img:app_validDiagram}
\end{figure}

\subsection*{Pantallas de las sentencias SQL equivalentes}

La figura~\ref{img:app_sqlSentences} muestra las pantallas del módulo de obtención de las sentencias equivalentes del diagrama que el usuario generó en la figura~\ref{img:app_diagrammerER}; este paso solo es posible después de haber pasado por el proceso de validación para el diagrama entidad-relación.
Del lado derecho en la figura~\ref{img:app_sqlScript} se aprecia el código en el lenguaje SQL necesario para crear la base de datos relacional en el sistema gestor de base de datos MySQL, y del lado izquierdo en la figura~\ref{img:app_dbName} se muestra el modal que el usuario visualiza al hacer click en el botón ``Obtener sentencias SQL'' en el cual deberá colocar el nombre que tendrá la base de datos a generar.

Si el usuario necesita exportar el script de sql a un archivo, cuenta con un botón en la parte superior para realizar esta acción.

\begin{figure}[H]
    \begin{subfigure}[b]{0.49\textwidth}
        \includegraphics[width=\textwidth]{interfaz/sql_sentences.png}
        \caption{Sentencencias SQL equivalentes al diagram ER.}
        \label{img:app_sqlScript}
      \end{subfigure}
      \hfill
      \begin{subfigure}[b]{0.49\textwidth}
        \includegraphics[width=\textwidth]{interfaz/get_sql_sentences.png}
        \caption{Modal para nombrar la base de datos sql.}
        \label{img:app_dbName}
      \end{subfigure}
    \caption{Pantallas de las sentencias SQL equivalentes al diagrama ER.}
    \label{img:app_sqlSentences}
\end{figure}

\subsection*{Pantallas de las consultas de acceso}

La figura~\ref{img:app_simpleQuery} es lo que el usuario visualiza al ingresar a este módulo, este paso solo es posible después de haber pasado por el proceso de validación para el diagrama entidad-relación. Es aquí donde puede agregar las consultas de acceso que desea que sea utilicen en el proceso de tranformación al modelo noSQL teniendo del lado izquierdo el diagrama ER en modo de solo lectura y al hacer click derecho en los atributos clave de una entidad vusualizará un menú con las opciones para generar dicha consulta.

Las consultas de acceso pueden ser tantas con crea necesitarlas como se aprecia en la figura~\ref{img:app_multipleQueries}, es importante mencionar que puede agregar tantos elementos al apartado ``Respecto al atributo'' como quiera pero debe existir al menos un elemento en el apartado ``Encontrar'' para que la consulta tenga sentido.

\begin{figure}[H]
    \centering
    \includegraphics[width=\textwidth]{interfaz/queries_simple.png}
    \caption{Pantalla para agregar una consulta de acceso.}
    \label{img:app_simpleQuery}
\end{figure}

\begin{figure}[H]
    \centering
    \includegraphics[width=\textwidth]{interfaz/queries_multiple.png}
    \caption{Pantalla con multiples consultas de acceso.}
    \label{img:app_multipleQueries}
\end{figure}

\section{Conclusiones}

La arquitectura cliente-servidor es apta para un desarrollo ágil de la propuesta de solución; asimismo, la arquitectura elegida permite al equipo usar lenguajes de programación que estén consolidados en el desarrollo web o en el diagramado, como es el caso de los lenguajes elegidos en la sección~\ref{ref:conclusiones-cap3}.


El patrón MVC está bien integrado con el \textit{framework} Nuxt y Flask, permitiendo que desde etapas tempranas del desarrollo de la propuesta de solución haya un prototipo funcional que puede revisar en la sección~\ref{ref:prototipo}.


Finalmente, el diagrama de clases y el diagrama de la base de datos también están integrados, porque desde el diseño conceptual es fácil usar el concepto de la anidación de datos y es solo cuestión de implementarlo en MongoDB.

%\chapter{Caso de estudio}
% 
De acuerdo con De la Vega\cite{de_la_vega_mortadelo_2020}, el modelo entidad-relación más usado en la literatura NoSQL es el modelo Venues que puede encontrar en el \textit{paper} de Chebotko~\cite{chebotko_big_2015}.

\section{Diagramado modelo entidad-relación básico}

El diagramado de Venues se ha hecho con la propuesta de solución desarrollada y en la figura~\ref{img:venues-er}

\begin{figure}[H]
    \centering
    \includegraphics[width=0.75\textwidth]{casoEstudio/venues-er.png}
    \caption{Diagrama entidad-relación de Venues}
    \label{img:venues-er}
\end{figure}

\section{Validación modelo entidad-relación básico}

 Una vez que el usuario cree tener un diagrama válido o quiere conocer si su diagrama actual presenta algún error, puede hacer click en el boton ``validar diagrama'' con lo que se le mostrará una ventana con los errores que la aplicación detecta. La figura~\ref{img:ex_diagrama1} muestra un ejemplo de diagrama sencillo mientras que en la figura~\ref{img:errors_diagrama1} se aprecian los errores estructurales del mismo.

 \begin{figure}[H]
  \begin{subfigure}[b]{0.49\textwidth}
      \includegraphics[width=\textwidth]{casoEstudio/ex_diagrama1.png}
      \caption{Ejemplo de un diagram er básico.}
      \label{img:ex_diagrama1}
    \end{subfigure}
    \hfill
    \begin{subfigure}[b]{0.49\textwidth}
      \includegraphics[width=\textwidth]{casoEstudio/errors_diagrama1.png}
      \caption{Ventana de errores del diagrama er.}
      \label{img:errors_diagrama1}
    \end{subfigure}
\end{figure}

 Siguiendo con el ejemplo del modelo Venues la figura~\ref{img:venues_validationER} muestra la validación del diagrma y como se espera este es estructuralmente válido.

 \begin{figure}[H]
  \centering
  \includegraphics[width=0.75\textwidth]{casoEstudio/vuenues_validation.png}
  \caption{Validación del modelo Venues.}
  \label{img:venues_validationER}
\end{figure}

\section{Generación sentencias SQL}
 Una vez validado el diagrama entidad-relación básico se pueden generar las sentencias SQL que se muestra en la figura~\ref{img:venues-sql} 

 \begin{figure}[H]
    \centering
    \includegraphics[width=0.75\textwidth]{casoEstudio/venues-sql.png}
    \caption{Sentencias SQL de Venues.}
    \label{img:venues-sql}
\end{figure}

\subsection*{Prueba de sentencias SQL en MySQL}

Como se muestra en la figura~\ref{img:mysql_workbench} se ha probado la ejecución de las sentencias sql en el gestor gráfico MySQL Workbench en su versión 8.0.22 y al no apreciar en la parte inferior de la figura ningún error se concluye que la ejecución ha sido exitosa.

\begin{figure}[H]
  \centering
  \includegraphics[width=0.75\textwidth]{casoEstudio/mysql_workbench.png}
  \caption{Ejecución de las sentencias SQL en MySQL Workbench.}
  \label{img:mysql_workbench}
\end{figure}

\section{Modelo conceptual NoSQL}

Con un diagrama entidad-relación básico válido es posible generar las entidades del modelo conceptual NoSQL como se muestra en la figura~\ref{img:venues-conceptual}.

\begin{figure}[H]
    \centering
    \includegraphics[width=0.75\textwidth]{casoEstudio/venues-conceptual.png}
    \caption{Entidades del modelo conceptual GDM}
    \label{img:venues-conceptual}
\end{figure}

Una vez generadas las entidades del modelo conceptual GDM es necesario que el usuario defina las consultas en la sección de consultas del GDM como se muestra en la figura~\ref{img:venues-conceptual-queries}.

\begin{figure}[H]
    \centering
    \includegraphics[width=0.75\textwidth]{casoEstudio/venues-conceptual-queries.png}
    \caption{Entidades y consultas del modelo conceptual GDM}
    \label{img:venues-conceptual-queries}
\end{figure}

El modelo Venues de  Chebotko contiene nueve consultas, que en el GDM son las siguientes:

\begin{verbatim}
    query Q1_artifactsByVenue:
    select venue.venueName, venue.year,
           artifacts.artifactId, artifacts.artifactTitle, artifacts.avgRating,
           artifacts.authors, artifacts.keywords
    from Venue as venue
    including venue.featuresArtifact as artifacts
    where venue.venueName = "?"
      and venue.year > "?"
  
  query Q2_artifactsByAuthor:
    select artifact.artifactId, artifact.artifactTitle, artifact.avgRating,
           artifact.authors, artifact.keywords, artifact.authors as author
    from Artifact as artifact
    where artifact.authors = "?"
  
  query Q3_usersByLikedArtifact:
    select user.userId, user.userName, user.userEmail
    from User as user
    including user.likesArtifact as artifacts
    where artifacts.artifactId = "?"
  
  query Q4_expertsByLikedArtifact:
    select user.userId, user.areasOfExpertise, user.userName, user.userEmail,
           user.areasOfExpertise as areaOfExpertise
    from User as user
    including user.likesArtifact as artifacts
    where artifacts.artifactId = "?"
      and user.areasOfExpertise = "?"
  
  query Q5_ratingByArtifact:
    select artifact.artifactId, artifact.avgRating
    from Artifact as artifact
    where artifact.artifactId = "?"
  
  query Q6_venuesLikedByUser:
    select user.userId, venues.venueName, venues.year,
           venues.country, venues.homepage
    from User as user
    including user.likesVenue as venues
    where user.userId = "?"
  
  query Q7_artifactsLikedByUser:
    select user.userId,
           likesArtifacts.artifactId,
           likesArtifacts.artifactTitle, likesArtifacts.authors,
           venue.venueName
    from User as user
    including user.likesArtifact as likesArtifacts,
              user.likesArtifact.venue as venue
    where user.userId = "?"
      and venue.year > "?"
  
  query Q8_reviewsByUser:
    select user.userId,
           reviews.reviewId, reviews.reviewTitle, reviews.body,
           artifact.artifactId, artifact.artifactTitle
    from User as user
    including user.postsReview as reviews,
              user.postsReview.artifact as artifact
    where user.userId = "?"
      and reviews.rating > "?"
  
  query Q9_artifacts:
    select artifact.artifactId, artifact.artifactTitle, artifact.avgRating,
           artifact.authors, artifact.keywords,
           venue.venueName
    from Artifact as artifact
    including artifact.venue as venue
    where artifact.artifactId = "?"
\end{verbatim}

\section{Modelo lógico NoSQL}

A partir de un modelo conceptual GDM, que está conformado por entidades y consultas, se puede generar el modelo lógido orientado a documentos. En la figura~\ref{img:venues-logical} se muestra el diagrama correspondiente al modelo lógico orientado a documentos de Venues.

\begin{figure}[H]
    \centering
    \includegraphics[width=0.75\textwidth]{casoEstudio/venues-logical.png}
    \caption{Modelo lógico orientado a documentos de Venues}
    \label{img:venues-logical}
\end{figure}

\section{Modelo físico NoSQL}

Tambien se pueden obtener las sentencias para MongoDB. Para el modelo Venues se muestra a continuación las sentencias correspondientes a su modelo lógico.

\begin{minted}[linenos,tabsize=2,breaklines, fontsize=\small]{json}
    db.createCollection("Venue",
{
  validator: {
    $jsonSchema: {    bsonType: "object",
    properties: {
venueName: { bsonType: "string"
        },year: { bsonType: "double"
        },venueId: { bsonType: "objectId"
        },homepage: { bsonType: "string"
        },country: { bsonType: "string"
        },topics: { bsonType: "string"
        },artifactRefArray: {
    bsonType: [
            "array"
          ],
    items: {
         properties: {
artifactRef: { bsonType: "objectId"
              }
            }
          }
        }
      }
    }
  }
})
db.createCollection("Artifact",
{
  validator: {
    $jsonSchema: {    bsonType: "object",
    properties: {
artifactId: { bsonType: "objectId"
        },authors: { bsonType: "string"
        },numRatings: { bsonType: "double"
        },avgRating: { bsonType: "double"
        },sumRatings: { bsonType: "double"
        },artifactTitle: { bsonType: "string"
        },keywords: { bsonType: "string"
        },venueRef: { bsonType: "objectId"
        }
      }
    }
  }
})
db.createCollection("User",
{
  validator: {
    $jsonSchema: {    bsonType: "object",
    properties: {
userId: { bsonType: "objectId"
        },userName: { bsonType: "string"
        },userEmail: { bsonType: "string"
        },areasOfExpertise: { bsonType: "string"
        },artifactRefArray: {
    bsonType: [
            "array"
          ],
    items: {
         properties: {
artifactRef: { bsonType: "objectId"
              }
            }
          }
        },venueRefArray: {
    bsonType: [
            "array"
          ],
    items: {
         properties: {
venueRef: { bsonType: "objectId"
              }
            }
          }
        },reviewArray: {
    bsonType: [
            "array"
          ],
    items: {
         properties: {
rating: { bsonType: "double"
              },body: { bsonType: "string"
              },reviewTitle: { bsonType: "string"
              },reviewId: { bsonType: "objectId"
              },artifactRef: { bsonType: "objectId"
              }
            }
          }
        }
      }
    }
  }
})
\end{minted}


\subsection*{Prueba de sentencias mongo en MongoDB}

Como se muestra en la figura~\ref{img:venues-mongo} se ha probado crear las colecciones en el manejador MongoDB y se solicita a MongoDB la información sobre la colección creada. Posteriormente, MongoDB muestra correctamente las reglas de validación para la colección Venue.

\begin{figure}[H]
    \centering
    \includegraphics[width=0.75\textwidth]{casoEstudio/venues-mongo.png}
    \caption{Prueba de sentencias del modelo físico orientado a documentos de Venues en MongoDB}
    \label{img:venues-mongo}
\end{figure}

\section{Conclusiones}
Como se muestra en este caso de estudio, la aplicación diagrama correctamente el modelo Venues, genera sus sentencias SQL que están probadas en MySQL, también genera correctamente el modelo conceptual, lógico y físico de un modelo NoSQL. Se diagrama el modelo lógico y se prueban las sentencias del modelo físico en MongoDB.


\appendix
\chapter{Apéndice}
\section{Unified Modeling Language}

De acuerdo con Fowler\cite{fowler_brief_2003}, UML (\textit{Unified Modeling Language}) es una familia de notaciones gráficas, respaldada por un metamodelo único, que ayuda a describir y diseñar sistemas de \textit{software}, particularmente sistemas de \textit{software} descritos utilizando el estilo orientado a objetos.

\subsection{Diagramas de clases}

De acuerdo con Fowler\cite{fowler_brief_2003}, un diagrama de clases describe los tipos de objetos en el sistema y los diversos tipos de relaciones estáticas que existen entre ellos. 


Los diagramas de clase también muestran las propiedades y operaciones de una clase y las restricciones que se aplican a la forma en que se conectan los objetos.


UML usa el término característica como un término general que cubre las propiedades y operaciones de una clase.


\subsubsection*{Miembros}
UML proporciona mecanismos para representar los miembros de la clase, como atributos y métodos, así como información adicional sobre ellos.


\paragraph*{Visibilidad}
Para especificar la visibilidad de un miembro de la clase (es decir, cualquier atributo o método), se coloca uno de los siguientes signos delante de ese miembro:

\begin{enumerate}
    \item +	Público
    \item -	Privado
    \item \# Protegido
    \item /	Derivado (se puede combinar con otro)
    \item \~ \;	Paquete
\end{enumerate}

\paragraph*{Ámbitos}

UML especifica dos tipos de ámbitos para los miembros: instancias y clasificadores y estos últimos se representan con nombres subrayados.

\begin{enumerate}
    \item Los miembros clasificadores se denotan comúnmente como “estáticos” en muchos lenguajes de programación; su ámbito es la propia clase.
    \begin{enumerate}
        \item Los valores de los atributos son los mismos en todas las instancias.
        \item La invocación de métodos no afecta al estado de las instancias.
    \end{enumerate}
    \item Los miembros instancias tienen como ámbito una instancia específica.
\begin{enumerate}
    \item Los valores de los atributos pueden variar entre instancias.
    \item La invocación de métodos afecta al estado de las instancias (es decir, cambiar el valor de sus atributos).
\end{enumerate}
\end{enumerate}
Para indicar que un miembro posee un ámbito de clasificador, hay que subrayar su nombre; de lo contrario, se asume por defecto que tendrá ámbito de instancia.



La figura~\ref{img:uml-asociacion} es una asociación entre dos clases, la clase \textit{Person} y la clase \textit{Magazine}.

La figura~\ref{img:uml-agregacion} es una agregación entre dos clases, la clase \textit{Professor} y la clase \textit{Class} donde la cardinalidad es una a muchos.

La figura~\ref{img:uml-composicion} muestra una  asociación y composición entre clases, la clase Almacen está asociada a la clase Cuentas y con la clase Cliente hay una relación de composición.


\paragraph*{Relaciones}
Una relación es un término general que abarca los tipos específicos de conexiones lógicas que se encuentran en los diagramas de clases y objetos. UML presenta las siguientes relaciones:

\paragraph*{Enlace}
Un enlace es la relación más básica entre objetos.

\paragraph*{Asociación}



\begin{figure}[H] 
    \centering
    \includegraphics[width=0.65\textwidth]{uml/01.png}
    \caption{Asociación}
    \label{img:uml-asociacion}
\end{figure}

Una asociación como la de la figura~\ref{img:uml-asociacion} representa una familia de enlaces; una asociación binaria (entre dos clases) normalmente se representa con una línea continua; una misma asociación relaciona cualquier número de clases y una asociación que relacione tres clases se llama asociación ternaria.

A una asociación se le asigna un nombre y en sus extremos se hacen indicaciones, como el rol que desempeña la asociación, los nombres de las clases relacionadas, su multiplicidad, su visibilidad, y otras propiedades.

Hay cuatro tipos diferentes de asociación: bidireccional, unidireccional, agregación (en la que se incluye la composición) y reflexiva; las asociaciones unidireccional y bidireccional son las más comunes.


\subsubsection*{Agregación}


\begin{figure}[H] 
    \centering
    \includegraphics[width=0.65\textwidth]{uml/02.png}
    \caption{Agregación}
    \label{img:uml-agregacion}
\end{figure}


La agregación o agrupación como la de la figura~\ref{img:uml-agregacion} es una variante de la relación de asociación ``tiene un'': la agregación es más específica que la asociación; se trata de una asociación que representa una relación de tipo parte-todo o parte-de.


Al ser un tipo de asociación, una agregación es posible que tenga un nombre y las mismas indicaciones en los extremos de la línea; sin embargo, una agregación no debe incluir más de dos clases, debe ser una asociación binaria.


Una agregación es posible que se dé cuando una clase es una colección o un contenedor de otras clases, pero a su vez, el tiempo de vida de las clases contenidas no tienen una dependencia fuerte del tiempo de vida de la clase contenedora (de el todo). 


En UML, como está en la figura~\ref{img:uml-agregacion} se representa gráficamente con un rombo hueco junto a la clase contenedora con una línea que lo conecta a la clase contenida; todo este conjunto es, semánticamente, un objeto extendido que es tratado como una única unidad en muchas operaciones, aunque físicamente está hecho de varios objetos más pequeños.

\begin{figure}[H] 
    \centering
    \includegraphics[width=0.65\textwidth]{uml/03.png}
    \caption{Asociación rombo sin rellenar y composición rombo negro}
    \label{img:uml-composicion}
\end{figure}

\subsubsection*{Composición}

La representación en UML de una relación de composición es mostrada en la figura~\ref{img:uml-composicion} como un diamante rellenado del lado del la clase contenedora, es decir, al final de la línea que conecta la clase contenido con la clase contenedora.


\paragraph*{Diferencias entre Composición y Agregación}

\subparagraph*{Relación de Composición}

\begin{enumerate}
    \item Cuando se intenta representar un todo y sus partes.
    \item Cuando se elimina el contenedor, el contenido también es eliminado.
\end{enumerate}


\subparagraph*{Relación de Agrupación}

\begin{enumerate}
    \item Cuando se representan las relaciones en un \textit{software} o base de datos. 
    \item Cuando el contenedor es eliminado, el contenido usualmente no es destruido.
\end{enumerate}
\section{Prototipo funcional}\label{ref:prototipo}

El prototipo funcional de la propuesta de solución usa para la autenticación de usuarios \textit{web tokens}; se dispone del código fuente y un \textit{live demo} de la aplicación.


El \textit{front end} (Nuxt/JS/GoJS) lo puede consultar en \href{https://github.com/martinez-acosta/TT-2019-B052}{repositorio front}, el \textit{back end} (Python/Flask) en \href{https://github.com/omaraparicio07/api-tt-2019-b052}{repositorio back}, el documento de TT (Latex) en \href{https://github.com/martinez-acosta/DOCS-TT-2019-B052}{repositorio documento} y el \textit{live demo} está disponible en \href{https://serene-haibt-2239b4.netlify.app/}{dirección}.


El \textit{live demo} del \textit{front end} está usando el \textit{deploy} automático en Netlify de la rama master y el código de desarrollo está en dev.


Asimismo, en la página de GitHub del proyecto de \textit{front end} es posible encontrar de manera breve un resumen del propósito de la aplicación web, estado actual del proyecto, tecnologías usadas y cosas que faltan por implementar.


El \textit{live demo} del \textit{back end} está usando el \textit{deploy} automático en Heroku de la rama master y el código de desarrollo está en las ramas \textit{feature}.


Para probar la aplicación, es necesario darse de alta en la aplicación, pero el equipo se ha tomado la molestia de crear credenciales temporales para los sinodales, directores y el maestro de seguimiento (las cuentas de usuario y contraseñas se hicieron llegar a cada uno de los involucrados); para los correos temporales de las cuentas se ha usado el servicio de: \href{https://temp-mail.org/es/}{temp-mail}.

\subsection*{Front end}

De acuerdo con el sitio de Netlify~\cite{netlify_netlify_nodate}, Netlify ofrece funcionalidad HTTPS para todos sus sitios, tiene una mitigación activa en contra de ataques DDoS, todo su tráfico está cifrado en redes TLS y los \textit{tokens} están cifrados.

De acuerdo con el sitio de JWT~\cite{jwt_web_2020}, un JSON Web Token (JWT) es un estándar abierto (RFC 7519) que define una forma compacta y autónoma para transmitir información de forma segura entre las partes como un objeto JSON.


Esta información se puede verificar y confiar porque está firmada digitalmente; los JWT se firman usando un clave denominada ``secreto'' (con el algoritmo HMAC) o un par de claves pública/privada usando RSA o ECDSA.

Aunque los JWT se pueden cifrar para proporcionar ``secreto'' entre las partes, los \textit{jwt web tokens} se enfocan en \textit{tokens} firmados; los \textit{tokens} firmados pueden verificar la integridad de los reclamos que contiene, mientras que los \textit{tokens} cifrados ocultan esos reclamos de otras partes; cuando los \textit{tokens} se firman utilizando pares de claves públicas / privadas, la firma también certifica que solo la parte que posee la clave privada es la que la firmó.


La figura~\ref{img:prototipo-welcome} muestra la pantalla de bienvenida de la aplicación, donde el mensaje de iniciar sesión o registarse es visible si el usuario no ha iniciado sesión.

\begin{figure}[H]
    \centering
    \includegraphics[width=0.75\textwidth]{prototipo/welcome.png}
    \caption{Pantalla de bienvenida}
    \label{img:prototipo-welcome}
\end{figure}

La figura~\ref{img:prototipo-login} muestra la pantalla de inicio de sesión de la aplicación, donde se pide el correo y la contraseña para iniciar sesión.


\begin{figure}[H]
    \centering
    \includegraphics[width=0.75\textwidth]{prototipo/login.png}
    \caption{Pantalla de login}
    \label{img:prototipo-login}
\end{figure}
La figura~\ref{img:prototipo-signup} muestra el formulario de registro de la aplicación; para la validación de los campos se ha usado Vuelidate y muestra mensajes de error en caso de que algún campo esté rellenado incorrectamente.


\begin{figure}[H]
    \centering
    \includegraphics[width=0.75\textwidth]{prototipo/signup.png}
    \caption{Pantalla de alta de usuario}
    \label{img:prototipo-signup}
\end{figure}
La figura~\ref{img:prototipo-er} muestra la vista del diagramador entidad-relación básico, del lado izquierdo están los botones para guardar y cargar un diagrama en formato .json válido para la aplicación; en la segunda columna está la zona de diagramado para crear un diagrama entidad-relación y del lado derecho está la paleta de elementos donde están los elementos de un diagrama entidad-relación básico; en la parte de abajo está el JSON equivalente del diagrama que está en la zona de diagramado y a su derecha está un menú para modificar algunas propiedades de los elementos del diagramador.

\begin{figure}[H]
    \centering
    \includegraphics[width=0.50\textwidth]{prototipo/er.png}
    \caption{Pantalla del diagramador entidad-relación}
    \label{img:prototipo-er}
\end{figure}



Las figuras~\ref{img:prototipo-welcome},~\ref{img:prototipo-login},~\ref{img:prototipo-signup} y~\ref{img:prototipo-er} son capturas de pantalla del prototipo de la propuesta de solución, como es de notar el prototipo permite editar, guardar y cargar un diagrama entidad-relación básico.


\section{Back end}


De acuerdo con el sitio de Heroku~\cite{heroku_heroku_2020}, la plataforma ofrece distintos mecanismos para la seguridad de los proyectos alojadas en ella, dentro de estas se encuentra el uso por defecto del protocolo HTTPS, que asegura el cifrado de los datos en la Internet.


De igual manera ofrece un constante escaneo de las aplicaciones en búsqueda de vulnerabilidades para mitigar los ataques DDoS, además de utilizar la infraestructura de la empresa Amazon, las cuales se encuentran acreditadas por diversos estándares de seguridad; esto da la confianza para que la aplicación Flask se ejecute de manera segura y tener salvaguardados los datos de los usuarios.


Los métodos o funciones implementadas en la aplicación \textit{flask} constan de un CRUD (\textit{Create, Read, Update, Delete}) para el manejo de usuarios entre otras para lograr la funcionalidad completa del proyecto. Para la mantener separado el proyecto front-end y los servicos web se utilizó  la herramienta \textit{flask\_restplus} la cual utiliza la interfaz de swagger para diseñar, crear, documentación y utilizar servicios web \textit{RESTful}, de esta manera los desarrolladores y usuarios pueden probar los servicios sin la necesidad del proyecto mostrado en \href{https://serene-haibt-2239b4.netlify.app}{netlify} o alguna herramienta para realizar peticiones rest como \href{https://www.postman.com/downloads/}{postman}, \href{https://install.advancedrestclient.com/install}{Advanced REST client}, etc. , la figura~\ref{img:swaggerApi} muestra la vista que el usuario pude ver al entrar en la aplicación \href{https://api-tt-2019-b052.herokuapp.com}{heroku}.

\begin{figure}[H]
  \centering
  \includegraphics[width=0.9\textwidth]{apiFlask/api_swagger_home.png}
  \caption{Pantalla inicial con la interfaz swagger}
  \label{img:swaggerApi}
\end{figure}


Para mantener la seguridad de los datos de los usuarios la mayoria de los servicios expuesto necesitan de de un token de autenticación para poder realizar operaciones de modificación o consulta, esto se logra habilidanto la característica de logueo para la interfaz swagger, donde se solicita un token para poder realizar peticiones que contienen datos del usuario. la figura~\ref{img:tokenSwagger} muestra la ventana para ingresar el token de autenticación de usuario generado por el endpoint \textit{login} de la figura~\ref{img:loginApi} con un usuario previamente registrado.

\begin{figure}[H]
  \begin{subfigure}[b]{0.49\textwidth}
      \includegraphics[width=\textwidth]{apiFlask/token_swagger.png}
      \caption{Solicitud de token de autenticación.}
      \label{img:tokenSwagger}
    \end{subfigure}
    \hfill
    \begin{subfigure}[b]{0.49\textwidth}
      \includegraphics[width=\textwidth]{apiFlask/login_api.png}
      \caption{Generación del token de autenticación.}
      \label{img:loginApi}
    \end{subfigure}
    \caption{Login y autenticación en el api.}
\end{figure}

Para la construcción de la aplicación se creo la estructura mostrada en la figura~\ref{img:apiStructure} donde la carpeta \textit{config} contiene un archivo con el nombre \textit{app.conf} con las variables de configuración de la aplicación como son la llave secreta para validar el \textit{JWT}, las credenciales para conexión a la base de datos, un usuario de pruebas y característica de la interfaz swagger, mientras que el archivo \textit{database.py} es una clase singleton para mantener una sola instancia de conexión a la base de datos para toda la aplicación.

\begin{figure}[H]
  \centering
  \includegraphics[width=0.9\textwidth]{apiFlask/api_structure.png}
  \caption{Estructura del proyecto backend.}
  \label{img:apiStructure}
\end{figure}

Las carpetas \textit{apis y service} contiene toda la lógica de negocio para las operaciones como son la creación de usuario, recuperación de la contraseña, la obtención de sentencias SQL, etc. El archivo \textit{server.py} es el encargado de cargar la configuración y ejecutar el servidor de flask para poner en marcha la aplicación, mientras que el archivo \textit{util.py} contiene los métodos para validar la dirección de correo electrónico y el envío de correos utilizando el servidor \textit{SMTP} de google.

A continuación se listan los endpoints creados para los servicios web junto a una breve descripción de cada uno de ellos:

\begin{itemize}
    \item /user (POST): Crea un nuevo usuario, requiere de un nombre de usuario, un email valido y no registrdo en la aplicación y una contraseña que será cifrada.
    \item /user (GET): Recupera un listado de todos los usuarios registrados de la aplicación.
    \item /user/{email} (GET): Recupera al información de un usuario en especifico, requiere de un email valido y registrado en la aplicación.
    \item /login (POST): Genera un token de autenticación de usuario, requiere de un email y la contraseña de un usuario registrado en la aplicación.
    \item /login (PUT): Cambios la contraseña de un usuarios, requiere el email de un usuario registrado y la nueva contraseña que será cifrada para actualizarla en la base de datos al usuario asociado.
    \item /diagram (POST): Guarda un diagrama en la base de datos, requiere de un diagrama en formato json para asociarlo al usuario asociado al token de autenticación.
    \item /diagram (GET): Recupera el diagrama asociado a un usuario, requiere de un token de autenticación para determinar el usuario del cual recuperar su diagrama.
    \item /relational (POST): Transforma un diagram ER a un script sql, requiere de un diagrama en formato js para obtener las sentencias SQL equivalentes.
    \item /relational/validate (POST): Valida la estructura un diagrama ER, requiere de un diagrama en formato js para aplicar las reglas de validación de la sección \ref{cap:validationER} y determinar si el diagrama es valido o en su defecto mostrar los errores que este presenta.
\end{itemize}


De acuerdo con el sitio de MongoDB\cite{mongodb_mongodb_2020}, MongoDB es una plataforma de base de datos alojada en la nube que garantiza la disponibilidad, escalabilidad y cumplimiento de estándares de seguridad; por estos motivos es que el almacenamiento de los datos está a cargo de este servicio, ofreciendo la conectividad con la aplicación Flask, ya que Heroku permite la conexión con los servicios de MongoDB Atlas; la configuración para la conexión no cambia si se están realizando pruebas en un ambiente local antes de exponer los últimos cambios al público.


\printbibliography

\end{document}
