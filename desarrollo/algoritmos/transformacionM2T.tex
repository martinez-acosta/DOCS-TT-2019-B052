\subsection{Transformación del modelo lógico orientado a documentos a MongoDB}\label{sec:script-mongoDB}
La generación de las sentencias de MongoDB es desde el modelo lógico orientado a documentos y se usa un validador para definir la estructura de las colecciones.


\begin{algorithm}[H]
    \SetKwInOut{Input}{Entrada}
    \SetKwInOut{Output}{Salida}
  
    \Input{modelo DDM, $ddm$}
    \Output{sentencias MongoDB, $mongo$}
        $mongo \gets parse(ddm)$\\
        $mongo.save()$
    \caption{Generar las sentencias de MongoDB desde un modelo lógico orientado a documentos}
  \end{algorithm}


  \begin{code}
    \captionof{listing}{Generación de sentencias MongoDB desde el DDM}
    \label{code:er-to-gdm}
    \begin{minted}[linenos,tabsize=2,breaklines, fontsize=\small]{python}
        import langs.ddmLang as ddm
        from pyecore.resources import ResourceSet, URI
        from pyecore.resources.xmi import XMIResource
        
        def loadModelDDM(input_file):
            rset = ResourceSet()
            # register the metamodel (available in the generated files)
            rset.metamodel_registry[ddm.nsURI] = ddm
            rset.resource_factory['ddm'] = lambda uri: XMIResource(uri)  
            resource = rset.get_resource(URI(input_file))
            model = resource.contents[0]
            return model
        
        def getMongoDBType(type_):
            if type_.name == "INT":
                tipo = "int"
            elif type_.name == "FLOAT":
                tipo = "double"
            elif type_.name == "TEXT":
                tipo = "string"
            elif type_.name == "DATE":
                tipo = "date"
            elif type_.name == "TIMESTAMP":
                tipo = "timestamp"
            elif type_.name == "BOOLEAN":
                tipo = "boolean"
            elif type_.name == "ID":
                tipo = "objectId"
            else:
                tipo = "string"
        
            return tipo
        
        def generateField(output_file,field):
            if isinstance(field, ddm.PrimitiveField):
                generatePrimitiveField(output_file,field)
            elif isinstance(field, ddm.Document):
                generateDocumentField(output_file,field)
            elif isinstance(field, ddm.ArrayField):
                generateArrayField(output_file,field)
            return
        
        def generateArrayField(output_file, field):
            output_file.write(field.name + ": {\n")
            output_file.write("    bsonType: [\"array\"],\n")
            output_file.write("    items: {\n")
            output_file.write("         properties: {\n")
            generateField(output_file,field.type)
            output_file.write("         }\n")
            output_file.write("    }\n")
            output_file.write("  }\n")
            return
        
        def generatePrimitiveField(output_file, field):
            output_file.write(field.name + ": { bsonType: \"" + getMongoDBType(field.type) + "\"}\n")
            return
        
        def generateDocumentField(output_file,document):
        
            for field in document.fields:
                if isinstance(field, ddm.PrimitiveField):
                    generatePrimitiveField(output_file,field)
                elif isinstance(field, ddm.Document):
                    generateDocumentField(output_file,field)
                elif isinstance(field, ddm.ArrayField):
                    generateArrayField(output_file,field)
                if field != document.fields[-1]:
                        output_file.write(",")
        
            return
        
        def writeFile(output_file,collection):
            output_file.write("db.createCollection(\"" + collection.name + "\", {\n")
            output_file.write("  validator: {\n")
            output_file.write("    $jsonSchema: {")
            output_file.write("    bsonType: \"object\",\n")
            output_file.write("    properties: {\n")
            
            generateDocumentField(output_file,collection.root)
            
            output_file.write("    }\n")
            output_file.write("    }\n}\n})\n")
            output_file.close()
            return
        
        def main():
            ddmModel = loadModelDDM("prueba_ddm.xmi")
            ofile = "laloMongo.json"
            count = 0 
            for collection in ddmModel.collections:
                if count == 0:
                    count += 1
                    with open(ofile, 'w') as output_file:
                        writeFile(output_file,collection)
                else:
                    with open(ofile, 'a') as output_file:
                        writeFile(output_file,collection)
        
        
                        
            return
    \end{minted}
\end{code}