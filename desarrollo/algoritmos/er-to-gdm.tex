\subsection{Transformación modelo entidad-relación a entidades del modelo conceptual GDM}
La generación de las entidades del modelo GDM desde el modelo entidad-relación básico es de le siguiente manera: 


A continuación se explica brevente la implementación.

\begin{algorithm}[H]
  \SetKwInOut{Input}{Entrada}
  \SetKwInOut{Output}{Salida}

  \Input{diagrama entidad-relación básico, $er$}
  \Output{entidades del Generic Datametamodel, $entities_{gdm}$}
  $entities \gets  list(filter(lambda\ item:\ item['type'] == 'entity',nodeData))$\\
    $entities2pair \gets  list(map(lambda\ entity: (entity, getConnectedNodes(entity['key'], nodeData, linkData)),entities))$\\
    $entities_{gdm} \gets  array$\\
    \ForEach{$pair \in entities2pair:$}{
      $references \gets references(pair)$\\
      $features \gets features(pair)$\\
      $entity pair.key$\\
      $entity.references \gets references$\\
      $entity.features \gets features$\\
      $entities_{gdm}.append(entity)$
    }
    
  \caption{Generar los \textit{features} y las \textit{references} del GDM desde el modelo entidad-relación.}
\end{algorithm}

\begin{code}
    \captionof{listing}{Generación de entidades del GDM desde el modelo entidad-relación básico}
    \label{code:er-to-gdm}
    \begin{minted}[linenos,tabsize=2,breaklines, fontsize=\small]{python}
      def main():
    
      input_file = open('er.json') 
      data = json.load(input_file) 
      input_file.close() 
      
      nodeData = data['nodeDataArray']
      linkData = data['linkDataArray']
    
      # Obtenemos entidades del modelo ER
      entities = list(filter(lambda item: item['type'] == "entity",nodeData))
    
      # Obtenemos lista de pares, donde la clave es una entidad y su valor son los nodos a los que está conectada la entidad
      entities2pair = list(map(lambda entity: (entity, getConnectedNodes(entity['key'], nodeData, linkData)),entities))
      count = 0
      for pair in entities2pair:
          entity = pair[0]
          features = pair[1]
          attributes = []
          references = []
          for feature in features:
              if (feature['type'] == "relation"):
                  reference = {}
                  relation = getRelationInfo( feature["key"], linkData, nodeData)
    
                  if (relation["from"]["key"] != entity["key"]):
                      reference["entity"] = relation["from"]["text"]
                      reference["cardinality"] = relation["fromC"]
                      if relation["fromC"] != '1':
                          reference["name"] = feature["text"] + relation["from"]["text"]
                      else:
                          reference["name"] = toFirstLower(relation["from"]["text"])
                  else:
                      reference["entity"] = relation["to"]["text"]
                      reference["cardinality"] = relation["toC"]
                      # Si la relación es a N, va el nombre de la relación concatenado con el nombre de la entidad
                      if relation["toC"] != '1':
                          reference["name"] = feature["text"] + relation["to"]["text"]
                      else:
                          reference["name"] = toFirstLower(relation["to"]["text"])
    
    
                  references.append(reference)
              if ( isAttribute(feature['type']) ):
                  attribute = {}
                  attribute["name"] = feature['text']
                  attribute["type"] =  feature["gdmType"]
                  attribute["array"] = True if feature["type"] == "multivalueAttribute" else False
                  attribute["unique"] = True if feature["type"] == "keyAttribute" else False
                  if attribute["unique"] == True:
                      attribute["type"] = "id"
                  attributes.append(attribute)
          
          # Escribimos el archivo de texto
          if count == 0:
              count += 1
              with open("laloVenues.gdm", 'w') as output_file:
                  output_file.write("entity " + entity["text"] + " {\n")
                  for attribute in reversed(attributes):
                      output_file.write(parseAttribute(attribute))
                  for reference in reversed(references):
                      output_file.write(parseReference(reference))
                  output_file.write("}\n\n")
                  output_file.close()
          else:
              with open("laloVenues.gdm", 'a') as output_file:
                  output_file.write("entity " + entity["text"] + " {\n")
                  for attribute in reversed(attributes):
                      output_file.write(parseAttribute(attribute))
                  for reference in reversed(references):
                      output_file.write(parseReference(reference))
                  output_file.write("}\n\n")
                  output_file.close()
      return
    \end{minted}
\end{code}



El modelo GDM del código~\ref{code:er-to-gdm} está compuesta por entidades y consultas, cada entidad contiene \textit{features}, y cada \textit{feature} puede ser atributo o una referencia a otra entidad. Todos los atributos tienen un tipo y una cardinalidad.


Este algoritmo solo obtiene las entidades del modelo GDM. Se necesitan además consultas definidas por el usuario para generar el modelo lógico orientado a documetnos.