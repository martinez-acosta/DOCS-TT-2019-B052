\subsection{Transformación de una instancia GDM al modelo lógico orientado a documentos}
De acuerdo con la sección~\ref{alg:gdm-to-logic} y la sección~\ref{sec:xtend}, se implementará el algoritmo usando Xtend.

Cualquier archivo de texto con una definición correcta del lenguaje GDM será convertido por Mortadelo a un modelo orientado a documentos.


Como se ha explicado en la sección~\ref{alg:gdm-to-logic}, el algoritmo a implementar es una transformación entre modelos. A continuación se adjunta una breve descripción del algoritmo y su implementación.

\begin{algorithm}[H]
    \SetKwInOut{Input}{Input}
    \SetKwInOut{Output}{Output}

    %\underline{function Euclid} $(a,b)$\;
    \Input{una instancia del modelo GDM, $gdm$}
    \Output{un modelo lógico orientado a documentos, $ddm$}
    $mainEntities \gets gdm.queries.collect((q)|q.from);$\\
    \ForEach{$me \in mainEntities$}{
        $collection \gets new Collection();$\\
        $collection.name \gets me.name;$\\
        $accessTree \gets allQueryPaths(me,gdm.queries);$\\
        $collection \gets populateDocumentType(collection.root,accessTree);$\\
        $ddm.collections.add(collection);$
    }
    
    \caption{Transformación del modelo conceptual GDM al modelo lógico orientado a documentos}
\end{algorithm}

Donde la función populateDocumentType() es otro algoritmo de la forma:

\begin{algorithm}[H]
    \SetKwInOut{Input}{Input}
    \SetKwInOut{Output}{Output}

    %\underline{function Euclid} $(a,b)$\;
    \Input{Un ``document type", $dt$}
    \Output{un nodo del arbol de acceso}
    $ nodeAttributes \gets node.entity.features.select(f|f.isTypeOf(Attribute))\; $\\
    $ nodeReferences \gets node.entity.features.select(f|f.isTypeOf(Reference))\; $\\
    \ForEach{$attr \in nodeAttributes$}{
        $pf \gets new PrimitiveField()\;$\\
        $pf.name \gets attr.name\;$\\
        $pf.type \gets attr.type\;$\\
        $dt.fields.add(pf)\;$\\
    }
    \ForEach{$ref \in nodeReferences$}{
        $targetNode \gets node.arcs.find(a|a.name = ref.name).target\;$\\
        \uIf{exists(targetNode)}{
            $baseType \gets new DocumentType()\;$\\
            $populateDocumentType(baseType,targetNode)\;$\\
        }
        \Else{
            $baseType \gets new PrimitiveField()\;$\\
            $baseType.type \gets findIdType(ref.entity)\;$\\
        }
        $baseType.name \gets ref.name\;$\\

        \uIf{$ref.cardinality$ == 1}{
            $dt.field.add(baseType)\;$\\
            
        }
        \Else{
            $arrayField \gets new ArrayField()\;$\\
            $arrayField.type \gets baseType\;$\\
            $dt.fields.add(arrayField)\;$
        }
        
    }
    \caption{Generar el contenido de un \textit{DocumentType} dado un árbol de acceso}
\end{algorithm}

\begin{code}
    \captionof{listing}{Transformación modelo a modelo: GDM a modelo lógico orientado a documentos}
    \label{code:mortadelo-gdm-m2m}
    \begin{minted}[linenos,tabsize=2,breaklines, fontsize=\small]{python}

      def main():
    
      gdmModel = loadModel("Static_model_test.xmi")
      
      ddmModel = ddm.DocumentDataModel()
  
      # Obtenemos las entidades de los elementos From
      # mainEntities = gdm.queries.map[q | q.from.entity].toSet    
      mainEntities = set(map(lambda q: q.from_.entity, gdmModel.queries))
  
      # entityToQueries =  mainEntities.map[me | me -> gdm.queries.filter[q | q.from.entity.equals(me)]]
      entityToQueries = list(map(lambda me: (me, list(filter(lambda q: q.from_.entity.name == me.name, gdmModel.queries))), mainEntities))
          
      # val entity2accessTree = newImmutableMap(entityToQueries.map[e2q | e2q.key -> createAccessTree(e2q.value)])
      entity2accessTree = list(map(lambda e2q: ( e2q[0], createAccessTree(e2q[1])), entityToQueries))
  
      # Completamos cada árbol de acceso
      for entity in mainEntities:
        tree = getTree(entity2accessTree,entity)
        othersTrees = getAllTrees(entity2accessTree)
        completeAccessTree(entity, tree, othersTrees)
      
      # Generamos las colecciones
      for entity in mainEntities:
          tree = getTree(entity2accessTree,entity)
          collection = ddm.Collection()
          collection.name = entity.name
          docType = ddm.Document()
          docType.name = "root"
          collection.root = docType
          populateDocument(docType,entity,tree,mainEntities)
          ddmModel.collections.append(collection)
          saveModelDDM(ddmModel)
  
      saveModelDDM(ddmModel)

\end{minted}
\end{code}

La implementación del código fuente~\ref{code:mortadelo-gdm-m2m} toma de entrada una instancia del modelo GDM en XML (un .model generado por el \textit{parser implementado}) y de salida genera otro .model correspondiente al modelo lógico orientado a documentos.


Cada entidad principal tiene un arbol de acceso asociado, para generar los documentos anidados se hace uso de ese arbol de acceso.