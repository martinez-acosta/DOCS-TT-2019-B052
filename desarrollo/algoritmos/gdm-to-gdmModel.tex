\subsection{Parser del archivo de texto simple GDM a su .model}
Una vez obtenido el archivo de texto simple GDM que está conformado por las entidades (atributos y referencias) y las consultas definidas por el usuario es necesario generar un modelo que entienda PyEcore. A continuación se muestra el algoritmo y parte de su implementación.

\begin{algorithm}[H]
    \SetKwInOut{Input}{Entrada}
    \SetKwInOut{Output}{Salida}
  
    \Input{modelo GDM en texto simple, $gdmTextoSimple$}
    \Output{modelo GDM en XML, $gdmXML$}
        $entities \gets Entities(gdmTextoSimple)$\\
        $queries \gets Queries(gdmTextoSimple)$\\
        $gdm \gets gdm.Model(entities,queries)$\\
        $gdm.saveXML()$
     
      
    \caption{Generar el .model desde un modelo GDM en archivo de texto simple}
  \end{algorithm}


  \begin{code}
    \captionof{listing}{Parser GDM de texto simple a XML del GDM}
    \label{code:er-to-gdm}
    \begin{minted}[linenos,tabsize=2,breaklines, fontsize=\small]{python}
        def main():

        # Creamos el modelo GDM
        model = gdm.Model()
        
        input_file = open('laloVenues.gdm', 'r') 
        lines = input_file.readlines()
        
        # Primero creamos las entidades y las consultas, porque las necesitamos para crear las referencias
        for line in lines:
            if "entity" in line:
                entity = gdm.Entity(name=line.split()[1])
                model.entities.append(entity)
            if "query" in line:
                query = gdm.Query(name=line.split()[1].strip(":"))
                model.queries.append(query)
    
        # Parseamos el documento para agregar llenar las entidades
        for i in range(len(lines)):
            line = lines[i]
            
            # Ignoramos los comentarios
            if "/*" in line or "* " in line or "*/" in line:
                continue
            
            # Si es una entidad
            if "entity" in line:
                populateEntity(model, lines, i)
    
        # Parseamos el documento para agregar llenar las consultas
        for i in range(len(lines)):
            line = lines[i]
            # Si es una consulta            
            if "query" in line:
                populateQuery(model,lines,i)
    
        saveModel(model)
    \end{minted}
\end{code}