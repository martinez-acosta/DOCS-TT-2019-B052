\subsection{Diagramado del modelo entidad-relación}
De acuerdo con Shahzad~\cite{shahzad_review_2016}, GoJS es una biblioteca de JavaScript para implementar diagramas interactivos y visualizaciones en plataformas y navegadores web modernos. 



\begin{code}
\captionof{listing}{Diagrama GoJS}
\label{code:diagram-gojs}
\begin{minted}[linenos,tabsize=2,breaklines, fontsize=\small]{js}
    // Definimos el diagrama
    this.myDiagram = $(
       go.Diagram,
       'myDiagramDiv', 
       {
         grid: $(go.Panel, 'Grid'),
         'draggingTool.dragsLink': false,
         'draggingTool.isGridSnapEnabled': true,
         'linkingTool.isUnconnectedLinkValid': false,
         'linkingTool.portGravity': 10,
         'relinkingTool.isUnconnectedLinkValid': false,
         'relinkingTool.portGravity': 10,
         'relinkingTool.fromHandleArchetype': $(go.Shape, 'Diamond', {
           segmentIndex: 0,
           cursor: 'pointer',
           desiredSize: new go.Size(8, 8),
           fill: 'tomato',
           stroke: 'darkred'
         })
       }
     )

      /*******************Enlaces*************************/
      this.myDiagram.linkTemplate = $(
        go.Link, // definición de un enlace
        {
          selectionAdorned: true,
          selectionAdornmentTemplate: linkSelectionAdornmentTemplate,
          layerName: 'Foreground',
          reshapable: true,
          routing: go.Link.AvoidsNodes,
          curve: go.Link.JumpOver
        },
        $(
          go.TextBlock, // la cardinalidad del enlace
          {
            text: '',
            textAlign: 'center',
            font: 'bold 14px sans-serif',
            stroke: '#1967B3',
            editable: true,
            segmentOffset: new go.Point(0, -10),
            segmentOrientation: go.Link.OrientUpright
          },
          new go.Binding('text', 'cardinality').makeTwoWay()
        )
      )
      /*******************Paleta*************************/
      model: new go.GraphLinksModel([
        {
          type: 'entity',
          text: 'Entidad',
          figure: 'Rectangle',
          dataType: 'varchar',
          fill: 'white'
        },
        {
          type: 'weakEntity',
          text: 'Entidad débil',
          figure: 'FramedRectangle',
          dataType: 'varchar',
          fill: 'white'
        },
        {
          type: 'relation',
          text: 'Relación',
          figure: 'Diamond',
          dataType: 'varchar',
          fill: 'white'
        }
      ])
 \end{minted}
\end{code}


Como se especificó en la sección~\ref{ref:sec-gojs}, el diagrama se implementó en GoJS y parte del código fuente en~\ref{code:diagram-gojs} muestra una instancia de cómo se configura el \textit{canvas} del diagramador, los enlaces entre nodos y la definición de la paleta de elementos; como se nota todos son objetos \textit{json} desde su creación y para la manipulación de los objetos se hace uso de las bibliotecas integradas de GoJS para \textit{json}.

