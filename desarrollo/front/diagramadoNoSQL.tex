\subsection{Modelo lógico y físico NoSQL}

El módulo del modelo lógico y físico de un modelo de datos orientado a documentos permite ver gráficamente la estructura del modelo lógico por medio de la biblioteca de GoJS.

Respecto a los algoritmos usados, el algoritmo para la generación de sentencias de MongoDB desde la definición de un modelo lógico orientado a documentos revise la sección y el algoritmo usado para la generación del diagrama se muestra a continuación en el código fuente~\ref{code:ddm-to-gojs} que como entrada recibe un XML serializado por PyEcore de una instancia del modelo lógico orientado a documentos y de salida genera los nodos y links necesarios para diagramar con GoJS.

\begin{code}
    \captionof{listing}{Generación de los nodos y links del modelo orientado a documentos para el diagramador NoSQL}
    \label{code:ddm-to-gojs}
    \begin{minted}[linenos,tabsize=2,breaklines, fontsize=\small]{python}
        with open(input_file) as xml_file: 
      
        data_dict = xmltodict.parse(xml_file.read()) 
        xml_file.close() 
      
        # generate the object using json.dumps()  
        # corresponding to json data 
          
        #json_data = json.dumps(data_dict) 
    
        collections = data_dict['documentDataModel:DocumentDataModel']['collections']
        nodeDataArray = []
        linksDataArray = []
    
        for collection in collections:
            # Obtenemos el documento raíz
            root = collection['root']
            collection_data = {}
            collection_data['name'] = collection['@name'] 
            collection_data['subtype'] = 'Collection' 
            collection_data['key'] = generateKey()
            stack = LifoQueue(maxsize = 1000)
            collection_data['items'] = []
            
            # Cada elemento del documento raíz puede ser de tipo primitivo, documento o un arreglo
            for field in root['fields']:
                
                # Si es un tipo primitivo
                if ( field['@xsi:type'] == 'documentDataModel:PrimitiveField'):
                    collection_data['items'].append(getPrimitiveField(field))
                    print(field)
    
                # Si es un documento
                if ( field['@xsi:type'] == 'documentDataModel:Document'):
                    # Añadimos la referencia del documento
                    collection_data['items'].append({ 
                        'name': field['@name'],
                        'type': field['@xsi:type'],
                        'subtype': 'Document',
                        'parentKey': collection_data['key'], 
                        #'key': generateKey(),
                        'figure': 'Rectangle',
                        'color': '#FFD700'
                        }
                    )
                    field['parentKey'] = collection_data['key']
                    stack.put(field)
                    
                    print(field)
                
                # Si es un arreglo
                if ( field['@xsi:type'] == 'documentDataModel:ArrayField'):
                    
                     # Añadimos la referencia del arreglo
                     collection_data['items'].append({ 
                         'name': field['@name'],
                         'type': field['@xsi:type'],
                         'subtype': 'Array',
                         'parentKey': collection_data['key'], 
                         #'key': generateKey(),
                         'figure': 'Hexagon',
                         'color': '#6ea5f8'
                         }
                     )
                     field['parentKey'] = collection_data['key']
                     stack.put(field)
                     print(field)         
            
            while not stack.empty():
                child =  stack.get()
                
                # Si es un documento
                if ( child['@xsi:type'] == 'documentDataModel:Document'):
                    # Añadimos la referencia del documento
                    items = getItemsDocument(child)
                    new_key_parent = generateKey()
                    nodeDataArray.append({ 
                        'name': child['@name'],
                        'type': child['@xsi:type'],
                        'items': items,
                        'subtype': 'Document',
                        'parentKey': child['parentKey'],
                        'key': new_key_parent,
                        'figure': 'Rectangle',
                        'color': '#FFD700'
                        }
                    )
                    for i in child['fields']:
                        i['parentKey'] = new_key_parent
                        stack.put(i)
                print(field)
                
                # Si es un arreglo
                if ( child['@xsi:type'] == 'documentDataModel:ArrayField'):  
                    items = getItemsArray(child)
                     # Añadimos la referencia del arreglo
    
                    new_key_parent = generateKey()
                    nodeDataArray.append({ 
                         'name': child['@name'],
                         'type': child['@xsi:type'],
                         'items': items,
                         'subtype': 'Array',
                         'parentKey': child['parentKey'],
                         'key': new_key_parent,
                         'figure': 'Hexagon',
                         'color': '#6ea5f8'
                         }
                     )
                    child['type']['parentKey'] = new_key_parent
                    stack.put(child['type'])
    
                # Obtenemos contenidos de cada hijo recursivamente
    
    
    
            
            nodeDataArray.append(collection_data) 
        
        for node in nodeDataArray:
            if node['subtype'] != 'Collection':
                linksDataArray.append({
                    'to': node['parentKey'] , 'from': node['key']
                    })
            
        # Write the json data to output  
        # json file 
        data = {}
        json_nodeDataArray = json.dumps(nodeDataArray)
        json_linksDataArray = json.dumps(linksDataArray) 
        data["nodeDataArray"] = nodeDataArray 
        data["linkDataArray"] = linksDataArray
        ofile = "venuesLalo.json"
        with open(ofile, "w") as json_file:
            json_file.write(json.dumps(data))             
            json_file.close() 
    
\end{minted}
\end{code}