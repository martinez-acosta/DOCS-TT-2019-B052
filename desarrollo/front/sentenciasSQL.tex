\subsection{Sentencias SQL}
De acuerdo con los objetivos específicos de la sección~\ref{sec:objSpecifics}, se implementó la obtención de sentencias SQL desde el modelo entidad-relación. Para revisar el algoritmo asociado, por favor, revise la sección~\ref{sec;desarrollo-sql}. 
\begin{code}
    \captionof{listing}{Diagrama GoJS}
    \label{code:sentencias-sql}
    \begin{minted}[linenos,tabsize=2,breaklines, fontsize=\small]{js}
getSentencesSQL() {
    const diagram = this.currentDiagram
    this.$store
      .dispatch('axiosER/convertToSQL', {
        diagram,
        dbName: this.db_name
      })
      .then((response) => {
        this.sentences = response.data
      })
      .catch((error) => {
        if (error.response.status === 500) {
          this.$snotify.warning(
            '¡Algo ocurrió! No fue posible obtener las sentencias SQ del diagrama, intente más tarde.'
          )
        }
      })
    this.convertToSQLDialog = false
  },
  downloadScript() {
    const scriptData = encodeURIComponent(this.sentences)
    this.urlFile = `data:text/plain;charset=utf-8,${scriptData}` // application/sql
    const dbname = this.db_name ? this.db_name : 'tt2019-B052'
    this.scriptName = dbname + '.sql'
    this.$snotify.success('Archivo descargado. ')
  }

\end{minted}
\end{code}


El fragmento de código~\ref{code:sentencias-sql} obtiene el evento \textit{listener} del botón obtener sentencias SQL y realiza una llamada por medio de Axios al \textit{back end}. Posteriormente, recibe el esquema de sentencias SQL y lo muestra al usuario para que pueda descargarlo y probarlo en MySQL.