
\section{Front end}


Como se especificó en la sección~\ref{ref:sec-nuxt-framework}, el \textit{web framework} a utilizar es Nuxt; Nuxt usa un  ``ciclo de vida'' para cada componente Vue y se muestra en la figura~\ref{img:nuxt-lifecycle}.

\begin{figure}[H]
    \centering
    \includegraphics[width=0.5\textwidth]{nuxt/lifecycle-vue.png}
    \caption{Ciclo de vida de Vue/Nuxt}
    \label{img:nuxt-lifecycle}
  \end{figure}
  
Por la reactividad de Nuxt/Vue y la necesidad de implementar GoJS por sobre Nuxt/Vue, todas las instancias de GoJS se crean en el \textit{hook mounted} y se restaura el estado de cada componente usando Vuex; se eliminan todos los escuchas necesarios y se limpian los componentes de ser necesario en el \textit{hook beforeDestroy}.


De acuerdo con el sitio de Netlify~\cite{netlify_netlify_nodate}, Netlify ofrece funcionalidad HTTPS para todos sus sitios, tiene una mitigación activa en contra de ataques DDoS, todo su tráfico está cifrado en redes TLS y los \textit{tokens} están cifrados.

De acuerdo con el sitio de JWT~\cite{jwt_web_2020}, un JSON Web Token (JWT) es un estándar abierto (RFC 7519) que define una forma compacta y autónoma para transmitir información de forma segura entre las partes como un objeto JSON.


Esta información se puede verificar y confiar porque está firmada digitalmente; los JWT se firman usando un clave denominada ``secreto'' (con el algoritmo HMAC) o un par de claves pública/privada usando RSA o ECDSA.

Aunque los JWT se pueden cifrar para proporcionar ``secreto'' entre las partes, los \textit{jwt web tokens} se enfocan en \textit{tokens} firmados; los \textit{tokens} firmados pueden verificar la integridad de los reclamos que contiene, mientras que los \textit{tokens} cifrados ocultan esos reclamos de otras partes; cuando los \textit{tokens} se firman utilizando pares de claves públicas / privadas, la firma también certifica que solo la parte que posee la clave privada es la que la firmó.


La figura~\ref{img:prototipo-welcome} muestra la pantalla de bienvenida de la aplicación, donde el mensaje de iniciar sesión o registarse es visible si el usuario no ha iniciado sesión.

\begin{figure}[H]
    \centering
    \includegraphics[width=0.75\textwidth]{interfaz/home.png}
    \caption{Pantalla de bienvenida}
    \label{img:prototipo-welcome}
\end{figure}

La figura~\ref{img:prototipo-login} muestra la pantalla de inicio de sesión de la aplicación, donde se pide el correo y la contraseña para iniciar sesión.


\begin{figure}[H]
    \centering
    \includegraphics[width=0.75\textwidth]{interfaz/login.png}
    \caption{Pantalla de login}
    \label{img:prototipo-login}
\end{figure}
La figura~\ref{img:prototipo-signup} muestra el formulario de registro de la aplicación; para la validación de los campos se ha usado Vuelidate y muestra mensajes de error en caso de que algún campo esté rellenado incorrectamente.


\begin{figure}[H]
    \centering
    \includegraphics[width=0.75\textwidth]{interfaz/register.png}
    \caption{Pantalla de alta de usuario}
    \label{img:prototipo-signup}
\end{figure}
La figura~\ref{img:prototipo-er} muestra la vista del diagramador entidad-relación básico, del lado izquierdo están los botones para guardar y cargar un diagrama en formato .json válido para la aplicación; en la segunda columna está la zona de diagramado para crear un diagrama entidad-relación y del lado derecho está la paleta de elementos donde están los elementos de un diagrama entidad-relación básico; en la parte de abajo está el JSON equivalente del diagrama que está en la zona de diagramado y a su derecha está un menú para modificar algunas propiedades de los elementos del diagramador.

\begin{figure}[H]
    \centering
    \includegraphics[width=0.50\textwidth]{interfaz/er_diagramer.png}
    \caption{Pantalla del diagramador entidad-relación}
    \label{img:prototipo-er}
\end{figure}



Las figuras~\ref{img:prototipo-welcome},~\ref{img:prototipo-login},~\ref{img:prototipo-signup} y~\ref{img:prototipo-er} son capturas de pantalla del prototipo de la propuesta de solución, como es de notar el prototipo permite editar, guardar y cargar un diagrama entidad-relación básico.

\subsection*{Pantallas del diagramador entidad-relación}

La figura~\ref{img:app_diagrammerER} muestra lo que visualiza el usuario una vez que concluyó su registro o inició sesión; se puede apreciar las herramientas para crear/editar un diagrama entidad-relación, además de la zona de trabajo conocida como \textit{canvas}.

\begin{figure}[H]
    \centering
    \includegraphics[width=0.75\textwidth]{interfaz/er_diagramer.png}
    \caption{Pantalla del diagramador entidad-relación}
    \label{img:app_diagrammerER}
\end{figure}


La figura~\ref{img:app_errorDiagram} muestra los errores del diagrama entidad-relación después que el usuario hace click en el  botón ``validar diagrama'', en caso que el diagrama no cumpla con las reglas mencionadas en la sección \ref{cap:validationER}.

\begin{figure}[H]
    \centering
    \includegraphics[width=0.75\textwidth]{interfaz/invalid_diagramER.png}
    \caption{Pantalla de erroresal validar un diagrama.}
    \label{img:app_errorDiagram}
\end{figure}

La figura~\ref{img:app_validDiagram} muestra el modal que el usuario visualiza después de hacer click en el botón ``validar diagrama'' y este cumple con todas las reglas de validación estructural.

\begin{figure}[H]
    \centering
    \includegraphics[width=0.75\textwidth]{interfaz/valid_diagramER.png}
    \caption{Pantalla de erroresal validar un diagrama.}
    \label{img:app_validDiagram}
\end{figure}

\subsection*{Pantallas de las sentencias SQL equivalentes}

La figura~\ref{img:app_sqlSentences} muestra las pantallas del módulo de obtención de las sentencias equivalentes del diagrama que el usuario generó en la figura~\ref{img:app_diagrammerER}; este paso solo es posible después de haber pasado por el proceso de validación para el diagrama entidad-relación.
Del lado derecho en la figura~\ref{img:app_sqlScript} se aprecia el código en el lenguaje SQL necesario para crear la base de datos relacional en el sistema gestor de base de datos MySQL, y del lado izquierdo en la figura~\ref{img:app_dbName} se muestra el modal que el usuario visualiza al hacer click en el botón ``Obtener sentencias SQL'' en el cual deberá colocar el nombre que tendrá la base de datos a generar.

Si el usuario necesita exportar el script de sql a un archivo, cuenta con un botón en la parte superior para realizar esta acción.

\begin{figure}[H]
    \begin{subfigure}[b]{0.49\textwidth}
        \includegraphics[width=\textwidth]{interfaz/sql_sentences.png}
        \caption{Sentencencias SQL equivalentes al diagram ER.}
        \label{img:app_sqlScript}
      \end{subfigure}
      \hfill
      \begin{subfigure}[b]{0.49\textwidth}
        \includegraphics[width=\textwidth]{interfaz/get_sql_sentences.png}
        \caption{Modal para nombrar la base de datos sql.}
        \label{img:app_dbName}
      \end{subfigure}
    \caption{Pantallas de las sentencias SQL equivalentes al diagrama ER.}
    \label{img:app_sqlSentences}
\end{figure}

\subsection*{Pantallas de las consultas de acceso}

La figura~\ref{img:app_simpleQuery} es lo que el usuario visualiza al ingresar a este módulo, este paso solo es posible después de haber pasado por el proceso de validación para el diagrama entidad-relación. Es aquí donde puede agregar las consultas de acceso que desea que sea utilicen en el proceso de tranformación al modelo noSQL teniendo del lado izquierdo el diagrama ER en modo de solo lectura y al hacer click derecho en los atributos clave de una entidad vusualizará un menú con las opciones para generar dicha consulta.

Las consultas de acceso pueden ser tantas con crea necesitarlas como se aprecia en la figura~\ref{img:app_multipleQueries}, es importante mencionar que puede agregar tantos elementos al apartado ``Respecto al atributo'' como quiera pero debe existir al menos un elemento en el apartado ``Encontrar'' para que la consulta tenga sentido.

\begin{figure}[H]
    \centering
    \includegraphics[width=\textwidth]{interfaz/queries_simple.png}
    \caption{Pantalla para agregar una consulta de acceso.}
    \label{img:app_simpleQuery}
\end{figure}

\begin{figure}[H]
    \centering
    \includegraphics[width=\textwidth]{interfaz/queries_multiple.png}
    \caption{Pantalla con multiples consultas de acceso.}
    \label{img:app_multipleQueries}
\end{figure}


