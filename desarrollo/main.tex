En este capítulo se muestra cómo se implementa cada uno de los objetivos~\ref{sec:objSpecifics} de la propuesta de solución, explicando cada función principal del \textit{front end}, del \textit{back end}, y explicando los algoritmos de la sección~\ref{sec:analisis-algoritmos}, empezando por la tranformación del diagrama entidad-relación al esquema relacional de sentencias SQL, la transformación modelo a modelo y modelo a texto del modelo NoSQL.

Asimismo, también se muestran fragmentos del código fuente del \textit{front end}, \textit{back end} y algoritmos. 


Para cumplir con los objetivos~\ref{sec:objSpecifics}. la propuesta de solución usa para la autenticación de usuarios \textit{web tokens}; se dispone del código fuente y un \textit{live demo} de la aplicación.


El \textit{front end} (Nuxt/JS/GoJS) lo puede consultar en \href{https://github.com/martinez-acosta/TT-2019-B052}{repositorio front}, el \textit{back end} (Python/Flask) en \href{https://github.com/omaraparicio07/api-tt-2019-b052}{repositorio back}, el documento de TT (Latex) en \href{https://github.com/martinez-acosta/DOCS-TT-2019-B052}{repositorio documento} y el \textit{live demo} está disponible en \href{https://serene-haibt-2239b4.netlify.app/}{dirección}.


El \textit{live demo} del \textit{front end} está usando el \textit{deploy} automático en Netlify de la rama master y el código de desarrollo está en la rama dev o en las ramas \textit{feature}.


Asimismo, en la página de GitHub del proyecto de \textit{front end} es posible encontrar de manera breve un resumen del propósito de la aplicación web, estado actual del proyecto, tecnologías usadas y cosas que faltan por implementar.


El \textit{live demo} del \textit{back end} está usando el \textit{deploy} automático en Heroku de la rama master y el código de desarrollo está en las ramas \textit{feature}.



\section{Front end}


Como se especificó en la sección~\ref{ref:sec-nuxt-framework}, el \textit{web framework} a utilizar es Nuxt; Nuxt usa un  ``ciclo de vida'' para cada componente Vue y se muestra en la figura~\ref{img:nuxt-lifecycle}.

\begin{figure}[H]
    \centering
    \includegraphics[width=0.5\textwidth]{nuxt/lifecycle-vue.png}
    \caption{Ciclo de vida de Vue/Nuxt}
    \label{img:nuxt-lifecycle}
  \end{figure}
  
Por la reactividad de Nuxt/Vue y la necesidad de implementar GoJS por sobre Nuxt/Vue, todas las instancias de GoJS se crean en el \textit{hook mounted} y se restaura el estado de cada componente usando Vuex; se eliminan todos los escuchas necesarios y se limpian los componentes de ser necesario en el \textit{hook beforeDestroy}.


De acuerdo con el sitio de Netlify~\cite{netlify_netlify_nodate}, Netlify ofrece funcionalidad HTTPS para todos sus sitios, tiene una mitigación activa en contra de ataques DDoS, todo su tráfico está cifrado en redes TLS y los \textit{tokens} están cifrados.

De acuerdo con el sitio de JWT~\cite{jwt_web_2020}, un JSON Web Token (JWT) es un estándar abierto (RFC 7519) que define una forma compacta y autónoma para transmitir información de forma segura entre las partes como un objeto JSON.


Esta información se puede verificar y confiar porque está firmada digitalmente; los JWT se firman usando un clave denominada ``secreto'' (con el algoritmo HMAC) o un par de claves pública/privada usando RSA o ECDSA.

Aunque los JWT se pueden cifrar para proporcionar ``secreto'' entre las partes, los \textit{jwt web tokens} se enfocan en \textit{tokens} firmados; los \textit{tokens} firmados pueden verificar la integridad de los reclamos que contiene, mientras que los \textit{tokens} cifrados ocultan esos reclamos de otras partes; cuando los \textit{tokens} se firman utilizando pares de claves públicas / privadas, la firma también certifica que solo la parte que posee la clave privada es la que la firmó.


La figura~\ref{img:prototipo-welcome} muestra la pantalla de bienvenida de la aplicación, donde el mensaje de iniciar sesión o registarse es visible si el usuario no ha iniciado sesión.

\begin{figure}[H]
    \centering
    \includegraphics[width=0.75\textwidth]{interfaz/home.png}
    \caption{Pantalla de bienvenida}
    \label{img:prototipo-welcome}
\end{figure}

La figura~\ref{img:prototipo-login} muestra la pantalla de inicio de sesión de la aplicación, donde se pide el correo y la contraseña para iniciar sesión.


\begin{figure}[H]
    \centering
    \includegraphics[width=0.75\textwidth]{interfaz/login.png}
    \caption{Pantalla de login}
    \label{img:prototipo-login}
\end{figure}
La figura~\ref{img:prototipo-signup} muestra el formulario de registro de la aplicación; para la validación de los campos se ha usado Vuelidate y muestra mensajes de error en caso de que algún campo esté rellenado incorrectamente.


\begin{figure}[H]
    \centering
    \includegraphics[width=0.75\textwidth]{interfaz/register.png}
    \caption{Pantalla de alta de usuario}
    \label{img:prototipo-signup}
\end{figure}
La figura~\ref{img:prototipo-er} muestra la vista del diagramador entidad-relación básico, del lado izquierdo están los botones para guardar y cargar un diagrama en formato .json válido para la aplicación; en la segunda columna está la zona de diagramado para crear un diagrama entidad-relación y del lado derecho está la paleta de elementos donde están los elementos de un diagrama entidad-relación básico; en la parte de abajo está el JSON equivalente del diagrama que está en la zona de diagramado y a su derecha está un menú para modificar algunas propiedades de los elementos del diagramador.

\begin{figure}[H]
    \centering
    \includegraphics[width=0.50\textwidth]{interfaz/er_diagramer.png}
    \caption{Pantalla del diagramador entidad-relación}
    \label{img:prototipo-er}
\end{figure}



Las figuras~\ref{img:prototipo-welcome},~\ref{img:prototipo-login},~\ref{img:prototipo-signup} y~\ref{img:prototipo-er} son capturas de pantalla del prototipo de la propuesta de solución, como es de notar el prototipo permite editar, guardar y cargar un diagrama entidad-relación básico.

\subsection*{Pantallas del diagramador entidad-relación}

La figura~\ref{img:app_diagrammerER} muestra lo que visualiza el usuario una vez que concluyó su registro o inició sesión; se puede apreciar las herramientas para crear/editar un diagrama entidad-relación, además de la zona de trabajo conocida como \textit{canvas}.

\begin{figure}[H]
    \centering
    \includegraphics[width=0.75\textwidth]{interfaz/er_diagramer.png}
    \caption{Pantalla del diagramador entidad-relación}
    \label{img:app_diagrammerER}
\end{figure}


La figura~\ref{img:app_errorDiagram} muestra los errores del diagrama entidad-relación después que el usuario hace click en el  botón ``validar diagrama'', en caso que el diagrama no cumpla con las reglas mencionadas en la sección \ref{cap:validationER}.

\begin{figure}[H]
    \centering
    \includegraphics[width=0.75\textwidth]{interfaz/invalid_diagramER.png}
    \caption{Pantalla de erroresal validar un diagrama.}
    \label{img:app_errorDiagram}
\end{figure}

La figura~\ref{img:app_validDiagram} muestra el modal que el usuario visualiza después de hacer click en el botón ``validar diagrama'' y este cumple con todas las reglas de validación estructural.

\begin{figure}[H]
    \centering
    \includegraphics[width=0.75\textwidth]{interfaz/valid_diagramER.png}
    \caption{Pantalla de erroresal validar un diagrama.}
    \label{img:app_validDiagram}
\end{figure}

\subsection*{Pantallas de las sentencias SQL equivalentes}

La figura~\ref{img:app_sqlSentences} muestra las pantallas del módulo de obtención de las sentencias equivalentes del diagrama que el usuario generó en la figura~\ref{img:app_diagrammerER}; este paso solo es posible después de haber pasado por el proceso de validación para el diagrama entidad-relación.
Del lado derecho en la figura~\ref{img:app_sqlScript} se aprecia el código en el lenguaje SQL necesario para crear la base de datos relacional en el sistema gestor de base de datos MySQL, y del lado izquierdo en la figura~\ref{img:app_dbName} se muestra el modal que el usuario visualiza al hacer click en el botón ``Obtener sentencias SQL'' en el cual deberá colocar el nombre que tendrá la base de datos a generar.

Si el usuario necesita exportar el script de sql a un archivo, cuenta con un botón en la parte superior para realizar esta acción.

\begin{figure}[H]
    \begin{subfigure}[b]{0.49\textwidth}
        \includegraphics[width=\textwidth]{interfaz/sql_sentences.png}
        \caption{Sentencencias SQL equivalentes al diagram ER.}
        \label{img:app_sqlScript}
      \end{subfigure}
      \hfill
      \begin{subfigure}[b]{0.49\textwidth}
        \includegraphics[width=\textwidth]{interfaz/get_sql_sentences.png}
        \caption{Modal para nombrar la base de datos sql.}
        \label{img:app_dbName}
      \end{subfigure}
    \caption{Pantallas de las sentencias SQL equivalentes al diagrama ER.}
    \label{img:app_sqlSentences}
\end{figure}

\subsection*{Pantallas de las consultas de acceso}

La figura~\ref{img:app_simpleQuery} es lo que el usuario visualiza al ingresar a este módulo, este paso solo es posible después de haber pasado por el proceso de validación para el diagrama entidad-relación. Es aquí donde puede agregar las consultas de acceso que desea que sea utilicen en el proceso de tranformación al modelo noSQL teniendo del lado izquierdo el diagrama ER en modo de solo lectura y al hacer click derecho en los atributos clave de una entidad vusualizará un menú con las opciones para generar dicha consulta.

Las consultas de acceso pueden ser tantas con crea necesitarlas como se aprecia en la figura~\ref{img:app_multipleQueries}, es importante mencionar que puede agregar tantos elementos al apartado ``Respecto al atributo'' como quiera pero debe existir al menos un elemento en el apartado ``Encontrar'' para que la consulta tenga sentido.

\begin{figure}[H]
    \centering
    \includegraphics[width=\textwidth]{interfaz/queries_simple.png}
    \caption{Pantalla para agregar una consulta de acceso.}
    \label{img:app_simpleQuery}
\end{figure}

\begin{figure}[H]
    \centering
    \includegraphics[width=\textwidth]{interfaz/queries_multiple.png}
    \caption{Pantalla con multiples consultas de acceso.}
    \label{img:app_multipleQueries}
\end{figure}



\subsection{Diagramado del modelo entidad-relación}
De acuerdo con Shahzad~\cite{shahzad_review_2016}, GoJS es una biblioteca de JavaScript para implementar diagramas interactivos y visualizaciones en plataformas y navegadores web modernos. 



\begin{code}
\captionof{listing}{Diagrama GoJS}
\label{code:diagram-gojs}
\begin{minted}[linenos,tabsize=2,breaklines, fontsize=\small]{js}
    // Definimos el diagrama
    this.myDiagram = $(
       go.Diagram,
       'myDiagramDiv', 
       {
         grid: $(go.Panel, 'Grid'),
         'draggingTool.dragsLink': false,
         'draggingTool.isGridSnapEnabled': true,
         'linkingTool.isUnconnectedLinkValid': false,
         'linkingTool.portGravity': 10,
         'relinkingTool.isUnconnectedLinkValid': false,
         'relinkingTool.portGravity': 10,
         'relinkingTool.fromHandleArchetype': $(go.Shape, 'Diamond', {
           segmentIndex: 0,
           cursor: 'pointer',
           desiredSize: new go.Size(8, 8),
           fill: 'tomato',
           stroke: 'darkred'
         })
       }
     )

      /*******************Enlaces*************************/
      this.myDiagram.linkTemplate = $(
        go.Link, // definición de un enlace
        {
          selectionAdorned: true,
          selectionAdornmentTemplate: linkSelectionAdornmentTemplate,
          layerName: 'Foreground',
          reshapable: true,
          routing: go.Link.AvoidsNodes,
          curve: go.Link.JumpOver
        },
        $(
          go.TextBlock, // la cardinalidad del enlace
          {
            text: '',
            textAlign: 'center',
            font: 'bold 14px sans-serif',
            stroke: '#1967B3',
            editable: true,
            segmentOffset: new go.Point(0, -10),
            segmentOrientation: go.Link.OrientUpright
          },
          new go.Binding('text', 'cardinality').makeTwoWay()
        )
      )
      /*******************Paleta*************************/
      model: new go.GraphLinksModel([
        {
          type: 'entity',
          text: 'Entidad',
          figure: 'Rectangle',
          dataType: 'varchar',
          fill: 'white'
        },
        {
          type: 'weakEntity',
          text: 'Entidad débil',
          figure: 'FramedRectangle',
          dataType: 'varchar',
          fill: 'white'
        },
        {
          type: 'relation',
          text: 'Relación',
          figure: 'Diamond',
          dataType: 'varchar',
          fill: 'white'
        }
      ])
 \end{minted}
\end{code}


Como se especificó en la sección~\ref{ref:sec-gojs}, el diagrama se implementó en GoJS y parte del código fuente en~\ref{code:diagram-gojs} muestra una instancia de cómo se configura el \textit{canvas} del diagramador, los enlaces entre nodos y la definición de la paleta de elementos; como se nota todos son objetos \textit{json} desde su creación y para la manipulación de los objetos se hace uso de las bibliotecas integradas de GoJS para \textit{json}.


\subsection{Modelo conceptual NoSQL}

El módulo del modelo conceptual NoSQL está implementado en Nuxt, haciendo uso de Vuex para guardar las entidades del GDM.

El algoritmo y el código fuente de la transformación del modelo entidad-relación básico a las entidades del GDM lo puede consultar en .

\subsection{Sentencias SQL}
De acuerdo con los objetivos específicos de la sección~\ref{sec:objSpecifics}, se implementó la obtención de sentencias SQL desde el modelo entidad-relación. Para revisar el algoritmo asociado, por favor, revise la sección~\ref{sec;desarrollo-sql}. 
\begin{code}
    \captionof{listing}{Diagrama GoJS}
    \label{code:sentencias-sql}
    \begin{minted}[linenos,tabsize=2,breaklines, fontsize=\small]{js}
getSentencesSQL() {
    const diagram = this.currentDiagram
    this.$store
      .dispatch('axiosER/convertToSQL', {
        diagram,
        dbName: this.db_name
      })
      .then((response) => {
        this.sentences = response.data
      })
      .catch((error) => {
        if (error.response.status === 500) {
          this.$snotify.warning(
            '¡Algo ocurrió! No fue posible obtener las sentencias SQ del diagrama, intente más tarde.'
          )
        }
      })
    this.convertToSQLDialog = false
  },
  downloadScript() {
    const scriptData = encodeURIComponent(this.sentences)
    this.urlFile = `data:text/plain;charset=utf-8,${scriptData}` // application/sql
    const dbname = this.db_name ? this.db_name : 'tt2019-B052'
    this.scriptName = dbname + '.sql'
    this.$snotify.success('Archivo descargado. ')
  }

\end{minted}
\end{code}


El fragmento de código~\ref{code:sentencias-sql} obtiene el evento \textit{listener} del botón obtener sentencias SQL y realiza una llamada por medio de Axios al \textit{back end}. Posteriormente, recibe el esquema de sentencias SQL y lo muestra al usuario para que pueda descargarlo y probarlo en MySQL.
\subsection{Modelo lógico y físico NoSQL}

El módulo del modelo lógico y físico de un modelo de datos orientado a documentos permite ver gráficamente la estructura del modelo lógico por medio de la biblioteca de GoJS.

Respecto a los algoritmos usados, el algoritmo para la generación de sentencias de MongoDB desde la definición de un modelo lógico orientado a documentos revise la sección y el algoritmo usado para la generación del diagrama se muestra a continuación en el código fuente~\ref{code:ddm-to-gojs} que como entrada recibe un XML serializado por PyEcore de una instancia del modelo lógico orientado a documentos y de salida genera los nodos y links necesarios para diagramar con GoJS.

\begin{code}
    \captionof{listing}{Generación de los nodos y links del modelo orientado a documentos para el diagramador NoSQL}
    \label{code:ddm-to-gojs}
    \begin{minted}[linenos,tabsize=2,breaklines, fontsize=\small]{python}
        with open(input_file) as xml_file: 
      
        data_dict = xmltodict.parse(xml_file.read()) 
        xml_file.close() 
      
        # generate the object using json.dumps()  
        # corresponding to json data 
          
        #json_data = json.dumps(data_dict) 
    
        collections = data_dict['documentDataModel:DocumentDataModel']['collections']
        nodeDataArray = []
        linksDataArray = []
    
        for collection in collections:
            # Obtenemos el documento raíz
            root = collection['root']
            collection_data = {}
            collection_data['name'] = collection['@name'] 
            collection_data['subtype'] = 'Collection' 
            collection_data['key'] = generateKey()
            stack = LifoQueue(maxsize = 1000)
            collection_data['items'] = []
            
            # Cada elemento del documento raíz puede ser de tipo primitivo, documento o un arreglo
            for field in root['fields']:
                
                # Si es un tipo primitivo
                if ( field['@xsi:type'] == 'documentDataModel:PrimitiveField'):
                    collection_data['items'].append(getPrimitiveField(field))
                    print(field)
    
                # Si es un documento
                if ( field['@xsi:type'] == 'documentDataModel:Document'):
                    # Añadimos la referencia del documento
                    collection_data['items'].append({ 
                        'name': field['@name'],
                        'type': field['@xsi:type'],
                        'subtype': 'Document',
                        'parentKey': collection_data['key'], 
                        #'key': generateKey(),
                        'figure': 'Rectangle',
                        'color': '#FFD700'
                        }
                    )
                    field['parentKey'] = collection_data['key']
                    stack.put(field)
                    
                    print(field)
                
                # Si es un arreglo
                if ( field['@xsi:type'] == 'documentDataModel:ArrayField'):
                    
                     # Añadimos la referencia del arreglo
                     collection_data['items'].append({ 
                         'name': field['@name'],
                         'type': field['@xsi:type'],
                         'subtype': 'Array',
                         'parentKey': collection_data['key'], 
                         #'key': generateKey(),
                         'figure': 'Hexagon',
                         'color': '#6ea5f8'
                         }
                     )
                     field['parentKey'] = collection_data['key']
                     stack.put(field)
                     print(field)         
            
            while not stack.empty():
                child =  stack.get()
                
                # Si es un documento
                if ( child['@xsi:type'] == 'documentDataModel:Document'):
                    # Añadimos la referencia del documento
                    items = getItemsDocument(child)
                    new_key_parent = generateKey()
                    nodeDataArray.append({ 
                        'name': child['@name'],
                        'type': child['@xsi:type'],
                        'items': items,
                        'subtype': 'Document',
                        'parentKey': child['parentKey'],
                        'key': new_key_parent,
                        'figure': 'Rectangle',
                        'color': '#FFD700'
                        }
                    )
                    for i in child['fields']:
                        i['parentKey'] = new_key_parent
                        stack.put(i)
                print(field)
                
                # Si es un arreglo
                if ( child['@xsi:type'] == 'documentDataModel:ArrayField'):  
                    items = getItemsArray(child)
                     # Añadimos la referencia del arreglo
    
                    new_key_parent = generateKey()
                    nodeDataArray.append({ 
                         'name': child['@name'],
                         'type': child['@xsi:type'],
                         'items': items,
                         'subtype': 'Array',
                         'parentKey': child['parentKey'],
                         'key': new_key_parent,
                         'figure': 'Hexagon',
                         'color': '#6ea5f8'
                         }
                     )
                    child['type']['parentKey'] = new_key_parent
                    stack.put(child['type'])
    
                # Obtenemos contenidos de cada hijo recursivamente
    
    
    
            
            nodeDataArray.append(collection_data) 
        
        for node in nodeDataArray:
            if node['subtype'] != 'Collection':
                linksDataArray.append({
                    'to': node['parentKey'] , 'from': node['key']
                    })
            
        # Write the json data to output  
        # json file 
        data = {}
        json_nodeDataArray = json.dumps(nodeDataArray)
        json_linksDataArray = json.dumps(linksDataArray) 
        data["nodeDataArray"] = nodeDataArray 
        data["linkDataArray"] = linksDataArray
        ofile = "venuesLalo.json"
        with open(ofile, "w") as json_file:
            json_file.write(json.dumps(data))             
            json_file.close() 
    
\end{minted}
\end{code}
\section{Back end}


De acuerdo con el sitio de Heroku~\cite{heroku_heroku_2020}, la plataforma ofrece distintos mecanismos para la seguridad de los proyectos que aloja, uno de estos mecanismos es el uso por defecto del protocolo HTTPS, que asegura el cifrado de los datos en la Internet.

De igual manera, ofrece un constante escaneo de las aplicaciones en búsqueda de vulnerabilidades para mitigar los ataques DDoS, además de utilizar la infraestructura de la empresa Amazon, las cuales se encuentran acreditadas por diversos estándares de seguridad; esto da la confianza para que la aplicación Flask se ejecute de manera segura y tener salvaguardados los datos de los usuarios.

De acuerdo con el sitio Pallets Projects~\cite{noauthor_pallets_2020}, flask es un \textit{framework} web escrito en python. Está diseñado para que el desarrollo de una aplicación web sea rápida y sencilla sin imponer ninguna dependencia o diseño sobre el proyecto, porque depende del desarrollador implementar las bibliotecas o herramientas que desee utilizar y, al contar con una comunidad amplia, hay una gran cantidad de extensiones para ampliar sus funcionalidades.
\subsection{El api flask} \label{sec:apiFlask}

Como se mencionó en la sección~\ref{sec:flask}, flask es un \textit{microframework}, pero es posible extender sus capacidades por medio de bibliotecas. Para el desarrollo de la api de servicios web se utiliza la biblioteca \textit{flask restplus}, que de acuerdo con su documentación oficial~\cite{noauthor_flask-restplus_2020}, es una extensión de flask con soporte para el desarrollo de API REST de manera rápida, fomenta buenas prácticas con una configuración mínima y proporciona una colección de decoradores y herramientas para describir un API de manera sencilla y exponer su documentación correctamente usando Swagger.

Para el desarrollo de la aplicación, se creó la estructura mostrada en la figura~\ref{img:backStructure} donde la carpeta \textit{config} contiene un archivo con el nombre \textit{app.conf} con las variables de configuración de la aplicación, como son la llave secreta para validar el JWT, credenciales para conexión a la base de datos, un usuario de pruebas y características de la interfaz swagger. El archivo \textit{database.py} es una clase \textit{singleton} para mantener una sola instancia de conexión a la base de datos para toda la aplicación.

\begin{figure}[H]
  \centering
  \includegraphics[width=0.5\textwidth]{apiFlask/api_structure.png}
  \caption{Estructura del proyecto backend.}
  \label{img:backStructure}
\end{figure}

Para lograr la implementación de flask con restplus es necesaria la siguiente configuración: se deben importar de la biblioteca \textit{flask} en el archivo principal del proyecto las clases \textit{Flask, jsonify, request, make\_response, Response} y, al mismo tiempo, la configuración descrita en el archivo \textit{config/app.conf} para tener disponible en toda la aplicación la conexión a la base de datos como la llave secreta del \textit{token} de autenticación y las credenciales del servidore SMTP para el envío de correo. Dado como resultado el fragmento de código \ref{code:apiInit}.

\begin{code}
\captionof{listing}{Clase principal del api flask}
\label{code:apiInit}
\begin{minted}[linenos,tabsize=2,breaklines, fontsize=\small]{python}
  from flask import Flask, jsonify, request, make_response, Response
  from flask_cors import CORS
  from flask_pymongo import PyMongo
  import os
  from apis import api
  from config.database import initialize_db

  app = Flask(__name__)
  app.config.from_pyfile(os.path.join(".", "config/app.conf"), silent=False)

  mongo = PyMongo(app)
  CORS(app, resources={r'/*': {'origins': '*'}})
  initialize_db(app)
  api.init_app(app)

  if __name__=='__main__':
    app.run(debug=True)

\end{minted}
\end{code}

\subsection{Los servicios web}

Como se muestra en la sección \ref{sec:apiFlask} eso basta para tener una aplicación flask ejecutandose, pero para poder exponer servicios web y que estos puedan ser consultados por otras aplicaciones son necesarios pasos adicionales descrios en esta sección. Para poder registrar un servicio en la aplicación flask primero se debe configurar la documentación de swagger, para lograr esto se recomienda en un archivo separado mantener la importación de los distintos namespaces de los servicios que desea exponer y es la clase \textbf{API} de restplus quien se encargará de registrarlos en el api y mosntrarlos al usuario como se muestra en el fragmento de código \ref{code:namespaces}, también es posible apreciar en este fragmento de código la configuración del \textit{header} de autenticación para poder hacer uso de los servicios así como la infomación del api, como el nombre de los desarrolladores, su email de contacto, la versión del api, si utiliza algún mecanismo de segirdad en la autenticación, etc.

\begin{code}
\captionof{listing}{Registro de los namespaces del api}
\label{code:namespaces}
\begin{minted}[linenos,tabsize=2,breaklines, fontsize=\small]{python}
from flask_restplus import Api
from .user import api as ns_user
from .login import api as ns_login
from .diagram import api as ns_diagram
from .relational_model import api as ns_relational

#Authorization
authorizations = {
  'Bearer Auth': {
    'type': 'apiKey',
    'in': 'header',
    'name': 'Authorization'
  }
}

api = Api(
  title='API TT 2019-B052',
  version='0.10.0',
  description='api de servicios para el trabajo termnal 2019-B052 <style>.models {display: none !important}</style>',
  contact="Omar Aparicio Quiroz, Eduardo Acosta Martinez",
  contact_email="omaraparicio07@gmail.com, fx2013630461@gmail.com",
  security='Bearer Auth',
  authorizations=authorizations,
  RESTPLUS_MASK_SWAGGER=False
  # All API metadatas
)


api.add_namespace(ns_user)
api.add_namespace(ns_login)
api.add_namespace(ns_diagram)
api.add_namespace(ns_relational)
\end{minted}
\end{code}

De la misma manera es recomendable tener separados cada uno de los namespaces en archivos separados, esto hace mas legible la integración de nuevos servicios al api. Como se aprecia el figura~\ref{img:apiStrucure} todos estos archivos se encuentran en la carpeta \textit{apis/} tanto la configuración como cada uno de los servicios expuestos asociados por contexto, es decir, todos los servicios relacionados con los datos del usuario se encuentran en \textit{apis/user.py}.

Para crear un recurso de un nuevo servicio web debemos agregar el decorador \textit{@api.route()} antes de una clase o método para mapear la ruta por la que el usuario podrá acceder a ese servicio, la clase debe tener como parámetro la clase \textit{Resource} de restplus para identificarlo como un elemento del api. Una de las buenas practicas que implementa la herramienta restplus es la limitante que por cada clase solamente es posible tener definido un método por cada tipo de petición HTTP, es decir, por cada clase solo se permite una método \textit{def get(self):} que atenderá la peticion http que tenga en el header \textit{Request Method} la opción \textit{GET}.

De manera adicional se implemento un decorador personalizado para validar que las peticiones que tengan que hacer uso de los datos del usuario o la base de datos cuenten con el token de autenticación en los headers de la petición. Este decorador se muestra en el fragmento de código \ref{code:tokenWrap}, una vez creado el decorador basta con colocarlo previamente a la definición de la clase o método que requiera de su uso.

\begin{code}
\captionof{listing}{Registro de los namespaces del api}
\label{code:tokenWrap}
\begin{minted}[linenos,tabsize=2,breaklines, fontsize=\small]{python}
  def token_required(f):
  @wraps(f)
  def decorated(*args, **kwargs):
    try:
      token = request.headers['Authorization']
    except:
      response = {
        'message': 'Token no encontrado en la petición'
      }
      return response,401

    if "Bearer " in token:
      token = token[token.index(' ')+1:]

    try:
      jwt.decode(token, current_app.config['SECRET_KEY'])
    except:
      response = {
        'message': 'Token inválido'
      }
      return response,401

    return f(*args, **kwargs)

  return decorated
\end{minted}
\end{code}
\subsection{Servicios para login}

Este \textit{namespace} tiene los servicios relacionados con la autenticación del usuario. Como es de esperarse, este \textit{namespace} no requiere del uso del decorador \ref{code:tokenWrap}, ya que es aquí donde un usuario previamente registrado generará un \textit{token} de autenticación haciendo uso de su correo electrónico y contraseña. De la misma forma, si este quiere cambiar su contraseña puede hacerlo. La figura~\ref{img:putLogin} muestra el uso del método \textit{PUT} para el cambio de contraseña, proporcionando un correo electrónico registrado en la base de datos. Una vez que el cambio de contraseña sea efectuado, el usuario recibirá un correo de confirmación como se muestra en la figura~\ref{img:changePass}.

\begin{figure}[H]
  \centering
  \includegraphics[width=0.9\textwidth]{apiFlask/putLogin.png}
  \caption{Método PUT del servicio para cambio de contraseña de un usuario.}
  \label{img:putLogin}
\end{figure}

\begin{figure}[H]
  \centering
  \includegraphics[width=0.9\textwidth]{apiFlask/emailChangePass.png}
  \caption{Correo de confirmación para cambio de contraseña.}
  \label{img:changePass}
\end{figure}

El servicio de login regresa al usuario un token de autenticación como se muestra en la figura~\ref{img:loginApi}. El usuario debe almacenar el token para poder proporcionarlo al api en cada petición que requiera de permisos para realizar las operaciones.


\begin{figure}[H]
  \centering
  \includegraphics[width=0.9\textwidth]{apiFlask/login_api.png}
  \caption{Generación del token de autenticación.}
  \label{img:loginApi}
\end{figure}
\subsection{Servicios para el usuario}

Los métodos o funciones implementadas en este \textit{namespace} constan de un CRUD (\textit{Create, Read, Update, Delete}) para el manejo de usuarios. Es aquí donde se permite el registro de un usuario nuevo, así como la consulta de todos los usuarios registrados en la base de datos, así como la consulta de un usuario en específico. Para limitar los datos que el servicio acepta en una petición se puede hacer uso del método model que proporciona la clase \textit{Namespace} de restplus, como se aprecia en el fragmento de código \ref{code:modelUser} el modelo de un usuario conta de los campos \textit{\_id, username, name, email, password, diagram} de los cuales \textit{\_id} y \textit{diagram} no son obligatorios para el registro de un nuevo usuario. Toda la lógica de esta operaciones se encuentra definida en el archivo \textit{apis/user.py} dando como resultado la figura~\ref{img:userEndpoints} en la cual se muestran los servicios disponibles para los usuarios.


\begin{code}
\captionof{listing}{Modelo para registrar un usuario.}
\label{code:modelUser}
\begin{minted}[linenos,tabsize=2,breaklines, fontsize=\small]{python}
  user = api.model('User', {
    '_id': fields.String(required=False, readonly=True),
    'username': fields.String(required=False, readonly=True, description="a username", example="username"),
    'name': fields.String(required=True ),
    'email': fields.String(required=True, example="email@domain.com"),
    'password': fields.String(required=True, example="P4ssw0rd*"),
    'diagram': fields.String(example="{}", readonly=True),
  })

\end{minted}
\end{code}

\begin{figure}[H]
  \centering
  \includegraphics[width=0.9\textwidth]{apiFlask/user_namespaces.png}
  \caption{Pantalla de los servicios relacionados a un usuario.}
  \label{img:userEndpoints}
\end{figure}
\subsection{Servicios para el diagrama}

Este namespaces tienes los servicios relacionados al diagram entidad-relación del usuario. Por razones de seguridad este namespaces si requiere del uso del decorador \ref{code:tokenWrap}, para asegurar que solamente usuarios registrados en la base de datos puedan consultar su diagram o en su defecto guardar un nuevo diagrama en su cuenta, es en estos casos donde entra en uso la configuración de restplus para la solicitud del token de autenticación de usuario para poder hacer uso de estos servicios como se muestra en la figura~\ref{img:tokenSwagger}.

\begin{figure}[H]
  \begin{subfigure}[b]{0.49\textwidth}
      \includegraphics[width=\textwidth]{apiFlask/token_swagger.png}
      \caption{Solicitud de token de autenticación.}
      \label{img:tokenSwagger}
    \end{subfigure}
    \hfill
    \begin{subfigure}[b]{0.49\textwidth}
      \includegraphics[width=\textwidth]{apiFlask/unauthorized.png}
      \caption{Token no encontrado en la petición.}
      \label{img:unauthorized}
    \end{subfigure}
\end{figure}

En caso de no proporcionar un token valido la operación no podrá ser completada y el api mostrara un mensaje como el de la figura~\ref{img:unauthorized}, de igual forma que al registrar un usuario estos servicio requieren que en la petición HTTP se encuentren ciertos parámetros. En este caso solamente se requiere que se encuentre un objeto json para guardar un diagrama, el cual sera el mismo que se vevolverá al usuario al consultar el servicio \textit{/diagram} por medio de una petición \textit{GET}.
\subsection{Servicio para el modélo relacional}

Este namespaces tienes los servicios que tienen una relación con el modélo relacional como son la obtención de las sentencias SQL o la validación del diagram entidad-relación. Para estos servicios tambien es necesario un token de autenticación para validar un diagrama ER así como la obtención de las sentencias SQL equicalentes, en ambos caso los servicios se deben hacer por una petición del métodp \textit{POST} para que el protocolo \textit{HETTPS} mantenga cifrado la información que se transmita por la red.

La figura~\ref{img:validateER} muestra el servicio de validación de diagram ER, con las reglas que este debe cumplir para ser considerado valido en su estructura. En caso que el diagrama no cumpla con alguna de estas reglas el servicio le mustra al usuario una lista de los errores presentes en el diagrama como se aprecia en la figura~\ref{img:invalidDiagram}

\begin{figure}[H]
  \centering
  \includegraphics[width=0.9\textwidth]{apiFlask/endpointValidate.png}
  \caption{Pantalla del servicio para validar un diagrama ER con las reglas en la descripción.}
  \label{img:validateER}
\end{figure}

\begin{figure}[H]
  \centering
  \includegraphics[width=0.9\textwidth]{apiFlask/invalidDiagram.png}
  \caption{Respuesta para un diagrama ER no valido.}
  \label{img:invalidDiagram}
\end{figure}

Para obtener las sentencias SQL al diagrama ER se creo el servicio \textit{/relational} definido en el archivo \textit{apis/relacional.py}. Este servicio solo debe ser usado para un diagrama ER valido, este servicio no ejecuta ninguna validación sobre el diagram recibido por que ese no es su proposito para eso se tiene el servicio de validación previamente mencionado.

Para conocer las sobre el proceso de obtención de las sentencias SQL equivalentes consulte la sección \ref{sec:desarrollo-sql}. Una vez realizado el proceso el servicio muestra al usuario un script de sql como se muestra en la figura~\ref{img:sqlSentences} que puede descargar o copiar para ejecutar en el sistema gestor de \textit{MySQL}


\begin{figure}[H]
  \centering
  \includegraphics[width=0.9\textwidth]{apiFlask/endpointRelational.png}
  \caption{Respuesta con las sentencias SQL equivalentes al diagram ER.}
  \label{img:sqlSentences}
\end{figure}

\section{Algoritmos}
Como se mencionó en la sección~\ref{sec:analisis-algoritmos}, un algoritmo es cualquier procedimiento computacional definido que toma algún valor, o conjunto de valores, como entrada y produce algún valor, o conjunto de valores, como salida; por lo tanto, un algoritmo es una secuencia de pasos computacionales que transforman la entrada en la salida.


En esta sección se describen cómo se implementan los algoritmos de la sección~\ref{sec:analisis-algoritmos}, empezando con el algoritmo para la validación estructural diagrama entidad-relación; se sigue con algoritmo para el mapeo modelo entidad-relación a relacional; después está la obtención de esquema SQL desde modelo relacional, el mapeo modelo entidad-relación a \textit{Generic Data Metamodel}; el mapeo del \textit{Generic Data Metamodel} a modelo lógico NoSQL y se finaliza con el algoritmo para el modelo lógico NoSQL a modelo físico en MongoDB.
\subsection{Validación estructural del diagrama entidad-relación}

Para cumplir con los objetivos del proyecto es importante que la validación estructural de la sección~\ref{cap:validationER} sea el primer paso a realizar, ya que no tiene sentido obtener el modelo NoSQL de un diagrama no válido; de igual manera, las sentencias SQL equivalentes pueden ser generadas, pero sin una validación estructural no se asegura la coherencia de la estructura de la base de datos relacional resultante.

\paragraph*{Validaciones generales}

Es necesario comprobar que los elementos del diagrama se encuentran conectados entre sí y no existen conexiones libres. Se debe recorrer cada conexión en el arreglo \textit{linksDataArrray} del objeto \textit{json diagram} y comprobar que existen los elementos \textit{'from'} y \textit{'to'} en cada uno de los nodos del arreglo.  En caso de encontrarse ambas propiedades en alguno de los nodos, indica que es una conexión libre y se agrega a una lista de conexiones libres que es mostrada al usuario para corregir esos detalles.


Para obtener los identificadores de cada elemento del diagrama se recorre el arreglo \textit{nodeDataArrray} del objeto \textit{json diagram} y el buscar su correspondencia en el arreglo  \textit{linksDataArrray} asegura que ambos elementos estén conectados entre sí.


En caso de no encontrar el identificador del elemento en el arreglo de conexiones, se agrega a una lista de elementos desconectados que es mostrada al usuario para que corrija esos detalles.


\begin{algorithm}[H]
  \SetKwInOut{Input}{Entrada}
  \SetKwInOut{Output}{Salida}

  \Input{una instancia del diagrama entidad-relación básico en formato \textit{json}, $diagram$}
  \Output{una lista de conexiones libres, $unconnectedLinks$}
  $linksList \gets diagram['linksDataArrray']$\\
  \ForEach{$link \in linksList:$}{
    \If{$!(link['from'] != ''$  \&  $link['to'] != ''):$}{
      $unconnectedLinks \gets link$\\
    }
  }
  
  \caption{Lista de conexiones libres en el diagrama.}
\end{algorithm}

\begin{algorithm}[H]
  \SetKwInOut{Input}{Entrada}
  \SetKwInOut{Output}{Salida}

  \Input{una instancia del diagrama entidad-relación básico en formato \textit{json}, $diagram$}
  \Output{una lista de elementos sin conexión, $unconnectedElements$}
  $linksList \gets diagram['linksDataArrray']$\\
  $itemsList \gets diagram['nodeDataArrray']$\\
  \ForEach{$item \in itemsList:$}{
    \ForEach{$link \in linksList:$}{
      \If{$!(link['from'] == itemKey$  $OR$  $link['to'] == itemKey):$}{
        $unconnectedElements \gets item$\\
      }
    }
  }
  
  \caption{Lista de elementos desconectados.}
\end{algorithm}


\paragraph*{Validaciones en las entidades}


Se comprueba que cada entidad en el arreglo \textit{nodeDataArrray} cuente con al menos un elemento atributo del tipo clave o del tipo simple que no se encuentre conectado a ninguna otra entidad; se realiza un proceso similar a la búsqueda de elemtnos desconectados con la excepción de que se requieren los identificadoes de los elementos del tipo atributo. En caso de no encontrar elementos del tipo atributo o atributo clave, la entidad se agrega a una lista de entidades sin atributos. En caso contrario, si se encuentran conexiones diresctas entre atributos, se genera una lista con las entidades conectadas. Ambas listas se muestran al usuario para que corrija esos detalles.

\begin{algorithm}[H]
  \SetKwInOut{Input}{Entrada}
  \SetKwInOut{Output}{Salida}

  \Input{una instancia del diagrama entidad-relación básico en formato \textit{json}, $diagram$}
  \Output{una lista de entidades sin atributos o sin clave primaria, $entitiesWithoutAttrs$}
  $entitiesWithAttrs \gets diagram['nodeDataArrray']$\\
  \ForEach{$entity \in entitiesWithAttrs:$}{
    \If{$not entity['attributes] :$}{
      $entitiesWithoutAttrs \gets entity_s, entity_T$\\
    }
    \If{$not entity['primaryKey] :$}{
      $entitiesWithoutAttrs \gets entity_s, entity_T$\\
    }
  }
  
  \caption{Lista de entidades sin atributos.}
\end{algorithm}

\begin{algorithm}[H]
  \SetKwInOut{Input}{Entrada}
  \SetKwInOut{Output}{Salida}

  \Input{una instancia del diagrama entidad-relación básico en formato \textit{json}, $diagram$}
  \Output{una lista de entidades conectadas directamente, $connetionBetweenEntities$}
  $linksList \gets diagram['linksDataArrray']$\\
  $entitiesKeyList \gets diagram['nodeDataArrray']$\\
  \ForEach{$link \in linksList:$}{
    \If{$link['from'] \in  entitiesKeyList $ \&  $link['to'] \in entitiesKeyList:$}{
      $connetionBetweenEntities \gets entity_s, entity_T$\\
    }
  }
  
  \caption{Lista de entidades conectadas directamente.}
\end{algorithm}

\paragraph*{Validaciones en los atributos}

Los atributos solo pueden tener una conexion a una entidad o a una relación, no se pueden asociar a más de un elemento o entre sí mismos de forma directa. Para comprobar estos puntos es necesario inspeccionar la propiedad \textit{nodeDataArrray} del objeto \textit{json diagram} en conjunto con la propiedad \textit{linksDataArrray} para determinar las conexiones entre elementos. En caso de encontrar una conexion directa entre atributos, agregarla a una lista de conexión entre atributos. De la misma forma, si el identificador del atributo es encontrado en más de una ocasión en los nodos de \textit{linksDataArrray}, entonces indica que el atributo está conectado a más de un elemento, por lo que se agrega a una lista de atributo con multiconexiones para mostrar ambas listas al usuario y que pueda corregir esos detalles.

\begin{algorithm}[H]
  \SetKwInOut{Input}{Entrada}
  \SetKwInOut{Output}{Salida}

  \Input{una instancia del diagrama entidad-relación básico en formato \textit{json}, $diagram$}
  \Output{una lista de atributos con conexiones múltiples, $attrMulticonnections$}
  $linksList \gets diagram['linksDataArrray']$\\
  $attrsKeyList \gets diagram['nodeDataArrray']$\\
  \ForEach{$link \in linksList:$}{
    \If{$link['from'] \in  attrsKeyList $ $OR$  $link['to'] \in attrsKeyList:$}{
      $count ++$
    }
  }
  \If{$count > 1$}{
    $attrMulticonnections \gets attr$\\
  }

  \caption{Lista de atributos con conexiones múltiples.}
\end{algorithm}

\paragraph*{Validaciones en las relaciones}

Las relaciones del diagrama están limitadas por una restricción de la propuesta de solución, esta no permite relaciones de grado superior a 2 y por obvias razones no tiene caso que exista una relación con una sola conexión. Esto no es lo mismo que las relaciones unarias o recursivas, ya que para que una relación pueda ser considerada recursiva deben existir 2 conexiones en la propiedad \textit{linksDataArrray} del objeto \textit{json diagram}. En ambas deben existir el identificador de la relación y la entidad, por lo tanto la validación puede limitarse a comprobar que los elementos del tipo relación en el diagrama cuenten con maximo 2 nodos en \textit{linksDataArrray}. De ser encontrado una relación con menos o más conexiones, se agrega a una lista de relaciones inválidas que se muestra al usuario para que pueda corregir esos detalles.

\begin{algorithm}[H]
  \SetKwInOut{Input}{Entrada}
  \SetKwInOut{Output}{Salida}

  \Input{una instancia del diagrama entidad-relación básico en formato \textit{json}, $diagram$}
  \Output{una lista de relaciones inválidas, $invalidrelationship$}
  $linksList \gets diagram['linksDataArrray']$\\
  $relationsKeyList \gets diagram['nodeDataArrray']$\\
  \ForEach{$link \in linksList:$}{
    \If{$link['from'] \in  relationsKeyList $ $OR$  $link['to'] \in relationsKeyList:$}{
      $count ++$
    }
  }
  \If{$count != 1$}{
    $invalidrelationship \gets relationship$\\
  }

  \caption{Lista de atributos con conexiones múltiples.}
\end{algorithm}
\subsection{Tranformación del diagrama entidad-relación al modelo relacional y generar las sentencias SQL}\label{sec:desarrollo-sql}

Para lograr la transformación del modelo entidad-relación del diagramador de la propuesta de solución al modelo relacional, se implementó el algoritmo de 7 pasos descrito en la sección \ref{sec:erToSQL}, a excepción del paso siete que por restricciones autoimpuestas en la propuesta de solución del proyecto no aplica. El séptimo paso hace referencia a la transformación de relaciones de grado 3 o superior, pero el diagramador no permite dichas relaciones, por lo cual este paso es omitido.

Cabe destacar que por simplicidad se implementó en un solo algoritmo la transformación del modelo entidad-relación al modelo relacional y la genración de las sentencias SQL a partir del modelo relacional.

\paragraph*{Paso 1: Mapeado de los tipos de entidad regulares}

\begin{algorithm}[H]
  \SetKwInOut{Input}{Entrada}
  \SetKwInOut{Output}{Salida}

  \Input{una instancia del diagram ER en formato \textit{json}, $diagram$}
  \Output{una lista de entidades con sus atributos, $entityWithAttrs$}
  $entities \gets diagram['nodeDataArrray']$\\
  $attrsList \gets diagram['nodeDataArrray']$\\
  $linksList \gets diagram['linksDataArrray']$\\
  \ForEach{$entity \in entities:$} {
    \ForEach{$link \in linksList:$}{
      \ForEach{$attr \in attrsList:$}{
        \If{$link = attrKey$  \&  $link = entityKey:$}{
          $entityWithAttrs \gets {entity:attr}$\\
        }
      }
    }
  }
  \caption{Asociar entidades con sus atributos.}
\end{algorithm}

\paragraph*{Paso 2: Mapeado de los tipos de entidad débiles}

\begin{algorithm}[H]
  \SetKwInOut{Input}{Entrada}
  \SetKwInOut{Output}{Salida}

  \Input{una instancia del diagram ER en formato \textit{json}, $diagram$}
  \Output{una lista de entidades débiles con sus atributos, $weakEntityWithAttrs$}
  $weakEntities \gets diagram['nodeDataArrray']$\\
  $attrsList \gets diagram['nodeDataArrray']$\\
  $linksList \gets diagram['linksDataArrray']$\\
  \ForEach{$entity \in entities:$} {
    \ForEach{$link \in linksList:$}{
      \ForEach{$attr \in attrsList:$}{
        \If{$link = attrKey$ \& $link = entityKey:$}{
          $weakEntityWithAttrs \gets entity.attr$\\
        }
      }
    }
  }
  \caption{Asociar entidades débiles con sus atributos.}
\end{algorithm}

\paragraph*{Paso 3: Mapeado de los tipos de relación 1:1 binaria}

\begin{algorithm}[H]
  \SetKwInOut{Input}{Entrada}
  \SetKwInOut{Output}{Salida}

  \Input{una instancia del diagram ER en formato \textit{json}, $diagram$}
  \Output{una lista de relaciones 1:1 con los identificadores de las entidades participantes, $relations1\_1$}
  $linksList \gets diagram['linksDataArrray']$\\
    \ForEach{$link \in linksList:$}{
        \If{$link['cardinality'] = 1:$}{
          $relations1\_1 \gets (link, cardinality)$\\
      }
    }
  \caption{Asociar entidades que participan en una relación 1:1 binaria.}
\end{algorithm}

\begin{algorithm}[H]
  \SetKwInOut{Input}{Entrada}
  \SetKwInOut{Output}{Salida}

  \Input{una lista de relaciones 1:1 binaria, $relations1\_1$}
  \Output{una lista de entidades con atributos y claves foráneas, $entiesWithAttrs$}
    \ForEach{$relation \in relations1\_1:$}{
      $entity_S \gets entiesWithAttrs[relationKey[0]]$\\
      $entity_T \gets entiesWithAttrs[relationKey[1]]$\\
      $pk_S \gets entity_S$\\
      $entity_T.attrs \gets pk_S$\\
      $entity_T.fks \gets pk_S$\\
    }
  \caption{Agregar el atributo clave de la entidad S a los atributos de la entidad T como clave foránea en una relación 1:1 binaria.}
\end{algorithm}

\paragraph*{Paso 4: Mapeado de tipos de relaciones 1:N binarias}

\begin{algorithm}[H]
  \SetKwInOut{Input}{Entrada}
  \SetKwInOut{Output}{Salida}

  \Input{una instancia del diagram ER en formato \textit{json}, $diagram$}
  \Output{una lista de relaciones 1:N con los identificadores de las entidades participantes, $relations1\_N$}
  $linksList \gets diagram['linksDataArrray']$\\
    \ForEach{$link \in linksList:$}{
        \If{$(link['cardinality'] \in [1, N]:$}{
          $relations1\_N \gets (link, cardinality)$\\
      }
    }
  \caption{Asociar entidades que participan en una relación 1:N binaria.}
\end{algorithm}

\begin{algorithm}[H]
  \SetKwInOut{Input}{Entrada}
  \SetKwInOut{Output}{Salida}

  \Input{una lista de relaciones 1:N binaria, $relations1\_N$}
  \Output{una lista de entidades con atributos y claves foráneas, $entiesWithAttrs$}
    \ForEach{$relation \in relations1\_1:$}{
      \If{$(link['cardinality'] = N :$}{
        $entity_S \gets entiesWithAttrs[relationKey$\\
      }
      \If{$(link['cardinality'] = 1 :$}{
        $entity_T \gets entiesWithAttrs[relationKey$\\
      }
    }
    $entity_S.attrs \gets pk_S$\\
    $entity_S.fks \gets pk_S$\\
  \caption{Agregar el atributo clave de la entidad T a los atributos de la entidad S como clave foránea en una relación 1:N binaria.}
\end{algorithm}

\paragraph*{Paso 5: Mapeado de tipos de relaciones N:M binarias}

\begin{algorithm}[H]
  \SetKwInOut{Input}{Entrada}
  \SetKwInOut{Output}{Salida}

  \Input{una instancia del diagram ER en formato \textit{json}, $diagram$}
  \Output{una lista de relaciones 1:N con los identificadores de las entidades participantes, $relationsM\_N$}
  $linksList \gets diagram['linksDataArrray']$\\
    \ForEach{$link \in linksList:$}{
        \If{$(link['cardinality'] \in [M, N]:$}{
          $relationsM\_N \gets (link, cardinality)$\\
      }
    }
  \caption{Asociar entidades que participan en una relación N:M binaria.}
\end{algorithm}

\begin{algorithm}[H]
  \SetKwInOut{Input}{Entrada}
  \SetKwInOut{Output}{Salida}

  \Input{una lista de relaciones N:N binaria, $relationsN\_M$}
  \Output{una entidad con atributos y claves foráneas, $entiesWithAttrs$}
    $attrsList \gets diagram['nodeDataArrray']$
    $linksList \gets diagram['linksDataArrray']$
    \ForEach{$relation \in relationsN\_M:$}{
      \ForEach{$link \in linksList:$}{
      \ForEach{$attr \in attrsList:$}{
        \If{$link = attrKey$  \&  $link = relationKey:$}{
          $entity_S \gets {entity:attr}$\\
          $entity_S.pk = entiesWithAttrs[link]$
          $entity_S.fk = entiesWithAttrs[link]$
        }
      }
    }
    }
  \caption{Agregar los atributo simples, atributos claves y claves foráneas  en una relación 1:N binaria.}
\end{algorithm}

\paragraph*{Paso 6: Mapeado de atributos multivalor}

\begin{algorithm}[H]
  \SetKwInOut{Input}{Entrada}
  \SetKwInOut{Output}{Salida}

  \Input{una instancia del diagram ER en formato \textit{json}, $diagram$}
  \Output{una lista de entidades del tipo multivalor con sus atributos, $emvWithAttrs$}
  $entitiesMultivalue \gets diagram['nodeDataArrray']$\\
  $attrsList \gets diagram['nodeDataArrray']$\\
  $linksList \gets diagram['linksDataArrray']$\\
  \ForEach{$entity \in entities:$} {
    \ForEach{$link \in linksList:$}{
      \ForEach{$attr \in attrsList:$}{
        \If{$entity = 'multiValue'$  \&  $link = entityKey:$}{
          $emvWithAttrs \gets {entity:attr}$\\
          $emvWithAttrs.pk = entity.pk$\\
        }
      }
    }
  }
  \caption{Asociar entidades multivalor con sus atributos.}
\end{algorithm}

Con los pasos anteriores tenemos de manera separada los elementos necesarios para obtener las sentencias SQL equivalentes al diagrama ER de entrada. Como todos los componentes se encuentran en listas resulta fácil iterarlos y poder construir los bloques del \textit{script} (con la ayuda de un template previamente generado ) en el lenguaje de consultas SQL para que puedan ser ejecutadas en el SGDB MySQL.

Se hizo uso de los siguientes \textit{templates} para crear las sentencias correspondientes y crear las tablas con sus atributos como se aprecia en la figura~\ref{img:templateTable} y elementos de distinción como son la clave primaria de la figura~\ref{img:pkTemplate} y las claves foráneas de la figura~\ref{img:fkTemplate} dependiendo del tipo de relación entre los elementos del diagrama ER.


\begin{figure}[H]
  \centering
  \includegraphics[width=\textwidth]{apiFlask/tableTemplate.png}
  \caption{Fragmento de código del template para construir la sentencia \textit{CREATE TABLE}.}
  \label{img:templateTable}
\end{figure}

\begin{figure}[H]
  \centering
  \includegraphics[width=\textwidth]{apiFlask/pkTemplate.png}
  \caption{Fragmento de código del template para construir la sentencia \textit{PRIMARY KEY}.}
  \label{img:pkTemplate}
\end{figure}

\begin{figure}[H]
  \centering
  \includegraphics[width=\textwidth]{apiFlask/fkTemplate.png}
  \caption{Fragmento de código del template para construir la sentencia \textit{FOREING KEY}.}
  \label{img:fkTemplate}
\end{figure}
\subsection{Transformación modelo entidad-relación a entidades del modelo conceptual GDM}
La generación de las entidades del modelo GDM desde el modelo entidad-relación básico es de le siguiente manera: 


A continuación se explica brevente la implementación.

\begin{algorithm}[H]
  \SetKwInOut{Input}{Entrada}
  \SetKwInOut{Output}{Salida}

  \Input{diagrama entidad-relación básico, $er$}
  \Output{entidades del Generic Datametamodel, $entities_{gdm}$}
  $entities \gets  list(filter(lambda\ item:\ item['type'] == 'entity',nodeData))$\\
    $entities2pair \gets  list(map(lambda\ entity: (entity, getConnectedNodes(entity['key'], nodeData, linkData)),entities))$\\
    $entities_{gdm} \gets  array$\\
    \ForEach{$pair \in entities2pair:$}{
      $references \gets references(pair)$\\
      $features \gets features(pair)$\\
      $entity pair.key$\\
      $entity.references \gets references$\\
      $entity.features \gets features$\\
      $entities_{gdm}.append(entity)$
    }
    
  \caption{Generar los \textit{features} y las \textit{references} del GDM desde el modelo entidad-relación.}
\end{algorithm}

\begin{code}
    \captionof{listing}{Generación de entidades del GDM desde el modelo entidad-relación básico}
    \label{code:er-to-gdm}
    \begin{minted}[linenos,tabsize=2,breaklines, fontsize=\small]{python}
      def main():
    
      input_file = open('er.json') 
      data = json.load(input_file) 
      input_file.close() 
      
      nodeData = data['nodeDataArray']
      linkData = data['linkDataArray']
    
      # Obtenemos entidades del modelo ER
      entities = list(filter(lambda item: item['type'] == "entity",nodeData))
    
      # Obtenemos lista de pares, donde la clave es una entidad y su valor son los nodos a los que está conectada la entidad
      entities2pair = list(map(lambda entity: (entity, getConnectedNodes(entity['key'], nodeData, linkData)),entities))
      count = 0
      for pair in entities2pair:
          entity = pair[0]
          features = pair[1]
          attributes = []
          references = []
          for feature in features:
              if (feature['type'] == "relation"):
                  reference = {}
                  relation = getRelationInfo( feature["key"], linkData, nodeData)
    
                  if (relation["from"]["key"] != entity["key"]):
                      reference["entity"] = relation["from"]["text"]
                      reference["cardinality"] = relation["fromC"]
                      if relation["fromC"] != '1':
                          reference["name"] = feature["text"] + relation["from"]["text"]
                      else:
                          reference["name"] = toFirstLower(relation["from"]["text"])
                  else:
                      reference["entity"] = relation["to"]["text"]
                      reference["cardinality"] = relation["toC"]
                      # Si la relación es a N, va el nombre de la relación concatenado con el nombre de la entidad
                      if relation["toC"] != '1':
                          reference["name"] = feature["text"] + relation["to"]["text"]
                      else:
                          reference["name"] = toFirstLower(relation["to"]["text"])
    
    
                  references.append(reference)
              if ( isAttribute(feature['type']) ):
                  attribute = {}
                  attribute["name"] = feature['text']
                  attribute["type"] =  feature["gdmType"]
                  attribute["array"] = True if feature["type"] == "multivalueAttribute" else False
                  attribute["unique"] = True if feature["type"] == "keyAttribute" else False
                  if attribute["unique"] == True:
                      attribute["type"] = "id"
                  attributes.append(attribute)
          
          # Escribimos el archivo de texto
          if count == 0:
              count += 1
              with open("laloVenues.gdm", 'w') as output_file:
                  output_file.write("entity " + entity["text"] + " {\n")
                  for attribute in reversed(attributes):
                      output_file.write(parseAttribute(attribute))
                  for reference in reversed(references):
                      output_file.write(parseReference(reference))
                  output_file.write("}\n\n")
                  output_file.close()
          else:
              with open("laloVenues.gdm", 'a') as output_file:
                  output_file.write("entity " + entity["text"] + " {\n")
                  for attribute in reversed(attributes):
                      output_file.write(parseAttribute(attribute))
                  for reference in reversed(references):
                      output_file.write(parseReference(reference))
                  output_file.write("}\n\n")
                  output_file.close()
      return
    \end{minted}
\end{code}



El modelo GDM del código~\ref{code:er-to-gdm} está compuesta por entidades y consultas, cada entidad contiene \textit{features}, y cada \textit{feature} puede ser atributo o una referencia a otra entidad. Todos los atributos tienen un tipo y una cardinalidad.


Este algoritmo solo obtiene las entidades del modelo GDM. Se necesitan además consultas definidas por el usuario para generar el modelo lógico orientado a documetnos.
\subsection{Transformación de una instancia GDM al modelo lógico orientado a documentos}
De acuerdo con la sección~\ref{alg:gdm-to-logic} y la sección~\ref{sec:xtend}, se implementará el algoritmo usando Xtend.

Cualquier archivo de texto con una definición correcta del lenguaje GDM será convertido por Mortadelo a un modelo orientado a documentos.


Como se ha explicado en la sección~\ref{alg:gdm-to-logic}, el algoritmo a implementar es una transformación entre modelos. A continuación se adjunta una breve descripción del algoritmo y su implementación.

\begin{algorithm}[H]
    \SetKwInOut{Input}{Input}
    \SetKwInOut{Output}{Output}

    %\underline{function Euclid} $(a,b)$\;
    \Input{una instancia del modelo GDM, $gdm$}
    \Output{un modelo lógico orientado a documentos, $ddm$}
    $mainEntities \gets gdm.queries.collect((q)|q.from);$\\
    \ForEach{$me \in mainEntities$}{
        $collection \gets new Collection();$\\
        $collection.name \gets me.name;$\\
        $accessTree \gets allQueryPaths(me,gdm.queries);$\\
        $collection \gets populateDocumentType(collection.root,accessTree);$\\
        $ddm.collections.add(collection);$
    }
    
    \caption{Transformación del modelo conceptual GDM al modelo lógico orientado a documentos}
\end{algorithm}

Donde la función populateDocumentType() es otro algoritmo de la forma:

\begin{algorithm}[H]
    \SetKwInOut{Input}{Input}
    \SetKwInOut{Output}{Output}

    %\underline{function Euclid} $(a,b)$\;
    \Input{Un ``document type", $dt$}
    \Output{un nodo del arbol de acceso}
    $ nodeAttributes \gets node.entity.features.select(f|f.isTypeOf(Attribute))\; $\\
    $ nodeReferences \gets node.entity.features.select(f|f.isTypeOf(Reference))\; $\\
    \ForEach{$attr \in nodeAttributes$}{
        $pf \gets new PrimitiveField()\;$\\
        $pf.name \gets attr.name\;$\\
        $pf.type \gets attr.type\;$\\
        $dt.fields.add(pf)\;$\\
    }
    \ForEach{$ref \in nodeReferences$}{
        $targetNode \gets node.arcs.find(a|a.name = ref.name).target\;$\\
        \uIf{exists(targetNode)}{
            $baseType \gets new DocumentType()\;$\\
            $populateDocumentType(baseType,targetNode)\;$\\
        }
        \Else{
            $baseType \gets new PrimitiveField()\;$\\
            $baseType.type \gets findIdType(ref.entity)\;$\\
        }
        $baseType.name \gets ref.name\;$\\

        \uIf{$ref.cardinality$ == 1}{
            $dt.field.add(baseType)\;$\\
            
        }
        \Else{
            $arrayField \gets new ArrayField()\;$\\
            $arrayField.type \gets baseType\;$\\
            $dt.fields.add(arrayField)\;$
        }
        
    }
    \caption{Generar el contenido de un \textit{DocumentType} dado un árbol de acceso}
\end{algorithm}

\begin{code}
    \captionof{listing}{Transformación modelo a modelo: GDM a modelo lógico orientado a documentos}
    \label{code:mortadelo-gdm-m2m}
    \begin{minted}[linenos,tabsize=2,breaklines, fontsize=\small]{python}

      def main():
    
      gdmModel = loadModel("Static_model_test.xmi")
      
      ddmModel = ddm.DocumentDataModel()
  
      # Obtenemos las entidades de los elementos From
      # mainEntities = gdm.queries.map[q | q.from.entity].toSet    
      mainEntities = set(map(lambda q: q.from_.entity, gdmModel.queries))
  
      # entityToQueries =  mainEntities.map[me | me -> gdm.queries.filter[q | q.from.entity.equals(me)]]
      entityToQueries = list(map(lambda me: (me, list(filter(lambda q: q.from_.entity.name == me.name, gdmModel.queries))), mainEntities))
          
      # val entity2accessTree = newImmutableMap(entityToQueries.map[e2q | e2q.key -> createAccessTree(e2q.value)])
      entity2accessTree = list(map(lambda e2q: ( e2q[0], createAccessTree(e2q[1])), entityToQueries))
  
      # Completamos cada árbol de acceso
      for entity in mainEntities:
        tree = getTree(entity2accessTree,entity)
        othersTrees = getAllTrees(entity2accessTree)
        completeAccessTree(entity, tree, othersTrees)
      
      # Generamos las colecciones
      for entity in mainEntities:
          tree = getTree(entity2accessTree,entity)
          collection = ddm.Collection()
          collection.name = entity.name
          docType = ddm.Document()
          docType.name = "root"
          collection.root = docType
          populateDocument(docType,entity,tree,mainEntities)
          ddmModel.collections.append(collection)
          saveModelDDM(ddmModel)
  
      saveModelDDM(ddmModel)

\end{minted}
\end{code}

La implementación del código fuente~\ref{code:mortadelo-gdm-m2m} toma de entrada una instancia del modelo GDM en XML (un .model generado por el \textit{parser implementado}) y de salida genera otro .model correspondiente al modelo lógico orientado a documentos.


Cada entidad principal tiene un arbol de acceso asociado, para generar los documentos anidados se hace uso de ese arbol de acceso.
\subsection{Transformación del modelo lógico orientado a documentos a MongoDB}


\section{Conclusiones}
Durante el proceso de desarrollo de este trabajo terminal se han presentado algunos retos que no se tenían contemplados en un inicio, desde la creación o separación del proyecto en 2 aplicaciones para un mejor manejo del trabajo en equipo hasta las tecnologías a utilizar. 

El resolver cada uno esos retos ha dejado un aprendizaje para el equipo de desarrollo, ya que nos enseña que aun teniendo un plan de trabajo y fechas establecidas, siempre existen inconvenientes con los que uno no cuenta, por lo que el equipo se debe adaptar a estos cambios para tratar de obtener los resultados y, en caso de ser necesario, modificar el plan inicial.

Uno de los mas grandes retos fue la transformación del diagrama entidad-relación al modelo NoSQL, ya que se en el plan inicial mp se contempló la falta de tecnologáis disponibles para transformaciones entre gramáticas en distintos niveles de abstrancción. Al no tener experiencia ni el contexto de todo lo que implica la ingeniería orientada a modelos, dado por las limitantes en tiempo para este trabajo terminal, el equipo se vio en la necesidad de utilizar parte del proyecto de un tercero para lograr el alcance esperado. De ser posible, como un trabajo a futuro se realizará una implementación propia para la transformaciones entre modelos. 