\section{Back end}


De acuerdo con el sitio de Heroku~\cite{heroku_heroku_2020}, la plataforma ofrece distintos mecanismos para la seguridad de los proyectos alojadas en ella, dentro de estas se encuentra el uso por defecto del protocolo HTTPS, que asegura el cifrado de los datos en la Internet.

De igual manera ofrece un constante escaneo de las aplicaciones en búsqueda de vulnerabilidades para mitigar los ataques DDoS, además de utilizar la infraestructura de la empresa Amazon, las cuales se encuentran acreditadas por diversos estándares de seguridad; esto da la confianza para que la aplicación Flask se ejecute de manera segura y tener salvaguardados los datos de los usuarios.

De acuerdo con el sitio Pallersprojects(REFFLASKDEF), flask es un \textit{framework} web ligero escrito en python. Esta diseñado para que el desarrollo de una aplicación web sea rápida y sencilla, con la capacidad de escalar a aplicaciones complejas, ofrece sugerencias pero no impone ninguna dependencia o diseño sobre el proyecto depende del desarrollador las bibliotecas o herramientas que desee utilizar y al contar con una cominidad amplia cuenta con una gran cantidad de extensiones para ampliar sus funcionalidades.