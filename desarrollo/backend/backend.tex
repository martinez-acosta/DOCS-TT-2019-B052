\section{Back end}


De acuerdo con el sitio de Heroku~\cite{heroku_heroku_2020}, la plataforma ofrece distintos mecanismos para la seguridad de los proyectos que aloja, uno de estos mecanismos es el uso por defecto del protocolo HTTPS, que asegura el cifrado de los datos en la Internet.

De igual manera, ofrece un constante escaneo de las aplicaciones en búsqueda de vulnerabilidades para mitigar los ataques DDoS, además de utilizar la infraestructura de la empresa Amazon, las cuales se encuentran acreditadas por diversos estándares de seguridad; esto da la confianza para que la aplicación Flask se ejecute de manera segura y tener salvaguardados los datos de los usuarios.

De acuerdo con el sitio Pallets Projects~\cite{noauthor_pallets_2020}, flask es un \textit{framework} web escrito en python. Está diseñado para que el desarrollo de una aplicación web sea rápida y sencilla sin imponer ninguna dependencia o diseño sobre el proyecto, porque depende del desarrollador implementar las bibliotecas o herramientas que desee utilizar y, al contar con una comunidad amplia, hay una gran cantidad de extensiones para ampliar sus funcionalidades.