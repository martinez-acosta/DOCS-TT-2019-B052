\subsection{Diagramado del modelo entidad-relación}
De acuerdo con Shahzad~\cite{shahzad_review_2016}, GoJS es una biblioteca de JavaScript para implementar diagramas interactivos y visualizaciones en plataformas y navegadores web modernos. 

\newenvironment{code}{\captionsetup{type=listing}}{}
\SetupFloatingEnvironment{listing}{name=Código fuente}

\begin{code}
\captionof{listing}{Diagrama GoJS}
\label{code:diagram-gojs}
\begin{minted}[linenos,tabsize=2,breaklines]{js}
    // Definimos el diagrama
    this.myDiagram = $(
       go.Diagram,
       'myDiagramDiv', 
       {
         grid: $(go.Panel, 'Grid'),
         'draggingTool.dragsLink': false,
         'draggingTool.isGridSnapEnabled': true,
         'linkingTool.isUnconnectedLinkValid': false,
         'linkingTool.portGravity': 10,
         'relinkingTool.isUnconnectedLinkValid': false,
         'relinkingTool.portGravity': 10,
         'relinkingTool.fromHandleArchetype': $(go.Shape, 'Diamond', {
           segmentIndex: 0,
           cursor: 'pointer',
           desiredSize: new go.Size(8, 8),
           fill: 'tomato',
           stroke: 'darkred'
         }),
         'relinkingTool.toHandleArchetype': $(go.Shape, 'Diamond', {
           segmentIndex: -1,
           cursor: 'pointer',
           desiredSize: new go.Size(8, 8),
           fill: 'darkred',
           stroke: 'tomato'
         }),
         'linkReshapingTool.handleArchetype': $(go.Shape, 'Diamond', {
           desiredSize: new go.Size(7, 7),
           fill: 'lightblue',
           stroke: 'deepskyblue'
         }),
         'undoManager.isEnabled': true
       }
     )
 \end{minted}
\end{code}


Como se especificó en la sección~\ref{ref:sec-gojs}, el diagrama se implementó en GoJS y el código fuente~\ref{code:diagram-gojs} muestra una instancia de cómo se configura el \textit{canvas} del diagramador.