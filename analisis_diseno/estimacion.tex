\subsection{Factibilidad económica}

De acuerdo con Pressman en \textit{Ingeniería de software, un enfoque práctico}\cite{pressman_software_2005}, en su sección de estimación, se utiliza una métrica por puntos de función como se muestra en la tabla \ref{tab:function_point_metrics}.


Es por este medio que se calcula o se realiza una estimación del costo total del proyecto, incluyendo los salarios de los desarrolladores que llevarán a cabo la implementación del sistema, así como los gastos por pagos de servicios que sean necesarios.

\subsubsection{Métricas orientadas a la función}


Para este proyecto, se considera que todas las funciones identificadas son de complejidad media con excepción de las entradas que tienen la complejidad más alta del sistema.


\begin{table}[h!]
	\begin{tabular}{|l|l|l|l|l|l|}
	\hline
	\multirow{2}{*}{Parámetro} & \multirow{2}{*}{Cuenta} & \multicolumn{3}{|l|}{Factores de ponderación} & \multirow{2}{*}{Total} \\ \cline{3-5}
														 &                         & baja       	& media       & alta		       &                        \\ \hline
	Entradas                   & 6                       & 3            & 4           & 6              & 36                     \\ \hline
	Salidas                     & 5                       & 4            & 5           & 7              & 25                     \\ \hline
	Tablas                     & 1                       & 3            & 4           & 6              & 4                      \\ \hline
	Interfaces                 & 4                       & 7            & 10          & 15             & 40                     \\ \hline
	Consultas                  & 4                       & 5            & 7           & 10             & 28                     \\ \hline
	Conteo total               &                         &              &             &		             & 133                    \\ \hline
	\end{tabular}
	\caption{Cálculo de las métricas por puntos de función}
	\label{tab:function_point_metrics}
	\end{table}


	\textbf{Fi (i = 1..14)} son factores de ajuste de valor basados en las respuetas de las preguntas de la tabla \ref{tab:questions_adjusment}. Los valores pueden ir de 0 (no importante o aplicable) a 5 (absolutamente esencial).

	\begin{table}[h!]
		\begin{tabular}{|p{9cm}|c|}
		\hline
		Pregunta                                                                                                                 & Ponderación \\ \hline
		¿Requiere el sistema métodos de seguridad y recuperación fiables?                                                       & 3           \\ \hline
		¿Se requiere comunicación especializada?                                                                              & 5           \\ \hline
		¿Existen funciones de procesamiento distribuido?                                                                         & 2           \\ \hline
		¿Es crítico el rendimiento?                                                                                              & 4           \\ \hline
		¿Se ejecutará el sistema en un entorno operativo existente y fuertemente utilizado?                                      & 4           \\ \hline
		¿Requiere el sistema una entrada de datos interactiva?                                                                    & 5           \\ \hline
		¿Requiere la entrada de datos interactiva que las transacciones de entrada se lleven a cabo sobre múltiples operaciones? & 5           \\ \hline
		¿Se actualizan los archivos maestros de forma interactiva?                                                               & 3           \\ \hline
		¿Son complejas las entradas, salidas, archivos o consultas?                                                               & 4           \\ \hline
		¿Es complejo el procesamiento interno?                                                                                   & 4           \\ \hline
		¿Se ha diseñado el código para ser reutilizable?                                                                         & 4           \\ \hline
		¿Están incluidas en el diseño la instalación y conversión?                                                               & 3           \\ \hline
		¿Se ha diseñado el sistema para soportar múltiples instalaciones en diferentes organizaciones?                           & 4           \\ \hline
		¿Se ha diseñado el sistema para facilitar los cambios y para ser fácilmente utilizable?                                  & 4           \\ \hline
		\centering $\sum Fi=$                                                                                                  	 & 54          \\ \hline
		\end{tabular}
		\caption{Factores de ajuste}
		\label{tab:questions_adjusment}
		\end{table}


\subsubsection{Puntos de función}

La fórmula para obtener los puntos de función con los factores de ajuste es la siguiente:


\begin{center}
	$
	PF = conteo\ total *  (0.65 + (0.01 * \sum Fi))
	$


	Se deben sustituir los valores del conteo total y los factores de ajuste.


	\begin{equation}
		PF =  133 *  (0.65 + (0.01 * 54))
		PF = 158.27 \approx 159
	\end{equation}
\end{center}

De lo anterior, aproximadamente se obtienen \textbf{159} puntos de función. Una vez obtenidos utilizando la llamada “Ball-Park” o “Estimación Indicativa“, que es la técnica de macro-estimación que se utiliza en situaciones de falta de información sobre el proyecto. El autor del artículo \textit{Applied Software Measurement}\cite{abran_applied_2006}, Carper Jones, propone la siguiente ecuación para determinar el esfuerzo de desarrollo de un proyecto:


\begin{center}
	$
	Esfuerzo = (\frac{PF}{150})*PF^{0.4}
	$


	Donde:


	PF : Puntos de función.


	Al sustituir los valores, el resultado es:

	$
	Esfuerzo = (\frac{159}{150})*159^{0.4}
	Esfuerzo = 8.05 meses
	$
\end{center}


De estos 8.05 meses, considerando un total de 40 horas a la semana de trabajo y 4.34 semanas por mes, el total de horas para el desarrollo y conclusión del proyecto se obtiene de esta manera:

\begin{center}
	Tiempo de desarrollo = horas/semana * semanas/mes * Esfuerzo
	Tiempo de desarrollo = $ 40 * 4.34 * 8.05 $
	Tiempo de desarrollo = $ 1397.48 \approx 1398 $ horas
\end{center}


Por ejemplo, una sola persona trabajando en el desarrollo del proyecto debería invertir 1398 horas con una jornada de 8 horas diarias de lunes a viernes hasta su finalización, por lo que si un equipo de desarrollo es de 2 personas con un horario de lunes a viernes de 4 horas diarias, el proyecto concluiría en 8.05 meses.

