\section{Metodología}

La metodología, es un marco de trabajo usado para estructurar, planificar y controlar el proceso de desarrollo en sistemas de información. En un proyecto de desarrollo de software la metodología ayuda a definir: Quién debe hacer Qué Cuándo y Cómo debe hacerlo. La metodología para el desarrollo de software es un modo sistemático de realizar, gestionar y administrar un proyecto para llevarlo a cabo con altas posibilidades de éxito. Una metodología para el desarrollo de software comprende actividades a seguir para idear, implementar y mantener un producto de software desde que surge la necesidad del producto hasta que se cumple el objetivo por el cual fue creado.[https://www.uladech.edu.pe/images/stories/universidad/documentos/2018/metodologia-desarrollo-software-v001.pdf]

Dentro del campo del desarrollo de software, nos encontramos con dos grupos de metodologías, las tradicionales y las agiles.


Las tradicionales, exigen una documentación exhaustiva y se centran en cumplir con el plan del proyecto definido totalmente en la fase inicial del desarrollo del mismo.


Dado a que cualquier cambio en el proceso generaba la necesidad de una reconstrucción en la metodología, surgieron las llamadas metodologías ágiles, las cuales permiten realizar cambios en los requerimientos conforme se va desarrollando el proyecto basada en las habilidades del equipo y una buena relación con el usuario mostrando avances funcionales en cortos periodos de tiempo con la posibilidad de que este realice una evaluacion y de ser necesario, sugiera cambios.

Se han propuesto muchos modelos ágiles de proceso y están en uso en toda la industria. Entre ellos se encuentran los siguientes:


\begin{itemize}
	\item Desarrollo adaptativo de software (DAS)
	\item Scrum
	\item Método de desarrollo de sistemas dinámicos (MDSD)
	\item Cristal
	\item Desarrollo impulsado por las características (DIC)
	\item Desarrollo esbelto de software (DES)
	\item Modelado ágil (MA)
	\item Proceso unificado ágil (PUA)
\end{itemize}

\subsection{SCRUM}
Para la factibilidad técnica se realiza una evaluación de las herramientas de hardware  y software que el equipo de trabajo tiene disponible.

Scrum se basa en la teoría de control de procesos empírica o empirismo. El empirismo asegura que el conocimiento procede de la experiencia y de tomar decisiones basándose en lo que se conoce. Scrum emplea un enfoque iterativo e incremental para optimizar la predictibilidad y el control del riesgo.

\noindent Los tres pilares que soportan toda la implementación del control de procesos:
\begin{itemize}
	\item Transparencia
	\item Inspección
	\item Adaptación
\end{itemize}

\noindent Scrum prescribe cuatro eventos formales, contenidos dentro del Sprint, para la inspección y adaptación.[K. Schwaber and J. Sutherland, The Scrum Guide, Julio 2016,]
\begin{itemize}
	\item Planificación del Sprint (\textit{Sprint Planning})
	\item Scrum Diario (\textit{Daily Scrum})
	\item Revisión del Sprint (\textit{Sprint Review})
	\item Retrospectiva del Sprint (\textit{Sprint Retrospective})
\end{itemize}

\noindent El Equipo Scrum consiste en los siguientes roles:
\begin{itemize}
	\item El Dueño de Producto (\textit{Product Owner}):\\
	Responsable de maximizar el valor del producto y el trabajo del Equipo de Desarrollo. Es la única persona responsable de gestionar la Lista del Producto (\textit{Product Backlog}).
	\item El Equipo de Desarrollo (\textit{Development Team}):\\
	Profesionales que realizan el trabajo de entregar un Incremento de producto “Terminado” que potencialmente se pueda poner en producción al final de cada Sprint. Solo los miembros del Equipo de Desarrollo participan en la creación del Incremento.
	\item  El Scrum Master:\\
	Responsable de asegurar que Scrum se entienda y se adopte, asegurándose de que el Equipo Scrum trabaja ajustándose a la teoría, prácticas y reglas de Scrum. Ayuda a las personas externas al Equipo Scrum a entender qué interacciones con el Equipo Scrum pueden ser útiles y cuáles no.
\end{itemize}

\noindent La metodología Scrum cuenta con los elementos descritos a continuación:

\begin{itemize}
	\item Lista de Producto (\textit{Product Backlog}):\\
	Es el listado de todas las tareas que necesita el proyecto para alcanzar su realización. Al iniciar el desarrollo del
	proyecto esta lista no se encuentra completa y conforme avanzan los \textit{sprints} se le añaden tareas para solventar las necesidades que	van surgiendo gracias a la retroalimentación del cliente.
	\item Lista de Pendientes del Sprint (\textit{Sprint Backlog}) : \\
	Es la lista de tareas seleccionadas del product backlog que se planifica realizar durante el periodo del \textit{sprints} y se definen a los responsables de cada tarea.
	\item Sprint:\\
	Es un periodo de tiempo determinado donde el equipo completa conjuntos de tareas incluidas en el \textit{backlog}.
	\item Incremento : \\
	El Incremento es la suma de todos los elementos de la Lista de Producto completados durante un \textit{sprints} y el valor de los incrementos de todos los Sprints anteriores
\end{itemize}

La metodología nos permite definir un periodo de hasta un mes para cada sprint y se ha optado por un periodo de 30 días, contemplándose un total de ocho sprints, donde al término de cada uno se tendrá un avance del sistema.

\subsubsection{Historias de usuario}

De acuerdo con [https://www.scrum.mx/informate/historias-de-usuario], son la técnica por la cuál el usuario especifica de manera general
los requerimientos que el sistema debe cumplir.


Normalmente estas redacciones se llevan a cabo en tarjetas de papel donde se describe de brevemente las funciones que el producto final
debe poseer, sean requisitos fucionales o no funcionales.


El tratamiento de las historias de ususario en flexible y dinámico, cada una de ellas es lo suficientemente detallada y delimitada para que el equipo de desarrollo pueda implementarla durante la duración del sprint. 

Es habitual que se siga una plantilla para estas tarjetas como la siguiente:

\begin{itemize}
	\item Como \textbf{<Usuario>}
	\item Quiero \textbf{<algún objetivo>}
	\item Para \textbf{<motivo>}
\end{itemize}


Una de sus grandes ventajas es que dado el caso que un usuario no sea lo suficientemente detallista con la historia, esta se puede partir en historias más pequeñas antes de el equipo empiece a trabajar en ella.


Un ejemplo de historia de usuario para el desarrollo de este proyecto, siguiento la plantilla seria:



\begin{itemize}
	\item Como usuario
	\item Quiero poder ingresar al sistema con mi correo y contraseña
	\item Para tener acceso a sus funciones
\end{itemize}

Otra forma de darle detalle a las historias de usuario es mediante el añadido de un criterio de aceptación. Un criterio de aceptación es una pruebaque será cierta cuando el equipo de desarrollo complete la descripción de la tarjeta.


Siguiendo con esta linea a continuacion se listan las principales historias de usuario que se consideraron para el desarrollo de este trabajo terminal, tenga en cuenta que algunas de ellas tiene criterios de aceptación pero en otras no se considero necesaria ya que son explicitamente claras.


\noindent\rule{\textwidth}{1pt}
\begin{itemize}
	\item No. 1
	\item Como usuario
	\item Quiero poder ingresar al sistema con mi correo y contraseña
	\item Para tener acceso a sus funciones.
	\item Criterios de aceptación
	\begin{itemize}
		\item El usuario recibirá un correo electrónico de confirmación de su alta en el sistema con el correo y contraseña que ingreso para tenerlos de respaldo.
	\end{itemize}
\end{itemize}
\noindent\rule{\textwidth}{1pt}
\begin{itemize}
	\item No. 2
	\item Como usuario
	\item Quiero poder recuperar mi contraseña en caso de olvidarla
	\item Para no perder el trabajo realizado en el sistema.
	\item Criterios de aceptación
	\begin{itemize}
		\item El usuario podra ingresar una nueva contraseña, siempre y cuando recuerde el correo electrónico con el que se dio de alta en el sistema.
		\item Al ingresar una nueva contraseña, recibirá un correo de confirmación del cambio de contraseña y sus datos permaneceran intactos.
	\end{itemize}
\end{itemize}
\noindent\rule{\textwidth}{1pt}
\begin{itemize}
	\item No. 3
	\item Como usuario del sistema quiero darme de alta con una contraseña facil de recordar
	\item Pero que este segura en la base de datos,
	\item Para no tener comprometido los diagramas que yo pueda generar en el sistema.

	\item Criterios de aceptación
	\begin{itemize}
		\item Asegurarse que usuario ingrese una contraseña de al menos 6 caracteres.
		\item Se le solicitará al usuario que ingrese 2 veces la misma contraseña para asegurarse le es fácil recordarla.
		\item Antes de guardar la contraseña, esta debera pasar por un metodo que la haga ilegible para el usuario(algun algoritmo de digestion o cifrado).
	\end{itemize}
\end{itemize}
\noindent\rule{\textwidth}{1pt}
\begin{itemize}
	\item No. 4
	\item Como usuario quiero poder crear un diagrama entidad-relacióno o diagrama ER arrastrando y soltando elementos de una paleta de iconos
	\item Para hacerlo de manera mas facíl e intuitiva.
	\item Criterios de aceptación
	\begin{itemize}
		\item El usuario podra empezar un nuevo diagrama al seleccionar la opción "Entidad-Relación".
		\item Tendrá a su disposición un menú de iconos con todos los elementos permitidos en un diagrama ER estandar.
		\item Podra arrastrar y soltar los iconos a un area delimitada para empezar con el diseño de su diagrama.
	\end{itemize}
\end{itemize}
\noindent\rule{\textwidth}{1pt}
\begin{itemize}
	\item No. 5
	\item Como usuario quiero poder guardar mi último trabajo realizado en el diagramador ER.
	\item Para poder consultarlo en otro momento.
	\item Criterios de aceptación
	\begin{itemize}
		\item Dispondra de un botón para poder guardar en la base de datos el diagrama que este creando/editando.
		\item Antes de almacenar el diagrama en el canvas, se le mostrará un mensaje de confirmación para guardar su diagrama actual.
	\end{itemize}m
\end{itemize}
\noindent\rule{\textwidth}{1pt}
\begin{itemize}
	\item No. 6
	\item Como usuario me gustaría poder ver mi último trabajo que realice
	\item Cuando seleccione la opción "Entidad-Relación"
	\item Para poder modificar el diseño.
\end{itemize}
\noindent\rule{\textwidth}{1pt}
\begin{itemize}
	\item No. 7
	\item Como usuario quiero tener la opción de cargar un diagrama a partir de un archivo
	\item Para hacer modificación en dicho diagrama y guardarlo de ser necesario.
	\item Criterios de aceptación
	\begin{itemize}
		\item El usuario tendra un botón "cargar" en el menú del diagramador ER para poder importar un archivo con extensión .json
		\item Al importar el archivo, este pasara por un proceso de validación para asegurarse que es un archivo json valido.
		\item Durante el proceso de validación se verificara que el contenido del archivo es un diagrama compatible con la estructura de los generados por el diagramador ER.
		\item Al contener un información compatible, se mostrará en el canvas el contenido del archivo.
	\end{itemize}
\end{itemize}
\noindent\rule{\textwidth}{1pt}
\begin{itemize}
	\item No. 8
	\item Como usuario quiero poder descargar el diagrama que este visible en la pagina web
	\item Para poder distribuirlo como yo desee.
	\item Criterios de aceptación
	\begin{itemize}
		\item El usuario dispondrá de un botón "Descargar" en el diagramador ER para obtener un archivo con el contenido del diagrama visible en el canvas.
		\item El archivo generado será de extensión .json con la información necesaria para que el diagramador pueda cargarlo en otro momento.
	\end{itemize}
\end{itemize}
\noindent\rule{\textwidth}{1pt}
\begin{itemize}
	\item No. 9
	\item Como usuario quiero tener una forma de validar mi diagrama ER
	\item Por que es importante saber si el diagrama que estoy creando cumple con las reglas de validación.
	\item Criterios de aceptación
	\begin{itemize}
		\item El usuario tendra disponible un botón que al darle click iniciara un proceso de validación del diagrama actual del canvas.
		\item Al termino del proceso de validación se le mostrará un mensaje al usuario indicando si el diagrama cumple o no las reglas de un diagrama ER.
	\end{itemize}
\end{itemize}
\noindent\rule{\textwidth}{1pt}
\begin{itemize}
	\item No. 10
	\item Como usuario en caso de tener un diagrama valido
	\item Me gustaría poder tranformar el diagrama ER en su versión del modelo relacional.
	\item Para poder obtener las sentencias SQL equivalentes.
	\item Criterios de aceptación
	\begin{itemize}
		\item El usuario dispondra de la opción de tranformar al modelo relacional solamente despues de haber validado que el diagrama ER cumple con las reglas.
		\item Posterior a la validación se le mostrará al ususario en mensaje de confirmación y un botón para disparar el proceso de transformación a su equivalente relacional.
		\item Al terminar el proceso de transformación equivalente se le redirijira al menú "Relacional" donde podra visualizar el equivalente al modelo relacional.
	\end{itemize}
\end{itemize}
\noindent\rule{\textwidth}{1pt}
\begin{itemize}
	\item No. 11
	\item Como usuario despues de observar el esquema relacional
	\item Quiero poder obtener las sentencias SQL equivalentes
	\item Para poder crear el mismo esquema de base de datos relacional en un SBGB
	\item Criterios de aceptación
	\begin{itemize}
		\item Las sentencias SQl solo podran ser descargadas en el menú ``Relacional" a un archivo con extension .sql dando click a un botón con la leyenda ``Descargar SQL".
		\item Solo podran ser obtenidas las sentencias SQL de un diagrama ER creado y/o validado por el sistema.
	\end{itemize}
\end{itemize}
\noindent\rule{\textwidth}{1pt}
\begin{itemize}
	\item No. 12
	\item Como usuario una vez observado el equivalente relacional del diagrama ER
	\item Quiero poder iniciar el proceso de transformación al modelo no relacional
	\item Para poder observar el cambio entre modelos.
	\item Criterios de aceptación
	\begin{itemize}
		\item Al dar click al botón "Transformar a NoSQl" el usuario iniciará el proceso para obtener el equivalente del modelo relacional al modelo NR.
		\item Al termino del proceso de transformación se le redirigirá la menú "No Relacional" donde observará el modelo NR equivalente a su diagrama ER.
	\end{itemize}
\end{itemize}
\noindent\rule{\textwidth}{1pt}
\begin{itemize}
	\item No. 13
	\item Como usuario quiero poder transformar mi diagrama ER en su equivalente modelo no relacional
	\item Para poder observar el cambio entre modelos.
	\item Criterios de aceptación
	\begin{itemize}
		\item El usuario dispondra de la opción de tranformar al modelo relacional solamente despues de haber validado que el diagrama ER cumple con las reglas.
		\item Posterior a la validación se le mostrará al ususario en mensaje de confirmación y un botón para disparar el proceso de transformación al modelo no relacional o modelo NR.
		\item Una vez validado el diagrama ER e iniciado el proceso para la transformación al modelo NR se le indicará al usuario que el proceso puede ser tardado ya que primero realizará la transformación al modelo relacional.
		\item Al termino del proceso de transformación se le redirigirá la menú "No Relacional" donde observará el modelo NR equivalente a su diagrama ER.
	\end{itemize}
\end{itemize}
\noindent\rule{\textwidth}{1pt}
\begin{itemize}
	\item No. 14
	\item Como usuario me gustaría tener un reporte técnico y
	\item Quiero que la redacción sea legible y referenciada
	\item Para compartirlo en el futuro con equipos de desarrollo y ver la posibilidad de agregar nuevas funciones al sistema.
\end{itemize}
\noindent\rule{\textwidth}{1pt}




Teniendo en cuenta que se esta trabajando con una metodología ágil, estas historias de usuario pueden aumentar o en su defecto dividirse en historias mas pequeñas dependiendo de las criterios del equipo de desarrollo durante el proceso de la implementación de cada una de ellas.




\subsubsection{Lista de Producto (\textit{Product Backlog})}
La Lista de Producto es una lista ordenada de todo lo que podría ser necesario en el producto y es la fuente de requisitos para cualquier cambio a realizarse en el producto, enumera todas las características, funcionalidades, requisitos, mejoras y correcciones que constituyen cambios a realizarse sobre el producto para entregas futuras.[http://openaccess.uoc.edu/webapps/o2/bitstream/10609/17885/1/mtrigasTFC0612memoria.pdf]

Se consideran la siguiente lista como el product backlog con las tareas necesarias para complir con todas las historias de usuario mencionadas en la sección anterior.
Considerando que estas pueden cambiar conforme avancen los sprints y añadirse nuevas tareas.


\begin{longtable}{ p{2cm} | p{10cm} | p{2cm} }

	\hline
	No. Historia de Usuario & requerimiento/Tarea &  Responsable \\[0.5cm]
	\hline
	\hline

	\endfirsthead

	\multicolumn{3}{c}{Continuación de tabla product backlog }\\
	\hline
	\hline
	\endhead

	\hline
	\hline
	%\caption{Product Backlog}
	\endlastfoot

	% template row table
	% \centering & & & \\[0.5cm]
	% \hline

	\centering 14 & Investigación Bases de datos relacionales. & Eduardo/Omar \\[0.5cm]
	\hline
	\centering 14 & Redacción y selección de las tecnologías a utilizar para el desarrollo de la plataforma.  & Eduardo \\[0.5cm]
	\hline
	\centering 14 & Investigación Bases de datos relacionales.  & Eduardo/Omar \\[0.5cm]
	\hline
	\centering 14 & Redacción bases de datos relacionales	en el documento técnico.  & Eduardo \\[0.5cm]
	\hline
	\centering 14 & Investigación Bases de datos no relacionales  & Eduardo/Omar \\[0.5cm]
	\hline
	\centering 14 & Redacción bases de datos no relacionales en el documento técnico.  & Eduardo \\[0.5cm]
	\hline
	\centering 14 & Investigación y selección del modelo de base de datos no relacional a utilizar junto a las	tecnologías a utilizar  & Eduardo/Omar \\[0.5cm]
	\hline
	\centering 14 & Análisis y diseño de la aquitectura Web  & Eduardo/Omar \\[0.5cm]
	\hline
	\centering 1 & Desarrollo de la estructura básica del backend.  & Omar \\[0.5cm]
	\hline
	\centering 1 & Desarrollo de la estructura	básica del frontend.  & Eduardo \\[0.5cm]
	\hline
	\centering 1 & Agregar servicio backend para registrar un usuario & Omar \\[0.5cm]
	\hline
	\centering 1 & Agregar formulario para captura de datos de registro de un usuario en el frontend. & Eduardo \\[0.5cm]
	\hline
	\centering 2 & Agregar servicio backend para recuperar contraseña del usuario. & Omar \\[0.5cm]
	\hline
	\centering 2 & Agregar servicio backend para envio de email al usuario registrado y de recuperación de contraseña. & Omar \\[0.5cm]
	\hline
	\centering 2 & Agregar vista con formulario para recuperación de contraseña del usuario enel frontend. & Eduardo \\[0.5cm]
	\hline
	\centering 2 & Integración de los servicios de registro y recuaperación de contraseña en el frontend & Eduardo \\[0.5cm]
	\hline
	\centering 3 & Agregar servicio backend para hacer ilegible la contraseña del usuario en la base de datos. & Omar \\[0.5cm]
	\hline
	\centering 4 & Planteamiento de escenarios de los esquemas entidad-relación.  & Eduardo \\[0.5cm]
	\hline
	\centering 4 & Agregar a la interfaz gráfica de la aplicación web el menú "Entidad-Relaciín". & Eduardo \\[0.5cm]
	\hline
	\centering 4 & Agregar iconos de los elementos basicos de un diagrama ER en el diagramador. & Eduardo \\[0.5cm]
	\hline
	\centering 5 & Agregar servicio backend para guardar un diagrama ER en formato json en la base de datos e integrarlo al frontend. & Omar \\[0.5cm]
	\hline
	\centering 5 & Agregar servicio backend para recuperar el diagrama guardado del usuario de la base de datos y regresarlo en formato json.  & Omar \\[0.5cm]
	\hline
	\centering 6 & Recuperar el último diagrama del usuario del backend y monstrarlo en el frontend & Eduardo \\[0.5cm]
	\hline
	\centering 6 & Manejar el estado de la intefaz web para no perder el diagrama ER que esta editando el usuario. & Eduardo \\[0.5cm]
	\hline
	\centering 6 & Definición de las reglas del modelo	entidad-relación. & Eduardo \\[0.5cm]
	\hline
	\centering 4 & Implementar la edición de diagramas ER en el frontend.  & Eduardo \\[0.5cm]
	\hline
	\centering 7 & Habilitar la carga de un archivo en la aplicación web.  & Omar \\[0.5cm]
	\hline
	\centering 7 & Agregar la función para validar el contenido del archivo json y pintarlo en el canvas. & Eduardo \\[0.5cm]
	\hline
	\centering 8 & Agregar la descarga del diagrama visible en el canvas a un archivo json. & Eduardo \\[0.5cm]
	\hline
	\centering 4 & Agregar elementos del modelo ER extendido al diagramador ER.  & Eduardo \\[0.5cm]
	\hline
	\centering 9 & Agregar botón de validar al frontend y mostrar loader mientras se procesa el diagrama ER. & Eduardo/Omar \\[0.5cm]
	\hline
	\centering 9 & Agregar servicio backend para la validación del diagrama entidad-relación. & Eduardo/Omar \\[0.5cm]
	\hline
	\centering 9 & implementación de algoritmo para validación del diagrama ER en el backend. & Eduardo/Omar \\[0.5cm]
	\hline
	\centering 9 & Pruebas de captura de distintos diagramas entidad-relación.  & Eduardo/Omar \\[0.5cm]
	\hline
	\centering 9 & Pruebas para validar el algoritmo de validación. & Eduardo/Omar \\[0.5cm]
	\hline
	\centering 10 & Agregar servicio al backend para transformación del esquema entidad-relación al modelo relacional.  & Omar \\[0.5cm]
	\hline
	\centering 10 & Implementación del algoritmo de transformación ER -> relacional & Eduardo/Omar \\[0.5cm]
	\hline
	\centering 10 & Agregar menú "Relacional" a la intefaz gráfica. & Eduardo/Omar \\[0.5cm]
	\hline
	\centering 10 & Prueba de transformación de distintos diagramas E-R al modelo relacional. & Eduardo/Omar \\[0.5cm]
	\hline
	\centering 10 & Visualización de la transformación del modelo ER al modelo Relacinal. & Eduardo \\[0.5cm]
	\hline
	\centering 14 & Revision de la redacción del reporte técnico para presentación de TT1  & Eduardo \\[0.5cm]
	\hline
	\centering 11 & Agregar servicio backend para la descarga del archivo .sql con las sentencias equivalentes. & Eduardo/Omar \\[0.5cm]
	\hline
	\centering 11 & Pruebas de coherencia de las sentencias equivalentes de distintos en el SGBD. & Eduardo/Omar \\[0.5cm]
	\hline
	\centering 12 & Definición de las reglas de transformación al modelo noSQL  & Eduardo/Omar \\[0.5cm]
	\hline
	\centering 12 & Pruebas de distintos escenarios del modelo relacional al modelo noSQL.  & Eduardo/Omar \\[0.5cm]
	\hline
	\centering 12 & Agregar servicio al backend para transformación del esquema relacional al modelo noSQL.  & Omar \\[0.5cm]
	\hline
	\centering 12 & Agregar servicio al backend para guardar el modelo noSQL en la base de datos.  & Omar \\[0.5cm]
	\hline
	\centering 12 & Agregar menú "No Relacional" a la intefaz gráfica. & Omar \\[0.5cm]
	\hline
	\centering 12 & Implementación del algortimo de transformación de modelo relacional al modelo noSQL. & Eduardo \\[0.5cm]
	\hline
	\centering 12 & Comprobación de la coherencia de la transformación entre modelos relacional a no relacional.  & Eduardo/Omar \\[0.5cm]
	\hline
	\centering 13 & Agregar servicio backend para transformación del modelo ER al modelo no relacional. & Eduardo/Omar \\[0.5cm]
	\hline
	\centering 13 & Ajustar la interfaz del menu "Entidad-Relación" para mostrar mensaje tranformación la modelo no relacioal. & Omar \\[0.5cm]
	\hline
	\centering 13 & Manejar el estado del diagrama ER y redireccionarl al menú "No Relacional" al terminar la tranformación. & Eduardo \\[0.5cm]
	\hline
	\centering 13 & Pruebas de caso de estudio para verificar la correcta transformación y coherencia en los datos.  & Eduardo/Omar \\[0.5cm]
	\hline
	\centering 14 & Revisión de la redacción del reporte técnico para presentación de TT2 & Eduardo \\[0.5cm]
	\hline
\end{longtable}