\section{Metodología}

De acuerdo a la Universidad Católica los Ángeles\cite{universidad_catolica_los_angeles_metodologidesarrollo_2020}, en el campo del desarrollo de \textit{software} hay dos grupos de metodologías, las tradicionales y las ágiles.


Las tradicionales suelen ser rígidas en cuanto a la documentación, exigiendo que sea exhaustiva y centrándose en cumplir con un plan de trabajo establecido en la etapa inicial del proyecto, mientas que las ágiles permiten realizar cambios en los requerimientos conforme a los avances del mismo.




Dado que cualquier cambio en el proceso de una metodologia tradicional genera la necesidad de una reconstrucción del plan de trabajo y requiere invertir tiempo de desarrollo, surgieron las llamadas metodologías ágiles que permiten realizar cambios en los requerimientos conforme a los avances en el proyecto, tomando en cuenta la habilidades del equipo de desarrollo y una relación con el cliente; esta forma de trabajo permite mostrar avances funcionales en el producto en un periodo de tiempo corto para realizar una evaluación y, en caso de ser requerido, se sugieran cambios.


Se han propuesto muchos modelos ágiles de proceso y están en uso en toda la industria. Entre ellos se encuentran los siguientes:


\begin{itemize}
	\item Desarrollo adaptativo de software (DAS).
	\item Scrum.
	\item Método de desarrollo de sistemas dinámicos (MDSD).
	\item Cristal.
	\item Desarrollo impulsado por las características (DIC).
	\item Desarrollo esbelto de software (DES).
	\item Modelado ágil (MA).
	\item Proceso unificado ágil (PUA).
\end{itemize}

\subsection{Scrum}

De acuerdo con \textit{The Scrum Guide}\cite{the_scrum_guide_definitive_2020}, Scrum es un marco de trabajo para la entrega de productos incrementales y de máximo valor productivo.

Un artefacto es un elemento que garantiza la transparencia, es el registro de la información fundamental del proceso Scrum y a continuación se describen los 4 artefactos principales de Scrum:

\begin{itemize}
	\item Lista de producto (\textit{product backlog}):	es el listado de todas las tareas que necesita el proyecto para alcanzar su realización. Al iniciar el desarrollo del
	proyecto esta lista no se encuentra completa y conforme avanzan los \textit{sprints} se le añaden tareas para solventar las necesidades que	van surgiendo gracias a la retroalimentación del cliente.
	\item Lista de pendientes del \textit{sprint} (\textit{sprint backlog}): es la lista de tareas seleccionadas del product backlog que se planifica realizar durante el periodo del \textit{sprint} y se definen a los responsables de cada tarea.
	\item \textit{Sprint}: es el corazón de Scrum; es de un periodo de tiempo determinado de un mes o menos, donde el equipo completa conjuntos de tareas incluidas en el \textit{backlog} para crear un incremento del producto utilizable.
	\item Incremento: es la suma de todos los elementos de la lista de producto completados durante un \textit{sprint} unido con los incrementos de los \textit{sprints} anteriores. Al finalizar el \textit{sprint} el nuevo incremento debe estar en condiciones de ser utilizado.
\end{itemize}

La metodología nos permite definir un periodo de hasta un mes para cada \textit{sprint} y se ha optado por un periodo de 30 días, contemplándose un total de ocho sprints, donde al término de cada uno se tendrá un avance del sistema.

Asimismo, Scrum tiene cuatro eventos formales contenidos dentro del \textit{sprint} para la inspección y adaptación del producto que se describen a continuación:
\begin{itemize}
	\item Planificación del \textit{sprint} (\textit{sprint planning}): es una reunión con todo el equipo Scrum con una duración máxima de 8 horas para el \textit{sprint} de un mes en la que el Scrum master es el encargado de que los asistentes entiendan el propósito de dicha reunión.
	\item Scrum diario (\textit{daily scrum}): es una reunión de máximo 15 minutos en la cual el equipo de desarrollo expone sus actividades y planifica las tareas de las próximas 24 horas.
	\item Revisión del \textit{sprint} (\textit{sprint review}): al finalizar cada \textit{sprint} se lleva a cabo un reunión para la revisión del incremento del producto y en caso de ser necesario realizar ajustes a la lista de producto.
	\item Retrospectiva del \textit{sprint} (\textit{sprint retrospective}): es el momento para el equipo Scrum de pensar en mejoras para el próximo \textit{sprint} que contribuyan al proyecto.
\end{itemize}

El equipo Scrum consiste en los siguientes roles:
\begin{itemize}
	\item El dueño de producto (\textit{product owner}): es la persona responsable de maximizar el valor del producto y el trabajo del equipo de desarrollo; es la única responsable de gestionar la lista de producto y cualquier cambio a esa lista debe ser revisada y aprobada por él.
	\item El equipo de desarrollo (\textit{development team}): es el conjunto de profesionales que realizan el trabajo para la entrega de un incremento en producto en cada \textit{sprint}; son un grupo autoorganizado y multifuncional donde cada miembro del equipo tiene habilidades especializadas pero que la responsabilidad de las tareas completadas, incrementos del producto o retrasos recaen en el equipo como un todo.
	\item  El Scrum master: es la persona responsable de asegurar que Scrum es entendido y adoptado por todos los involucrados en el proyecto, asegurándose de ayudar a las personas externas al equipo Scrum a entender qué interacciones con el equipo pueden ser útiles y cuáles no.
\end{itemize}


\subsubsection{Historias de usuario}

De acuerdo con Scrum México\cite{scrum_mexico_scrum_2020}, las historias de usuario son la técnica por la que el usuario especifica de manera general los requerimientos que el sistema debe cumplir.


Normalmente estas redacciones se llevan a cabo en tarjetas de papel donde se describen brevemente las funciones que el producto final debe poseer sean o no requisitos fucionales.


El tratamiento de las historias de usuario es flexible y dinámico, cada una de ellas es lo suficientemente detallada y delimitada para que el equipo de desarrollo pueda implementarla durante la duración del \textit{sprint}.


Es habitual que se siga una plantilla para estas tarjetas como la siguiente:

\begin{itemize}
	\item Como \textbf{<Usuario>}
	\item Quiero \textbf{<algún objetivo>}
	\item Para \textbf{<motivo>}
\end{itemize}


Una de sus grandes ventajas dado el caso de que un usuario no sea lo suficientemente detallista con la historia, esta se puede partir en historias más pequeñas antes de que el equipo empiece a trabajar en ella.


Un ejemplo de historia de usuario para el desarrollo seria:

\begin{itemize}
	\item Como usuario
	\item Quiero poder ingresar al sistema con mi correo y contraseña
	\item Para tener acceso a sus funciones
\end{itemize}


Otra forma de darle detalle a las historias de usuario es mediante el añadido de un criterio de aceptación. Un criterio de aceptación es una prueba que será cierta cuando el equipo de desarrollo complete la descripción de la tarjeta.


A continuación se listan las principales historias de usuario que se consideraron para el desarrollo de la propuesta de solución; tenga en cuenta que algunas de ellas tienen criterios de aceptación pero en otras no se consideró necesarias porque son explicitamente claras.


\noindent\rule{\textwidth}{1pt}
\begin{itemize}
	\item No. 1
	\item Como usuario
	\item Quiero poder ingresar al sistema con mi correo y contraseña
	\item Para tener acceso a sus funciones
	\item Criterios de aceptación
	\begin{itemize}
		\item El usuario recibirá un correo electrónico de confirmación de su alta en el sistema con el correo y contraseña que ingresó para tenerlos de respaldo.
	\end{itemize}
\end{itemize}
\noindent\rule{\textwidth}{1pt}
\begin{itemize}
	\item No. 2
	\item Como usuario
	\item Quiero poder recuperar mi contraseña en caso de olvidarla
	\item Para no perder el trabajo realizado en el sistema
	\item Criterios de aceptación
	\begin{itemize}
		\item El usuario podrá ingresar una nueva contraseña, siempre y cuando recuerde el correo electrónico con el que se dio de alta en el sistema.
		\item Al ingresar una nueva contraseña, recibirá un correo de confirmación del cambio de contraseña y sus datos permanecerán intactos.
	\end{itemize}
\end{itemize}
\noindent\rule{\textwidth}{1pt}
\begin{itemize}
	\item No. 3
	\item Como usuario del sistema quiero darme de alta con una contraseña facil de recordar
	\item Pero que esté segura en la base de datos,
	\item Para no tener comprometidos los diagramas que yo pueda generar en el sistema.

	\item Criterios de aceptación
	\begin{itemize}
		\item Asegurarse que usuario ingrese una contraseña de al menos 8 caracteres.
		\item Se le solicitará al usuario que ingrese 2 veces la misma contraseña para asegurarse que le es fácil recordarla y que efectivamente es la misma.
		\item Antes de guardar la contraseña, esta deberá pasar por un método que la haga ilegible para el usuario (algún algoritmo de digestión o cifrado).
	\end{itemize}
\end{itemize}
\noindent\rule{\textwidth}{1pt}
\begin{itemize}
	\item No. 4
	\item Como usuario quiero poder crear un diagrama ER arrastrando y soltando elementos de una ``paleta"
	\item Para hacerlo de manera mas fácil e intuitiva.
	\item Criterios de aceptación
	\begin{itemize}
		\item El usuario podrá empezar un nuevo diagrama al seleccionar la opción de ``diagramador ER".
		\item Tendrá a su disposición una paleta con los elementos permitidos en un diagrama ER estándar.
		\item Podrá arrastrar y soltar los elementos de la paleta a un área delimitada para empezar con el diseño de su diagrama.
	\end{itemize}
\end{itemize}
\noindent\rule{\textwidth}{1pt}
\begin{itemize}
	\item No. 5
	\item Como usuario quiero poder guardar mi último trabajo realizado en el diagramador ER.
	\item Para poder consultarlo en otro momento.
	\item Criterios de aceptación
	\begin{itemize}
		\item Dispondrá de un botón para poder guardar en la base de datos el diagrama que esté creando/editando.
		\item Antes de almacenar el diagrama en el \textit{canvas} o zona de diagramado, se le mostrará un mensaje de confirmación para guardar su diagrama actual.
	\end{itemize}
\end{itemize}
\noindent\rule{\textwidth}{1pt}
\begin{itemize}
	\item No. 6
	\item Como usuario me gustaría poder ver mi último trabajo que realicé
	\item Cuando seleccione la opción ``Entidad-Relación"
	\item Para poder modificar el diseño.
\end{itemize}
\noindent\rule{\textwidth}{1pt}
\begin{itemize}
	\item No. 7
	\item Como usuario quiero tener la opción de cargar un diagrama a partir de un archivo
	\item Para hacer modificación de dicho diagrama y guardarlo de ser necesario.
	\item Criterios de aceptación
	\begin{itemize}
		\item El usuario tendra un botón ``cargar" en el menú del diagramador ER para poder importar un archivo con extensión .json
		\item Al importar el archivo, este pasará por un proceso de validación para asegurarse que es un archivo JSON válido.
		\item Durante el proceso de validación se verificará que el contenido del archivo es un diagrama compatible con la estructura de los generados por el diagramador ER.
		\item Al contener un información compatible, se mostrará en la zona de diagramado el contenido del archivo.
	\end{itemize}
\end{itemize}
\noindent\rule{\textwidth}{1pt}
\begin{itemize}
	\item No. 8
	\item Como usuario quiero poder descargar el diagrama que esté visible en la página web
	\item Para poder distribuirlo como yo desee.
	\item Criterios de aceptación
	\begin{itemize}
		\item El usuario dispondrá de un botón ``Descargar" en el diagramador ER para obtener un archivo con el contenido del diagrama visible en la zona de diagramado.
		\item El archivo generado será de extensión .json con la información necesaria para que el diagramador pueda cargarlo en otro momento.
	\end{itemize}
\end{itemize}
\noindent\rule{\textwidth}{1pt}
\begin{itemize}
	\item No. 9
	\item Como usuario quiero tener una forma de validar mi diagrama ER
	\item Por que es importante saber si el diagrama que estoy creando es un diagrama válido del modelo ER.
	\item Criterios de aceptación
	\begin{itemize}
		\item El usuario tendrá disponible un botón que al darle click iniciará un proceso de validación del diagrama actual de la zona de diagramado.
		\item Al término del proceso de validación se le mostrará un mensaje al usuario indicando si el diagrama cumple o no las reglas del modelo ER.
	\end{itemize}
\end{itemize}
\noindent\rule{\textwidth}{1pt}
\begin{itemize}
	\item No. 10
	\item Como usuario en caso de tener un diagrama ER válido
	\item Me gustaría poder tranformar el diagrama ER en su versión del modelo relacional.
	\item Para poder ver el equivalente del diagrama ER en el modelo Relacional.
	\item Criterios de aceptación
	\begin{itemize}
		\item El usuario dispondrá de la opción de tranformar al modelo relacional solamente despues de haber validado que el diagrama ER cumple con las reglas.
		\item Posterior a la validación se le mostrará al ususario en mensaje de confirmación y un botón para disparar el proceso de transformación a su equivalente relacional.
		\item Al terminar el proceso de transformación equivalente se le redirijira al menú ``Relacional" donde podrá visualizar el equivalente al modelo relacional.
	\end{itemize}
\end{itemize}
\noindent\rule{\textwidth}{1pt}
\begin{itemize}
	\item No. 11
	\item Como usuario después de observar el diagrama relacional
	\item Quiero poder obtener las sentencias SQL equivalentes
	\item Para poder crear el esquema de base de datos relacional en un SBGB
	\item Criterios de aceptación
	\begin{itemize}
		\item Las sentencias SQl solo podrán ser descargadas en el menú ``Relacional" a un archivo con extension .sql dando click a un botón con la leyenda ``Descargar SQL".
		\item Solo podrán ser obtenidas las sentencias SQL de un diagrama ER creado y/o validado por el sistema.
	\end{itemize}
\end{itemize}
%\noindent\rule{\textwidth}{1pt}
%\begin{itemize}
	%\item No. 12
	%\item Como usuario una vez observado el equivalente relacional del diagrama ER
	%\item Quiero poder iniciar el proceso de transformación al modelo no relacional
	%\item Para poder observar el cambio entre modelos.
	%\item Criterios de aceptación
	%\begin{itemize}
		%\item Al dar click al botón ``Transformar a NoSQl" el usuario iniciará el proceso para obtener el equivalente del modelo relacional al modelo NR.
		%\item Al término del proceso de transformación se le redirigirá al menú ``No Relacional" donde observará el modelo NoSQL equivalente a su diagrama ER.
%	\end{itemize}
%\end{itemize}
\noindent\rule{\textwidth}{1pt}
\begin{itemize}
	\item No. 12
	\item Como usuario quiero poder transformar mi diagrama ER en su equivalente modelo concepcual NoSQL
	\item Para poder observar el cambio entre modelos.
	\item Criterios de aceptación
	\begin{itemize}
		\item El usuario dispondrá de la opción de tranformar al modelo NoSQL solamente después de haber validado que el diagrama ER cumple con las reglas.
		\item Posterior a la validación se le mostrará al usuario un mensaje de confirmación y un botón para disparar el proceso de transformación al modelo conceptual NoSQL.
		\item Una vez validado el diagrama ER e iniciado el proceso para la transformación al modelo conceptual NoSQL se le indicará al usuario que el proceso puede ser tardado.
		\item Al término del proceso de transformación se le redirigirá al menú ``No Relacional" donde observará el modelo conceptual NoSQL equivalente a su diagrama ER.
	\end{itemize}
\end{itemize}
\noindent\rule{\textwidth}{1pt}
\begin{itemize}
	\item No. 13
	\item Como usuario quiero poder obtener el \textit{script} desde el modelo conceptual NoSQL
	\item Para poder generar la base de datos en un gestor de base de datos NoSQL orientado a documentos
	\item Criterios de aceptación
	\begin{itemize}
		\item El usuario dispondrá de la opción de obtener el \textit{script} solamente después de haber validado que el diagrama ER cumple con las reglas.
		\item Posterior a la validación se le mostrará al usuario un mensaje de confirmación y un botón para disparar el proceso de generación de \textit{scripts} para el gestor de base de datos orientado a documentos.
		\item Una vez empezado el proceso de generación de \textit{scripts} se le indicará al usuario que el proceso puede ser tardado.
		\item Al término del proceso de transformación se le redirigirá al menú ``No Relacional" donde observará los \textit{scripts} NoSQL.
	\end{itemize}
\end{itemize}

\noindent\rule{\textwidth}{1pt}
\begin{itemize}
	\item No. 14
	\item Como usuario me gustaría tener un reporte técnico y
	\item Quiero que la redacción sea legible y referenciada
	\item Para compartirlo en el futuro con equipos de desarrollo y ver la posibilidad de agregar nuevas funciones al sistema.
\end{itemize}
\noindent\rule{\textwidth}{1pt}




Teniendo en cuenta que se está trabajando con una metodología ágil, estas historias de usuario pueden aumentar o en su defecto dividirse en historias más pequeñas dependiendo de los criterios del equipo de desarrollo durante el proceso de la implementación de cada historia.




\subsubsection{Lista de producto (\textit{product backlog})}

De acuerdo a Trigas Gallego\cite{manuel_trigas_gallego_metodologiscrum_2020}, la lista de producto es una lista ordenada de todo lo que podría ser necesario en el producto y es la fuente de requisitos para cualquier cambio a realizarse en el producto, enumera todas las características, funcionalidades, requisitos, mejoras y correcciones que constituyen cambios a realizarse sobre el producto para entregas futuras.


Se consideran la siguiente lista como la lista de producto con las tareas necesarias para complir con todas las historias de usuario mencionadas en la sección anterior.


Considerando que estas pueden cambiar conforme avancen los \textit{sprints} y añadirse nuevas tareas.


\begin{longtable}{ p{2cm} | p{10cm} | p{2cm} }

	\hline
	No. Historia de Usuario & requerimiento/Tarea &  Responsable \\[0.5cm]
	\hline
	\hline

	\endfirsthead

	\multicolumn{3}{c}{Continuación de tabla de lista de producto }\\
	\hline
	\hline
	\endhead

	\hline
	\hline
	\caption{Lista de producto}
	\endlastfoot

	% template row table
	% \centering & & & \\[0.5cm]
	% \hline

	\centering 14 & Investigación Bases de datos relacionales. & Eduardo/Omar \\[0.5cm]
	\hline
	\centering 14 & Redacción y selección de las tecnologías a utilizar para el desarrollo de la plataforma.  & Eduardo \\[0.5cm]
	\hline
	\centering 14 & Investigación Bases de datos relacionales.  & Eduardo/Omar \\[0.5cm]
	\hline
	\centering 14 & Redacción bases de datos relacionales en el documento técnico.  & Eduardo \\[0.5cm]
	\hline
	\centering 14 & Investigación Bases de datos no relacionales.  & Eduardo/Omar \\[0.5cm]
	\hline
	\centering 14 & Redacción bases de datos no relacionales en el documento técnico.  & Eduardo \\[0.5cm]
	\hline
	\centering 14 & Investigación y selección del modelo de base de datos no relacional a utilizar junto a las	tecnologías a utilizar.  & Eduardo/Omar \\[0.5cm]
	\hline
	\centering 14 & Análisis y diseño de la aquitectura web.  & Eduardo/Omar \\[0.5cm]
	\hline
	\centering 1 & Desarrollo de la estructura básica del \textit{backend}.  & Omar \\[0.5cm]
	\hline
	\centering 1 & Desarrollo de la estructura	básica del \textit{frontend}.  & Eduardo \\[0.5cm]
	\hline
	\centering 1 & Agregar servicio \textit{backend} para registrar un usuario. & Omar \\[0.5cm]
	\hline
	\centering 1 & Agregar formulario para captura de datos de registro de un usuario en el \textit{frontend}. & Eduardo \\[0.5cm]
	\hline
	\centering 2 & Agregar servicio \textit{backend} para recuperar contraseña del usuario. & Omar \\[0.5cm]
	\hline
	\centering 2 & Agregar servicio \textit{backend} para envío de email al usuario registrado y de recuperación de contraseña. & Omar \\[0.5cm]
	\hline
	\centering 2 & Agregar vista con formulario para recuperación de contraseña del usuario en el \textit{frontend}. & Eduardo \\[0.5cm]
	\hline
	\centering 2 & Integración de los servicios de registro y recuaperación de contraseña en el \textit{frontend}. & Eduardo \\[0.5cm]
	\hline
	\centering 3 & Agregar servicio \textit{backend} para hacer ilegible la contraseña del usuario en la base de datos. & Omar \\[0.5cm]
	\hline
	\centering 4 & Planteamiento de escenarios de los esquemas entidad-relación.  & Eduardo \\[0.5cm]
	\hline
	\centering 4 & Agregar a la interfaz gráfica de la aplicación web el menú ``Entidad-Relación". & Eduardo \\[0.5cm]
	\hline
	\centering 4 & Agregar iconos de los elementos basicos de un diagrama ER en el diagramador. & Eduardo \\[0.5cm]
	\hline
	\centering 5 & Agregar servicio \textit{backend} para guardar un diagrama ER en formato JSON en la base de datos e integrarlo al \textit{frontend}. & Omar \\[0.5cm]
	\hline
	\centering 5 & Agregar servicio \textit{backend} para recuperar el diagrama guardado del usuario de la base de datos y regresarlo en formato JSON.  & Omar \\[0.5cm]
	\hline
	\centering 6 & Recuperar el último diagrama del usuario del \textit{backend} y monstrarlo en el \textit{frontend}. & Eduardo \\[0.5cm]
	\hline
	\centering 6 & Manejar el estado de la intefaz web para no perder el diagrama ER que está editando el usuario. & Eduardo \\[0.5cm]
	\hline
	\centering 6 & Definición de las reglas del modelo entidad-relación. & Eduardo \\[0.5cm]
	\hline
	\centering 4 & Implementar la edición de diagramas ER en el \textit{frontend}.  & Eduardo \\[0.5cm]
	\hline
	\centering 7 & Habilitar la carga de un archivo en la aplicación web.  & Omar \\[0.5cm]
	\hline
	\centering 7 & Agregar la función para validar el contenido del archivo JSON y pintarlo en la zona de diagramado. & Eduardo \\[0.5cm]
	\hline
	\centering 8 & Agregar la descarga del diagrama visible en la zona de diagramado a un archivo JSON. & Eduardo \\[0.5cm]
	\hline
	\centering 9 & Agregar botón de validar al \textit{frontend} y mostrar el \textit{loader} mientras se procesa el diagrama ER. & Eduardo/Omar \\[0.5cm]
	\hline
	\centering 9 & Agregar servicio \textit{backend} para la validación del diagrama entidad-relación. & Eduardo/Omar \\[0.5cm]
	\hline
	\centering 9 & implementación de algoritmo para validación del diagrama ER en el \textit{backend}. & Eduardo/Omar \\[0.5cm]
	\hline
	\centering 9 & Pruebas de captura de distintos diagramas entidad-relación.  & Eduardo/Omar \\[0.5cm]
	\hline
	\centering 9 & Pruebas para validar el algoritmo de validación. & Eduardo/Omar \\[0.5cm]
	\hline
	\centering 10 & Agregar servicio al \textit{backend} para transformación del esquema entidad-relación al modelo relacional.  & Omar \\[0.5cm]
	\hline
	\centering 10 & Implementación del algoritmo de transformación ER -> relacional & Eduardo/Omar \\[0.5cm]
	\hline
	\centering 10 & Agregar menú relacional a la intefaz gráfica. & Eduardo/Omar \\[0.5cm]
	\hline
	\centering 10 & Prueba de transformación de distintos diagramas E-R al modelo relacional. & Eduardo/Omar \\[0.5cm]
	\hline
	\centering 10 & Visualización de la transformación del modelo ER al modelo Relacional. & Eduardo \\[0.5cm]
	\hline
	\centering 14 & Revision de la redacción del reporte técnico para presentación de TT1  & Eduardo \\[0.5cm]
	\hline
	\centering 11 & Agregar servicio \textit{backend} para la descarga del archivo .sql con las sentencias equivalentes. & Eduardo/Omar \\[0.5cm]
	\hline
	\centering 11 & Pruebas de coherencia de las sentencias equivalentes de distintos en el SGBD. & Eduardo/Omar \\[0.5cm]
	\hline
	\centering 12 & Definición de las reglas de transformación al modelo NoSQL.  & Eduardo/Omar \\[0.5cm]
	\hline
	\centering 12 & Pruebas de distintos escenarios del modelo relacional al modelo NoSQL.  & Eduardo/Omar \\[0.5cm]
	\hline
	\centering 12 & Agregar servicio al \textit{backend} para transformación del esquema relacional al modelo NoSQL.  & Omar \\[0.5cm]
	\hline
	\centering 12 & Agregar servicio al \textit{backend} para guardar el modelo NoSQL en la base de datos.  & Omar \\[0.5cm]
	\hline
	\centering 12 & Agregar menú no relacional a la intefaz gráfica. & Omar \\[0.5cm]
	\hline
	\centering 12 & Implementación del algortimo de transformación de modelo relacional al modelo conceptual NoSQL. & Eduardo \\[0.5cm]
	\hline
	\centering 12 & Comprobación de la coherencia de la transformación entre modelos relacional a no relacional.  & Eduardo/Omar \\[0.5cm]
	\hline
	\centering 13 & Agregar servicio \textit{backend} para transformación del modelo ER al modelo no relacional. & Eduardo/Omar \\[0.5cm]
	\hline
	\centering 13 & Ajustar la interfaz del menú ER para mostrar mensaje tranformación la modelo NoSQL. & Omar \\[0.5cm]
	\hline
	\centering 13 & Manejar el estado del diagrama ER y redireccionar al menú no relacional al terminar la tranformación. & Eduardo \\[0.5cm]
	\hline
	\centering 13 & Pruebas de caso de estudio para verificar la correcta transformación y coherencia en los datos.  & Eduardo/Omar \\[0.5cm]
	\hline
	\centering 14 & Revisión de la redacción del reporte técnico para presentación de TT2 & Eduardo \\[0.5cm]
	\hline
\end{longtable}
