
\subsection*{Factibilidad operativa}
La herramienta va dirigida hacia cualquier estudiante que se encuentre trabajando con modelos de bases de datos desde un enfoque conceptual como lo requieren las primeras etapas del aprendizaje sobre el tema, buscando principalmente lograr un  impacto en los estudiantes que comienzan a tener una aproximación a los modelos no relacionales de bases de datos.

Se cuenta con una interfaz intuitiva para que el estudiante pueda visualizar la información de una manera ya conocida, como lo son los diagramas E-R y las opciones que ésta les brinde de manera comprensible, no requiriendo de un usuario o personal especializado para su utilización.
El usuario contará con registro ligado a su cuenta de correo electrónico que le permitirá acceder a la herramienta de forma única brindándole un almacenamiento de los proyectos que con esta puedan haberse realizado.

Dentro de la herramienta se encuentran tres funciones principales, en un desarrollo primario se cuenta con un diagramador entidad-relación el cual permite una visualización abstracta de la construcción de una base de datos que apoya en su correcta descripción gracias a mensajes sobre la conexión de elementos por medio de la validación del mismo.

El segundo bloque podrá recibir el diagrama entidad-relación anteriormente tratado, o bien, trabajar sobre uno cuyo desarrollo y carga se haya realizado con antelación. Posteriormente el diagrama podrá ser transformado al modelo relacional con la posibilidad de obtener el esquema de la base de datos en sentencias SQL. 
Finalmente si es requerido, se obtendrá el esquema de base de datos en un modelo de datos no relacional y tendrá la posibilidad de implementar el modelo no relacional a un SGBDNR (Sistema Gestor de Base de Datos No Relacional).

Basándonos en las necesidades y tecnologías mencionadas a utilizar para el desarrollo de la herramienta, dentro del personal se deberá contar con un perfil como se describe a continuación:

\textbf{Analista de sistemas:}

Se entiende por analista a la persona que trabaja con los requerimientos, en el diseño global y en el diseño detallado. 
Los principales atributos que deberían considerarse en un analista son:
\begin{itemize}	
	\item Liderazgo
	\item Dedicación
	\item Habilidad para el diseño, 
	\item Análisis de requerimientos
	\item Correcta comunicación 
	\item Cooperación entre sus pares. 
\end{itemize}

El salario promedio para un puesto de Analista de sistema en México es de \$11,708 al mes. Las estimaciones de salarios se basan en 564 salarios que empleados y usuarios que trabajan de Analista de sistema enviaron a Indeed de forma anónima, y en los salarios que recopilamos de los anuncios de empleo que se publicaron en Indeed en los últimos 36 meses. [https://www.indeed.com.mx/salaries/analista-de-sistema-Salaries]

\textbf{Desarrollador}

Las tendencias actuales siguen enfatizando la importancia de la capacidad de los analistas. Sin embargo, debido a que la productividad se ve afectada notablemente por la habilidad del programador en el uso de las herramientas actuales, existe una tendencia a darle mayor importancia a la capacidad del programador. 

\begin{itemize}	
	\item Front end.
	
	\noindent El desarrollador front-end es un especialista encargado de diseñar la interfaz de usuario de los sitios web, es decir, es el encargado de traducir las definiciones de diseño y estilo visual realizadas en etapas previas a código semántico. Diseña la estructura, la tipografía, la colorimetría, imágenes, banners, etc. 
	
	El salario promedio de Front End en México es de \$27.500-Mes Basado en 777 salarios [https://neuvoo.com.mx/salario/?job=front+end]
	
	\item Back end.
	
	\noindent El desarrollador back-end es el encargado de implementar la interacción entre el usuario y el sitio web. Se trata del responsable de la programación de un sitio web y todos sus componentes, coordinando páginas, formularios, funcionalidades, bases de datos y servidores web, evitando problemas en las capas más profundas del proyecto. 
	
	El salario promedio de Back End en México es de \$29.000-Mes Basado en 429 salarios [https://neuvoo.com.mx/salario/?job=back+end]
\end{itemize}
