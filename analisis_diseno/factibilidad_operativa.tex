
\subsection*{Factibilidad operativa}


La factibilidad operativa permite predecir de cierta forma si el sistema propuesto podrá ponerse en marcha, aprovechando todos los beneficios que puede ofrecer a todos los usuarios involucrados en ella. La herramienta va dirigida a los estudiante que se encuentre en un primer acercamiento a los modelos de bases de datos entidad-relación o relacional desde un enfoque conceptual, buscando principalmente mostrarles una aproximación a los modelos no relacionales de bases de datos. El sistema propuesto cuenta con una interfaz intuitiva para que el usuario final(estudiantes) pueda visualizar, crear y editar un diagrama ER y las opciones que ésta les brinde de manera comprensible.


Es por estos motivos que el sistema propuesto tiene una alta probabilidad de aceptación por parte de los usuarios finales al encontrarse en un entorno en el que se trabaja con software continuamente. Además del beneficio que aporta al plan de estudios actual al ofrecer una forma practica de ver aplicado los conceptos recibidos en el curso de base de datos,el cuál solo contempla un alcance hasta la normalización de bases de datos relacionales, y tener una introducción a los modelos no relacionales (noSQL). Al haber cursado la misma asignaruta por experiencia nos hemos dado cuenta que el tiempo disponible durante el curso es ajustado con la cantidad de módulos que pretender cubrir y en muchas ocaciones los docentes deben presindir de ciertos temas para completar el temario.

Con la implantación de la aplicación web que proponemos los estudiantes que cursen la asignatura de de base de datos tendrán la oportunidad de conocer una opción mas en cuanto a tecnologías de almacenamiento de datos para implementar en sus propios sistemas. De igual manera puede impulsarlos a generar propuestas para la apertura de una asignatura optativa si el interes por estos modelos de datos resulta interesante para ellos.

Teniendo en cuenta los puntos mencionados anteriormente podemos concluir que el sistema propuesto tendrá un uso en la institución y un potencial beneficio para los estudiantes y los involucrados en ella.

