\subsection*{Material Design}
De acuerdo a la documentación de Google\cite{noauthor_introduction_nodate}, Material Design es es un lenguaje visual que sintetiza los principios clásicos del buen diseño respecto a las ideas de Google.

Material design es una normativa de diseño enfocado en la visualización del sistema operativo Android, además en la web y en cualquier plataforma. Fue desarrollado por Google y anunciado en la conferencia Google I/O celebrada el 25 de junio de 2014. 


Sus objetivos son crear un lenguaje visual que sintetice los principios clásicos del buen diseño, unificar el desarrollo de un único sistema subyacente que unifique la experiencia del usuario en plataformas y dispositivos, así como personalizar el lenguaje visual de Material Design.


Material Design está inspirado en el mundo físico y sus texturas, incluida la forma en que reflejan la luz y proyectan sombras. El objetivo de Material es hacer imaginar los medios de papel y tinta.


El diseño de materiales se guía por métodos de diseño de impresión (tipografía, cuadrículas, espacio, escala, color e imágenes) para crear jerarquía, significado y enfoque que sumergen a los espectadores en la experiencia.


Material Design mantiene la misma interfaz de usuario en todas las plataformas, utilizando componentes compartidos en Android, iOS, Flutter y la web.
