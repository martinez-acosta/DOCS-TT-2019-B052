\subsection{Costos de desarrollo}

De acuerdo con Software Guru\cite{pedro_galvan_estudio_2020}, en una publicación que recopila los datos de salarios en el área del desarrollo de software para febrero de 2020, un desarrollador con 0 a 2 años de experiencia como es el caso de un estudiante tiene en un sueldo de $15,000$ mensuales para una jornada completa, considerando que este proyecto contempla jornadas de medio tiempo (4 horas) de lunes a viernes la cifra antes mencionada se reduce. Teniendo en cuenta estos datos y un periodo de 9 meses que es el tiempo aproximado de duración del proyecto tenemos como costo total por sueldos para el equipo de desarrollo desglosado de la tabla \ref{tab:devs_salary}.

\begin{table}[h!]
    \centering
    \begin{tabular}{|c|c|c|c|}
    \hline
        Concepto & Costo Aprox. semanal & Costo Aprox. Mensual & Monto total \\ \hline
        Sueldo & 1850 & 7500 & 67500 \\ \hline
    \end{tabular}
    \caption{Costos del personal}
    \label{tab:devs_salary}
\end{table}


Esto da como resultado un total final de \$ 135000 por los sueldos de los 2 integrantes del equipo de desarrollo. Se consideraron 9 meses para todos los gastos, un mes extra a la obtenida en estimación para utilizarse en el caso de estudio del sistema una vez concluido.

Los gastos por pagos de licencia de software quedan excluidos ya que las tecnologías seleccionadas son libres o gratuitas lo cuál no supone un costo para su uso, de igual manera en la tabla \ref{tab:sw_licences} se ejemplifican.

\begin{table}
    \centering
    \begin{tabular}{|c|c|c|c|}
    \hline
        Software & Licencia & Costo \\ \hline
        Visual Studio Code  & MIT & 0  \\ \hline
        Gunicorn(flask Server) & MIT & 0 \\ \hline
        MongoDB Atlas & Apache v2 & 0 \\ \hline
    \end{tabular}
    \caption{Costos por licencias de software}
    \label{tab:sw_licences}
\end{table}

Otros gastos que necesarios son los pagos por servicios requeridos listados en la tabla \ref{tab:services_costs}

\begin{table}
    \centering
    \begin{tabular}{|c|c|c|c|}
    \hline
        Concepto & Costo Mensual & Monto total \\ \hline
        Luz & 250 & 2250  \\ \hline
        Internet & 349 & 3141 \\ \hline
        Heroku hosting & 0 (free plan) & 0 \\ \hline
        Netlify hosting & 0 (free plan) & 0 \\ \hline
    \end{tabular}
    \caption{Costos por servicios}
    \label{tab:services_costs}
\end{table}

Realizando la suma de todas las cantidades antes mencionadas tenemos un total final de :

\begin{center}

	sueldos = 135000.00
	servicios = (2250 + 3141) * 9 = 48519.00

	Gastos totales = 135000.00 + 48519.00 = 183519.00 pesos mexicanos

\end{center}

