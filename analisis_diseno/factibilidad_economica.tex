
\subsection*{Factibilidad economica}
Dentro de la factibilidad económica se encuentran los costos necesarios que llevará consigo el desarrollo de la aplicación.


Costo de hardware
\begin{longtable}{  p{2.5cm} | p{6cm} }	
	
	\hline
	
	\textbf{Costo } & \textbf{Equipo }  \\
	\hline
	\hline
	
	\endfirsthead
	
	\multicolumn{2}{c}{Continuación de Tabla \ref{long}}\\
	\hline
	Continuación de tabla\\
	\hline
	\endhead
	
	\hline
	\endfoot
		
	\$10,000.00 & Toshiba Satellite Radius P55W-B, Windows 10, Intel Core i7-10U, 8GB RAM\\
	\$13,999.00 & Dell Laptop Latitude 3400, Ubuntu, Intel Core I5 8265, 8GB RAM.\\
	\$16,499.00 & HP Pavilion15", Windows 10, Core i5-9300H, 8GB RAM.\\
	\$1,599.00 & DeskJet Ink Advantage 3775\\
	
\end{longtable}


Costo de software
\begin{longtable}{ l | l }	
	
	\hline
	
	\textbf{Nombre } & \textbf{Costo } \\
	\hline
	\hline
	
	\endfirsthead
	
	\multicolumn{2}{c}{Continuación de Tabla \ref{long}}\\
	\hline
	Continuación de tabla\\
	\hline
	\endhead
	
	\hline
	\endfoot
	
	 Windows 10 & \$3,599.00\\
	 Ubuntu & Gratuito\\
	 MongoDB Atlas & \\
	 Visual Studio Code & Gratuito\\
	 Flask & Software libre\\
	 Vue.js & Software libre\\
	 Nuxt & Software libre\\
	
\end{longtable}


Es necesario detallar que el costo de la herramienta no será determinado puramente con los costos de software y hardware descritos dentro de esta sección, ya que son requeridos valores tales como el tiempo de desarrollo y el esfuerzo requeridos que serán especificados gracias al Modelo Constructivo de Costos (COCOMO II)


