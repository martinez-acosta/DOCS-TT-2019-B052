\subsection{Costos de desarrollo}

De acuerdo con \textit{Software Guru}\cite{pedro_galvan_estudio_2020}, en una publicación que recopila los datos de salarios en el área de desarrollo de \textit{software} para febrero de 2020, un desarrollador con 0 a 2 años de experiencia, como es el caso de un estudiante, tiene en un salario de \$ 15 000 mensuales en una jornada completa, considerando que este proyecto contempla jornadas de medio tiempo (4 horas) de lunes a viernes, se reduce la cifra antes mencionada. Teniendo en cuenta estos datos y un periodo de 9 meses, que es el tiempo aproximado de duración del proyecto, el costo total por salarioss para el equipo de desarrollo está desglosado en la tabla \ref{tab:devs_salary}.

\begin{table}[h!]
    \centering
    \begin{tabular}{|c|c|c|c|}
    \hline
        Concepto & Costo Aprox. Semanal & Costo Aprox. Mensual & Monto total \\ \hline
        Salario & 1850 & 7500 & 67500 \\ \hline
    \end{tabular}
    \caption{Costos del personal}
    \label{tab:devs_salary}
\end{table}


Esto da como resultado un total final de \$ 135 000 por los salarios de los 2 integrantes del equipo de desarrollo. Se tomaron en cuenta 9 meses para todos los gastos, un mes extra a lo obtenido en estimación para utilizarse en el caso de estudio del sistema una vez concluido.

Los gastos por pagos de licencia de \textit{software} quedan excluidos, ya que las tecnologías seleccionadas son libres o gratuitas, lo cual no supone un costo para su uso. De igual manera, esto se encuentra simplificado en la tabla \ref{tab:sw_licences}.

\begin{table}
    \centering
    \begin{tabular}{|c|c|c|c|}
    \hline
        Software & Licencia & Costo \\ \hline
        Visual Studio Code  & MIT & 0  \\ \hline
        Gunicorn(flask Server) & MIT & 0 \\ \hline
        MongoDB Atlas & Apache v2 & 0 \\ \hline
    \end{tabular}
    \caption{Costos por licencias de software}
    \label{tab:sw_licences}
\end{table}

Otros gastos necesarios son los pagos por servicios requeridos listados en la tabla \ref{tab:services_costs}.

\begin{table}
    \centering
    \begin{tabular}{|c|c|c|c|}
    \hline
        Concepto & Costo Mensual & Monto total \\ \hline
        Luz & 250 & 2250  \\ \hline
        Internet & 349 & 3141 \\ \hline
        Heroku hosting & 0 (free plan) & 0 \\ \hline
        Netlify hosting & 0 (free plan) & 0 \\ \hline
    \end{tabular}
    \caption{Costos por servicios}
    \label{tab:services_costs}
\end{table}

Habiendo realizado la suma de todas las cantidades antes mencionadas, el total final es:

\begin{center}

	salarios = 135 000.00
	servicios = (2250 + 3141) * 9 = 48 519.00

	Gastos totales = 135 000.00 + 48 519.00 = 183 519.00 pesos mexicanos.

\end{center}

