\subsection*{Hypertext Transfer Protocol}
De acuerdo a la W3C\cite{noauthor_http_nodate}, Hypertext Transfer Protocol, HTTP, o protocolo de transferencia de hipertexto, es el protocolo de comunicación que permite las transferencias de información en la World Wide Web.


HTTP fue desarrollado por el World Wide Web Consortium y la Internet Engineering Task Force, colaboración que culminó en 1999 con la publicación de una serie de RFC, siendo el más importante de ellos el RFC 2616 que especifica la versión 1.1. HTTP define la sintaxis y la semántica que utilizan los elementos de software de la arquitectura web (clientes, servidores, proxies) para comunicarse.


HTTP es un protocolo sin estado, es decir, no guarda ninguna información sobre conexiones anteriores. El desarrollo de aplicaciones web necesita frecuentemente mantener estado. Para esto se usan las cookies, que es información que un servidor puede almacenar en el sistema cliente.


Es un protocolo orientado a transacciones y sigue el esquema petición-respuesta entre un cliente y un servidor. Al cliente se le suele llamar ``agente de usuario" o \textit{user agent}, que realiza una petición enviando un mensaje con cierto formato al servidor. Al servidor se le suele llamar servidor web y envía un mensaje de respuesta. 


Los mensajes HTTP son en texto plano lo que lo hace más legible y fácil de depurar. Esto tiene el inconveniente de hacer los mensajes más largos.

Los mensajes tienen la siguiente estructura:

\begin{enumerate}
    \item Línea inicial (termina con retorno de carro y un salto de línea).
    \begin{enumerate}
        \item Para las peticiones: la acción requerida por el servidor (método de petición) seguido de la URL del recurso y la versión HTTP que soporta el cliente.
        \item Para respuestas: La versión del HTTP usado seguido del código de respuesta (que indica que ha pasado con la petición seguido de la URL del recurso) y de la frase asociada a dicho retorno.        
    \end{enumerate}
    \item Las cabeceras del mensaje que terminan con una línea en blanco. Son metadatos. Estas cabeceras le dan gran flexibilidad al protocolo.
    \item Cuerpo del mensaje: es opcional. Su presencia depende de la línea anterior del mensaje y del tipo de recurso al que hace referencia la URL. Típicamente tiene los datos que se intercambian cliente y servidor. Por ejemplo para una petición podría contener ciertos datos que se quieren enviar al servidor para que los procese. Para una respuesta podría incluir los datos que el cliente ha solicitado.
\end{enumerate}

HTTP define una serie predefinida de métodos de petición que pueden utilizarse. El protocolo tiene flexibilidad para ir añadiendo nuevos métodos y para así añadir nuevas funcionalidades. El número de métodos de petición se ha ido aumentando según se avanzaba en las versiones.


 Cada método indica la acción que desea que se efectúe sobre el recurso identificado. Lo que este recurso representa depende de la aplicación del servidor. Por ejemplo, el recurso puede corresponderse con un archivo que reside en el servidor.

\begin{enumerate}
    \item GET: el método GET solicita una representación del recurso especificado. Las solicitudes que usan GET solo deben recuperar datos y no deben tener ningún otro efecto. 
    \item HEAD: pide una respuesta idéntica a la que correspondería a una petición GET, pero en la respuesta no se devuelve el cuerpo. Esto es útil para poder recuperar los metadatos de los encabezados de respuesta, sin tener que transportar todo el contenido.
    \item POST: envía los datos para que sean procesados por el recurso identificado. Los datos se incluirán en el cuerpo de la petición. Esto puede resultar en la creación de un nuevo recurso o de las actualizaciones de los recursos existentes o ambas cosas.
    \item PUT:  sube, carga o realiza un upload de un recurso especificado (archivo o fichero) y es un camino más eficiente ya que POST utiliza un mensaje multiparte y el mensaje es decodificado por el servidor. En contraste, el método PUT permite escribir un archivo en una conexión socket establecida con el servidor. La desventaja del método PUT es que los servidores de alojamiento compartido no lo tienen habilitado.
\end{enumerate}