
\section{Factibilidad técnica}
Para la factibilidad técnica se realiza una evaluación de las herramientas de hardware  y software que el equipo de trabajo tiene disponible.


Recursos técnicos de software
\begin{longtable}{ p{3cm}| l | l | l | l }	
	
	\hline
	
	\textbf{Equipo } & \textbf{Tipo } & \textbf{Sistema operativo } & \textbf{Procesador } & \textbf{RAM } \\
	\hline
	\hline
	
	\endfirsthead
	
	\multicolumn{5}{c}{Continuación de Tabla \ref{long}}\\
	\hline
	Continuación de tabla\\
	\hline
	\endhead
	
	\hline
	\endfoot
	
	Toshiba Satellite Radius P55W-B & Laptop & Windows 10 & Intel Core i7-10U & 8GB\\
	Dell Laptop Latitude 3400 & Laptop & Windows 10 & Intel Core I5 8265 &  8GB\\
	HP Pavilion15" & Laptop & Windows 10 & Core i5-9300H &  8GB\\
	DeskJet Ink Advantage 3775  & Impresora Multifuncional & - & - & -\\
	
\end{longtable}


Recursos técnicos de hardware
\begin{longtable}{ l | l }	
	
	\hline
	
	\textbf{Nombre } & \textbf{Tipo } \\
	\hline
	\hline
	
	\endfirsthead
	
	\multicolumn{2}{c}{Continuación de Tabla \ref{long}}\\
	\hline
	Continuación de tabla\\
	\hline
	\endhead
	
	\hline
	\endfoot
	
	Windows 10 & Sistema operativo\\
	Ubuntu & Sistema operativo\\
	MongoDB Atlas & Base de datos en la nube\\
	Visual Studio Code & IDE\\
	Flask & Framework\\
	Vue.js & Framework\\
	Nuxt & Framework\\
	
\end{longtable}

Con todo esto, se puede concluir que se cuenta con todo el software y hardware necesario para la realización del proyecto.


