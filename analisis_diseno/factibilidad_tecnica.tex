\subsection{Factibilidad técnica}

Mediante esta fase del estudio se determinará si el equipo de desarrollo cuenta con los recursos técnicos necesarios para la realización del sistema propuesto. Esto se realiza considerando la disponibilidad de los recursos tanto de \textit{hardware}, \textit{software} y recurso humano.

\subsubsection{Sistema operativo}

Este es un elemento importante, ya que debe cumplir con las características de estabilidad, velocidad, seguridad y escalabilidad para soportar la instalación del sistema.


La tabla \ref{tab:options_so} presenta diferentes alternativas de sistemas operativos que cumplen con las características mencionadas y que son suficientes para albergar el sistema:

\begin{table}[h!]
    \centering
    \begin{tabular}{|l|l|l|}
    \hline
    Nombre                & Licencia                       & Costo                         \\ \hline
    Microsoft Server 2019 & \multicolumn{1}{c|}{Privativa} & \multicolumn{1}{c|}{972 USD}  \\ \hline
    Red Hat Enterprise 8  & \multicolumn{1}{c|}{GPL}       & \multicolumn{1}{c|}{349 USD}  \\ \hline
    \rowcolor[HTML]{66BB6A}
    Ubuntu Server 19.04   & \multicolumn{1}{c|}{\cellcolor[HTML]{66BB6A}GPL}       & \multicolumn{1}{c|}{\cellcolor[HTML]{66BB6A}gratuito} \\ \hline
    \end{tabular}
    \caption{Comparativa de sistemas operativos}
    \label{tab:options_so}
    \end{table}

La opción que se utilizará es ubuntu server por el precio del producto además de ser un sistema conocido para el equipo y contar con una comunidad amplia para temas de soporte.

\subsubsection{Lenguaje de desarrollo}

El lenguaje de desarrollo debe de cumplir con las siguientes características:

\begin{itemize}
    \item Soporte para conexión a base de datos.
    \item Facilidad para el desarrollo.
    \item En continua mejora.
    \item Fácil de administrar.
    \item Contar con algún \textit{framework} web.
\end{itemize}

La tabla \ref{tab:options_lenguage} presenta una Comparativa de lenguajes de desarrollo que cumplen dichas características:

\begin{table}[h!]
    \centering
    \begin{tabular}{|l|l|c|c|}
    \hline
    Nombre & Sistema Operativo & \multicolumn{1}{l|}{Licencia}                       & \multicolumn{1}{l|}{Costo}                            \\ \hline
    Java   & Multiplataforma   & GPL v2                                              & Gratuito                                              \\ \hline
    Ruby   & Multiplataforma   & GPL v2                                              & Gratuito                                              \\ \hline
    C\#    & Microsoft         & GPL                                                 & Gratuito                                              \\ \hline
    \rowcolor[HTML]{66BB6A}
    Python & Multiplataforma   & \multicolumn{1}{l|}{\cellcolor[HTML]{66BB6A}GPL v2} & \multicolumn{1}{l|}{\cellcolor[HTML]{66BB6A}Gratuito} \\ \hline
    \end{tabular}
    \caption{Comparativa de lenguajes de programación}
    \label{tab:options_lenguage}
    \end{table}


Además de tener una curva de aprendizaje rápida, se utilizará python por ser un lenguage conocido por el equipo y ofrecer una gran cantidad de bibliotecas para agregar funcionalidades de una forma práctica ahorrando tiempo de desarrollo.

\subsubsection{Sistema gestor de base de datos}

Este es un factor muy importante, ya que determinará cómo se almacenará la información del sistema, por lo tanto debe cumplir con las siguientes características:

\begin{itemize}
    \item Escalable.
    \item Seguro.
    \item Contar con soporte para grandes cantidades de información.
    \item Contar con soporte para conexión con distintos lenguajes de programación.
\end{itemize}


La lista \ref{tab:tab:options_database} presenta una lista de sistemas gestores de bases de datos que cumplen dichas características:

\begin{table}[h!]
    \centering
    \begin{tabular}{|l|l|c|c|}
    \hline
    Nombre           & Entorno         & \multicolumn{1}{l|}{Licencia} & \multicolumn{1}{l|}{Costo} \\ \hline
    MySQL            & Multiplataforma & GPL                           & Gratuito                   \\ \hline
    MariaDB          & Multiplataforma & GPL v2                        & Gratuito                   \\ \hline
    Oracle Database  & Multiplataforma & OTN                           & 350 USD                    \\ \hline
    \rowcolor[HTML]{66BB6A}
    MongoDB Atlas     & Nube            & AGPL v3                       & Gratuito                   \\ \hline
    DynamoDB         & Nube            & N/D                           & 5.33 USD/mes               \\ \hline
    Apache Cassandra & Nube            & Apache License v2             & 0.33 UDS/GB                \\ \hline
    \end{tabular}
    \caption{Comparativa de sistemas gestores de bases de datos}
    \label{tab:options_database}
    \end{table}


La base de datos MongoDB atlas ofrece una opción gratuita que cumple con los requerimientos para el desarrollo del proyecto, ademas de la ventaja de estar alojada en la nube para tener una alta disponibilidad de los datos independiente del sistema operativo y entorno de despliegue de la aplicación web.


Las características de los equipos de cómputo con los que se dispone actualmente para el desarrollo del sistema se muestran en la tabla \ref{tab:hw_devices}.

\begin{table}[h!]
    \begin{tabular}{|c|c|c|c|}
        \hline
        Equipo & Elemento & Capacidad & Costo\\ \hline
        \multirow{3}{*}{Laptop 1} & Memoria RAM & 8 GB & \\
        & Almacenamiento & 500 GB HDD & \$ 22,500.00\\
        & Procesador & Intel Core i5 6ta gen. &\\ \hline
        \multirow{3}{*}{Laptop 2} & Memoria RAM & 8 GB & \\
        & Almacenamiento & 256 GB SSD & \$ 31,999.00\\
        & Procesador & Intel Core i5 8va gen. &\\ \hline
    \end{tabular}
    \caption{Equipo de cómputo}
    \label{tab:hw_devices}
\end{table}


Con los datos anteriormente mencionados, se concluye que la tecnología para el desarrollo del sistema existe y se cuenta con los recursos de \textit{hardware} suficientes para iniciar con su implementación.