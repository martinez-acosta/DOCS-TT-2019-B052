
\section{Metodología}
De acuerdo con la Universidad Católica los Ángeles\cite{universidad_catolica_los_angeles_metodologidesarrollo_2020}, en el campo del desarrollo de \textit{software} hay dos grupos de metodologías: las tradicionales y las ágiles.


Las tradicionales se centran en cumplir con un plan rígido de trabajo establecido en la etapa inicial del proyecto, mientras que las ágiles permiten realizar cambios en los requerimientos conforme avance el mismo.


Dado que cualquier cambio en el proceso de una metodologia tradicional genera la necesidad de una reconstrucción del plan de trabajo (invirtiendo tiempo que se podría usar para desarrollar), surgieron las metodologías ágiles, que permiten realizar cambios en los requerimientos conforme avance el proyecto. 


Tomando en cuenta la experiencia del equipo, esta forma de trabajo permite mostrar avances funcionales en el producto en un periodo de tiempo corto para realizar una evaluación y en caso de ser requerido se sugieran cambios.


Se han propuesto muchos modelos ágiles de proceso y están en uso en toda la industria; entre ellos se encuentran los siguientes:


\begin{itemize}
	\item DAS (Desarrollo Adaptativo de Software): ASD tiene como fundamento la teoría de sistemas adaptativos complejos; por ello, interpreta los proyectos de \textit{software} como sistemas adaptativos complejos compuestos
    por agentes (los interesados), entornos (organizacional o tecnológico) y salidas (el producto desarrollado)\cite{cadavid_revision_2013}.
	\item Scrum: La metodología Scrum para el \textit{desarrollo} ágil de software es un marco de trabajo diseñado para lograr la colaboración eficaz de equipos en proyectos, que emplea un conjunto de reglas y artefactos y define roles que generan la estructura necesaria para su correcto funcionamiento\cite{cadavid_revision_2013}.
	\item MDSD (Método de Desarrollo de Sistemas Dinámicos): DSDM es un marco de trabajo creado para entregar la solución correcta en el momento correcto; utiliza un ciclo de vida iterativo, fragmenta el proyecto en periodos cortos de tiempo y define entregables para cada uno de estos periodos; tiene roles claramente definidos y especifica su trabajo dentro de periodos de tiempo\cite{cadavid_revision_2013}.
	\item Crystal: la filosofía de Crystal define el desarrollo como un juego cooperativo de invención y comunicación cuya meta principal es entregar \textit{software} útil, que funcione y su objetivo secundario es preparar el próximo juego\cite{cadavid_revision_2013}.
\end{itemize}

\subsection{Scrum}

De acuerdo con Ken Schwaber\cite{the_scrum_guide_definitive_2020}, Scrum es un marco de trabajo para la entrega de productos incrementales y de máximo valor productivo.

Un artefacto es un elemento que garantiza la transparencia, es el registro de la información fundamental del proceso Scrum y a continuación se describen sus cuatro artefactos principales:

\begin{itemize}
	\item Lista de producto (\textit{product backlog}):	es el listado de todas las tareas que necesita el proyecto para alcanzar su realización; al iniciar el desarrollo del proyecto, esta lista no se encuentra completa y conforme avanzan los \textit{sprints} se le añaden tareas para solventar las necesidades que van surgiendo gracias a la retroalimentación del cliente.
	\item Lista de pendientes del \textit{sprint} (\textit{sprint backlog}): es la lista de tareas seleccionadas del \textit{product backlog} que se planifica realizar durante el periodo del \textit{sprint} y define a los responsables de cada tarea.
	\item \textit{Sprint}: es el corazón de Scrum, tiene un periodo de tiempo determinado de un mes o incluso menos donde el equipo completa conjuntos de tareas incluidas en el \textit{backlog} para crear un incremento del producto utilizable.
	\item Incremento: es la suma de todos los elementos de la lista de productos completados durante un \textit{sprint} unido con los incrementos de los \textit{sprints} anteriores; al finalizar el \textit{sprint}, el nuevo incremento debe estar en condiciones de ser utilizado.
\end{itemize}

El equipo Scrum (\textit{scrum team}) consiste en los siguientes roles:
\begin{itemize}
	\item El dueño de producto (\textit{product owner}): es la persona responsable de maximizar el valor del producto y el trabajo del equipo de desarrollo; es el único responsable de gestionar la lista de producto y cualquier cambio a esa lista debe ser revisada y aprobada por él.
	\item El equipo de desarrollo (\textit{development team}): son los profesionales que realizan el trabajo para la entrega de un incremento en el producto en cada \textit{sprint}; es un grupo autoorganizado y multifuncional donde cada miembro del equipo tiene habilidades especializadas, pero que la responsabilidad de las tareas completadas, incrementos del producto o retrasos recaen en el equipo como un todo.
	\item  El Scrum master: es la persona responsable de asegurar que Scrum es entendido y adoptado por todos los involucrados en el proyecto, asegurándose de ayudar a las personas externas al equipo a entender qué interacciones son útiles con el equipo de desarrollo.
\end{itemize}


Scrum tiene cuatro eventos principales en  un \textit{sprint}, que sirven para la inspección y adaptación del producto que se describen a continuación:
\begin{itemize}
	\item Planificación del \textit{sprint} (\textit{sprint planning}): es una reunión con el equipo de desarrollo que tiene una duración máxima de 8 horas para el \textit{sprint} de un mes; el Scrum Master es el encargado de que los asistentes entiendan el propósito de dicha reunión.
	\item Scrum diario (\textit{daily scrum}): es una reunión de un máximo de 15 minutos en la cual el equipo expone sus actividades y planifica las tareas de las próximas 24 horas.
	\item Revisión del \textit{sprint} (\textit{sprint review}): al finalizar cada \textit{sprint} se lleva a cabo una reunión para la revisión del incremento del producto y en caso de ser necesario realizar ajustes a la lista de producto.
	\item Retrospectiva del \textit{sprint} (\textit{sprint retrospective}): es cuando el equipo de desarrollo tiene la oportunidad de pensar en mejoras para el próximo \textit{sprint}.
\end{itemize}

por qué Scrum?

cómo vamos a aplicar Scrum?

Como la metodología permite definir un periodo de hasta un mes para cada \textit{sprint}, se ha optado por un periodo de 30 días, contemplándose un total de ocho \textit{sprints}, donde al término de cada uno se tendrá un avance del sistema.