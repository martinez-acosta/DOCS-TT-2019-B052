De acuerdo con Pressman~\cite{pressman_software_2005}, el objetivo del diseño del \textit{software} es aplicar un conjunto de principios, conceptos y prácticas que llevan al desarrollo de un sistema o producto de alta calidad; la meta del diseño es crear un modelo de \textit{software} que implantará correctamente todos los requerimientos del usuario y causará placer a quienes lo utilicen; los diseñadores del \textit{software} deben elegir entre muchas alternativas de diseño y llegar a la solución que mejor se adapte a las necesidades de
los participantes en el proyecto.


En este capítulo se muestra el diseño del sistema; se ve el funcionamiento de los módulos del sistema, así como el diagrama de clases y el esquema de la base de datos; además, se muestra como el usuario interactúa con la interfaz gráfica.

\section{Diagrama de clases}

De acuerdo con Visual Paradigm~\cite{visual_paradigm_visual_2020}, el diagrama de clases es un modelo conceptual que permite visualizar la estructura de las clases de un sistema, sus atributos, métodos y las relaciones entre objetos.


El diagrama de clases de la propuesta de solución se muestra en la figura~\ref{img:classes_diagram}; en la cual se aprecia la clase dominante \textit{usuario} y las demás clases dependen de esta.


\begin{figure}[H]
  \centering
  \includegraphics[width=0.75\textwidth]{diseño/diagrama_de_clases.png}
  \caption{Diagrama de clases}
  \label{img:classes_diagram}
\end{figure}

Es posible interpretar el significado del diagrama de clases de la figura~\ref{img:classes_diagram} leyendo los puntos de la siguiente manera:

\begin{itemize}
  \item El usuario es la clase principal.
  \item Existe una agregación entre Usuario y DiagramaER.
  \item El ModeloGDM es parte del ModeloLogicoNoSQL; el ModeloGDM puede existir por sí mismo.
  \item El GDM depende del DiagramaER; sin embargo, el DiagramaER no depende del GDM.
  \item La ConsultaDeAccessoSimple depende del Diagrama ER.
  \item El ModeloLogicoRelacional depende del DiagramaRelacional.
\end{itemize}


\section{Diagrama de la base de datos}
Como MongoDB es una base de datos NoSQL orientada a documentos, los datos se almacenan en documentos JSON; teniendo en cuenta el diagrama de clases de la figura~\ref{img:classes_diagram}, se aprecia que las relaciones son por composición de 1 a 1 y la clase dominante es Usuario, es decir, las clases DiagramaER, DiagramaRelacional y DiagramaNoSQL no pueden existir si no existe una clase Usuario asociada a ellas.


Por esto la mejor manera de almacenar los datos es en documentos anidados, donde el documento principal es el usuario y este contendrá los documentos de las otras clases; en la figura \ref{img:database_schema} se representa la relación por composición de las clases de la figura~\ref{img:classes_diagram}.


\begin{figure}[H]
    \centering
    \includegraphics[width=0.75\textwidth]{diseño/diagrama_DB.png}
    \caption{Diagrama de la base de datos.}
    \label{img:database_schema}
\end{figure}


En la figura~\ref{img:database_schema} se pone en práctica el comportamiento del diagrama de clases de la figura~\ref{img:classes_diagram}, porque al eliminar un usuario internamente se eliminan los diagramas asociados a este; es por este motivo que el usuario debe estar 100 \% seguro al momento de eliminar su cuenta, ya que será imposible recuperar los diagramas que haya realizado un vez completada la acción.
\section{Arquitectura del sistema}

De acuerdo con Pressman\cite{pressman_software_2005} en su sección Diseño de la arquitectura , el diseño arquitectónico de un software es la representación de los componentes o módulos del sistema y cómo estos interactúan entre sí. En esta sección se muestra que diseño arquitectónico utilizará el sistema como parte de su implementación.

\subsection{Arquitectura cliente-servidor}

La arquitectura cliente servidor es un modelo de diseño del software compuesta por 2 partes, el clientes, es quien solicita información por medio de una petición y el servidor quien se encarga de proveer los datos que le son requeridos. Las aplicaciones web utilizan frecuentemente este modelo teniendo como clientes a un navegador web y el servidor es quien provee la información que solicitante, pudiendo atender a más de un cliente al mismo tiempo.

La figura \ref{img:arq_client_server} describe la arquitectura cliente servidor del sistema.

\begin{figure}[H]
    \centering
    \includegraphics[width=0.75\textwidth]{diseño/cliente_servidor.png}
    \caption{Arquitectura cliente-servidor.}
    \label{img:arq_client_server}
\end{figure}


En la imagen anterior se observa que para poder hacer uso del sistema el usuario debe contar con un dispositivo capaz de ejecutar una navegador web, como una computadora o teléfono inteligente( smartphone). El servidor será una computadora que se encuentra en un instancia de la nube la cual contiene los requisitos necesarios para el funcionamiento del sistema como son: el sistema operativo GNU/Linux, el servidor web gunicorn, las aplicaciones flask y nuxt, conexión a internet y una configuración de red   que permita la comunicación con la base de datos MongoDB.


\subsection{Patrón de diseñó MVC}

De acuerdo con Dragos-Paul Pop \cite{dragos_2014} en su trabajo \textit{`Designing an MVC Model for Rapid Web Application Development'}, el patrón MVC fue concevido por primera vez en la década de los 70 en la empresa Xerox Parc. y su principal propósito era  cerrar la brecha entre el modelo mental del usuario y el modelo digital de las aplicaciones en una computadora.


Más tarde el paradigma MVC fue descrito por  Krasner y Pope\cite{pope_1988} donde destacan que se pueden obtener enormes beneficios si se piensa en la modularidad de las aplicaciones. El patrón se divide en 3 partes principalmente: el modelo, la vista y el controlador los cuales se describen adelante.

\begin{itemize}
    \item El modelo: es el dominio principal de la aplicación, es quien se encarga del manejo de los datos  y toda la lógica del negocio, es decir, no tiene que tener conocimiento sobre la interfaz gráfica ya sea una aplicación web, de escritorio o móvil. Idealmente el modelo debe ser independiente de la plataforma en que se ejecuta.
    \item La vista : contiene todos los elementos visibles al usuario y en contra parte al modelo, esta no debe tener conocimiento de la lógica del negocio de la aplicación. Sus responsabilidades deben limitarse a definir la estructura y apariencia de los datos presentados en la interfaz gráfica.
    \item El controlador : es el intermediario entre la vista y el modelo, es quien gestiona el flujo de la aplicación por medio de la comunicación entre los otros 2 componentes.
\end{itemize}

La figura \ref{img:mvc_implementation} describe la implementación del patrón MVC en la propuesta de solución junto a las tecnologías utilizadas en cada componente del paradigma. Como se aprecia la vista queda a cargo del framework Nuxt quien es el encargado del renderizado de la interfaz gráfica para el usuario mostrando elementos como el formulario de login y el diagramador entidad-relación.

\begin{figure}[H]
  \centering
  \includegraphics[width=0.75\textwidth]{diseño/patron_mvc.png}
  \caption{Esquema MVC.}
  \label{img:mvc_implementation}
\end{figure}


De igual manera el framework flask será el encargado del rol de controlador, es decir, quién llevará a cabo al comunicación de los datos que el usuario proporcione en la vista para transformarlos en algo que el modelo pueda entender como el guardado del objeto json del diagrama entidad-relación para ser almacenado en la base de datos. Finalmente el modelo queda a cargo del lenguaje python para el manejo de los datos y el encargado de mantener la comunicación con la base de datos, pudiéndose apoyar en bibliotecas de terceros como el propio framework flask para transformar los datos de la lógica de negocio y convertirlos en algo que la vista pueda mostrar al usuario.

\section{Interfaz de usuario}

En esta sección se muestran las imágenes del prototipo funcional de la propuesta de solución junto a una breve descripción de cada una de las pantallas.

\subsection*{Pantalla de inicio}

La figura~\ref{img:app_home} muestra la pantalla inicial que el usuario observa apenas entra a la \href{https://serene-haibt-2239b4.netlify.app/}{dirección}; donde tiene la opción de registrarse o iniciar sesión en caso de no haberlo hecho anteriormente.

\begin{figure}[H]
    \centering
    \includegraphics[width=0.75\textwidth]{interfaz/home.png}
    \caption{Pantalla inicial.}
    \label{img:app_home}
\end{figure}

\subsection*{Pantalla de registro}

La figura \ref{img:app_register} cuenta con un formulario para el registro de un usuario nuevo, se deben llenar todos los datos cumpliendo una serie de reglas entre ellas:

\begin{itemize}
    \item La contraseña debe tener al menos 6 caracteres.
    \item Se solicita que repitan la contraseña en otro campo y deben coincidir.
    \item Se debe colocar un \textit{email} válido.
\end{itemize}

Una vez llenado los datos solicitados y al dar clic en el botón \textit{submit}, el usuario recibirá un correo de bienvenida en la dirección de \textit{submit} que proporcionó; al finalizar el proceso de registro del usuario en la base de datos, será redirigido automáticamente a la pantalla del diagramador entidad-relación.

\begin{figure}[H]
    \centering
    \includegraphics[width=0.75\textwidth]{interfaz/register.png}
    \caption{Pantalla de registro.}
    \label{img:app_register}
\end{figure}

\subsection*{Pantalla de login}

La figura~\ref{img:app_login} cuenta con un formulario para que el usuario ingrese el \textit{email} y contraseña que proporcionó en el momento de su registro; una vez que el sistema valide la existencia del usuario en la base de datos, será redirigido automáticamente a la pantalla del diagramador entidad-relación.

\begin{figure}[H]
    \centering
    \includegraphics[width=0.75\textwidth]{interfaz/login.png}
    \caption{Pantalla de login.}
    \label{img:app_login}
\end{figure}


\subsection*{Pantallas del diagramador entidad-relación}

La figura~\ref{img:app_diagrammerER} muestra lo que visualiza el usuario una vez que concluyó su registro o inició sesión; se puede apreciar las herramientas para crear/editar un diagrama entidad-relación, además de la zona de trabajo conocida como \textit{canvas}.

\begin{figure}[H]
    \centering
    \includegraphics[width=0.75\textwidth]{interfaz/er_diagramer.png}
    \caption{Pantalla del diagramador entidad-relación}
    \label{img:app_diagrammerER}
\end{figure}


La figura~\ref{img:app_errorDiagram} muestra los errores del diagrama entidad-relación después que el usuario hace click en el  botón ``validar diagrama'', en caso que el diagrama no cumpla con las reglas mencionadas en la sección \ref{cap:validationER}.

\begin{figure}[H]
    \centering
    \includegraphics[width=0.75\textwidth]{interfaz/invalid_diagramER.png}
    \caption{Pantalla de erroresal validar un diagrama.}
    \label{img:app_errorDiagram}
\end{figure}

La figura~\ref{img:app_validDiagram} muestra el modal que el usuario visualiza después de hacer click en el botón ``validar diagrama'' y este cumple con todas las reglas de validación estructural.

\begin{figure}[H]
    \centering
    \includegraphics[width=0.75\textwidth]{interfaz/valid_diagramER.png}
    \caption{Pantalla de erroresal validar un diagrama.}
    \label{img:app_validDiagram}
\end{figure}

\subsection*{Pantallas de las sentencias SQL equivalentes}

La figura~\ref{img:app_sqlSentences} muestra las pantallas del módulo de obtención de las sentencias equivalentes del diagrama que el usuario generó en la figura~\ref{img:app_diagrammerER}; este paso solo es posible después de haber pasado por el proceso de validación para el diagrama entidad-relación.
Del lado derecho en la figura~\ref{img:app_sqlScript} se aprecia el código en el lenguaje SQL necesario para crear la base de datos relacional en el sistema gestor de base de datos MySQL, y del lado izquierdo en la figura~\ref{img:app_dbName} se muestra el modal que el usuario visualiza al hacer click en el botón ``Obtener sentencias SQL'' en el cual deberá colocar el nombre que tendrá la base de datos a generar.

Si el usuario necesita exportar el script de sql a un archivo, cuenta con un botón en la parte superior para realizar esta acción.

\begin{figure}[H]
    \begin{subfigure}[b]{0.49\textwidth}
        \includegraphics[width=\textwidth]{interfaz/sql_sentences.png}
        \caption{Sentencencias SQL equivalentes al diagram ER.}
        \label{img:app_sqlScript}
      \end{subfigure}
      \hfill
      \begin{subfigure}[b]{0.49\textwidth}
        \includegraphics[width=\textwidth]{interfaz/get_sql_sentences.png}
        \caption{Modal para nombrar la base de datos sql.}
        \label{img:app_dbName}
      \end{subfigure}
    \caption{Pantallas de las sentencias SQL equivalentes al diagrama ER.}
    \label{img:app_sqlSentences}
\end{figure}

\subsection*{Pantallas de las consultas de acceso}

La figura~\ref{img:app_simpleQuery} es lo que el usuario visualiza al ingresar a este módulo, este paso solo es posible después de haber pasado por el proceso de validación para el diagrama entidad-relación. Es aquí donde puede agregar las consultas de acceso que desea que sea utilicen en el proceso de tranformación al modelo noSQL teniendo del lado izquierdo el diagrama ER en modo de solo lectura y al hacer click derecho en los atributos clave de una entidad vusualizará un menú con las opciones para generar dicha consulta.

Las consultas de acceso pueden ser tantas con crea necesitarlas como se aprecia en la figura~\ref{img:app_multipleQueries}, es importante mencionar que puede agregar tantos elementos al apartado ``Respecto al atributo'' como quiera pero debe existir al menos un elemento en el apartado ``Encontrar'' para que la consulta tenga sentido.

\begin{figure}[H]
    \centering
    \includegraphics[width=\textwidth]{interfaz/queries_simple.png}
    \caption{Pantalla para agregar una consulta de acceso.}
    \label{img:app_simpleQuery}
\end{figure}

\begin{figure}[H]
    \centering
    \includegraphics[width=\textwidth]{interfaz/queries_multiple.png}
    \caption{Pantalla con multiples consultas de acceso.}
    \label{img:app_multipleQueries}
\end{figure}

\section{Conclusiones}

La arquitectura cliente-servidor es apta para un desarrollo ágil de la propuesta de solución; asimismo, la arquitectura elegida permite al equipo usar lenguajes de programación que estén consolidados en el desarrollo web o en el diagramado, como es el caso de los lenguajes elegidos en la sección~\ref{ref:conclusiones-cap3}.


El patrón MVC está bien integrado con el \textit{framework} Nuxt y Flask, permitiendo que desde etapas tempranas del desarrollo de la propuesta de solución haya un prototipo funcional que puede revisar en la sección~\ref{ref:prototipo}.


Finalmente, el diagrama de clases y el diagrama de la base de datos también están integrados, porque desde el diseño conceptual es fácil usar el concepto de la anidación de datos y es solo cuestión de implementarlo en MongoDB.
