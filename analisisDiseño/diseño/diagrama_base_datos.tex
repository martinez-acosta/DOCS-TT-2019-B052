\section{Diagrama de la base de datos}
Como MongoDB es una base de datos NoSQL orientada a documentos, los datos se almacenan en documentos JSON; teniendo en cuenta el diagrama de clases de la figura~\ref{img:classes_diagram}, se aprecia que las relaciones son por composición de 1 a 1 y la clase dominante es Usuario, es decir, las clases DiagramaER, DiagramaRelacional y DiagramaNoSQL no pueden existir si no existe una clase Usuario asociada a ellas.


Por esto la mejor manera de almacenar los datos es en documentos anidados, donde el documento principal es el usuario y este contendrá los documentos de las otras clases; en la figura \ref{img:database_schema} se representa la relación por composición de las clases de la figura~\ref{img:classes_diagram}.


\begin{figure}[H]
    \centering
    \includegraphics[width=0.75\textwidth]{diseño/diagrama_DB.png}
    \caption{Diagrama de la base de datos.}
    \label{img:database_schema}
\end{figure}


En la figura~\ref{img:database_schema} se pone en práctica el comportamiento del diagrama de clases de la figura~\ref{img:classes_diagram}, porque al eliminar un usuario internamente se eliminan los diagramas asociados a este; es por este motivo que el usuario debe estar 100 \% seguro al momento de eliminar su cuenta, ya que será imposible recuperar los diagramas que haya realizado un vez completada la acción.