\section{Diagrama de la base de datos}


Al utilizarse una base de datos no relacional orientada a documentos como lo es MongoDB los datos se almacenarán en documentos json. Teniendo en cuenta el diagrama de clases anterior se aprecia que las relaciones son por composición de 1 a 1 y la clase dominante es Usuario, es decir, las clases DiagramaER, DiagramaRelacional y DiagramaNoSQL no pueden existir si no existe una clase Usuario asociada a ellas.


Es por esto que la mejor manera de almacenar los datos es en documentos anidados donde el documento principal es el usuario y este contendrá los documentos de las otras clases, en la figura \ref{img:database_schema} se representa la relación por composición de las clases mencionada con anterioridad.


\begin{figure}[H]
    \centering
    \includegraphics[width=0.75\textwidth]{diseño/diagrama_DB.png}
    \caption{Diagrama de la base de datos.}
    \label{img:database_schema}
\end{figure}


De esta manera podemos asegurar el comportamiento del diagrama de clases y al eliminar un usuario internamente se eliminan los diagramas asociados a este. Es por este motivo que el usuario debe estar 100\% seguro al momento de eliminar su cuenta ya que será imposible recuperar los diagramas que haya realizado un vez completada la acción.