\section{Arquitectura del sistema}

De acuerdo con Pressman~\cite{pressman_software_2005} en su sección diseño de la arquitectura, el diseño arquitectónico de un \textit{software} es la representación de los componentes o módulos del sistema y cómo estos interactúan entre sí; en esta sección se muestra el diseño arquitectónico que usa el sistema como parte de su implementación.

\subsection{Arquitectura cliente-servidor}

Pressman menciona en su trabajo~\cite{pressman_software_2005} que la arquitectura cliente-servidor es un modelo de diseño del \textit{software} compuesta por 2 partes, el cliente que solicita información por medio de una petición y el servidor quien se encarga de proveer los datos que le son requeridos; las aplicaciones web utilizan frecuentemente este modelo teniendo como clientes a un navegador web y el servidor es quien provee la información que se solicite, pudiendo atender a más de un cliente al mismo tiempo.

La figura \ref{img:arq_client_server} describe la arquitectura cliente servidor del sistema.

\begin{figure}[H]
    \centering
    \includegraphics[width=0.75\textwidth]{diseño/cliente_servidor.png}
    \caption{Arquitectura cliente-servidor.}
    \label{img:arq_client_server}
\end{figure}


En la figura~\ref{img:arq_client_server} se observa que para poder hacer uso del sistema, el usuario debe contar con un dispositivo capaz de ejecutar un navegador web como una computadora o un \textit{smartphone}; el servidor será una computadora que se encuentra en un instancia de la nube que contiene los requisitos necesarios para el funcionamiento del sistema como son: el sistema operativo GNU/Linux, el servidor web, las aplicaciones en Flask y Nuxt, conexión a Internet y una configuración de red que permita la comunicación con la base de datos de MongoDB.

En la figura~\ref{img:desgloce_arquitectura} se muestran los componentes y módulos con los que cuentan tanto el cliente como el servidor. Se utilizas distintas bibliotecas para poder realizar las funcionalidades de la aplicación, por ejemplo Vuex para manejar el estado de los datos que el usuario ingresa en la interfaz gráfica o flask-pymongo para el manejo de la conexión a la base de datos.

\begin{figure}[H]
  \centering
  \includegraphics[width=0.75\textwidth]{diseño/desgloce_diagrama.png}
  \caption{Desgloce con los componentes de la arquitectura cliete-servidor.}
  \label{img:desgloce_arquitectura}
\end{figure}


\subsection{Patrón de diseño MVC}

Según Dragos-Paul Pop menciona en su trabajo~\cite{pop_designing_2014}, el patrón MVC fue concebido por primera vez en la década de los 70 en la empresa Xerox y su principal propósito era cerrar la brecha entre el modelo mental del usuario y el modelo digital de las aplicaciones en una computadora.


Más tarde, el paradigma MVC fue descrito por Krasner y Pope\cite{pope_cookbook_1988} donde destacan que se pueden obtener enormes beneficios si se piensa en la modularidad de las aplicaciones; el patrón se divide en 3 partes principalmente: el modelo, la vista y el controlador.

\begin{itemize}
    \item El modelo: es el dominio principal de la aplicación, es quien se encarga del manejo de los datos y toda la lógica del negocio, es decir, no tiene que tener conocimiento sobre la interfaz gráfica, ya sea una aplicación web, de escritorio o móvil; idealmente el modelo debe ser independiente de la plataforma en que se ejecuta.
    \item La vista: contiene todos los elementos visibles al usuario y a diferencia del modelo, esta no debe tener conocimiento de la lógica del negocio de la aplicación; sus responsabilidades deben limitarse a definir la estructura y apariencia de los datos presentados en la interfaz gráfica.
    \item El controlador: es el intermediario entre la vista y el modelo, es quien gestiona el flujo de la aplicación por medio de la comunicación entre los otros 2 componentes.
\end{itemize}

La figura \ref{img:mvc_implementation} describe la implementación del patrón MVC en la propuesta de solución junto a las tecnologías utilizadas en cada componente del paradigma; como se aprecia la vista queda a cargo del \textit{framework} Nuxt quien es el encargado del renderizado de la interfaz gráfica para el usuario mostrando elementos como el formulario de \textit{login} y el diagramador entidad-relación.

\begin{figure}[H]
  \centering
  \includegraphics[width=0.75\textwidth]{diseño/patron_mvc.png}
  \caption{Esquema MVC}
  \label{img:mvc_implementation}
\end{figure}


De igual manera, el \textit{framework flask} será el encargado del rol de controlador; llevará a cabo al comunicación de los datos que el usuario proporcione en la vista para transformarlos en algo que el modelo pueda entender como el guardado del objeto JSON del diagrama entidad-relación para ser almacenado en la base de datos; finalmente, el modelo queda a cargo del lenguaje Python para el manejo de los datos y es el encargado de mantener la comunicación con la base de datos, pudiéndose apoyar en bibliotecas de terceros como el propio \textit{framework flask} para transformar los datos de la lógica de negocio y convertirlos en algo que la vista pueda mostrar al usuario.
