\section{Diagrama de clases}

De acuerdo con Visual Paradigm~\cite{visual_paradigm_visual_2020}, el diagrama de clases es un modelo conceptual que permite visualizar la estructura de las clases de un sistema, sus atributos, métodos y las relaciones entre objetos.


El diagrama de clases de la propuesta de solución se muestra en la figura~\ref{img:classes_diagram}; en la cual se aprecia la clase dominante \textit{usuario} y las demás clases dependen de esta.


\begin{figure}[H]
  \centering
  \includegraphics[width=0.75\textwidth]{diseño/diagrama_de_clases.png}
  \caption{Diagrama de clases}
  \label{img:classes_diagram}
\end{figure}

Es posible interpretar el significado del diagrama de clases de la figura~\ref{img:classes_diagram} leyendo los puntos de la siguiente manera:

\begin{itemize}
  \item El usuario es la clase principal.
  \item Existe una agregación entre Usuario y DiagramaER.
  \item El ModeloGDM es parte del ModeloLogicoNoSQL; el ModeloGDM puede existir por sí mismo.
  \item El GDM depende del DiagramaER; sin embargo, el DiagramaER no depende del GDM.
  \item La ConsultaDeAccessoSimple depende del Diagrama ER.
  \item El ModeloLogicoRelacional depende del DiagramaRelacional.
\end{itemize}

