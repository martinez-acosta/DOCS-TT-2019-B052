\section{Interfaz de usuario}

En esta sección se muestran las imágenes del prototipo funcional de la propuesta de solución junto a una breve descripción de cada una de las pantallas.

\subsection*{Pantalla de inicio}

La figura~\ref{img:app_home} muestra la pantalla inicial que el usuario observa apenas entra a la \href{https://serene-haibt-2239b4.netlify.app/}{dirección}; donde tiene la opción de registrarse o iniciar sesión en caso de no haberlo hecho anteriormente.

\begin{figure}[H]
    \centering
    \includegraphics[width=0.75\textwidth]{interfaz/home.png}
    \caption{Pantalla inicial.}
    \label{img:app_home}
\end{figure}

\subsection*{Pantalla de registro}

La figura \ref{img:app_register} cuenta con un formulario para el registro de un usuario nuevo, se deben llenar todos los datos cumpliendo una serie de reglas entre ellas:

\begin{itemize}
    \item La contraseña debe tener al menos 6 caracteres.
    \item Se solicita que repitan la contraseña en otro campo y deben coincidir.
    \item Se debe colocar un \textit{email} válido.
\end{itemize}

Una vez llenado los datos solicitados y al dar clic en el botón \textit{submit}, el usuario recibirá un correo de bienvenida en la dirección de \textit{submit} que proporcionó; al finalizar el proceso de registro del usuario en la base de datos, será redirigido automáticamente a la pantalla del diagramador entidad-relación.

\begin{figure}[H]
    \centering
    \includegraphics[width=0.75\textwidth]{interfaz/register.png}
    \caption{Pantalla de registro.}
    \label{img:app_register}
\end{figure}

\subsection*{Pantalla de login}

La figura~\ref{img:app_login} cuenta con un formulario para que el usuario ingrese el \textit{email} y contraseña que proporcionó en el momento de su registro; una vez que el sistema valide la existencia del usuario en la base de datos, será redirigido automáticamente a la pantalla del diagramador entidad-relación.

\begin{figure}[H]
    \centering
    \includegraphics[width=0.75\textwidth]{interfaz/login.png}
    \caption{Pantalla de login.}
    \label{img:app_login}
\end{figure}


\subsection*{Pantallas del diagramador entidad-relación}

La figura~\ref{img:app_diagrammerER} muestra lo que visualiza el usuario una vez que concluyó su registro o inició sesión; se puede apreciar las herramientas para crear/editar un diagrama entidad-relación, además de la zona de trabajo conocida como \textit{canvas}.

\begin{figure}[H]
    \centering
    \includegraphics[width=0.75\textwidth]{interfaz/er_diagramer.png}
    \caption{Pantalla del diagramador entidad-relación}
    \label{img:app_diagrammerER}
\end{figure}


La figura~\ref{img:app_errorDiagram} muestra los errores del diagrama entidad-relación después que el usuario hace click en el  botón ``validar diagrama'', en caso que el diagrama no cumpla con las reglas mencionadas en la sección \ref{cap:validationER}.

\begin{figure}[H]
    \centering
    \includegraphics[width=0.75\textwidth]{interfaz/invalid_diagramER.png}
    \caption{Pantalla de erroresal validar un diagrama.}
    \label{img:app_errorDiagram}
\end{figure}

La figura~\ref{img:app_validDiagram} muestra el modal que el usuario visualiza después de hacer click en el botón ``validar diagrama'' y este cumple con todas las reglas de validación estructural.

\begin{figure}[H]
    \centering
    \includegraphics[width=0.75\textwidth]{interfaz/valid_diagramER.png}
    \caption{Pantalla de erroresal validar un diagrama.}
    \label{img:app_validDiagram}
\end{figure}

\subsection*{Pantallas de las sentencias SQL equivalentes}

La figura~\ref{img:app_sqlSentences} muestra las pantallas del módulo de obtención de las sentencias equivalentes del diagrama que el usuario generó en la figura~\ref{img:app_diagrammerER}; este paso solo es posible después de haber pasado por el proceso de validación para el diagrama entidad-relación.
Del lado derecho en la figura~\ref{img:app_sqlScript} se aprecia el código en el lenguaje SQL necesario para crear la base de datos relacional en el sistema gestor de base de datos MySQL, y del lado izquierdo en la figura~\ref{img:app_dbName} se muestra el modal que el usuario visualiza al hacer click en el botón ``Obtener sentencias SQL'' en el cual deberá colocar el nombre que tendrá la base de datos a generar.

Si el usuario necesita exportar el script de sql a un archivo, cuenta con un botón en la parte superior para realizar esta acción.

\begin{figure}[H]
    \begin{subfigure}[b]{0.49\textwidth}
        \includegraphics[width=\textwidth]{interfaz/sql_sentences.png}
        \caption{Sentencencias SQL equivalentes al diagram ER.}
        \label{img:app_sqlScript}
      \end{subfigure}
      \hfill
      \begin{subfigure}[b]{0.49\textwidth}
        \includegraphics[width=\textwidth]{interfaz/get_sql_sentences.png}
        \caption{Modal para nombrar la base de datos sql.}
        \label{img:app_dbName}
      \end{subfigure}
    \caption{Pantallas de las sentencias SQL equivalentes al diagrama ER.}
    \label{img:app_sqlSentences}
\end{figure}

\subsection*{Pantallas de las consultas de acceso}

La figura~\ref{img:app_simpleQuery} es lo que el usuario visualiza al ingresar a este módulo, este paso solo es posible después de haber pasado por el proceso de validación para el diagrama entidad-relación. Es aquí donde puede agregar las consultas de acceso que desea que sea utilicen en el proceso de tranformación al modelo noSQL teniendo del lado izquierdo el diagrama ER en modo de solo lectura y al hacer click derecho en los atributos clave de una entidad vusualizará un menú con las opciones para generar dicha consulta.

Las consultas de acceso pueden ser tantas con crea necesitarlas como se aprecia en la figura~\ref{img:app_multipleQueries}, es importante mencionar que puede agregar tantos elementos al apartado ``Respecto al atributo'' como quiera pero debe existir al menos un elemento en el apartado ``Encontrar'' para que la consulta tenga sentido.

\begin{figure}[H]
    \centering
    \includegraphics[width=\textwidth]{interfaz/queries_simple.png}
    \caption{Pantalla para agregar una consulta de acceso.}
    \label{img:app_simpleQuery}
\end{figure}

\begin{figure}[H]
    \centering
    \includegraphics[width=\textwidth]{interfaz/queries_multiple.png}
    \caption{Pantalla con multiples consultas de acceso.}
    \label{img:app_multipleQueries}
\end{figure}
