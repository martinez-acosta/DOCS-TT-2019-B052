\subsubsection{Generic Data Metamodel a modelo lógico NoSQL}\label{alg:gdm-to-logic}
De acuerdo con de la Vega\cite{de_la_vega_mortadelo_2020}, las bases de datos orientadas a documentos tienen como objetivo almacenar jerarquías de objetos que son probables que se consultarán juntas.


Estas jerarquías de objetos se conocen como documentos y se agrupan en colecciones; asimismo, una colección almacena documentos de la misma entidad y por lo general un documento está compuesto de pares clave-valor y contiene otros documentos incrustados.


\paragraph*{Modelo lógico orientado a documentos}

En la figura~\ref{img:mortadelo-gdm-logical-model-document-oriented} se muestra el modelo lógico orientado a documentos propuesto por Alfonso de la Vega.


\begin{figure}[H] 
    \centering
    \includegraphics[width=0.65\textwidth]{mortadelo/GDM-logical-model-document-oriented.png}
    \caption{Modelo lógico orientado a documentos propuesto por Alfonso de la Vega}
    \label{img:mortadelo-gdm-logical-model-document-oriented}
\end{figure}

A continuación se explica cada clase que conforma el modelo de la figura~\ref{img:mortadelo-gdm-logical-model-document-oriented}:

\begin{itemize}
    \item Clase \textit{document data model}: está compuesta de $n$ \textit{collections}.
    \item Clase \textit{collection}: tiene un nombre que la identifica y en una colección se almacenan documentos que, en general, comparten la misma estructura.
    \item Clase \textit{document type}: define la estructura de los documentos a guardar con un conjunto de instancias de la clase \textit{field}; es la estructura principal que se guardará en cada colección.
    \item Clase \textit{field}: tiene instancias de las clases \textit{primitive field} (atributos simples), \textit{array field} (más instancias de la clase \textit{field}) o \textit{document type} (otras estructuras de documentos), permitiendo guardar elementos anidados.
    \item Clase \textit{primitive field}: contiene un tipo primitivo de dato.
    \item Clase \textit{array field}: permite anidar más instancias de la clase \textit{field}.
\end{itemize}


La figura~\ref{img:mortadelo-gdm1} muestra el Generic Data Metamodel, que está compuesto por clases interrelacionadas entre sí con notación UML y consta de dos secciones principales: los elementos de la estructura (\textit{structure model elements}) y el cómo se realizarán las consultas (\textit{access queries elements}).


\begin{figure}[H] 
    \centering
    \includegraphics[width=0.75\textwidth]{mortadelo/GDM.png}
    \caption{Generic Data Metamodel}
    \label{img:mortadelo-gdm1}
\end{figure}



Sea $\mathrm{Queries}$ el conjunto de todas las queries $q_{n}$ en una instancia GDM como la de la figura~\ref{img:mortadelo-gdm1}:

\begin{equation} \label{eq:cap4.3.3.5-00}
    \mathrm{Queries} = \{q_{1}, q_{2},..., q_{n}\}\; \mathrm{donde}\; q_{n}\;\epsilon\;\mathrm{Queries}\; \mathrm{y}\; n=1,2,...
\end{equation}


Sea $\mathrm{Entities_{consulted}}$ el conjunto de entidades que son consultadas en una instancia GDM como la de la figura~\ref{img:mortadelo-gdm1} (es decir, las entidades que están asociadas a la clase \textit{query} desde un elemento \textit{from}): 


\begin{equation} \label{eq:cap4.3.3.5-01}
    \mathrm{Entities_{consulted}} = \{e_{c1},e_{c2},...,e_{cn}\}\; \mathrm{donde}\; e_{cn}\;\epsilon\;\mathrm{Entities_{consulted}}\; \mathrm{y}\; n=1,2,...
\end{equation}

Sea $\mathrm{Collections}$ el conjunto de colecciones $c_{n}$ donde cada $c_{n}$ corresponde a una entidad consultada:

\begin{equation} \label{eq:cap4.3.3.5-02}
    \mathrm{Collections} = \{c_{1},c_{2},...,c_{n}\}\; \mathrm{donde}\; c_{n}\;\epsilon\;\mathrm{Collections}\; \mathrm{y}\; n=1,2,...
\end{equation}


% $e_{fn}$ de $E_{f}$ (es decir que por cada $ef_{n}$ se crea un $c_{n}$ en la que cada $c_{n}$ contiene un único elemento \textit{DocumentType}, llamado ``documento raíz''):


Por cada $c_{n}$ en \eqref{eq:cap4.3.3.5-02} se generan los contenidos de cada documento raíz.


La figura~\ref{img:mortadelo-gdm-logical-model-access-tree} es un ejemplo del árbol de acceso de la consulta $q_{4}$ de la figura~\ref{img:mortadelo-gdm-logical-model-q4}; está formado por el único elemento \textit{from} de la consulta y con los elementos \textit{including} se forma el árbol, donde las aristas son los identificadores de cada \textit{including}.


La figura~\ref{img:mortadelo-gdm-logical-model-q4} es una consulta en el GDM en modo texto.


\begin{figure}[H] 
    \centering
    \includegraphics[width=0.65\textwidth]{mortadelo/GDM-access-tree.png}
    \caption{Access Tree - Modelo lógico orientado a documentos}
    \label{img:mortadelo-gdm-logical-model-access-tree}
\end{figure}


\begin{figure}[H] 
    \centering
    \includegraphics[width=0.65\textwidth]{mortadelo/GDM-q4.png}
    \caption{Access query Q4}
    \label{img:mortadelo-gdm-logical-model-q4}
\end{figure}


Por cada $q_{n}$ en \eqref{eq:cap4.3.3.5-00} se obtiene primero su único elemento \textit{from} y después todos los elementos \textit{including} para generar un árbol de acceso como el de la figura~\ref{img:mortadelo-gdm-logical-model-access-tree}.


Añadimos al documento raíz todos los atributos simples de su entidad correspondiente; nótese que se está analizando tanto la instancia de la clase \textit{entity} y el árbol de acceso creado.


Si el elemento ``ref'' de la \textit{entity} está en el árbol de acceso, se necesita saber su cardinalidad y la entidad objetivo de esa referencia.


Si la entidad objetivo está en el árbol de acceso, los datos de esa referencia se incluyen como un subdocumento y se llama recursivamente la función para generar el contenido de este subdocumento; por el contrario, en caso de que la referencia no esté en el árbol de acceso, se incluye la referencia como un identificador si la cardinalidad es 1 o como un arreglo de referencias cuando la carnidad es $n$.


Por último, el autor menciona dos optimizaciones si se quiere reducir el nivel de denormalización que las puede consultar en su investigación\cite{de_la_vega_mortadelo_2020}.

En resumen, el algoritmo quedaría de la forma:
\begin{algorithm}[H]
    \SetKwInOut{Input}{Input}
    \SetKwInOut{Output}{Output}

    %\underline{function Euclid} $(a,b)$\;
    \Input{una instancia del modelo GDM, $gdm$}
    \Output{un modelo lógico orientado a documentos, $ddm$}
    $mainEntities \gets gdm.queries.collect((q)|q.from);$\\
    \ForEach{$me \in mainEntities$}{
        $collection \gets new Collection();$\\
        $collection.name \gets me.name;$\\
        $accessTree \gets allQueryPaths(me,gdm.queries);$\\
        $collection \gets populateDocumentType(collection.root,accessTree);$\\
        $ddm.collections.add(collection);$
    }
    
    \caption{Transformación del modelo conceptual GDM al modelo lógico orientado a documentos}
\end{algorithm}

Donde la función populateDocumentType() es otro algoritmo de la forma:

\begin{algorithm}[H]
    \SetKwInOut{Input}{Input}
    \SetKwInOut{Output}{Output}

    %\underline{function Euclid} $(a,b)$\;
    \Input{Un ``document type", $dt$}
    \Output{un nodo del arbol de acceso}
    $ nodeAttributes \gets node.entity.features.select(f|f.isTypeOf(Attribute))\; $\\
    $ nodeReferences \gets node.entity.features.select(f|f.isTypeOf(Reference))\; $\\
    \ForEach{$attr \in nodeAttributes$}{
        $pf \gets new PrimitiveField()\;$\\
        $pf.name \gets attr.name\;$\\
        $pf.type \gets attr.type\;$\\
        $dt.fields.add(pf)\;$\\
    }
    \ForEach{$ref \in nodeReferences$}{
        $targetNode \gets node.arcs.find(a|a.name = ref.name).target\;$\\
        \uIf{exists(targetNode)}{
            $baseType \gets new DocumentType()\;$\\
            $populateDocumentType(baseType,targetNode)\;$\\
        }
        \Else{
            $baseType \gets new PrimitiveField()\;$\\
            $baseType.type \gets findIdType(ref.entity)\;$\\
        }
        $baseType.name \gets ref.name\;$\\

        \uIf{$ref.cardinality$ == 1}{
            $dt.field.add(baseType)\;$\\
            
        }
        \Else{
            $arrayField \gets new ArrayField()\;$\\
            $arrayField.type \gets baseType\;$\\
            $dt.fields.add(arrayField)\;$
        }
        
    }
    \caption{Generar el contenido de un \textit{DocumentType} dado un árbol de acceso}
\end{algorithm}
