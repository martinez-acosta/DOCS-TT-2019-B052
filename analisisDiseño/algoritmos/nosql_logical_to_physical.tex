\subsubsection{Modelo lógico NoSQL a modelo físico en MongoDB}\label{sec:logico-documentos-fisico}


De acuerdo con la documentación de MongoDB\cite{mongodb_mongodb_2020}, se puede escribir un \textit{script} para crear un esquema  de base de datos en un archivo con extensión .js y con el contenido de los datos en formato JSON válido; para generar el esquema completo se deben seguir los siguientes pasos:


\begin{enumerate}
    \item Crear un archivo con extensión .js.
    \item Abrir el archivo con los permisos de lectura y escritura para poder añadir contenido.
    \item Agregar al archivo una línea con la instrucción \textit{conn = new Mongo();}.
    \item Agregar una nueva línea al archivo .js con la instrucción \textit{db = conn.getDB(`DB\_NAME');} esto creará una base de datos con el nombre que se le indique en caso de que no exista.
    \item Agregar al archivo la instrucción \textit{db.dropDatabase()} para eliminar todos los registros del esquema de la base de datos en caso de que existan y así evitar datos erróneos.
    \item Agregar una línea con la instrucción \textit{db.createCollection(`COLLECTION\_N');} por cada entidad del GDM, donde \textit{COLLECTION\_N} es sustituido por el nombre de la entidad.
    \item Agregar una línea con el nombre de un elemento de la primera colección seguida de un signo igual (=) y un carácter de llave `\{'.
    \item Agregar una línea por cada atributo simple de la entidad del GDM con la instrucción \textit{``ATTR\_NAME'':``DEFAULT\_VALUE''}, sustituyendo \textit{ATTR\_NAME} por el nombre de cada atributo y \textit{DEFAULT\_VALUE} por un valor por defecto que corresponda al tipo de dato del atributo.
    \item Agrear una línea con el carácter `\}', para indicar el cierre del objeto JSON.
\end{enumerate}


Los pasos del 1 al 5 son suficientes para generar un esquema de base de datos vacío en MongoDB, solamente debe ejecutarse el archivo con el comando \textit{mongo file.js} y se puede comprobar que todo es correcto entrando a la consola de MongoDB y revisar los esquemas disponibles con el comando \textit{show dbs;}.


Los incisos posteriores generan un objeto de prueba para cada colección de datos, esto se hace principalmente para visualizar un elemento en cada colección; de lo contrario solamente se apreciarían colecciones vacías en MongoDB y estas pueden almacenar todo tipo de información siempre y cuando se trate de un objeto JSON válido.


Finalmente, una vez verificada que la estrucuctura es correcta, se deben truncar todas las colecciones y de esta manera tener un esquema de la base de datos NoSQL sin registros y empezar a almacenar los documentos; esto se logra con la instrucción \textit{db.COLLECTION\_NAME.remove(\{\})} sustituyendo \textit{COLLECTION\_NAME} por cada uno de los nombres de las colecciones.
