\subsubsection{Modelo entidad-relación básico a Generic Data Metamodel}\label{sec:er-to-gdm}

Se ha decidido que el algoritmo debe ser basado en \textit{querys} como el de Chebotko; por ello para obtener consultas válidas, debe existir una ruta en el modelo entidad-relación básico que permita llegar desde la entidad n a la entidad m recorriendo las rutas del diagrama.


La figura~\ref{img:simple-access-pattern} muestra el boceto de la consulta simple a implementar, está inspirada en el trabajo de Chebotko; cada consulta tiene un nombre, la columna \textit{find} representa el atributo a buscar respecto a los diferentes atributos de la columna \textit{given}. En la parte de abajo está un campo para poner una descripción breve de la consulta y botones para limpiar los respectivos campos de la consulta. En la figura~\ref{img:simple-access-pattern} \textit{User} y \textit{Artifact} son entidades y \textit{id} e \textit{email} son atributos de sus respectivas entidades.


\begin{figure}[H] 
    \centering
    \includegraphics[width=0.75\textwidth]{simple_access_patterns.png}
    \caption{Consulta básica}
    \label{img:simple-access-pattern}
\end{figure}


El algoritmo por entrada tiene una instancia del modelo conceptual entidad-relacíon básico $+$ consultas básicas como las de la figura~\ref{img:simple-access-pattern} y de salida una instancia del modelo GDM.


La definición del modelo orientado a documentos será la usada por Alfonso de la Vega en su publicación sobre el GDM.


Se crean primero los elementos de consultas del GDM a partir de las consultas básicas de la instancia del modelo entidad-relación básico como las de la figura~\ref{img:simple-access-pattern}:


\begin{enumerate}
    \item Por cada consulta se crea un elemento de la clase \textit{query} asignando el nombre de la consulta a realizar.
    \item Se crea un elemento \textit{select} con los parámetros de la columna \textit{given} del diagrama entidad-relación básico.
    \item Se crea el elemento \textit{from} desde la primera entidad en la que se realiza la consulta.
    \item Se recorre la ruta del diagrama entidad-relación básico y se van añadiendo cada entidad del recorrido como un elemento \textit{including} con el nombre de la relación como referencia.
    \item Finalmente, se crea un elemento de la clase \textit{boolean expression} para contener la expresión booleana de la consulta.
\end{enumerate}


Para los elementos del \textit{structure model elements} del GDM:

\begin{enumerate}
    \item Por cada entidad se crea una clase entidad correspondiente en el GDM. 
    \item Por cada atributo se crea un clase atributo correspondiente y se indica su tipo.
    \item Por cada relación se crea una referencia correspondiente.
\end{enumerate}

%El algoritmo necesita ser basado en querys como el de Chebotko.


%input: modelo entidad-relacion


%output: ¿modelo orientado a documentos o GDM?


% ¿Qué definición del modelo orientado a documentos se usará? ¿El del GDM o el de De Lima?


% ¿Cómo convertir las relaciones?
% Hay reglas para convertir las relaciones en bloques del documento orientado a objetos.
