\subsubsection{Obtención de esquema SQL desde modelo relacional}


Las sentencias SQL que generará la aplicación web son especificamente para el gestor MySQL, por lo cual de acuerdo a su guia de sintaxis [https://manuales.guebs.com/mysql-5.0/sql-syntax.html] para obtener las sentencias SQL del modelo relacional se seguiran los siguientes pasos:

\begin{enumerate}
  \item Por cada relación(tabla) del modelo relacional se agrega una sentencia CREATE TABLE seguida del nombre de la relación seguido de sus atributos encerrados entre paréntesis.
  \item Por cada atributo(columna) se crea un campo con su nombre correspondiente y se indica el tipo de dato, para hacerlo mas especifico se le puede añadir alguna de las siguientes caracteriticas:
  \begin{itemize}
    \item Los atributos que esten obligados a tener un valor se le agrega la palabra NOT NULL.
    \item Si la columna tomará un valor por defecto se agrega la instrucción AUTO\_INCREMENT siempre y cuando el tipo de dato sea INTEGER.
    \item Si la columna es la que identificará a la tabla de forma única en todo el esquema se debe agregar como llave primaria de la tabla.
  \end{itemize}
  \item La llave primaria de agrega como un atributo mas con la palabra PRIMARY KEY seguido del nombre de atributo,encerrada entre paréntesis, que indentifica de forma unica a la tabla.
  \item Todos los atributos de la tabla se encuentran encerrados entre paréntesis separados por comas.
  \item Al final de la lista de atrubutos de la tabla(despues del parentesis) se agrega el motor de almacenamiento de MySQL , ENGINE=InnoDB (restricción del proyecto