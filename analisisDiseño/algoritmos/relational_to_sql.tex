\subsubsection{Obtención de esquema SQL desde modelo relacional}\label{sec:esquema-sql}


De acuerdo con la sintaxis de la documentación de MySQL\cite{mysql_mysql_2020}, para obtener el esquema SQL del modelo relacional se realizarán los siguientes pasos:

\begin{enumerate}
  \item Por cada relación (tabla) del modelo relacional se agrega una sentencia CREATE TABLE, seguida del nombre de la relación y de sus atributos encerrados entre paréntesis.
  \item Por cada atributo (columna) se crea un campo con su nombre correspondiente y se indica el tipo de dato, para hacerlo más específico, se le puede añadir alguna de las siguientes características:
  \begin{itemize}
    \item Los atributos que estén obligados a tener un valor se les agrega la palabra NOT NULL.
    \item Si la columna tomará un valor por defecto, se agrega la instrucción AUTO\_INCREMENT siempre y cuando el tipo de dato sea INTEGER.
    \item Si la columna es la que identificará la tabla de forma única en todo el esquema, se debe agregar como llave primaria de la tabla.
  \end{itemize}
  \item La llave primaria se agrega como un atributo más con la palabra PRIMARY KEY, seguido del nombre de atributo (encerrada entre paréntesis) que indentifica de forma única a la tabla.
  \item Todos los atributos de la tabla se encuentran encerrados entre paréntesis separados por comas.
  \item Al final de la lista de atributos de la tabla (después del paréntesis) se agrega el motor de almacenamiento: MySQL,ENGINE=InnoDB.
\end{enumerate}
