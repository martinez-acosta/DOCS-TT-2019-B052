\subsubsection{Validación estructural diagrama entidad-relación}

Las reglas de validación básicas para el diagrama ER:

\paragraph*{Generales}
\begin{enumerate}
    \item No puede haber elementos sin conectar.
    \item Tampoco puede haber enlaces sin conectar.
    %\item Solo relación de participación binarias (restricción de la propuesta de solución). ¿esta va?
\end{enumerate}


\paragraph*{Entidad}
\begin{enumerate}
    \item Una entidad es válida si tiene atributos, porque no tiene propósito una entidad sin atributos.
    \item La clave primaria puede ser simple o compuesta.
    \item La clave primaria no es una clave foránea.
    \item La clave primaria debe ser un atributo clave asociado a la entidad (restricción de la propuesta de solución).
    \item Dos entidades solo pueden estar conectadas entre sí mediante una relación.
    \item Todas las entidades deben tener un atributo clave.
    \item En una entidad débil, un atributo solo puede estar asociado a un solo atributo o a una sola entidad.
    \item Una entidad débil no debe existir si no tiene una relación con otra entidad.
\end{enumerate}

\paragraph*{Atributo}

\begin{enumerate}    
\item Un atributo solo puede estar asociado a un solo atributo o a una sola entidad.
\item Un atributo puede ser compuesto.
\item Un atributo puede ser multivalor.
\item Un atributo puede ser derivado.
\item Un atributo debe tener un nombre.
\item Un atributo no puede usar una relación para asociarse a otro elemento.
\item Un atributo compuesto solo puede estar asociado a una entidad.
\item Un atributo derivado solo puede estar asociado a una entidad.
\end{enumerate}

\paragraph*{Relación}

\begin{enumerate}
    \item Una relación solo puede ser entre entidades.
    \item El grado de participación máximo es dos (restricción de la propuesta de solución).
    \item Una relación puede ser unaria (recursiva).
    \item No están permitidas relaciones ternarias o de grado n (restricción de la propuesta de solución).
\end{enumerate}

Para los diferentes tipos de relaciones, de acuerdo con Dullea\cite{dullea_analysis_2003}, el modelo ER está compuesto por entidades, las relaciones entre entidades y restricciones en esas relaciones.


La conectividad entre entidades y relaciones se denomina ruta; las rutas son los bloques de construcción de la validez estructural; definen visualmente la asociación semántica y estructural que cada entidad tiene simultáneamente con todas las demás entidades o consigo misma en una ruta.

En el modelado conceptual es posible clasificar la validez en dos tipos: validez semántica y validez estructural; un diagrama ER es válido cuando es válido semántica y estructuralmente; asimismo, los términos validez estructural y validez semántica se definen de la siguiente manera:


\begin{enumerate}
    \item Validez esctructural: un diagrama ER es estructuralmente válido solo cuando la consideración simultánea de todas las restricciones estructurales impuestas en el modelo no implica una inconsistencia lógica en ninguno de los posibles estados; 
    \item Validez semántica: un diagrama ER es semánticamente válido solo cuando cada relación representa exactamente el concepto del dominio del problema. 
\end{enumerate}


 Un diagrama ER semánticamente válido muestra la representación correcta del dominio de aplicación que se está modelando; sin embargo, dado que la validez semántica depende de la aplicación, no es posible definir criterios de validez generalizados, por lo que no se considerará la validez semántica, sino únicamente la validez estructural.


En términos generales, un diagrama ER es estructuralmente inválido si contiene construcciones que son contradictorias entre sí o al menos una restricción de cardinalidad es inconsistente.


Un diagrama ER representa la semántica de la aplicación en términos de restricciones de cardinalidad máxima y mínima. 


Cada conjunto de restricciones de cardinalidad en una sola relación debe ser coherente con todas las restricciones restantes en el modelo y en todos los estados posibles.


Una relación recursiva es estructuralmente inválida cuando las restricciones de participación y cardinalidad no respaldan la existencia de instancias de datos como lo requiere el usuario y hace que todo el diagrama sea inválido.


En general, un diagrama estructuralmente inválido refleja reglas de negocio semánticamente inconsistentes; para que un modelo sea válido, todas las rutas del modelo también deben ser válidas.


La tabla~\ref{tab:relaciones} muestra los distintos tipos de relaciones recursivas, mostrando el tipo de relación, la dirección de esa relación, sus restricciones de participación, de cardinalidad y un ejemplo.

En la tabla~\ref{tab:relaciones}, tabla~\ref{tab:reglas-validez-relaciones-recursivas} y tabla~\ref{tab:reglas-validez-relaciones-binarias} se usa la notación uno (1) y muchos (M) para la máxima cardinalidad; una sola línea para indicar la participación opcional y una línea doble para mostrar la participación obligatoria; las palabras ``obligatorio'' y ``opcional'' se utilizan en la tabla para indicar la cardinalidad mínima obligatoria (o total) y opcional (o parcial), respectivamente; además, la notación $|E|$ representa el número de instancias en la entidad $E$.

La tabla~\ref{tab:relaciones} resume cada relación recursiva por sus propiedades direccionales, la combinación de restricciones de cardinalidad mínima/máxima y ejemplos. 


La tabla~\ref{tab:reglas-validez-relaciones-recursivas} muestra primero reglas de validación para relaciones recursivas y un ejemplo válido; después se da un corolario de la validez de las relaciones recursivas y un ejemplo no válido.


La tabla~\ref{tab:reglas-validez-relaciones-binarias} tiene la misma estructura que la tabla~\ref{tab:reglas-validez-relaciones-recursivas} en la que muestra reglas de validez para las relaciones binarias y después un corolario con un ejemplo no válido.

% Please add the following required packages to your document preamble:
% \usepackage{multirow}
% \usepackage{graphicx}
\begin{table}[H]
    \centering
    \resizebox{\textwidth}{!}{%
    \begin{tabular}{lllllll}
    \hline
    \multirow{2}{*}{\begin{tabular}[c]{@{}l@{}}Tipo \\ de relación\end{tabular}} & \multirow{2}{*}{}                                             & \multirow{2}{*}{\begin{tabular}[c]{@{}l@{}}Dirección de\\  la relación\end{tabular}} & \multirow{2}{*}{\begin{tabular}[c]{@{}l@{}}Restricción de \\ participación\end{tabular}}                   & \multirow{2}{*}{\begin{tabular}[c]{@{}l@{}}Restricciones de\\ cardinalidad\end{tabular}} & \multicolumn{2}{l}{Ejemplo}                                                                                                             \\ \cline{6-7} 
                                                                                 &                                                               &                                                                                      &                                                                                                            &                                                                                          & Relación                                                                  & Roles                                                       \\ \hline
    \begin{tabular}[c]{@{}l@{}}Simétrica\\ (reflexiva)\end{tabular}              &                                                               & Bidireccional                                                                        & \begin{tabular}[c]{@{}l@{}}Opcional–opcional\\ Obligatoria-obligatoria\end{tabular}                        & \begin{tabular}[c]{@{}l@{}}1-1\\ M-N\end{tabular}                                        & Cónyuge de                                                                & Persona                                                     \\
    \begin{tabular}[c]{@{}l@{}}Asimétrica\\ (no reflexiva)\end{tabular}          & Jerárquica                                                    & Unidireccional                                                                       & Opcional-opcional                                                                                          & \begin{tabular}[c]{@{}l@{}}1-M\\ 1-1\end{tabular}                                        & \begin{tabular}[c]{@{}l@{}}Supervisa\\ Es supervisado \\ por\end{tabular} & \begin{tabular}[c]{@{}l@{}}Gerente-\\ empleado\end{tabular} \\
                                                                                 &                                                               &                                                                                      & \begin{tabular}[c]{@{}l@{}}Opcional-opcional\\ Opcional-obligatoria\\ Obligatoria-obligatoria\end{tabular} & M-N                                                                                      & \begin{tabular}[c]{@{}l@{}}Supervisa\\ Es supervisado\\ por\end{tabular}  & \begin{tabular}[c]{@{}l@{}}Gerente\\ Empleado\end{tabular}  \\
                                                                                 & Circular                                                      & Unidireccional                                                                       & \begin{tabular}[c]{@{}l@{}}Opcional-opcional\\ Obligatoria-obligatoria\end{tabular}                        & 1-1                                                                                      & \begin{tabular}[c]{@{}l@{}}Apoya\\ Es apoyado por\end{tabular}            & Servicio técnico                                            \\
                                                                                 & \begin{tabular}[c]{@{}l@{}}Jerárquica\\ Circular\end{tabular} & Unidireccional                                                                       & \begin{tabular}[c]{@{}l@{}}Opcional-opcional\\ Opcional-obligatoria\end{tabular}                           & 1-M                                                                                      & \begin{tabular}[c]{@{}l@{}}Apoya\\ Es apoyado por\end{tabular}            & Responsable                                                 \\
                                                                                 &                                                               &                                                                                      & \begin{tabular}[c]{@{}l@{}}Opcional-opcional\\ Opcional-obligatoria\\ Obligatoria-obligatoria\end{tabular} & M-N                                                                                      & Supervisa                                                                 & \begin{tabular}[c]{@{}l@{}}Gerente-\\ empleado\end{tabular} \\
                                                                                 & Reflejada                                                     & Unidireccional                                                                       & Opcional-opcional                                                                                          & 1-1                                                                                      & \begin{tabular}[c]{@{}l@{}}Gestiona\\ Se gestiona\end{tabular}            & CEO                                                        
    \end{tabular}%
    }
    \caption{Tipos de relación recursiva válidos según las restricciones de cardinalidad}
    \label{tab:relaciones}
    \end{table}




De la tabla~\ref{tab:relaciones}, una relación recursiva es simétrica o reflexiva cuando todas las instancias que participan en la relación toman un solo papel y el significado semántico de la relación es exactamente el mismo para todas las instancias que participan en la relación independientemente de la dirección en la que se ve; estos tipos de relación se denominan bidireccionales.


De la tabla~\ref{tab:relaciones}, una relación recursiva es asimétrica o no reflexiva cuando hay una asociación entre dos grupos de roles diferentes dentro de la misma entidad y el significado semántico de la relación es diferente dependiendo de la dirección en la que se ven las asociaciones entre los grupos de roles; estos tipos de relación se denominan unidireccionales.


De la tabla~\ref{tab:relaciones}, una relación recursiva es jerárquica cuando un grupo de instancias dentro de la misma entidad se clasifican en calificaciones, órdenes o clases, una encima de otra; implica un comienzo (o arriba) y un final (o abajo) para el esquema de clasificación de instancias.


De la tabla~\ref{tab:relaciones}, una relación recursiva es circular cuando una relación recursiva asimétrica tiene al menos una instancia que no cumple con la jerarquía de clasificación; la relación es unidireccional, ya que se puede ver desde dos direcciones con un significado semántico diferente.


De la tabla~\ref{tab:relaciones}, una relación reflejada existe cuando la semántica de una relación permite que una instancia de una entidad se asocie a sí misma a través de la relación.


% Please add the following required packages to your document preamble:
% \usepackage{multirow}
% \usepackage{graphicx}
\begin{table}[H]
    \centering
    \resizebox{\textwidth}{!}{%
    \begin{tabular}{lc}
    \hline
    \multicolumn{1}{c}{\multirow{2}{*}{Reglas de validación para relaciones recursivas}}                                                                                                                                                                                                          & \multirow{2}{*}{Ejemplo válido} \\  &                                 \\ \hline
    & \\ 
    \begin{tabular}[c]{@{}l@{}}Solo las relaciones recursivas 1:1 con restricciones de cardinalidad mínimas obligatorias-obligatorias u \\ opcionales son estructuralmente válidas; válido para relaciones simétricas y completamente circular.\end{tabular}                                      &           \begin{minipage}{.3\textwidth}\includegraphics[width=\linewidth]{images/validezER/tabla1/01.png}\end{minipage}                       \\
    \begin{tabular}[c]{@{}l@{}}Para las relaciones recursivas 1:M o M:1, la cardinalidad mínima opcional-opcional \\ es estructuralmente válida; válido solo para relaciones asimétricas.\end{tabular}                                                                                            & \begin{minipage}{.3\textwidth}\includegraphics[width=\linewidth]{images/validezER/tabla1/02.png}\end{minipage}                                \\
    \begin{tabular}[c]{@{}l@{}}Para 1:M las relaciones recursivas del tipo jerárquico-circular, \\ la cardinalidad mínima opcional-obligatoria son estructuralmente válidas; \\ válido solo para relaciones jerárquico-circulares.\end{tabular}                                                   & 
    \begin{minipage}{.3\textwidth}
        \includegraphics[width=\linewidth]{images/validezER/tabla1/03.png}
    \end{minipage}                                \\
    \begin{tabular}[c]{@{}l@{}}Todas las relaciones recursivas con cardinalidad máxima\\ de muchos a muchos son estructuralmente válidas independientemente de las restricciones\\ mínimas de cardinalidad; válido para relaciones simétricas, jerárquicas y jerárquicas-circulares.\end{tabular} &  \begin{minipage}{.3\textwidth}
        \includegraphics[width=\linewidth]{images/validezER/tabla1/04.png}
    \end{minipage}                                \\                               \\
    \begin{tabular}[c]{@{}l@{}}Todas las relaciones recursivas con cardinalidad mínima opcional–opcional son estructuralmente válidas;\\ válido para relaciones simétricas y asimétricas.\end{tabular}                                                                                            &  \begin{minipage}{.3\textwidth}
        \includegraphics[width=\linewidth]{images/validezER/tabla1/05.png}
    \end{minipage}                                \\                               \\ \hline
    \multicolumn{1}{c}{Colorarios de validez para relaciones recursivas}                                                                                                                                                                                                                          & Ejemplo no válido               \\ \hline
    & \\
    \begin{tabular}[c]{@{}l@{}}Todas las relaciones recursivas 1: 1 con restricciones de cardinalidad mínima obligatoria–opcional\\  u opcional-obligatoria son estructuralmente inválidas.\end{tabular}                                                                                          &   \begin{minipage}{.3\textwidth}
        \includegraphics[width=\linewidth]{images/validezER/tabla1/06.png}
    \end{minipage}                                \\                              \\
    \begin{tabular}[c]{@{}l@{}}Todas las relaciones recursivas 1:M o M:1 con restricciones de cardinalidad mínimas\\ obligatorias-obligatorias son estructuralmente inválidas.\end{tabular}                                                                                                       & \begin{minipage}{.3\textwidth}
        \includegraphics[width=\linewidth]{images/validezER/tabla1/07.png}
    \end{minipage}                                \\            \\
    \begin{tabular}[c]{@{}l@{}}Todas las relaciones recursivas 1:M o M:1 con restricción de participación obligatoria en ``uno''\\ y una restricción de participación opcional en las ``muchas'' restricciones son estructuralmente inválidas.\end{tabular}                                       & \begin{minipage}{.3\textwidth}
        \includegraphics[width=\linewidth]{images/validezER/tabla1/08.png}
    \end{minipage}                                \\           
    \end{tabular}%
    }
    \caption{Resumen de reglas de validez para relaciones recursivas con ejemplos}
    \label{tab:reglas-validez-relaciones-recursivas}
    \end{table}


% Please add the following required packages to your document preamble:
% \usepackage{multirow}
% \usepackage{graphicx}
\begin{table}[H]
    \centering
    \resizebox{\textwidth}{!}{%
    \begin{tabular}{lc}
    \hline
    \multicolumn{1}{c}{\multirow{2}{*}{Reglas de validez para relaciones binarias.}}                                                                                                                                                                                                                     & \multirow{2}{*}{Ejemplo válido} \\      
      & \\ \hline
      &                                 \\
    \begin{tabular}[c]{@{}l@{}}Una ruta acíclica que contiene todas las relaciones binarias siempre\\ es estructuralmente válida.\end{tabular}                                                                                                                                                           & \begin{minipage}{.3\textwidth}
        \includegraphics[width=\linewidth]{images/validezER/tabla2/01.png}
    \end{minipage}                               \\
    \begin{tabular}[c]{@{}l@{}}Una ruta cíclica que contiene todas las relaciones binarias y\\ una o más relaciones opcional–opcional siempre es\\ estructuralmente válida.\end{tabular}                                                                                                                 & \begin{minipage}{.3\textwidth}
        \includegraphics[width=\linewidth]{images/validezER/tabla2/02.png}
    \end{minipage}                               \\
    \begin{tabular}[c]{@{}l@{}}Una ruta cíclica que contiene todas las relaciones binarias y\\ una o más relaciones de muchos a uno con participación opcional\\ del lado ``uno'' siempre es estructuralmente válida.\end{tabular}                                                                       & \begin{minipage}{.3\textwidth}
        \includegraphics[width=\linewidth]{images/validezER/tabla2/03.png}
    \end{minipage}                               \\
    \begin{tabular}[c]{@{}l@{}}Una ruta cíclica que contiene todas las relaciones binarias y\\ una o más relaciones de muchos a muchos es siempre estructuralmente válida.\end{tabular}                                                                                                                  & \begin{minipage}{.3\textwidth}
        \includegraphics[width=\linewidth]{images/validezER/tabla2/04.png}
    \end{minipage}                               \\
    \begin{tabular}[c]{@{}l@{}}Las rutas cíclicas que contienen al menos un\\ conjunto de relaciones opuestas siempre son válidas.\end{tabular}                                                                                                                                                                                                    & \begin{minipage}{.3\textwidth}
        \includegraphics[width=\linewidth]{images/validezER/tabla2/05.png}
    \end{minipage}                               \\
    \multicolumn{1}{c}{\begin{tabular}[c]{@{}c@{}}Una ruta cíclica que contiene todas las relaciones binarias uno a uno que son todas obligatorias-obligatorias\\ o al menos una restricción de cardinalidad mínima opcional-opcional siempre es estructuralmente válida.\end{tabular}}                  & \begin{minipage}{.3\textwidth}
        \includegraphics[width=\linewidth]{images/validezER/tabla2/06.png}
    \end{minipage}                               \\ \hline
    \multicolumn{1}{c}{Corolarios de validez para relaciones binarias} & Ejemplo no válido               \\ 
    \hline
    & \\
    \begin{tabular}[c]{@{}l@{}}Las rutas cíclicas que no contienen relaciones opuestas ni relaciones autoajustables\\  son estructuralmente inválidas y se denominan relaciones circulares.\end{tabular}                                                                                                 & \begin{minipage}{.3\textwidth}
        \includegraphics[width=\linewidth]{images/validezER/tabla2/07.png}
    \end{minipage}           \\
    \begin{tabular}[c]{@{}l@{}}La presencia de una relación ``uno a uno obligatorio-obligatorio'' no tiene ningún \\ efecto sobre la validez estructural (o invalidez) de una ruta cíclica que contiene otros tipos de relación.\\  (Este corolario se aplica a todas las reglas anteriores).\end{tabular} & \multicolumn{1}{l}{}           
    \end{tabular}%
    }
    \caption{Resumen de reglas de validez para relaciones binarias con ejemplos}
    \label{tab:reglas-validez-relaciones-binarias}
    \end{table}

En la tabla~\ref{tab:reglas-validez-relaciones-recursivas} se pone un resumen de reglas válidas para relaciones recursivas y en la tabla~\ref{tab:reglas-validez-relaciones-binarias} reglas válidas para relaciones binarias.

