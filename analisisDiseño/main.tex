% Análisis del diseño
% 4.1 Metodología
% 4.1.1 Scrum qué es scrum, para qué sirve y por qué lo elegimos, cómo lo vamos a aplicar
% 4.2 factibilidad si es viable o no
% 4.2.4 conclusion si es factible ténicamente por ser un trabajo escolar y tenemos el personal capacitado
% 4.3 analisis del sistema
% 4.3.1 historias de usuario
% 4.3.2 lista de producto
% 4.3.3 diagrama de actividades, similar al de Omar (el de omar es de funcional) en la presentación
% 4.3.4 Algoritmos para el desarrollo del sistema (Capitulo cinco adjuntar en cap 4 en la parte de análisis)
% 4.4 conclusiones generales

% Capítulo 5: Diseño del sistema
% 5.1 diagrama de clases
% 5.2 diagrama de la base de datos
% 5.3 arquitectura del sistema
% 5.4 diseño de la UI
% 5.5 conclusiones generales

% agregar anexo para ver prototipo funcional, se menciona lo del calendario, cómo acceder, agregar cómo se añade la seguridad inherente por las herramientas que usamos (aspectos del modelado relacional como la normalización).

% algoritmos



En este capítulo se muestra la metodología a usar y el porqué; se ve si es factible o no la propuesta de solución con estudios como la factibilidad ténica, económica y costos de desarrollo; asimismo, en el análisis del sistema se muestan las historias de usuario y la lista del producto a desarrollar; se muestran los principales algoritmos a implementar, como la validación estructural, el mapeo del modelo entidad-relación básico al modelo relacional, la obtención del esquema relacional en sentencias SQL, el mapeo del modelo entidad-relación al GDM, el mapeo del GDM a un modelo lógico orientado a documentos, la obtención del esquema en sentencias de MongoDB y las conclusiones del capítulo.



\section{Metodología}
De acuerdo con la Universidad Católica los Ángeles\cite{universidad_catolica_los_angeles_metodologidesarrollo_2020}, en el campo del desarrollo de \textit{software} hay dos grupos de metodologías: las tradicionales y las ágiles.


Las tradicionales se centran en cumplir con un plan rígido de trabajo establecido en la etapa inicial del proyecto, mientras que las ágiles permiten realizar cambios en los requerimientos conforme avance el mismo.


Dado que cualquier cambio en el proceso de una metodologia tradicional genera la necesidad de una reconstrucción del plan de trabajo (invirtiendo tiempo que se podría usar para desarrollar), surgieron las metodologías ágiles, que permiten realizar cambios en los requerimientos conforme avance el proyecto. 


Tomando en cuenta la experiencia del equipo, esta forma de trabajo permite mostrar avances funcionales en el producto en un periodo de tiempo corto para realizar una evaluación y en caso de ser requerido se sugieran cambios.


Se han propuesto muchos modelos ágiles de proceso y están en uso en toda la industria; entre ellos se encuentran los siguientes:


\begin{itemize}
	\item DAS (Desarrollo Adaptativo de Software): ASD tiene como fundamento la teoría de sistemas adaptativos complejos; por ello, interpreta los proyectos de \textit{software} como sistemas adaptativos complejos compuestos
    por agentes (los interesados), entornos (organizacional o tecnológico) y salidas (el producto desarrollado)\cite{cadavid_revision_2013}.
	\item Scrum: La metodología Scrum para el \textit{desarrollo} ágil de software es un marco de trabajo diseñado para lograr la colaboración eficaz de equipos en proyectos, que emplea un conjunto de reglas y artefactos y define roles que generan la estructura necesaria para su correcto funcionamiento\cite{cadavid_revision_2013}.
	\item MDSD (Método de Desarrollo de Sistemas Dinámicos): DSDM es un marco de trabajo creado para entregar la solución correcta en el momento correcto; utiliza un ciclo de vida iterativo, fragmenta el proyecto en periodos cortos de tiempo y define entregables para cada uno de estos periodos; tiene roles claramente definidos y especifica su trabajo dentro de periodos de tiempo\cite{cadavid_revision_2013}.
	\item Crystal: la filosofía de Crystal define el desarrollo como un juego cooperativo de invención y comunicación cuya meta principal es entregar \textit{software} útil, que funcione y su objetivo secundario es preparar el próximo juego\cite{cadavid_revision_2013}.
\end{itemize}

\subsection{Scrum}

De acuerdo con Ken Schwaber\cite{the_scrum_guide_definitive_2020}, Scrum es un marco de trabajo para la entrega de productos incrementales y de máximo valor productivo.

Un artefacto es un elemento que garantiza la transparencia, es el registro de la información fundamental del proceso Scrum y a continuación se describen sus cuatro artefactos principales:

\begin{itemize}
	\item Lista de producto (\textit{product backlog}):	es el listado de todas las tareas que necesita el proyecto para alcanzar su realización; al iniciar el desarrollo del proyecto, esta lista no se encuentra completa y conforme avanzan los \textit{sprints} se le añaden tareas para solventar las necesidades que van surgiendo gracias a la retroalimentación del cliente.
	\item Lista de pendientes del \textit{sprint} (\textit{sprint backlog}): es la lista de tareas seleccionadas del \textit{product backlog} que se planifica realizar durante el periodo del \textit{sprint} y define a los responsables de cada tarea.
	\item \textit{Sprint}: es el corazón de Scrum, tiene un periodo de tiempo determinado de un mes o incluso menos donde el equipo completa conjuntos de tareas incluidas en el \textit{backlog} para crear un incremento del producto utilizable.
	\item Incremento: es la suma de todos los elementos de la lista de productos completados durante un \textit{sprint} unido con los incrementos de los \textit{sprints} anteriores; al finalizar el \textit{sprint}, el nuevo incremento debe estar en condiciones de ser utilizado.
\end{itemize}

El equipo Scrum (\textit{scrum team}) consiste en los siguientes roles:
\begin{itemize}
	\item El dueño de producto (\textit{product owner}): es la persona responsable de maximizar el valor del producto y el trabajo del equipo de desarrollo; es el único responsable de gestionar la lista de producto y cualquier cambio a esa lista debe ser revisada y aprobada por él.
	\item El equipo de desarrollo (\textit{development team}): son los profesionales que realizan el trabajo para la entrega de un incremento en el producto en cada \textit{sprint}; es un grupo autoorganizado y multifuncional donde cada miembro del equipo tiene habilidades especializadas, pero que la responsabilidad de las tareas completadas, incrementos del producto o retrasos recaen en el equipo como un todo.
	\item  El Scrum master: es la persona responsable de asegurar que Scrum es entendido y adoptado por todos los involucrados en el proyecto, asegurándose de ayudar a las personas externas al equipo a entender qué interacciones son útiles con el equipo de desarrollo.
\end{itemize}


Scrum tiene cuatro eventos principales en  un \textit{sprint}, que sirven para la inspección y adaptación del producto que se describen a continuación:
\begin{itemize}
	\item Planificación del \textit{sprint} (\textit{sprint planning}): es una reunión con el equipo de desarrollo que tiene una duración máxima de 8 horas para el \textit{sprint} de un mes; el Scrum Master es el encargado de que los asistentes entiendan el propósito de dicha reunión.
	\item Scrum diario (\textit{daily scrum}): es una reunión de un máximo de 15 minutos en la cual el equipo expone sus actividades y planifica las tareas de las próximas 24 horas.
	\item Revisión del \textit{sprint} (\textit{sprint review}): al finalizar cada \textit{sprint} se lleva a cabo una reunión para la revisión del incremento del producto y en caso de ser necesario realizar ajustes a la lista de producto.
	\item Retrospectiva del \textit{sprint} (\textit{sprint retrospective}): es cuando el equipo de desarrollo tiene la oportunidad de pensar en mejoras para el próximo \textit{sprint}.
\end{itemize}

por qué Scrum?

cómo vamos a aplicar Scrum?

Como la metodología permite definir un periodo de hasta un mes para cada \textit{sprint}, se ha optado por un periodo de 30 días, contemplándose un total de ocho \textit{sprints}, donde al término de cada uno se tendrá un avance del sistema.
\section{Factibilidad}\label{ref:sec-factibilidad}

De acuerdo con Sommerville\cite{sommerville_software_2011}, un estudio de factibilidad es un estudio breve que responde tres preguntas clave:

\begin{enumerate}
    \item ¿El sistema contribuye a los objetivos generales de la organización? 
    \item ¿Se puede implementar el sistema dentro del cronograma y el presupuesto utilizando las tecnologías actuales? 
    \item ¿Se puede integrar el sistema con otros sistemas que se utilizan? 
\end{enumerate}


Para responder estas preguntas se expone la factibilidad técnica, factibilidad económica y costos de desarrollo; en las conclusiones del capítulo están las respuestas a las preguntas anteriores.

\subsection{Factibilidad técnica}\label{ref:factibilidad-tecnica}

De acuerdo con Pressman\cite{pressman_software_2005}, este estudio determina si el equipo de desarrollo cuenta con los recursos técnicos necesarios para la realización del sistema propuesto; esto se realiza considerando la disponibilidad de los recursos tanto de \textit{hardware}, \textit{software} y recurso humano.

\subsubsection*{Sistema operativo}

De acuerdo con Stallings\cite{stallings_operating_2012}, un sistema operativo es el \textit{software} principal o conjunto de programas de un sistema informático que gestiona los recursos de \textit{hardware} y provee servicios a los programas de aplicación de \textit{software}, ejecutándose en modo privilegiado respecto de los restantes.


Este es un elemento importante, ya que debe cumplir con las características de estabilidad, velocidad, seguridad y escalabilidad para soportar la instalación del sistema.


A continuación se presentan diferentes sistemas operativos que cumplen con las características mencionadas y que son suficientes para albergar el sistema:

\begin{itemize}
    \item Windows: es un grupo de varias familias de sistemas operativos gráficos patentados, son desarrollados y comercializados por Microsoft; cada familia atiende a un determinado sector de la industria informática\cite{wilson_about_2015}.
    \item GNU/Linux: es una familia de sistemas operativos tipo Unix de código abierto basados en el núcleo de Linux creado por Linus Torvalds\cite{love_linux_2010}.
\end{itemize}

El sistema operativo elegido para el desarrollo de la propuesta de solución es GNU/Linux porque es de libre acceso, es gratuito y los integrantes del equipo tienen experencia con este sistema.


\subsubsection*{Lenguaje de desarrollo}

De acuerdo con Aaby\cite{aaby_introduction_1996}, un lenguaje de programación es un lenguaje formal (es decir, un lenguaje con reglas gramaticales definidas) que le proporciona a una persona, en este caso el programador, la capacidad de escribir una serie de instrucciones o secuencias de órdenes en forma de algoritmos con el fin de controlar el comportamiento físico o lógico de una computadora, de manera que es posible obtener diversas clases de datos o ejecutar determinadas tareas.


Se valora que el lenguaje de programación para el desarrollo de la propuesta de solución debe tener soporte para conexión a base de datos, sea posible usarlo para el desarrollo, sea vigente (que esté en continua mejora), sea fácil de administrar y se integre con algún \textit{framework} web.


A continuación se presenta una lista de lenguajes de desarrollo que cumplen dichas características:

\begin{itemize}
    \item Java: es un lenguaje de programación de propósito general que está basado en clases, orientado a objetos y diseñado para tener la menor cantidad posible de dependencias de implementación; su objetivo es permitir seguir el  \textit{write once, run anywhere}\cite{joy_java_2000}.
    \item Python: es un lenguaje de programación interpretado, de alto nivel y de propósito general; la filosofía de diseño de Python enfatiza la legibilidad del código con su uso notable de espacios en blanco significativos; Sus construcciones de lenguaje y su enfoque orientado a objetos tienen como objetivo ayudar a los programadores a escribir código claro y lógico para proyectos de pequeña y gran escala\cite{van_rossum_python_2007}.
    \item C\#: es un lenguaje de programación multiparadigma de propósito general como imperativo, declarativo, funcional, genérico, orientado a objetos y orientado a componentes\cite{hejlsberg_c_2003}.
    \item JavaScript: es un lenguaje de programación interpretado, de alto nivel y de propósito general; se escribe dinámicamente; es compatible con múltiples paradigmas de programación, incluida la programación orientada a objetos\cite{noauthor_javascript_nodate}.
\end{itemize}

Los lenguajes de programación Python y JavaScript se han elegido para desarrollar la propuesta de solución; para más detalles revise la sección~\ref{ref:conclusiones-cap3}

\subsubsection*{Sistema gestor de base de datos}

De acuerdo con Connolly\cite{connolly_database_2005}, es un sistema de \textit{software} que permite a los usuarios definir, crear, mantener y controlar el acceso a la base de datos.


Este es un factor muy importante, ya que determinará cómo se almacenará la información del sistema, por lo tanto debe ser escalable, seguro, contar con soporte para grandes cantidades de información y soporte para conexión con distintos lenguajes de programación.


A continuación se presenta una lista de sistemas gestores de bases de datos que cumplen dichas características:

\begin{itemize}
    \item MySQL: es un sistema de gestión de bases de datos relacionales de código abierto\cite{dubois_mysql_1999}.
    \item MongoDB: es un gestor de bases de datos orientado a documentos y utiliza documentos similares a JSON con esquemas opcionales\cite{banker_mongodb_2011}.
\end{itemize}

El gestor de base de datos que se ha elegido para desarrollar la propuesta de solución es MongoDB; para más detalles revise la sección~\ref{ref:databases}.


La tabla~\ref{tab:hw_devices} muestra las características de las computadoras con los que el equipo de desarrollo dispone; la primera columna hace referencia al número de computadora del que se dispone; la siguiente hace referencia a qué elementos contiene, la tercera columna sus especificaciones y la cuarta columna el costo de cada computadora.

\begin{table}[H]
    \centering
    \begin{tabular}{|c|c|c|c|}
        \hline
        Equipos & Elementos & Especificaciones & Costo \\ \hline
        \multirow{3}{*}{Laptop 1} & Memoria RAM & 8 GB & \\
        & Almacenamiento & 500 GB HDD & 22 500.00 mxn  \\
        & Procesador & Intel Core i5 6ta gen. & \\ \hline
        \multirow{3}{*}{Laptop 2} & Memoria RAM & 8 GB & \\
        & Almacenamiento & 256 GB SSD & 32 000.00 mxn \\
        & Procesador & Intel Core i5 8va gen. & \\ \hline
    \end{tabular}
    \caption{Computadoras con las que se cuenta}
    \label{tab:hw_devices}
\end{table}


Con los datos que están en la tabla~\ref{tab:hw_devices}, se concluye que la tecnología para el desarrollo del sistema existe y se cuenta con los recursos de \textit{hardware} suficientes para iniciar con su implementación.
\subsection{Factibilidad económica}

De acuerdo con Pressman\cite{pressman_software_2005}, el punto de función es una ``unidad de medida'' para expresar la cantidad de funcionalidad comercial que un sistema de información (como producto) proporciona a un usuario; los puntos de función se utilizan para calcular una medición de tamaño funcional (FSM) de \textit{software}.

Como señala Pressman en su libro, se utiliza una métrica por puntos de función para realizar una estimación del costo total del proyecto, incluyendo los salarios de los desarrolladores que harán la implementación del sistema, así como los gastos por pagos de servicios que sean necesarios como se muestra en la tabla \ref{tab:function_point_metrics}. 

\subsubsection*{Métricas orientadas a la función}


Para este proyecto, se considera que todas las funciones identificadas son de complejidad media con excepción de las entradas que tienen la complejidad más alta del sistema.


\begin{table}[H]
	\centering
	\begin{tabular}{|l|l|l|l|l|l|}
	\hline
	\multirow{2}{*}{Parámetro} & \multirow{2}{*}{Cuenta} & \multicolumn{3}{|l|}{Factores de ponderación} & \multirow{2}{*}{Total} \\ \cline{3-5}
														 &                         & baja       	& media       & alta		       &                        \\ \hline
	Entradas                   & 6                       & 3            & 4           & 6              & 36                     \\ \hline
	Salidas                     & 5                       & 4            & 5           & 7              & 25                     \\ \hline
	Tablas                     & 1                       & 3            & 4           & 6              & 4                      \\ \hline
	Interfaces                 & 4                       & 7            & 10          & 15             & 40                     \\ \hline
	Consultas                  & 4                       & 5            & 7           & 10             & 28                     \\ \hline
	Conteo total               &                         &              &             &		             & 133                    \\ \hline
	\end{tabular}
	\caption{Cálculo de las métricas por puntos de función}
	\label{tab:function_point_metrics}
	\end{table}


	\textbf{Fi (i = 1..14)} son factores de ajuste de valor basados en las respuetas de las preguntas de la tabla \ref{tab:questions_adjusment}. Los valores pueden ir de 0 (no importante o aplicable) a 5 (absolutamente esencial).

	\begin{table}
		\begin{tabular}{|p{9cm}|c|}
		\hline
		Pregunta                                                                                                                 & Ponderación \\ \hline
		¿Requiere el sistema métodos de seguridad y recuperación fiables?                                                       & 3           \\ \hline
		¿Se requiere comunicación especializada?                                                                              & 5           \\ \hline
		¿Existen funciones de procesamiento distribuido?                                                                         & 2           \\ \hline
		¿Es crítico el rendimiento?                                                                                              & 4           \\ \hline
		¿Se ejecutará el sistema en un entorno operativo existente y fuertemente utilizado?                                      & 4           \\ \hline
		¿Requiere el sistema una entrada de datos interactiva?                                                                    & 5           \\ \hline
		¿Requiere la entrada de datos interactiva que las transacciones de entrada se lleven a cabo sobre múltiples operaciones? & 5           \\ \hline
		¿Se actualizan los archivos maestros de forma interactiva?                                                               & 3           \\ \hline
		¿Son complejas las entradas, salidas, archivos o consultas?                                                               & 4           \\ \hline
		¿Es complejo el procesamiento interno?                                                                                   & 4           \\ \hline
		¿Se ha diseñado el código para ser reutilizable?                                                                         & 4           \\ \hline
		¿Están incluidas en el diseño la instalación y conversión?                                                               & 3           \\ \hline
		¿Se ha diseñado el sistema para soportar múltiples instalaciones en diferentes organizaciones?                           & 4           \\ \hline
		¿Se ha diseñado el sistema para facilitar los cambios y para ser fácilmente utilizable?                                  & 4           \\ \hline
		\centering $\sum Fi=$                                                                                                  	 & 54          \\ \hline
		\end{tabular}
		\caption{Factores de ajuste}
		\label{tab:questions_adjusment}
		\end{table}


\subsubsection*{Puntos de función}

La fórmula para obtener los puntos de función con los factores de ajuste es la siguiente:

\begin{equation} \label{eq:cap4-00.01}
	\mathrm{PF} = \mathrm{conteo\ total} *  (0.65 + (0.01 * \sum F_i)) 
\end{equation}
	
De \eqref{eq:cap4-00.01} se deben sustituir los valores del conteo total y los factores de ajuste:

\begin{equation} \label{eq:cap4-00.02}
	\mathrm{PF} =  133 *  (0.65 + (0.01 * 54))
\end{equation}
\begin{equation} \label{eq:cap4-00.03}
	\mathrm{PF} = 158.27 \approx 159
\end{equation}
	
		
De lo anterior, aproximadamente se obtienen \textbf{159} puntos de función; una vez obtenidos utilizando la llamada ``Ball-Park'' o ``Estimación Indicativa'', que es la técnica de macro-estimación que se utiliza en situaciones de falta de información sobre el proyecto.


De acuerdo con Carper\cite{abran_applied_2006}, la siguiente ecuación determina el esfuerzo de desarrollo de un proyecto:


\begin{equation} \label{eq:cap4-01}
	\mathrm{Esfuerzo} = (\frac{\mathrm{PF}}{150})*\mathrm{PF}^{0.4} 
\end{equation}
donde \eqref{eq:cap4-01} $\mathrm{PF}$ son puntos de función; al sustituir los valores en \eqref{eq:cap4-01}:

\begin{equation} \label{eq:cap4-02}
	\mathrm{Esfuerzo} = (\frac{159}{150})*159^{0.4}
\end{equation}
\begin{equation} \label{eq:cap4-03}
	\mathrm{Esfuerzo} = 8.05 \; \mathrm{meses}
\end{equation}
	



Los 8.05 meses de \eqref{eq:cap4-03}; considerando un total de 40 horas a la semana de trabajo y 4.34 semanas por mes, el total de horas para el desarrollo y conclusión del proyecto se obtiene de esta manera:

\begin{equation} \label{eq:cap4-04}
	\mathrm{Tiempo\; de\; desarrollo} =  40 * 4.34 * 8.05 
\end{equation}

\begin{equation} \label{eq:cap4-05}
	\mathrm{Tiempo\; de\; desarrollo} =  1397.48 \approx 1398\;  \mathrm{horas}
\end{equation}

	



Por ejemplo, una sola persona trabajando en el desarrollo del proyecto debería invertir 1398 horas con una jornada de 8 horas diarias de lunes a viernes hasta su finalización, por lo que si un equipo de desarrollo es de 2 personas con un horario de lunes a viernes de 4 horas diarias, el proyecto concluiría en 8.05 meses.


\subsection{Costos de desarrollo}

De acuerdo con \textit{Software Guru}\cite{pedro_galvan_estudio_2020}, en una publicación que recopila los datos de salarios en el área de desarrollo de \textit{software} para febrero de 2020, un desarrollador con 0 a 2 años de experiencia, como es el caso de un estudiante, tiene en un salario de \$ 15 000 mensuales en una jornada completa, considerando que este proyecto contempla jornadas de medio tiempo (4 horas) de lunes a viernes, se reduce la cifra antes mencionada. Teniendo en cuenta estos datos y un periodo de 9 meses, que es el tiempo aproximado de duración del proyecto, el costo total por salarioss para el equipo de desarrollo está desglosado en la tabla \ref{tab:devs_salary}.

\begin{table}[H]
    \centering
    \begin{tabular}{|c|c|c|c|}
    \hline
        Concepto & Costo Aprox. Semanal & Costo Aprox. Mensual & Monto total \\ \hline
        Salario & 1850 & 7500 & 67500 \\ \hline
    \end{tabular}
    \caption{Costos del personal}
    \label{tab:devs_salary}
\end{table}


Considerado a 2 personas en el equipo de desarrollo y un periodo de 9 meses(se agrego un mes mas para el caso de estudio) utilizando los salarios de la tabla \ref{tab:devs_salary} tenemos que los gastos totales se obtienen con la siguiente fórmula:

\begin{equation} \label{eq:cap4-07}
	\mathrm{salarios} =  \mathrm{No\; de\; integrantes} * \mathrm{salario/mes} * \mathrm{tiempo\; de\; desarrollo}  {pesos mexicanos}
\end{equation}

sustituyendo los calores en \eqref{eq:cap4-07}

\begin{equation} \label{eq:cap4-08}
	\mathrm{salarios} =  2 * 7500 * 9  {pesos mexicanos}
\end{equation}


Esto da como resultado un total final de \$ 135 000 por los salarios de los 2 integrantes del equipo de desarrollo. Se tomaron en cuenta 9 meses para todos los gastos, un mes extra a lo obtenido en estimación para utilizarse en el caso de estudio del sistema una vez concluido.

Los gastos por pagos de licencia de \textit{software} quedan excluidos, ya que las tecnologías seleccionadas son libres o gratuitas, lo cual no supone un costo para su uso. De igual manera, esto se encuentra simplificado en la tabla \ref{tab:sw_licences}.

\begin{table}
    \centering
    \begin{tabular}{|c|c|c|c|}
    \hline
        Software & Licencia & Costo \\ \hline
        Visual Studio Code  & MIT & 0  \\ \hline
        Gunicorn(flask Server) & MIT & 0 \\ \hline
        MongoDB Atlas & Apache v2 & 0 \\ \hline
    \end{tabular}
    \caption{Costos por licencias de software}
    \label{tab:sw_licences}
\end{table}

Otros gastos necesarios son los pagos por servicios requeridos listados en la tabla \ref{tab:services_costs}.

\begin{table}
    \centering
    \begin{tabular}{|c|c|c|c|}
    \hline
        Concepto & Costo Mensual & Monto total \\ \hline
        Luz & 250 & 2250  \\ \hline
        Internet & 349 & 3141 \\ \hline
        Heroku hosting & 0 (free plan) & 0 \\ \hline
        Netlify hosting & 0 (free plan) & 0 \\ \hline
    \end{tabular}
    \caption{Costos por servicios}
    \label{tab:services_costs}
\end{table}

Habiendo realizado la suma de todas las cantidades antes mencionadas, el total final se obtine de la siguiente manera:

\begin{equation} \label{eq:cap4-09}
    \mathrm{servicio_totales} = \mathrm{servicios_por_mes} * \mathrm{tiempo_de_desarrollo} {pesos mexicanos}
    \mathrm{servicio_totales} = 5391 * 9 {pesos mexicanos}
    \mathrm{servicio_totales} = 48 519.00 {pesos mexicanos}
\end{equation}

\begin{equation} \label{eq:cap4-10}
    \mathrm{Gastos\; totales} = \mathrm{salarios} + \mathrm{servicios\; totales} + \mathrm{costos_de_equipo_de_compumto}
    \mathrm{Gastos\; totales} = 135 000 + 48 519 + 54 500
    \mathrm{Gastos\; totales} = 238 019 {pesos mexicanos}
\end{equation}

\begin{center}

	salarios = 135 000.00
	servicios = (2250 + 3141) * 9 = 48 519.00

	Gastos totales = 135 000.00 + 48 519.00 = 183 519.00 pesos mexicanos.

\end{center}


%
%\subsection{Factibilidad operativa}
\subsection{Conclusiones}


La factibilidad operativa permite predecir si es posible poner en marcha el sistema propuesto, aprovechando todos los beneficios que se ofrecen a todos los usuarios involucrados en ello. La herramienta va dirigida a los estudiantes que se encuentren en un primer acercamiento a los modelos de bases de datos entidad-relación o relacional desde un enfoque conceptual, buscando principalmente mostrarles una aproximación a los modelos no relacionales de bases de datos. El sistema propuesto cuenta con una interfaz intuitiva para que el usuario final, los estudiantes, puedan visualizar, crear y editar un diagrama ER y las opciones que esta les brinde de manera comprensible.


Teniendo en cuenta los motivos anteriormente explicados, el sistema propuesto tiene una alta probabilidad de aceptación por parte de los usuarios finales al encontrarse en un entorno en el que se trabaja con \textit{software} continuamente, además del beneficio que aporta al plan de estudios actual al ofrecer una forma práctica de ver aplicado los conceptos adquiridos en el curso de base de datos, el cual solo contempla un alcance hasta la normalización de bases de datos relacionales y tener una introducción a los modelos no relacionales (noSQL). Un estudiante que ha cursado dicha asignatura se dará cuenta que el tiempo disponible durante el curso es limitado por la cantidad de módulos que pretende cubrir y en muchas ocasiones los docentes deben prescindir de ciertos temas para completar el temario.

Con la implantación de la aplicación web que se está proponiendo, los estudiantes que cursen la asignatura de de base de datos tendrán la oportunidad de conocer una opción más en cuanto a tecnologías de almacenamiento de datos para implementar en sus propios sistemas. De igual manera, puede impulsarlos a generar propuestas para la apertura de una asignatura optativa si el interes por estos modelos de datos resulta interesante para ellos.

Teniendo en cuenta los puntos mencionados anteriormente, se concluye que el sistema propuesto tendrá un uso en la institución y un potencial beneficio para los estudiantes y los involucrados en ello.

\begin{enumerate}
    \item ¿El sistema contribuye a los objetivos generales de la organización? la respuesta es si, ya que la organización centra su misión en formar profesionales en ingeniería, tecnologías y ciencias de la computación, para lograrlo de mantenerse a actualizado en las tecnologías emergentes y ofrecer a sus estudiantes un mayor abanico de opciones para su desarrollo profesional.
    \item ¿Se puede implementar el sistema dentro del cronograma y el presupuesto utilizando las tecnologías actuales? claro que es posible como se muestra en la factibilidad tecnica se cuenta con las tecnologías para el desarrollo del producto y el esfuerzo es ajjstado para el equipo de desarrollo pero queda en los limites del tiempo establecido en el cronograma, ademas de ser desarrollado con una metodología agíl lo que ofrece productos funcionales por cada iteración.
    \item ¿Se puede integrar el sistema con otros sistemas que se utilizan? esto es perfectamente viable, el sistema puede ser adaptado a otros sistemas por la facilidad de estar disponible en la web tiene una gran opoprtinidad para comunicarse con otros sistemas disponibles en la organización, por ejemplo puede integrarse directamente en la asignatura de base de datos como una opción mas para el modelado de diagramas ER con la ventaja de tener el acercamiento a los modelos no relacionales de manera practica.
\end{enumerate}
\section{Análisis del sistema}
De acuerdo con Pressman\cite{pressman_software_2005}, las condiciones del mercado cambian con rapidez, clientes y usuarios finales necesitan cambios constantes por nuevas amenazas competitivas; por ello los profesionales deben enfocar la ingeniería de \textit{software} en forma que les permita mantenerse ágiles para definir procesos maniobrables, adaptativos y esbeltos que satisfagan las necesidades de los negocios modernos.


Una filosofía ágil para la ingeniería de \textit{software} pone el énfasis en cuatro aspectos clave: la importancia de los equipos con organización propia que tienen el control sobre el trabajo que realizan, la comunicación y colaboración entre los miembros del equipo, profesionales y sus clientes, el reconocimiento de que el cambio representa una oportunidad y la insistencia en la entrega rápida de \textit{software} que satisfaga al consumidor.


\subsection{Historias de usuario}\label{sec:historias-usuario}

De acuerdo con Scrum México\cite{scrum_mexico_scrum_2020}, las historias de usuario conforman la técnica por la que el usuario especifica de manera general los requerimientos que el sistema debe cumplir.


Normalmente estas redacciones se llevan a cabo en tarjetas de papel donde se describen brevemente las funciones que el producto final debe poseer, ya sean requisitos funcionales o no.


El tratamiento de las historias de usuario es flexible y dinámico, cada una de ellas es lo suficientemente detallada y delimitada para que el equipo de desarrollo implemente durante la duración del \textit{sprint}.


Es habitual que se siga una plantilla para estas tarjetas, como la que se expone a continuación:

\begin{itemize}
	\item Como \textbf{<Usuario>}
	\item Quiero \textbf{<algún objetivo>}
	\item Para \textbf{<motivo>}
\end{itemize}


Una de sus grandes ventajas, dado el caso de que un usuario no sea lo suficientemente detallista con la historia, es que esta se puede partir en historias más pequeñas antes de que el equipo empiece a trabajar en ella.


Este es un ejemplo de historia de usuario para el desarrollo:

\begin{itemize}
	\item Como usuario,
	\item quiero ingresar al sistema con mi correo y contraseña
	\item para tener acceso a sus funciones.
\end{itemize}


Otra forma de darle detalle a las historias de usuario es mediante el añadido de un criterio de aceptación; un criterio de aceptación es una prueba que será cierta cuando el equipo de desarrollo complete la descripción de la tarjeta.


A continuación se listan las principales historias de usuario que se consideraron para el desarrollo de la propuesta de solución; tenga en cuenta que algunas de ellas tienen criterios de aceptación, pero en otras no se consideró necesarias porque son explícitamente claras.


\noindent\rule{\textwidth}{1pt}
\begin{itemize}
	\item N.° 1
	\item Como usuario,
	\item quiero ingresar al sistema con mi correo y contraseña,
	\item para tener acceso a sus funciones.
	\item Criterios de aceptación:
	\begin{itemize}
		\item El usuario recibirá un correo electrónico de confirmación de su alta en el sistema con el correo y contraseña que ingresó para tenerlos de respaldo.
	\end{itemize}
\end{itemize}
\noindent\rule{\textwidth}{1pt}
\begin{itemize}
	\item N.° 2
	\item Como usuario,
	\item quiero recuperar mi contraseña en caso de olvidarla,
	\item para no perder el trabajo realizado en el sistema.
	\item Criterios de aceptación:
	\begin{itemize}
		\item El usuario podrá ingresar una nueva contraseña siempre y cuando recuerde el correo electrónico con el que se dio de alta en el sistema.
		\item Al ingresar una nueva contraseña, recibirá un correo de confirmación del cambio de contraseña y sus datos permanecerán intactos.
	\end{itemize}
\end{itemize}
\noindent\rule{\textwidth}{1pt}
\begin{itemize}
	\item N.° 3
	\item Como usuario del sistema, quiero darme de alta con una contraseña fácil de recordar,
	\item pero que esté segura en la base de datos,
	\item para no tener comprometidos los diagramas que genere en el sistema.

	\item Criterios de aceptación:
	\begin{itemize}
		\item Asegurarse que el usuario ingrese una contraseña de al menos 8 caracteres.
		\item Se le solicitará al usuario que ingrese 2 veces la misma contraseña para asegurarse que le es fácil recordarla y que efectivamente es la misma.
		\item Antes de guardar la contraseña, esta deberá pasar por un método que la haga ilegible para el usuario (algún algoritmo de digestión o cifrado).
	\end{itemize}
\end{itemize}
\noindent\rule{\textwidth}{1pt}
\begin{itemize}
	\item N.° 4
	\item Como usuario quiero crear un diagrama ER arrastrando y soltando elementos de una ``paleta'',
	\item para hacerlo de manera más fácil e intuitiva.
	\item Criterios de aceptación:
	\begin{itemize}
		\item El usuario podrá empezar un nuevo diagrama al seleccionar la opción de diagramador ER.
		\item Tendrá a su disposición una paleta con los elementos permitidos en un diagrama ER básico.
		\item Podrá arrastrar y soltar los elementos de la paleta a un área delimitada para empezar con el diseño de su diagrama.
	\end{itemize}
\end{itemize}
\noindent\rule{\textwidth}{1pt}
\begin{itemize}
	\item N.° 5
	\item Como usuario quiero guardar mi último trabajo realizado en el diagramador ER,
	\item para poder consultarlo en otro momento.
	\item Criterios de aceptación:
	\begin{itemize}
		\item Dispondrá de un botón para poder guardar en la base de datos el diagrama que esté creando o editando.
		\item Antes de almacenar el diagrama en el \textit{canvas} o zona de diagramado, se le mostrará un mensaje de confirmación para guardar su diagrama actual.
	\end{itemize}
\end{itemize}
\noindent\rule{\textwidth}{1pt}
\begin{itemize}
	\item N.° 6
	\item Como usuario me gustaría poder ver el último trabajo que realice
	\item cuando seleccione la opción ``Entidad-Relación'',
	\item para poder modificar el diseño.
\end{itemize}
\noindent\rule{\textwidth}{1pt}
\begin{itemize}
	\item N.° 7
	\item Como usuario quiero tener la opción de cargar un diagrama a partir de un archivo,
	\item para hacer modificación de dicho diagrama y guardarlo de ser necesario.
	\item Criterios de aceptación:
	\begin{itemize}
		\item El usuario tendra un botón ``cargar'' en el menú del diagramador ER para poder importar un archivo con extensión .json.
		\item Al importar el archivo, este pasará por un proceso de validación para asegurarse que es un archivo .json válido.
		\item Durante el proceso de validación, se verificará que el contenido del archivo es un diagrama compatible con la estructura de los generados por el diagramador ER.
		\item Al contener información compatible, se mostrará en la zona de diagramado el contenido del archivo.
	\end{itemize}
\end{itemize}
\noindent\rule{\textwidth}{1pt}
\begin{itemize}
	\item N.° 8
	\item Como usuario quiero descargar el diagrama que esté visible en la página web,
	\item para poder distribuirlo como yo desee.
	\item Criterios de aceptación:
	\begin{itemize}
		\item El usuario dispondrá de un botón ``Descargar'' en el diagramador ER para obtener un archivo con el contenido del diagrama visible en la zona de diagramado.
		\item El archivo generado será de extensión .json con la información necesaria para que el diagramador lo cargue en otro momento.
	\end{itemize}
\end{itemize}
\noindent\rule{\textwidth}{1pt}
\begin{itemize}
	\item N.° 9
	\item Como usuario quiero tener una forma de validar mi diagrama ER,
	\item porque es importante saber si el diagrama que estoy creando es un diagrama válido del modelo ER.
	\item Criterios de aceptación:
	\begin{itemize}
		\item El usuario tendrá disponible un botón que al darle clic iniciará un proceso de validación del diagrama actual en la zona de diagramado.
		\item Al término del proceso de validación, se le mostrará un mensaje al usuario indicando si el diagrama cumple o no las reglas del modelo ER.
	\end{itemize}
\end{itemize}
\noindent\rule{\textwidth}{1pt}
\begin{itemize}
	\item N.° 10
	\item Como usuario, en caso de tener un diagrama ER válido,
	\item me gustaría poder tranformar el diagrama ER en su versión del modelo relacional,
	\item para poder ver el equivalente del diagrama ER en el modelo Relacional.
	\item Criterios de aceptación:
	\begin{itemize}
		\item El usuario dispondrá de la opción de tranformar al modelo relacional solamente después de haber validado que el diagrama ER cumple con las reglas.
		\item Posterior a la validación, se le mostrará al usuario un mensaje de confirmación y un botón para disparar el proceso de transformación a su equivalente relacional.
		\item Al terminar el proceso de transformación equivalente, se le redirijira al menú ``Relacional'' donde podrá visualizar el equivalente al modelo relacional.
	\end{itemize}
\end{itemize}
\noindent\rule{\textwidth}{1pt}
\begin{itemize}
	\item N.° 11
	\item Como usuario, después de observar el diagrama relacional,
	\item quiero obtener las sentencias SQL equivalentes,
	\item para poder crear el esquema de base de datos relacional en un DBMS.
	\item Criterios de aceptación:
	\begin{itemize}
		\item Las sentencias SQL solo podrán ser descargadas en el menú ``Relacional'' a un archivo con extension .sql dando clic a un botón con la leyenda ``Descargar SQL''.
		\item Solo se obtendrán las sentencias SQL de un diagrama ER creado y/o validado por el sistema.
	\end{itemize}
\end{itemize}
%\noindent\rule{\textwidth}{1pt}
%\begin{itemize}
	%\item N.° 12
	%\item Como usuario, una vez observado el equivalente relacional del diagrama ER,
	%\item quiero iniciar el proceso de transformación al modelo no relacional
	%\item para poder observar el cambio entre modelos.
	%\item Criterios de aceptación:
	%\begin{itemize}
		%\item al dar clic al botón “Transformar a NoSQl”, el usuario iniciará el proceso para obtener el equivalente del modelo relacional al modelo NR.
		%\item al término del proceso de transformación, se le redirigirá al menú ``No Relacional'' donde observará el modelo NoSQL equivalente a su diagrama ER.
%	\end{itemize}
%\end{itemize}
\noindent\rule{\textwidth}{1pt}
\begin{itemize}
	\item N.° 12
	\item Como usuario quiero transformar mi diagrama ER en su equivalente modelo conceptual NoSQL,
	\item para poder observar el cambio entre modelos.
	\item Criterios de aceptación:
	\begin{itemize}
		\item El usuario dispondrá de la opción de tranformar al modelo NoSQL solamente después de haber validado que el diagrama ER cumple con las reglas.
		\item Posterior a la validación, se le mostrará al usuario un mensaje de confirmación y un botón para disparar el proceso de transformación al modelo conceptual NoSQL.
		\item Una vez validado el diagrama ER e iniciado el proceso para la transformación al modelo conceptual NoSQL, se le indicará al usuario que el proceso tardará un tiempo.
		\item Al término del proceso de transformación, se le redirigirá al menú ``No Relacional'' donde observará el modelo conceptual NoSQL equivalente a su diagrama ER.
	\end{itemize}
\end{itemize}
\noindent\rule{\textwidth}{1pt}
\begin{itemize}
	\item N.° 13
	\item Como usuario quiero obtener el \textit{script} desde el modelo conceptual NoSQL,
	\item para poder generar la base de datos en un gestor de base de datos NoSQL orientado a documentos.
	\item Criterios de aceptación:
	\begin{itemize}
		\item El usuario dispondrá de la opción de obtener el \textit{script} solamente después de haber validado que el diagrama ER cumple con las reglas.
		\item Posterior a la validación, se le mostrará al usuario un mensaje de confirmación y un botón para disparar el proceso de generación de \textit{scripts} para el gestor de base de datos orientado a documentos.
		\item Una vez empezado el proceso de generación de \textit{scripts}, se le indicará al usuario que el proceso tardará un tiempo.
		\item Al término del proceso de transformación, se le redirigirá al menú ``No Relacional'' donde observará los \textit{scripts} NoSQL.
	\end{itemize}
\end{itemize}

\noindent\rule{\textwidth}{1pt}
\begin{itemize}
	\item N.° 14
	\item Como usuario me gustaría tener un reporte técnico y
	\item quiero que la redacción sea legible y referenciada,
	\item para compartirlo en el futuro con equipos de desarrollo y ver la posibilidad de agregar nuevas funciones al sistema.
\end{itemize}
\noindent\rule{\textwidth}{1pt}




Teniendo en cuenta que se está trabajando con una metodología ágil, estas historias de usuario pueden aumentar o en su defecto dividirse en historias más pequeñas dependiendo de los criterios del equipo de desarrollo durante el proceso de la implementación de cada historia.




\subsection{Lista de producto}

De acuerdo a Trigas Gallego\cite{manuel_trigas_gallego_metodologiscrum_2020}, la lista de producto es una lista ordenada de todo lo que sería necesario en el producto y es la fuente de requisitos para cualquier cambio a realizarse en el mismo; enumera las características, funcionalidades, requisitos, mejoras y correcciones que constituyen cambios a realizarse en el producto para entregas futuras.

La tabla~\ref{tab:lista-producto} muestra la lista de producto para el proyecto; muestra el número de historia, la tarea a realizar, así como su encargado.


\begin{longtable}{ p{2cm} | p{10cm} | p{2cm} }
	\hline
	N.° de Historia de Usuario & Requerimiento/Tarea & Responsable \\[0.5cm]
	\hline
	\hline

	\endfirsthead

	\multicolumn{3}{c}{Continuación de tabla de lista de producto }\\
	\hline
	\hline
	\endhead

	\hline
	\hline
	\caption{Lista de producto}
	\endlastfoot

	% template row table
	% \centering & & & \\[0.5cm]
	% \hline

	\centering 14 & Investigación de bases de datos relacionales. & Eduardo/Omar \\[0.5cm]
	\hline
	\centering 14 & Redacción y selección de las tecnologías a utilizar para el desarrollo de la plataforma.  & Eduardo \\[0.5cm]
	\hline
	\centering 14 & Investigación de bases de datos relacionales.  & Eduardo/Omar \\[0.5cm]
	\hline
	\centering 14 & Redacción de bases de datos relacionales en el documento técnico.  & Eduardo \\[0.5cm]
	\hline
	\centering 14 & Investigación de bases de datos no relacionales.  & Eduardo/Omar \\[0.5cm]
	\hline
	\centering 14 & Redacción de bases de datos no relacionales en el documento técnico.  & Eduardo \\[0.5cm]
	\hline
	\centering 14 & Investigación y selección del modelo de base de datos no relacional a utilizar junto a las tecnologías a utilizar.  & Eduardo/Omar \\[0.5cm]
	\hline
	\centering 14 & Análisis y diseño de la aquitectura web.  & Eduardo/Omar \\[0.5cm]
	\hline
	\centering 1 & Desarrollo de la estructura básica del \textit{backend}.  & Omar \\[0.5cm]
	\hline
	\centering 1 & Desarrollo de la estructura básica del \textit{frontend}.  & Eduardo \\[0.5cm]
	\hline
	\centering 1 & Agregar servicio \textit{backend} para registrar un usuario. & Omar \\[0.5cm]
	\hline
	\centering 1 & Agregar formulario para captura de datos de registro de un usuario en el \textit{frontend}. & Eduardo \\[0.5cm]
	\hline
	\centering 2 & Agregar servicio \textit{backend} para recuperar contraseña del usuario. & Omar \\[0.5cm]
	\hline
	\centering 2 & Agregar servicio \textit{backend} para envío de correo al usuario registrado y de recuperación de contraseña. & Omar \\[0.5cm]
	\hline
	\centering 2 & Agregar vista con formulario para recuperación de contraseña del usuario en el \textit{frontend}. & Eduardo \\[0.5cm]
	\hline
	\centering 2 & Integración de los servicios de registro y recuperación de contraseña en el \textit{frontend}. & Eduardo \\[0.5cm]
	\hline
	\centering 3 & Agregar servicio \textit{backend} para hacer ilegible la contraseña del usuario en la base de datos. & Omar \\[0.5cm]
	\hline
	\centering 4 & Planteamiento de escenarios de los esquemas entidad-relación.  & Eduardo \\[0.5cm]
	\hline
	\centering 4 & Agregar a la interfaz gráfica de la aplicación web el menú ``Entidad-Relación''. & Eduardo \\[0.5cm]
	\hline
	\centering 4 & Agregar íconos de los elementos basicos de un diagrama ER en el diagramador. & Eduardo \\[0.5cm]
	\hline
	\centering 5 & Agregar servicio \textit{backend} para guardar un diagrama ER en formato JSON en la base de datos e integrarlo al \textit{frontend}. & Omar \\[0.5cm]
	\hline
	\centering 5 & Agregar servicio \textit{backend} para recuperar el diagrama guardado del usuario de la base de datos y regresarlo en formato JSON.  & Omar \\[0.5cm]
	\hline
	\centering 6 & Recuperar el último diagrama del usuario del \textit{backend} y monstrarlo en el \textit{frontend}. & Eduardo \\[0.5cm]
	\hline
	\centering 6 & Manejar el estado de la intefaz web para no perder el diagrama ER que está editando el usuario. & Eduardo \\[0.5cm]
	\hline
	\centering 6 & Definición de las reglas del modelo entidad-relación. & Eduardo \\[0.5cm]
	\hline
	\centering 4 & Implementar la edición de diagramas ER en el \textit{frontend}.  & Eduardo \\[0.5cm]
	\hline
	\centering 7 & Habilitar la carga de un archivo en la aplicación web.  & Omar \\[0.5cm]
	\hline
	\centering 7 & Agregar la función para validar el contenido del archivo .json y pintarlo en la zona de diagramado. & Eduardo \\[0.5cm]
	\hline
	\centering 8 & Agregar la descarga del diagrama visible en la zona de diagramado a un archivo .json. & Eduardo \\[0.5cm]
	\hline
	\centering 9 & Agregar botón de validar al \textit{frontend} y mostrar el \textit{loader} mientras se procesa el diagrama ER. & Eduardo/Omar \\[0.5cm]
	\hline
	\centering 9 & Agregar servicio \textit{backend} para la validación del diagrama entidad-relación. & Eduardo/Omar \\[0.5cm]
	\hline
	\centering 9 & implementación de algoritmo para validación del diagrama ER en el \textit{backend}. & Eduardo/Omar \\[0.5cm]
	\hline
	\centering 9 & Pruebas de captura de distintos diagramas entidad-relación.  & Eduardo/Omar \\[0.5cm]
	\hline
	\centering 9 & Pruebas para validar el algoritmo de validación. & Eduardo/Omar \\[0.5cm]
	\hline
	\centering 10 & Agregar servicio al \textit{backend} para transformación del esquema entidad-relación al modelo relacional.  & Omar \\[0.5cm]
	\hline
	\centering 10 & Implementación del algoritmo de transformación ER -> relacional & Eduardo/Omar \\[0.5cm]
	\hline
	\centering 10 & Agregar menú relacional a la intefaz gráfica. & Eduardo/Omar \\[0.5cm]
	\hline
	\centering 10 & Prueba de transformación de distintos diagramas ER al modelo relacional. & Eduardo/Omar \\[0.5cm]
	\hline
	\centering 10 & Visualización de la transformación del modelo ER al modelo relacional. & Eduardo \\[0.5cm]
	\hline
	\centering 14 & Revision de la redacción del reporte técnico para presentación de TT1  & Eduardo \\[0.5cm]
	\hline
	\centering 11 & Agregar servicio \textit{backend} para la descarga del archivo .sql con las sentencias equivalentes. & Eduardo/Omar \\[0.5cm]
	\hline
	\centering 11 & Pruebas de coherencia de las sentencias equivalentes en el DBMS. & Eduardo/Omar \\[0.5cm]
	\hline
	\centering 12 & Definición de las reglas de transformación al modelo NoSQL.  & Eduardo/Omar \\[0.5cm]
	\hline
	\centering 12 & Pruebas de distintos escenarios del modelo relacional al modelo NoSQL.  & Eduardo/Omar \\[0.5cm]
	\hline
	\centering 12 & Agregar servicio al \textit{backend} para transformación del esquema relacional al modelo NoSQL.  & Omar \\[0.5cm]
	\hline
	\centering 12 & Agregar servicio al \textit{backend} para guardar el modelo NoSQL en la base de datos.  & Omar \\[0.5cm]
	\hline
	\centering 12 & Agregar menú no relacional a la intefaz gráfica. & Omar \\[0.5cm]
	\hline
	\centering 12 & Implementación del algortimo de transformación de modelo relacional al modelo conceptual NoSQL. & Eduardo \\[0.5cm]
	\hline
	\centering 12 & Comprobación de la coherencia de la transformación entre modelos relacional a no relacional.  & Eduardo/Omar \\[0.5cm]
	\hline
	\centering 13 & Agregar servicio \textit{backend} para transformación del modelo ER al modelo no relacional. & Eduardo/Omar \\[0.5cm]
	\hline
	\centering 13 & Ajustar la interfaz del menú ER para mostrar mensaje de transformación al modelo NoSQL. & Omar \\[0.5cm]
	\hline
	\centering 13 & Manejar el estado del diagrama ER y redireccionar al menú no relacional al terminar la tranformación. & Eduardo \\[0.5cm]
	\hline
	\centering 13 & Pruebas de caso de estudio para verificar la correcta transformación y coherencia de los datos.  & Eduardo/Omar \\[0.5cm]
	\hline
	\centering 14 & Revisión de la redacción del reporte técnico para presentación de TT2 & Eduardo \\[0.5cm]

    \label{tab:lista-producto}
\end{longtable}

Se considera la tabla~\ref{tab:lista-producto} como la lista de producto con las tareas necesarias para cumplir con todas las historias de usuario mencionadas en la sección anterior, considerando que es posible que cambien conforme avancen los \textit{sprints} y así añadir nuevas tareas.
%seccion algoritmos
\subsection{Algoritmos para el desarrollo del sistema}

De acuerdo con Cormen~\cite{cormen_introduction_2009}, un algoritmo es cualquier procedimiento computacional definido que toma algún valor, o conjunto de valores, como entrada y produce algún valor, o conjunto de valores, como salida; por lo tanto, un algoritmo es una secuencia de pasos computacionales que transforman la entrada en la salida.


En esta sección se describen los algoritmos a implementar; se empieza con el algoritmo para la validación estructural diagrama entidad-relación; se sigue con algoritmo para el mapeo modelo entidad-relación a relacional; después está la obtención de esquema SQL desde modelo relacional, el mapeo modelo entidad-relación a Generic Data Metamodel; el mapeo del Generic Data Metamodel a modelo lógico NoSQL y se finaliza con el algoritmo para el modelo lógico NoSQL a modelo físico en MongoDB.

%Respecto al algoritmo 1, en la literatura sobre validez estructural de un diagrama ER hay pocos trabajos, en específico solo hemos encontrado dos: el de Daniel L. Moody, Metrics for Evaluating the Quality of Entity Relationship Models - 1998 y el de James Dullea An analysis of structural validity in entity-relationship modeling  - 2003. Para la redacción del algoritmo nos hemos apoyado en el trabajo de Dullea.

%Respecto al algoritmo 2, hemos usado la propuesta de Elmasri que está en su libro Fundamentos de Sistemas de Bases de Datos.

%Del algoritmo 3, nos estamos basando en la guía de implementación de MySQL para realizar la obtención de las sentencias SQL a partir del modelo conceptual del modelo relacional, porque MySQL es el SGBD que utilizaremos.

%Del algoritmo 4, este es el más problemático, dado que no hay a día hoy algoritmo de mapeo en la literatura, porque la propuesta del Generic Data Metamodel es muy nueva (2018 & 2020).

%La primera versión del paper donde sale este metamodelo conceptual NoSQL es del 2018 por parte de la Vega, García-Saiz & compañía. Hay una segunda revisión del paper accesible desde diciembre de 2019 con algunas actualizaciones y publicado en abril de 2020.

%Seguimos pensando y redactando este algoritmo, pero lo que ya sabemos es:
%Necesita estar basado en querys como el trabajo de Chebotko.
%input: modelo ER & output: Generic Data Metamodel
%Las relaciones del modelo ER son referencias en el Generic Data Metamodel
%Además, para apoyarnos en redactar este algoritmo tenemos los trabajos de Chebotko - 2015 (que realiza un mapeo de un diagrama ER + consultas a un modelo lógico orientado a columnas) y el trabajo de Lima & Santos Mello - 2016 (que realiza el mapeo de un modelo ER + workload a un modelo lógico orientado a documentos).

%Del algoritmo 5, vamos a usar el algoritmo que de la Vega propone en Mortadelo: Automatic generation of NoSQL stores from platform-independent data models - 2020, en el que realiza la transformación de una instancia del GDM a un modelo lógico orientado a documentos.

%Finalmente, respecto al algoritmo 6. utilizando la propuesta de la Vega para obtener el modelo lógico NoSQL(M2M) obtendremos el modelo físico (M2T) adaptado a los lineamiento del SGDB NoSQL de MongoDB.


\subsubsection{Validación estructural diagrama entidad-relación}

Las reglas de validación básicas para el diagrama ER:

\paragraph*{Generales}
\begin{enumerate}
    \item No puede haber elementos sin conectar.
    \item Tampoco puede haber enlaces sin conectar.
    %\item Solo relación de participación binarias (restricción de la propuesta de solución). ¿esta va?
\end{enumerate}


\paragraph*{Entidad}
\begin{enumerate}
    \item Una entidad es válida si tiene atributos, porque no tiene propósito una entidad sin atributos.
    \item La clave primaria puede ser simple o compuesta.
    \item La clave primaria no es una clave foránea.
    \item La clave primaria debe ser un atributo clave asociado a la entidad (restricción de la propuesta de solución).
    \item Dos entidades solo pueden estar conectadas entre sí mediante una relación.
    \item Todas las entidades deben tener un atributo clave.
    \item En una entidad débil, un atributo solo puede estar asociado a un solo atributo o a una sola entidad.
    \item Una entidad débil no debe existir si no tiene una relación con otra entidad.
\end{enumerate}

\paragraph*{Atributo}

\begin{enumerate}    
\item Un atributo solo puede estar asociado a un solo atributo o a una sola entidad.
\item Un atributo puede ser compuesto.
\item Un atributo puede ser multivalor.
\item Un atributo puede ser derivado.
\item Un atributo debe tener un nombre.
\item Un atributo no puede usar una relación para asociarse a otro elemento.
\item Un atributo compuesto solo puede estar asociado a una entidad.
\item Un atributo derivado solo puede estar asociado a una entidad.
\end{enumerate}

\paragraph*{Relación}

\begin{enumerate}
    \item Una relación solo puede ser entre entidades.
    \item El grado de participación máximo es dos (restricción de la propuesta de solución).
    \item Una relación puede ser unaria (recursiva).
    \item No están permitidas relaciones ternarias o de grado n (restricción de la propuesta de solución).
\end{enumerate}

Para los diferentes tipos de relaciones, de acuerdo con Dullea\cite{dullea_analysis_2003}, el modelo ER está compuesto por entidades, las relaciones entre entidades y restricciones en esas relaciones.


La conectividad entre entidades y relaciones se denomina ruta; las rutas son los bloques de construcción de la validez estructural; definen visualmente la asociación semántica y estructural que cada entidad tiene simultáneamente con todas las demás entidades o consigo misma en una ruta.

En el modelado conceptual es posible clasificar la validez en dos tipos: validez semántica y validez estructural; un diagrama ER es válido cuando es válido semántica y estructuralmente; asimismo, los términos validez estructural y validez semántica se definen de la siguiente manera:


\begin{enumerate}
    \item Validez esctructural: un diagrama ER es estructuralmente válido solo cuando la consideración simultánea de todas las restricciones estructurales impuestas en el modelo no implica una inconsistencia lógica en ninguno de los posibles estados; 
    \item Validez semántica: un diagrama ER es semánticamente válido solo cuando cada relación representa exactamente el concepto del dominio del problema. 
\end{enumerate}


 Un diagrama ER semánticamente válido muestra la representación correcta del dominio de aplicación que se está modelando; sin embargo, dado que la validez semántica depende de la aplicación, no es posible definir criterios de validez generalizados, por lo que no se considerará la validez semántica, sino únicamente la validez estructural.


En términos generales, un diagrama ER es estructuralmente inválido si contiene construcciones que son contradictorias entre sí o al menos una restricción de cardinalidad es inconsistente.


Un diagrama ER representa la semántica de la aplicación en términos de restricciones de cardinalidad máxima y mínima. 


Cada conjunto de restricciones de cardinalidad en una sola relación debe ser coherente con todas las restricciones restantes en el modelo y en todos los estados posibles.


Una relación recursiva es estructuralmente inválida cuando las restricciones de participación y cardinalidad no respaldan la existencia de instancias de datos como lo requiere el usuario y hace que todo el diagrama sea inválido.


En general, un diagrama estructuralmente inválido refleja reglas de negocio semánticamente inconsistentes; para que un modelo sea válido, todas las rutas del modelo también deben ser válidas.


La tabla~\ref{tab:relaciones} muestra los distintos tipos de relaciones recursivas, mostrando el tipo de relación, la dirección de esa relación, sus restricciones de participación, de cardinalidad y un ejemplo.

En la tabla~\ref{tab:relaciones}, tabla~\ref{tab:reglas-validez-relaciones-recursivas} y tabla~\ref{tab:reglas-validez-relaciones-binarias} se usa la notación uno (1) y muchos (M) para la máxima cardinalidad; una sola línea para indicar la participación opcional y una línea doble para mostrar la participación obligatoria; las palabras ``obligatorio'' y ``opcional'' se utilizan en la tabla para indicar la cardinalidad mínima obligatoria (o total) y opcional (o parcial), respectivamente; además, la notación $|E|$ representa el número de instancias en la entidad $E$.

La tabla~\ref{tab:relaciones} resume cada relación recursiva por sus propiedades direccionales, la combinación de restricciones de cardinalidad mínima/máxima y ejemplos. 


La tabla~\ref{tab:reglas-validez-relaciones-recursivas} muestra primero reglas de validación para relaciones recursivas y un ejemplo válido; después se da un corolario de la validez de las relaciones recursivas y un ejemplo no válido.


La tabla~\ref{tab:reglas-validez-relaciones-binarias} tiene la misma estructura que la tabla~\ref{tab:reglas-validez-relaciones-recursivas} en la que muestra reglas de validez para las relaciones binarias y después un corolario con un ejemplo no válido.

% Please add the following required packages to your document preamble:
% \usepackage{multirow}
% \usepackage{graphicx}
\begin{table}[H]
    \centering
    \resizebox{\textwidth}{!}{%
    \begin{tabular}{lllllll}
    \hline
    \multirow{2}{*}{\begin{tabular}[c]{@{}l@{}}Tipo \\ de relación\end{tabular}} & \multirow{2}{*}{}                                             & \multirow{2}{*}{\begin{tabular}[c]{@{}l@{}}Dirección de\\  la relación\end{tabular}} & \multirow{2}{*}{\begin{tabular}[c]{@{}l@{}}Restricción de \\ participación\end{tabular}}                   & \multirow{2}{*}{\begin{tabular}[c]{@{}l@{}}Restricciones de\\ cardinalidad\end{tabular}} & \multicolumn{2}{l}{Ejemplo}                                                                                                             \\ \cline{6-7} 
                                                                                 &                                                               &                                                                                      &                                                                                                            &                                                                                          & Relación                                                                  & Roles                                                       \\ \hline
    \begin{tabular}[c]{@{}l@{}}Simétrica\\ (reflexiva)\end{tabular}              &                                                               & Bidireccional                                                                        & \begin{tabular}[c]{@{}l@{}}Opcional–opcional\\ Obligatoria-obligatoria\end{tabular}                        & \begin{tabular}[c]{@{}l@{}}1-1\\ M-N\end{tabular}                                        & Cónyuge de                                                                & Persona                                                     \\
    \begin{tabular}[c]{@{}l@{}}Asimétrica\\ (no reflexiva)\end{tabular}          & Jerárquica                                                    & Unidireccional                                                                       & Opcional-opcional                                                                                          & \begin{tabular}[c]{@{}l@{}}1-M\\ 1-1\end{tabular}                                        & \begin{tabular}[c]{@{}l@{}}Supervisa\\ Es supervisado \\ por\end{tabular} & \begin{tabular}[c]{@{}l@{}}Gerente-\\ empleado\end{tabular} \\
                                                                                 &                                                               &                                                                                      & \begin{tabular}[c]{@{}l@{}}Opcional-opcional\\ Opcional-obligatoria\\ Obligatoria-obligatoria\end{tabular} & M-N                                                                                      & \begin{tabular}[c]{@{}l@{}}Supervisa\\ Es supervisado\\ por\end{tabular}  & \begin{tabular}[c]{@{}l@{}}Gerente\\ Empleado\end{tabular}  \\
                                                                                 & Circular                                                      & Unidireccional                                                                       & \begin{tabular}[c]{@{}l@{}}Opcional-opcional\\ Obligatoria-obligatoria\end{tabular}                        & 1-1                                                                                      & \begin{tabular}[c]{@{}l@{}}Apoya\\ Es apoyado por\end{tabular}            & Servicio técnico                                            \\
                                                                                 & \begin{tabular}[c]{@{}l@{}}Jerárquica\\ Circular\end{tabular} & Unidireccional                                                                       & \begin{tabular}[c]{@{}l@{}}Opcional-opcional\\ Opcional-obligatoria\end{tabular}                           & 1-M                                                                                      & \begin{tabular}[c]{@{}l@{}}Apoya\\ Es apoyado por\end{tabular}            & Responsable                                                 \\
                                                                                 &                                                               &                                                                                      & \begin{tabular}[c]{@{}l@{}}Opcional-opcional\\ Opcional-obligatoria\\ Obligatoria-obligatoria\end{tabular} & M-N                                                                                      & Supervisa                                                                 & \begin{tabular}[c]{@{}l@{}}Gerente-\\ empleado\end{tabular} \\
                                                                                 & Reflejada                                                     & Unidireccional                                                                       & Opcional-opcional                                                                                          & 1-1                                                                                      & \begin{tabular}[c]{@{}l@{}}Gestiona\\ Se gestiona\end{tabular}            & CEO                                                        
    \end{tabular}%
    }
    \caption{Tipos de relación recursiva válidos según las restricciones de cardinalidad}
    \label{tab:relaciones}
    \end{table}




De la tabla~\ref{tab:relaciones}, una relación recursiva es simétrica o reflexiva cuando todas las instancias que participan en la relación toman un solo papel y el significado semántico de la relación es exactamente el mismo para todas las instancias que participan en la relación independientemente de la dirección en la que se ve; estos tipos de relación se denominan bidireccionales.


De la tabla~\ref{tab:relaciones}, una relación recursiva es asimétrica o no reflexiva cuando hay una asociación entre dos grupos de roles diferentes dentro de la misma entidad y el significado semántico de la relación es diferente dependiendo de la dirección en la que se ven las asociaciones entre los grupos de roles; estos tipos de relación se denominan unidireccionales.


De la tabla~\ref{tab:relaciones}, una relación recursiva es jerárquica cuando un grupo de instancias dentro de la misma entidad se clasifican en calificaciones, órdenes o clases, una encima de otra; implica un comienzo (o arriba) y un final (o abajo) para el esquema de clasificación de instancias.


De la tabla~\ref{tab:relaciones}, una relación recursiva es circular cuando una relación recursiva asimétrica tiene al menos una instancia que no cumple con la jerarquía de clasificación; la relación es unidireccional, ya que se puede ver desde dos direcciones con un significado semántico diferente.


De la tabla~\ref{tab:relaciones}, una relación reflejada existe cuando la semántica de una relación permite que una instancia de una entidad se asocie a sí misma a través de la relación.


% Please add the following required packages to your document preamble:
% \usepackage{multirow}
% \usepackage{graphicx}
\begin{table}[H]
    \centering
    \resizebox{\textwidth}{!}{%
    \begin{tabular}{lc}
    \hline
    \multicolumn{1}{c}{\multirow{2}{*}{Reglas de validación para relaciones recursivas}}                                                                                                                                                                                                          & \multirow{2}{*}{Ejemplo válido} \\  &                                 \\ \hline
    & \\ 
    \begin{tabular}[c]{@{}l@{}}Solo las relaciones recursivas 1:1 con restricciones de cardinalidad mínimas obligatorias-obligatorias u \\ opcionales son estructuralmente válidas; válido para relaciones simétricas y completamente circular.\end{tabular}                                      &           \begin{minipage}{.3\textwidth}\includegraphics[width=\linewidth]{images/validezER/tabla1/01.png}\end{minipage}                       \\
    \begin{tabular}[c]{@{}l@{}}Para las relaciones recursivas 1:M o M:1, la cardinalidad mínima opcional-opcional \\ es estructuralmente válida; válido solo para relaciones asimétricas.\end{tabular}                                                                                            & \begin{minipage}{.3\textwidth}\includegraphics[width=\linewidth]{images/validezER/tabla1/02.png}\end{minipage}                                \\
    \begin{tabular}[c]{@{}l@{}}Para 1:M las relaciones recursivas del tipo jerárquico-circular, \\ la cardinalidad mínima opcional-obligatoria son estructuralmente válidas; \\ válido solo para relaciones jerárquico-circulares.\end{tabular}                                                   & 
    \begin{minipage}{.3\textwidth}
        \includegraphics[width=\linewidth]{images/validezER/tabla1/03.png}
    \end{minipage}                                \\
    \begin{tabular}[c]{@{}l@{}}Todas las relaciones recursivas con cardinalidad máxima\\ de muchos a muchos son estructuralmente válidas independientemente de las restricciones\\ mínimas de cardinalidad; válido para relaciones simétricas, jerárquicas y jerárquicas-circulares.\end{tabular} &  \begin{minipage}{.3\textwidth}
        \includegraphics[width=\linewidth]{images/validezER/tabla1/04.png}
    \end{minipage}                                \\                               \\
    \begin{tabular}[c]{@{}l@{}}Todas las relaciones recursivas con cardinalidad mínima opcional–opcional son estructuralmente válidas;\\ válido para relaciones simétricas y asimétricas.\end{tabular}                                                                                            &  \begin{minipage}{.3\textwidth}
        \includegraphics[width=\linewidth]{images/validezER/tabla1/05.png}
    \end{minipage}                                \\                               \\ \hline
    \multicolumn{1}{c}{Colorarios de validez para relaciones recursivas}                                                                                                                                                                                                                          & Ejemplo no válido               \\ \hline
    & \\
    \begin{tabular}[c]{@{}l@{}}Todas las relaciones recursivas 1: 1 con restricciones de cardinalidad mínima obligatoria–opcional\\  u opcional-obligatoria son estructuralmente inválidas.\end{tabular}                                                                                          &   \begin{minipage}{.3\textwidth}
        \includegraphics[width=\linewidth]{images/validezER/tabla1/06.png}
    \end{minipage}                                \\                              \\
    \begin{tabular}[c]{@{}l@{}}Todas las relaciones recursivas 1:M o M:1 con restricciones de cardinalidad mínimas\\ obligatorias-obligatorias son estructuralmente inválidas.\end{tabular}                                                                                                       & \begin{minipage}{.3\textwidth}
        \includegraphics[width=\linewidth]{images/validezER/tabla1/07.png}
    \end{minipage}                                \\            \\
    \begin{tabular}[c]{@{}l@{}}Todas las relaciones recursivas 1:M o M:1 con restricción de participación obligatoria en ``uno''\\ y una restricción de participación opcional en las ``muchas'' restricciones son estructuralmente inválidas.\end{tabular}                                       & \begin{minipage}{.3\textwidth}
        \includegraphics[width=\linewidth]{images/validezER/tabla1/08.png}
    \end{minipage}                                \\           
    \end{tabular}%
    }
    \caption{Resumen de reglas de validez para relaciones recursivas con ejemplos}
    \label{tab:reglas-validez-relaciones-recursivas}
    \end{table}


% Please add the following required packages to your document preamble:
% \usepackage{multirow}
% \usepackage{graphicx}
\begin{table}[H]
    \centering
    \resizebox{\textwidth}{!}{%
    \begin{tabular}{lc}
    \hline
    \multicolumn{1}{c}{\multirow{2}{*}{Reglas de validez para relaciones binarias.}}                                                                                                                                                                                                                     & \multirow{2}{*}{Ejemplo válido} \\      
      & \\ \hline
      &                                 \\
    \begin{tabular}[c]{@{}l@{}}Una ruta acíclica que contiene todas las relaciones binarias siempre\\ es estructuralmente válida.\end{tabular}                                                                                                                                                           & \begin{minipage}{.3\textwidth}
        \includegraphics[width=\linewidth]{images/validezER/tabla2/01.png}
    \end{minipage}                               \\
    \begin{tabular}[c]{@{}l@{}}Una ruta cíclica que contiene todas las relaciones binarias y\\ una o más relaciones opcional–opcional siempre es\\ estructuralmente válida.\end{tabular}                                                                                                                 & \begin{minipage}{.3\textwidth}
        \includegraphics[width=\linewidth]{images/validezER/tabla2/02.png}
    \end{minipage}                               \\
    \begin{tabular}[c]{@{}l@{}}Una ruta cíclica que contiene todas las relaciones binarias y\\ una o más relaciones de muchos a uno con participación opcional\\ del lado ``uno'' siempre es estructuralmente válida.\end{tabular}                                                                       & \begin{minipage}{.3\textwidth}
        \includegraphics[width=\linewidth]{images/validezER/tabla2/03.png}
    \end{minipage}                               \\
    \begin{tabular}[c]{@{}l@{}}Una ruta cíclica que contiene todas las relaciones binarias y\\ una o más relaciones de muchos a muchos es siempre estructuralmente válida.\end{tabular}                                                                                                                  & \begin{minipage}{.3\textwidth}
        \includegraphics[width=\linewidth]{images/validezER/tabla2/04.png}
    \end{minipage}                               \\
    \begin{tabular}[c]{@{}l@{}}Las rutas cíclicas que contienen al menos un\\ conjunto de relaciones opuestas siempre son válidas.\end{tabular}                                                                                                                                                                                                    & \begin{minipage}{.3\textwidth}
        \includegraphics[width=\linewidth]{images/validezER/tabla2/05.png}
    \end{minipage}                               \\
    \multicolumn{1}{c}{\begin{tabular}[c]{@{}c@{}}Una ruta cíclica que contiene todas las relaciones binarias uno a uno que son todas obligatorias-obligatorias\\ o al menos una restricción de cardinalidad mínima opcional-opcional siempre es estructuralmente válida.\end{tabular}}                  & \begin{minipage}{.3\textwidth}
        \includegraphics[width=\linewidth]{images/validezER/tabla2/06.png}
    \end{minipage}                               \\ \hline
    \multicolumn{1}{c}{Corolarios de validez para relaciones binarias} & Ejemplo no válido               \\ 
    \hline
    & \\
    \begin{tabular}[c]{@{}l@{}}Las rutas cíclicas que no contienen relaciones opuestas ni relaciones autoajustables\\  son estructuralmente inválidas y se denominan relaciones circulares.\end{tabular}                                                                                                 & \begin{minipage}{.3\textwidth}
        \includegraphics[width=\linewidth]{images/validezER/tabla2/07.png}
    \end{minipage}           \\
    \begin{tabular}[c]{@{}l@{}}La presencia de una relación ``uno a uno obligatorio-obligatorio'' no tiene ningún \\ efecto sobre la validez estructural (o invalidez) de una ruta cíclica que contiene otros tipos de relación.\\  (Este corolario se aplica a todas las reglas anteriores).\end{tabular} & \multicolumn{1}{l}{}           
    \end{tabular}%
    }
    \caption{Resumen de reglas de validez para relaciones binarias con ejemplos}
    \label{tab:reglas-validez-relaciones-binarias}
    \end{table}

En la tabla~\ref{tab:reglas-validez-relaciones-recursivas} se pone un resumen de reglas válidas para relaciones recursivas y en la tabla~\ref{tab:reglas-validez-relaciones-binarias} reglas válidas para relaciones binarias.


\subsubsection{Modelo entidad-relación a relacional}
De acuerdo con la propuesta de Elmasri\cite{ramez_elmasri_fundamentos_nodate}, un algoritmo en siete pasos basta para convertir las estructuras de un modelo ER básico con tipos de entidades fuertes y débiles, relaciones binarias con distintas restricciones estructurales, relaciones de grado n y atributos (simples, compuestos y multivalor) en relaciones; a continuación se explica cada paso del algoritmo:

\paragraph*{Paso 1: Mapeado de los tipos de entidad regulares}
Por cada entidad (fuerte) regular $E$ del esquema ER cree una relación $R$ que incluya todos los atributos simples de $E$.


Incluya únicamente los atributos simples que conforman un atributo compuesto; seleccione uno de los atributos clave de $E$ como clave principal para $R$. 


Si la clave elegida de $E$ es compuesta, el conjunto de los atributos simples que la forman constituirán la clave principal de $R$.


Si durante el diseño conceptual se identificaron varias claves para $E$, la información que describe los atributos que forman cada clave adicional conserva su orden para especificar las claves (únicas) secundarias de la relación $R$. 

\paragraph*{Paso 2: Mapeado de los tipos de entidad débiles}
Por cada tipo de entidad débil $W$ del esquema ER con el tipo de entidad propietario $E$, cree una relación $R$ e incluya todos los atributos simples (o componentes simples de los atributos compuestos) de $W$ como atributos de $R$. 


Además, incluya como atributos de la \textit{foreign key} de $R$, los atributos de la o las relaciones que correspondan al o los tipos de entidad propietarios; esto se encarga de identificar el tipo de relación de $W$. 


La clave principal de $R$ es la combinación de las claves principales del o de los propietarios y la clave parcial del tipo de entidad débil $W$ si la hubiera.


Si hay un tipo de entidad débil $E_{2}$ cuyo propietario también es un tipo de entidad débil $E_{1}$, $E_{1}$ debe asignarse antes que $E_{2}$ para determinar primero su clave principal.


\paragraph*{Paso 3: Mapeado de los tipos de relación 1:1 binaria}

Por cada tipo de relación 1:1 binaria $R$ del esquema ER, identifique las relaciones $S$ y $T$ que corresponden a los tipos de entidad que participan en $R$; hay tres metodologías posibles: (1) la metodología de la foreign key, (2) la metodología de la relación mezclada y (3) la metodología de referencia cruzada o relación de relación; la primera metodología es la más útil y la que debe seguirse salvo que se den ciertas condiciones especiales, como las que se explican a continuación:

\begin{enumerate}
    \renewcommand\labelenumi{\bfseries\theenumi}
    \item \textbf{Metodología de la \textit{foreign key}}: seleccione una de las relaciones (por ejemplo, $S$) e incluya como \textit{foreign key} en $S$ la clave principal de $T$; lo mejor es elegir un tipo de entidad con participación total en $R$ en el papel de $S$. Incluya todos los tributos simples (o los componentes simples de los atributos compuestos) del tipo de relación 1:1 $R$ como atributos de $S$.
    
    \item \textbf{Metodología de la relación mezclada}:  una asignación alternativa de un tipo de relación 1:1 es posible al mezclar los dos tipos de entidad y la relación en una sola relación. 
    
    \item \textbf{Metodología de referencia cruzada o relación de relación}: consiste en configurar una tercera relación $R$ con el propósito de crear una referencia cruzada de las claves principales de las relaciones $S$ y $T$ que representan los tipos de entidad; como se verá, esta metodología es necesaria para las relaciones M:N binarias; la relación $R$ se denomina relación de relación (y, en algunas ocasiones, tabla de búsqueda), porque cada tupla de $R$ representa una instancia de relación que relaciona una tupla de $S$ con otra de $T$.
\end{enumerate}

\paragraph*{Paso 4: Mapeado de tipos de relaciones 1:N binarias}
Por cada relación 1:N binaria regular $R$, identifique la relación $S$ que representa el tipo de entidad participante en el lado $N$ del tipo de relación. 


Incluya como \textit{foreign key} en $S$ la clave principal de la relación $T$ que representa el otro tipo de entidad participante en $R$, porque cada instancia de entidad en el lado $N$ está relacionada, a lo sumo, con una instancia de entidad del lado 1 del tipo de relación. 


Incluya cualesquiera atributos simples (o componentes simples de los atributos compuestos) del tipo de relación 1:N como atributos de $S$.


De nuevo, una metodología alternativa es la opción de relación de relación (referencia cruzada), como en el
caso de las relaciones 1:1 binarias. 


Se crea una relación $R$ separada cuyos atributos son las claves de $S$ y $T$, y cuya clave principal es la misma que la clave de $S$; esta opción puede utilizarse si pocas tuplas de $S$ participan en la relación para evitar excesivos valores nulos en la \textit{foreign key}.


\paragraph*{Paso 5: Mapeado de tipos de relaciones M:N binarias}

Por cada tipo de relación M:N binaria $R$ cree una nueva relación $S$ para representar a R.


Incluya como atributos de la \textit{foreign key} en $S$ las claves principales de las relaciones que representan los tipos de entidad participantes; su combinación formará la clave principal de $S$.


Incluya también cualesquiera atributos simples del tipo de relación M:N (o los componentes simples de los atributos compuestos) como atributos de $S$.


No es posible representar un tipo de relación M:N con un atributo de \textit{foreign key} en una de las relaciones participantes (como se hizo para los tipos de relación 1:1 o 1:N) debido a la razón de cardinalidad M:N; se deben crear una relación de relación $S$ separada.

Siempre es posible asignar relaciones 1:1 o 1:N de un modo similar a las relaciones M:N utilizando la metodología de la referencia cruzada (relación de relación). 


Esta alternativa es particularmente útil cuando existen pocas instancias de relación, a fin de evitar valores nulos en las \textit{foreign keys}.


En este caso, la clave principal de la relación de relación sólo será una de las \textit{foreign keys} que hace referencia a las relaciones de entidad participantes. 


Para una relación 1:N, la clave principal de la relación de relación será la \textit{foreign key} que hace referencia a la relación de la entidad en el lado $N$.


En una relación 1:1, cada \textit{foreign key} es posible de utilizar como la clave principal de la relación de relación, siempre y cuando no haya entradas nulas en la relación.


\paragraph*{Paso 6: Mapeado de atributos multivalor}

Por cada atributo multivalor $A$, cree una nueva relación $R$.


Esta relación incluirá un atributo correspondiente a $A$, más el atributo clave principal $K$ (como \textit{foreign key} en $R$) de la relación que representa el tipo de entidad o tipo de relación que tiene $A$ como un atributo.


La clave principal de $R$ es la combinación de $A$ y $K$; si el atributo multivalor es compuesto, se incluyen sus componentes simples.


\paragraph*{Paso 7: Mapeado de los tipos de relación de grado n}
Por cada tipo de relación de grado n $R$, donde $n > 2$, cree una nueva relación $S$ para representar $R$.


Incluya como atributos de la \textit{foreign key} en $S$ las claves principales de las relaciones que representan los tipos de entidad participantes. 


Incluya también cualesquiera atributos simples del tipo de relación de grado n (o los componentes simples de los atributos compuestos) como atributos de $S$.


Normalmente, la clave principal de $S$ es una combinación de todas las \textit{foreign keys} que hacen referencia a las relaciones que representan los tipos de entidad participantes.


No obstante, si las restricciones de cardinalidad en cualquiera de los tipos de entidad $E$ que participan en $R$ es 1, entonces la clave principal de $S$ no debe incluir el atributo de la \textit{foreign key} que hace referencia a la relación $E'$ correspondiente a $E$.

\subsection*{Correspondencia entre los modelos ER y relacional}

%\begin{tabular}{l l }[H]
%    \centering
%    \hline \multicolumn{1}{c}{\textbf{Modelo entidad-relación}} & \multicolumn{1}{c}{\textbf{Modelo relacional}} \\ 
%    Tipo de entidad & Relación de entidad\\ \hline
%    Tipo de relación $1\colon 1$ o $1\colon N$ & \textit{foreign key} (o relación de relación)\\ \hline
%    Tipo de relación $M\colon N$ & Relación de relación y dos \textit{foreign keys}\\ \hline
%    Tipo de relación de grado n & Relación de relación y $n$ \textit{foreign keys}\\ \hline
%    Atributo simple & Atributo\\ \hline
%    Atributo compuesto & Conjunto de atributos simples\\ \hline
%    Atributo multivalor & Relación y \textit{foreign key}\\ \hline
%    Conjunto de valores & Dominio\\ \hline
%    Atributo clave & Clave principal (o secundaria)\\
%    \caption{Correspondencia entre los modelos ER y relacional.} \label{tab:modelos-er-relacional} \\
%\end{tabular}

\subsubsection{Obtención de esquema SQL desde modelo relacional}


Las sentencias SQL que generará la aplicación web son especificamente para el gestor MySQL, por lo cual de acuerdo a su guia de sintaxis [https://manuales.guebs.com/mysql-5.0/sql-syntax.html] para obtener las sentencias SQL del modelo relacional se seguiran los siguientes pasos:

\begin{enumerate}
  \item Por cada relación(tabla) del modelo relacional se agrega una sentencia CREATE TABLE seguida del nombre de la relación seguido de sus atributos encerrados entre paréntesis.
  \item Por cada atributo(columna) se crea un campo con su nombre correspondiente y se indica el tipo de dato, para hacerlo mas especifico se le puede añadir alguna de las siguientes caracteriticas:
  \begin{itemize}
    \item Los atributos que esten obligados a tener un valor se le agrega la palabra NOT NULL.
    \item Si la columna tomará un valor por defecto se agrega la instrucción AUTO\_INCREMENT siempre y cuando el tipo de dato sea INTEGER.
    \item Si la columna es la que identificará a la tabla de forma única en todo el esquema se debe agregar como llave primaria de la tabla.
  \end{itemize}
  \item La llave primaria de agrega como un atributo mas con la palabra PRIMARY KEY seguido del nombre de atributo,encerrada entre paréntesis, que indentifica de forma unica a la tabla.
  \item Todos los atributos de la tabla se encuentran encerrados entre paréntesis separados por comas.
  \item Al final de la lista de atrubutos de la tabla(despues del parentesis) se agrega el motor de almacenamiento de MySQL , ENGINE=InnoDB (restricción del proyecto
\subsubsection{Modelo entidad-relación básico a Generic Data Metamodel}\label{sec:er-to-gdm}

Se ha decidido que el algoritmo debe ser basado en \textit{querys} como el de Chebotko; por ello para obtener consultas válidas, debe existir una ruta en el modelo entidad-relación básico que permita llegar desde la entidad n a la entidad m recorriendo las rutas del diagrama.


La figura~\ref{img:simple-access-pattern} muestra el boceto de la consulta simple a implementar, está inspirada en el trabajo de Chebotko; cada consulta tiene un nombre, la columna \textit{find} representa el atributo a buscar respecto a los diferentes atributos de la columna \textit{given}. En la parte de abajo está un campo para poner una descripción breve de la consulta y botones para limpiar los respectivos campos de la consulta. En la figura~\ref{img:simple-access-pattern} \textit{User} y \textit{Artifact} son entidades y \textit{id} e \textit{email} son atributos de sus respectivas entidades.


\begin{figure}[H] 
    \centering
    \includegraphics[width=0.75\textwidth]{simple_access_patterns.png}
    \caption{Consulta básica}
    \label{img:simple-access-pattern}
\end{figure}


El algoritmo por entrada tiene una instancia del modelo conceptual entidad-relacíon básico $+$ consultas básicas como las de la figura~\ref{img:simple-access-pattern} y de salida una instancia del modelo GDM.


La definición del modelo orientado a documentos será la usada por Alfonso de la Vega en su publicación sobre el GDM.


Se crean primero los elementos de consultas del GDM a partir de las consultas básicas de la instancia del modelo entidad-relación básico como las de la figura~\ref{img:simple-access-pattern}:


\begin{enumerate}
    \item Por cada consulta se crea un elemento de la clase \textit{query} asignando el nombre de la consulta a realizar.
    \item Se crea un elemento \textit{select} con los parámetros de la columna \textit{given} del diagrama entidad-relación básico.
    \item Se crea el elemento \textit{from} desde la primera entidad en la que se realiza la consulta.
    \item Se recorre la ruta del diagrama entidad-relación básico y se van añadiendo cada entidad del recorrido como un elemento \textit{including} con el nombre de la relación como referencia.
    \item Finalmente, se crea un elemento de la clase \textit{boolean expression} para contener la expresión booleana de la consulta.
\end{enumerate}


Para los elementos del \textit{structure model elements} del GDM:

\begin{enumerate}
    \item Por cada entidad se crea una clase entidad correspondiente en el GDM. 
    \item Por cada atributo se crea un clase atributo correspondiente y se indica su tipo.
    \item Por cada relación se crea una referencia correspondiente.
\end{enumerate}

%El algoritmo necesita ser basado en querys como el de Chebotko.


%input: modelo entidad-relacion


%output: ¿modelo orientado a documentos o GDM?


% ¿Qué definición del modelo orientado a documentos se usará? ¿El del GDM o el de De Lima?


% ¿Cómo convertir las relaciones?
% Hay reglas para convertir las relaciones en bloques del documento orientado a objetos.

\subsubsection{Generic Data Metamodel a modelo lógico NoSQL}\label{alg:gdm-to-logic}
De acuerdo con de la Vega\cite{de_la_vega_mortadelo_2020}, las bases de datos orientadas a documentos tienen como objetivo almacenar jerarquías de objetos que son probables que se consultarán juntas.


Estas jerarquías de objetos se conocen como documentos y se agrupan en colecciones; asimismo, una colección almacena documentos de la misma entidad y por lo general un documento está compuesto de pares clave-valor y contiene otros documentos incrustados.


\paragraph*{Modelo lógico orientado a documentos}

En la figura~\ref{img:mortadelo-gdm-logical-model-document-oriented} se muestra el modelo lógico orientado a documentos propuesto por Alfonso de la Vega.


\begin{figure}[H] 
    \centering
    \includegraphics[width=0.65\textwidth]{mortadelo/GDM-logical-model-document-oriented.png}
    \caption{Modelo lógico orientado a documentos propuesto por Alfonso de la Vega}
    \label{img:mortadelo-gdm-logical-model-document-oriented}
\end{figure}

A continuación se explica cada clase que conforma el modelo de la figura~\ref{img:mortadelo-gdm-logical-model-document-oriented}:

\begin{itemize}
    \item Clase \textit{document data model}: está compuesta de $n$ \textit{collections}.
    \item Clase \textit{collection}: tiene un nombre que la identifica y en una colección se almacenan documentos que, en general, comparten la misma estructura.
    \item Clase \textit{document type}: define la estructura de los documentos a guardar con un conjunto de instancias de la clase \textit{field}; es la estructura principal que se guardará en cada colección.
    \item Clase \textit{field}: tiene instancias de las clases \textit{primitive field} (atributos simples), \textit{array field} (más instancias de la clase \textit{field}) o \textit{document type} (otras estructuras de documentos), permitiendo guardar elementos anidados.
    \item Clase \textit{primitive field}: contiene un tipo primitivo de dato.
    \item Clase \textit{array field}: permite anidar más instancias de la clase \textit{field}.
\end{itemize}


La figura~\ref{img:mortadelo-gdm1} muestra el Generic Data Metamodel, que está compuesto por clases interrelacionadas entre sí con notación UML y consta de dos secciones principales: los elementos de la estructura (\textit{structure model elements}) y el cómo se realizarán las consultas (\textit{access queries elements}).


\begin{figure}[H] 
    \centering
    \includegraphics[width=0.75\textwidth]{mortadelo/GDM.png}
    \caption{Generic Data Metamodel}
    \label{img:mortadelo-gdm1}
\end{figure}



Sea $\mathrm{Queries}$ el conjunto de todas las queries $q_{n}$ en una instancia GDM como la de la figura~\ref{img:mortadelo-gdm1}:

\begin{equation} \label{eq:cap4.3.3.5-00}
    \mathrm{Queries} = \{q_{1}, q_{2},..., q_{n}\}\; \mathrm{donde}\; q_{n}\;\epsilon\;\mathrm{Queries}\; \mathrm{y}\; n=1,2,...
\end{equation}


Sea $\mathrm{Entities_{consulted}}$ el conjunto de entidades que son consultadas en una instancia GDM como la de la figura~\ref{img:mortadelo-gdm1} (es decir, las entidades que están asociadas a la clase \textit{query} desde un elemento \textit{from}): 


\begin{equation} \label{eq:cap4.3.3.5-01}
    \mathrm{Entities_{consulted}} = \{e_{c1},e_{c2},...,e_{cn}\}\; \mathrm{donde}\; e_{cn}\;\epsilon\;\mathrm{Entities_{consulted}}\; \mathrm{y}\; n=1,2,...
\end{equation}

Sea $\mathrm{Collections}$ el conjunto de colecciones $c_{n}$ donde cada $c_{n}$ corresponde a una entidad consultada:

\begin{equation} \label{eq:cap4.3.3.5-02}
    \mathrm{Collections} = \{c_{1},c_{2},...,c_{n}\}\; \mathrm{donde}\; c_{n}\;\epsilon\;\mathrm{Collections}\; \mathrm{y}\; n=1,2,...
\end{equation}


% $e_{fn}$ de $E_{f}$ (es decir que por cada $ef_{n}$ se crea un $c_{n}$ en la que cada $c_{n}$ contiene un único elemento \textit{DocumentType}, llamado ``documento raíz''):


Por cada $c_{n}$ en \eqref{eq:cap4.3.3.5-02} se generan los contenidos de cada documento raíz.


La figura~\ref{img:mortadelo-gdm-logical-model-access-tree} es un ejemplo del árbol de acceso de la consulta $q_{4}$ de la figura~\ref{img:mortadelo-gdm-logical-model-q4}; está formado por el único elemento \textit{from} de la consulta y con los elementos \textit{including} se forma el árbol, donde las aristas son los identificadores de cada \textit{including}.


La figura~\ref{img:mortadelo-gdm-logical-model-q4} es una consulta en el GDM en modo texto.


\begin{figure}[H] 
    \centering
    \includegraphics[width=0.65\textwidth]{mortadelo/GDM-access-tree.png}
    \caption{Access Tree - Modelo lógico orientado a documentos}
    \label{img:mortadelo-gdm-logical-model-access-tree}
\end{figure}


\begin{figure}[H] 
    \centering
    \includegraphics[width=0.65\textwidth]{mortadelo/GDM-q4.png}
    \caption{Access query Q4}
    \label{img:mortadelo-gdm-logical-model-q4}
\end{figure}


Por cada $q_{n}$ en \eqref{eq:cap4.3.3.5-00} se obtiene primero su único elemento \textit{from} y después todos los elementos \textit{including} para generar un árbol de acceso como el de la figura~\ref{img:mortadelo-gdm-logical-model-access-tree}.


Añadimos al documento raíz todos los atributos simples de su entidad correspondiente; nótese que se está analizando tanto la instancia de la clase \textit{entity} y el árbol de acceso creado.


Si el elemento ``ref'' de la \textit{entity} está en el árbol de acceso, se necesita saber su cardinalidad y la entidad objetivo de esa referencia.


Si la entidad objetivo está en el árbol de acceso, los datos de esa referencia se incluyen como un subdocumento y se llama recursivamente la función para generar el contenido de este subdocumento; por el contrario, en caso de que la referencia no esté en el árbol de acceso, se incluye la referencia como un identificador si la cardinalidad es 1 o como un arreglo de referencias cuando la carnidad es $n$.


Por último, el autor menciona dos optimizaciones si se quiere reducir el nivel de denormalización que las puede consultar en su investigación\cite{de_la_vega_mortadelo_2020}.

En resumen, el algoritmo quedaría de la forma:
\begin{algorithm}[H]
    \SetKwInOut{Input}{Input}
    \SetKwInOut{Output}{Output}

    %\underline{function Euclid} $(a,b)$\;
    \Input{una instancia del modelo GDM, $gdm$}
    \Output{un modelo lógico orientado a documentos, $ddm$}
    $mainEntities \gets gdm.queries.collect((q)|q.from);$\\
    \ForEach{$me \in mainEntities$}{
        $collection \gets new Collection();$\\
        $collection.name \gets me.name;$\\
        $accessTree \gets allQueryPaths(me,gdm.queries);$\\
        $collection \gets populateDocumentType(collection.root,accessTree);$\\
        $ddm.collections.add(collection);$
    }
    
    \caption{Transformación del modelo conceptual GDM al modelo lógico orientado a documentos}
\end{algorithm}

Donde la función populateDocumentType() es otro algoritmo de la forma:

\begin{algorithm}[H]
    \SetKwInOut{Input}{Input}
    \SetKwInOut{Output}{Output}

    %\underline{function Euclid} $(a,b)$\;
    \Input{Un ``document type", $dt$}
    \Output{un nodo del arbol de acceso}
    $ nodeAttributes \gets node.entity.features.select(f|f.isTypeOf(Attribute))\; $\\
    $ nodeReferences \gets node.entity.features.select(f|f.isTypeOf(Reference))\; $\\
    \ForEach{$attr \in nodeAttributes$}{
        $pf \gets new PrimitiveField()\;$\\
        $pf.name \gets attr.name\;$\\
        $pf.type \gets attr.type\;$\\
        $dt.fields.add(pf)\;$\\
    }
    \ForEach{$ref \in nodeReferences$}{
        $targetNode \gets node.arcs.find(a|a.name = ref.name).target\;$\\
        \uIf{exists(targetNode)}{
            $baseType \gets new DocumentType()\;$\\
            $populateDocumentType(baseType,targetNode)\;$\\
        }
        \Else{
            $baseType \gets new PrimitiveField()\;$\\
            $baseType.type \gets findIdType(ref.entity)\;$\\
        }
        $baseType.name \gets ref.name\;$\\

        \uIf{$ref.cardinality$ == 1}{
            $dt.field.add(baseType)\;$\\
            
        }
        \Else{
            $arrayField \gets new ArrayField()\;$\\
            $arrayField.type \gets baseType\;$\\
            $dt.fields.add(arrayField)\;$
        }
        
    }
    \caption{Generar el contenido de un \textit{DocumentType} dado un árbol de acceso}
\end{algorithm}

\subsubsection{Modelo lógico NoSQL a modelo físico en MongoDB}


De acuerdo con la documentación de mongodb [https://docs.mongodb.com/manual/tutorial/write-scripts-for-the-mongo-shell/], se puede escribir un script para crear un esquema(schema)  de base de datos en un archivo con extensión .js y con el contenido de los datos en formato JSON valido. Para generar el schema completo se deben seguir los siguientes pasos:


\begin{enumerate}
    \item Crear un archivo con extensión .js.
    \item Abrir el archivo con los permisos de lectura y escritura para poder añadir contenido.
    \item Agregar al archivo una linea con la instrucción \textit{conn = new Mongo();}.
    \item Agregar una nueva linea al archivo .js con la instrucción \textit{db = conn.getDB(`DB\_NAME');} esto creara una base de datos con el nombre que se le indique en caso de que no exista.
    \item Agregar al archivo la instrucción \textit{db.dropDatabase()} para eliminar todos los registros del esquema de la base de datos en caso de que esta exista y así evitar datos erróneos.
    \item Agregar una linea con la instrucción \textit{db.createCollection(`COLLECTION\_N');} por cada entidad del GDM, donde \textit{COLLECTION\_N} es sustituido por el nombre de la entidad.
    \item Agregar una linea con el nombre de un elemento de la primera colección seguida de un signo igual (=) y un caracter de llave `\{'.
    \item Agregar una linea por cada atributo simple de la entidad del GDM con la instrucción \textit{``ATTR\_NAME'' : ``DEFAULT\_VALUE''}, sustituyendo \textit{ATTR\_NAME} por el nombre de cada atributo y \textit{DEFAULT\_VALUE} por un valor por defecto que corresponda al tipo de dato del atributo.
    \item Agrear una linea con el caracter `\}', para indicar el cierre del objeto JSON.
\end{enumerate}


Los pasos del 1 al 5 serían suficientes para generar un esquema de base de datos vació en mongoDB, solamente debe ejecutarse el archivo con el comando \textit{mongo file.js}, se puede comprobar que todo es correcto entrando a la consola de mongodb y revisar los esquemas disponibles con el comando \textit{show dbs;}.


Los incisos posteriores generan un objeto de prueba para cada colección de datos, esto se hace principalmente para visualizar un elemento en cada colección de lo contrario solamente se apreciarían colecciones vacias en mongoDB,y estas  pueden almacenar todo tipo de información siempre y cuando se trate de un objeto JSON valido.


Finalmente una vez verificada que la estrucuctura es correcta se deben truncar todas las colecciones y de esta manera tener un esquema de la base de datos NoSQL sin registros y empezar a almacenar los documentos. Esto se logra con la instrucción \textit{db.COLLECTION\_NAME.remove(\{\})} sustituyendo  \textit{COLLECTION\_NAME} por cada uno de los nombres de las colecciones.


\section{Conclusiones}

La factibilidad operativa permite predecir si es posible poner en marcha el sistema propuesto, aprovechando todos los beneficios que se ofrecen a todos los usuarios involucrados en ello.


La herramienta va dirigida a estudiantes de nivel medio o nivel superior que tengan un primer acercamiento a los modelos de bases de datos entidad-relación o relacional desde un enfoque conceptual, como es el caso en ESCOM, en la asignatura de Base de Datos; el sistema propuesto cuenta con una interfaz intuitiva para que el usuario final, los estudiantes, visualice, crear y editar un diagrama ER y las opciones que esta les brinde de manera comprensible.


Por lo explicado anteriormente, el sistema propuesto tiene una alta probabilidad de aceptación por parte de los usuarios finales al encontrarse en un entorno en el que se trabaja con \textit{software} continuamente, además del beneficio que aporta al plan de estudios actual al ofrecer una forma práctica de ver aplicado los conceptos adquiridos en la asignatura de Bases de Datos, el cual solo contempla un alcance hasta la normalización de bases de datos relacionales y tener una introducción a los modelos no relacionales NoSQL.


Un estudiante que ha cursado dicha asignatura se dará cuenta que el tiempo disponible durante el curso es limitado por la cantidad de módulos que pretende cubrir y en muchas ocasiones los docentes deben prescindir de ciertos temas para completar el temario.


Con la implementación de la propuesta de solución, los estudiantes que cursen la asignatura de Bases de Datos tendrán la oportunidad de conocer una opción más en cuanto a tecnologías de almacenamiento de datos para implementar en sus propios sistemas; de igual manera, los puede impulsar a solicitar la apertura de una asignatura optativa sobre los modelos de datos NoSQL si hay interés por estos temas.


Se concluye que el sistema propuesto tendrá un uso en la institución y un potencial beneficio para los estudiantes y los involucrados.


A continuación están las respuestas (con preguntas incluidas) de la sección~\ref{ref:sec-factibilidad}:


\begin{enumerate}
    \item ¿El sistema contribuye a los objetivos generales de la organización?\\ Sí, ya que la misión en ESCOM es formar profesionales líderes en saberes de ingeniería, tecnología y ciencias de la computación con una visión globalizada; así como contribuir con investigación y desarrollo tecnológico para el crecimiento del país; por lo que la propuesta de solución contribuye directamente a la visión de la ESCOM.
    \item ¿Se puede implementar el sistema dentro del cronograma y el presupuesto utilizando las tecnologías actuales? \\Es posible, tal como se muestra en la sección~\ref{ref:factibilidad-tecnica}, se cuenta con las tecnologías para el desarrollo del producto y el esfuerzo es ajustado para el equipo de desarrollo, pero queda en los límites del tiempo establecido en el cronograma; además de ser desarrollado con una metodología ágil que ofrece productos funcionales por cada iteración.
    \item ¿Se puede integrar el sistema con otros sistemas que se utilizan?\\ Sí, el sistema es integrable a otros sistemas por estar disponible en la web; por ejemplo, puede integrarse directamente en la asignatura de Bases de Datos como una opción más para el modelado de diagramas ER con la ventaja de tener el acercamiento a los modelos no relacionales de manera práctica.
\end{enumerate}


Como se ha mencionado en el documento, se ha considerado Scrum como metodología para desarrollar la propuesta de solución, porque el proyecto requiere de entregas regulares de su avance para realizar modificaciones con ayuda de la retroalimentación constante del cliente, en este caso las directoras del proyecto.


Esto otorga beneficios como poder responder con flexibilidad, adaptación a los requisitos de cliente, estrechar la relación con el mismo y mantener al equipo motivado con pequeñas entregas funcionales del producto; además, tomando en cuenta la experiencia del equipo, esta forma de trabajo permite mostrar avances funcionales en el producto en un periodo de tiempo corto para realizar una evaluación y en caso de ser requerido se sugieran cambios.


De la sección de algoritmos, para la validez estructural de un diagrama entidad-relación se hace uso del trabajo de Dullea~\cite{dullea_analysis_2003} para el análisis de las relaciones unarias y binarias; además, se proponen restricciones para las entidades, atributos y relaciones del modelo entidad-relación básico.


Del mapeo modelo entidad-relación básico a modelo relacional se ha optado por usar la propuesta de Elmasri~\cite{ramez_elmasri_fundamentos_nodate}; para la obtención del esquema SQL se ha decidido parsear el modelo lógico relacional obtenido por el algoritmo anterior y se indica en la sección~\ref{sec:esquema-sql} las reglas básicas para implementar el algoritmo en el DBMS MySQL.


Para el algoritmo de mapeo entre el modelo entidad-relación y el GDM se ha optado por proponer consultas similares al trabajo de Chebotko\cite{chebotko_big_2015} para generar las consultas del GDM; asimismo, las relaciones en el modelo entidad-relación son referencias en el GDM y una consulta válida en el modelo entidad-relación es toda consulta que permita llegar a un atributo de una entidad $Y$ desde una entidad $X$ recorriendo las rutas del diagrama entidad-relación.


Para generar el modelo lógico orientado a documentos a partir del GDM se hace uso del algoritmo propuesto por Alfonso de la Vega, en el que se generan árboles de acceso de las consultas en el GDM para generar los documentos anidados en el modelo lógico orientado a documentos.


Finalmente, para obtener el esquema de sentencias en MongoDB se da una serie de pasos en la sección~\ref{sec:logico-documentos-fisico} para implementar el algoritmo.
