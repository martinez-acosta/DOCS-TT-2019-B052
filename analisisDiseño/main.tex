% Análisis del diseño
% 4.1 Metodología
% 4.1.1 Scrum qué es scrum, para qué sirve y por qué lo elegimos, cómo lo vamos a aplicar
% 4.2 factibilidad si es viable o no
% 4.2.4 conclusion si es factible ténicamente por ser un trabajo escolar y tenemos el personal capacitado
% 4.3 analisis del sistema
% 4.3.1 historias de usuario
% 4.3.2 lista de producto
% 4.3.3 diagrama de actividades, similar al de Omar (el de omar es de funcional) en la presentación
% 4.3.4 Algoritmos para el desarrollo del sistema (Capitulo cinco adjuntar en cap 4 en la parte de análisis)
% 4.4 conclusiones generales

% Capítulo 5: Diseño del sistema
% 5.1 diagrama de clases
% 5.2 diagrama de la base de datos
% 5.3 arquitectura del sistema
% 5.4 diseño de la UI
% 5.5 conclusiones generales

% agregar anexo para ver prototipo funcional, se menciona lo del calendario, cómo acceder, agregar cómo se añade la seguridad inherente por las herramientas que usamos (aspectos del modelado relacional como la normalización).

% algoritmos



Aquí falta poner la introducción y el capítulo está organizado de la siguiente manera blablabla y tiene tal, tal y tal.


\section{Metodología}
De acuerdo con la Universidad Católica los Ángeles\cite{universidad_catolica_los_angeles_metodologidesarrollo_2020}, en el campo del desarrollo de \textit{software} hay dos grupos de metodologías: las tradicionales y las ágiles.


Las tradicionales se centran en cumplir con un plan rígido de trabajo establecido en la etapa inicial del proyecto, mientras que las ágiles permiten realizar cambios en los requerimientos conforme avance el mismo.


Dado que cualquier cambio en el proceso de una metodologia tradicional genera la necesidad de una reconstrucción del plan de trabajo (invirtiendo tiempo que se podría usar para desarrollar), surgieron las metodologías ágiles, que permiten realizar cambios en los requerimientos conforme avance el proyecto. 


Tomando en cuenta la experiencia del equipo, esta forma de trabajo permite mostrar avances funcionales en el producto en un periodo de tiempo corto para realizar una evaluación y en caso de ser requerido se sugieran cambios.


Se han propuesto muchos modelos ágiles de proceso y están en uso en toda la industria; entre ellos se encuentran los siguientes:


\begin{itemize}
	\item DAS (Desarrollo Adaptativo de Software): ASD tiene como fundamento la teoría de sistemas adaptativos complejos; por ello, interpreta los proyectos de \textit{software} como sistemas adaptativos complejos compuestos
    por agentes (los interesados), entornos (organizacional o tecnológico) y salidas (el producto desarrollado)\cite{cadavid_revision_2013}.
	\item Scrum: La metodología Scrum para el \textit{desarrollo} ágil de software es un marco de trabajo diseñado para lograr la colaboración eficaz de equipos en proyectos, que emplea un conjunto de reglas y artefactos y define roles que generan la estructura necesaria para su correcto funcionamiento\cite{cadavid_revision_2013}.
	\item MDSD (Método de Desarrollo de Sistemas Dinámicos): DSDM es un marco de trabajo creado para entregar la solución correcta en el momento correcto; utiliza un ciclo de vida iterativo, fragmenta el proyecto en periodos cortos de tiempo y define entregables para cada uno de estos periodos; tiene roles claramente definidos y especifica su trabajo dentro de periodos de tiempo\cite{cadavid_revision_2013}.
	\item Crystal: la filosofía de Crystal define el desarrollo como un juego cooperativo de invención y comunicación cuya meta principal es entregar \textit{software} útil, que funcione y su objetivo secundario es preparar el próximo juego\cite{cadavid_revision_2013}.
\end{itemize}

\subsection{Scrum}

De acuerdo con Ken Schwaber\cite{the_scrum_guide_definitive_2020}, Scrum es un marco de trabajo para la entrega de productos incrementales y de máximo valor productivo.

Un artefacto es un elemento que garantiza la transparencia, es el registro de la información fundamental del proceso Scrum y a continuación se describen sus cuatro artefactos principales:

\begin{itemize}
	\item Lista de producto (\textit{product backlog}):	es el listado de todas las tareas que necesita el proyecto para alcanzar su realización; al iniciar el desarrollo del proyecto, esta lista no se encuentra completa y conforme avanzan los \textit{sprints} se le añaden tareas para solventar las necesidades que van surgiendo gracias a la retroalimentación del cliente.
	\item Lista de pendientes del \textit{sprint} (\textit{sprint backlog}): es la lista de tareas seleccionadas del \textit{product backlog} que se planifica realizar durante el periodo del \textit{sprint} y define a los responsables de cada tarea.
	\item \textit{Sprint}: es el corazón de Scrum, tiene un periodo de tiempo determinado de un mes o incluso menos donde el equipo completa conjuntos de tareas incluidas en el \textit{backlog} para crear un incremento del producto utilizable.
	\item Incremento: es la suma de todos los elementos de la lista de productos completados durante un \textit{sprint} unido con los incrementos de los \textit{sprints} anteriores; al finalizar el \textit{sprint}, el nuevo incremento debe estar en condiciones de ser utilizado.
\end{itemize}

El equipo Scrum (\textit{scrum team}) consiste en los siguientes roles:
\begin{itemize}
	\item El dueño de producto (\textit{product owner}): es la persona responsable de maximizar el valor del producto y el trabajo del equipo de desarrollo; es el único responsable de gestionar la lista de producto y cualquier cambio a esa lista debe ser revisada y aprobada por él.
	\item El equipo de desarrollo (\textit{development team}): son los profesionales que realizan el trabajo para la entrega de un incremento en el producto en cada \textit{sprint}; es un grupo autoorganizado y multifuncional donde cada miembro del equipo tiene habilidades especializadas, pero que la responsabilidad de las tareas completadas, incrementos del producto o retrasos recaen en el equipo como un todo.
	\item  El Scrum master: es la persona responsable de asegurar que Scrum es entendido y adoptado por todos los involucrados en el proyecto, asegurándose de ayudar a las personas externas al equipo a entender qué interacciones son útiles con el equipo de desarrollo.
\end{itemize}


Scrum tiene cuatro eventos principales en  un \textit{sprint}, que sirven para la inspección y adaptación del producto que se describen a continuación:
\begin{itemize}
	\item Planificación del \textit{sprint} (\textit{sprint planning}): es una reunión con el equipo de desarrollo que tiene una duración máxima de 8 horas para el \textit{sprint} de un mes; el Scrum Master es el encargado de que los asistentes entiendan el propósito de dicha reunión.
	\item Scrum diario (\textit{daily scrum}): es una reunión de un máximo de 15 minutos en la cual el equipo expone sus actividades y planifica las tareas de las próximas 24 horas.
	\item Revisión del \textit{sprint} (\textit{sprint review}): al finalizar cada \textit{sprint} se lleva a cabo una reunión para la revisión del incremento del producto y en caso de ser necesario realizar ajustes a la lista de producto.
	\item Retrospectiva del \textit{sprint} (\textit{sprint retrospective}): es cuando el equipo de desarrollo tiene la oportunidad de pensar en mejoras para el próximo \textit{sprint}.
\end{itemize}

por qué Scrum?

cómo vamos a aplicar Scrum?

Como la metodología permite definir un periodo de hasta un mes para cada \textit{sprint}, se ha optado por un periodo de 30 días, contemplándose un total de ocho \textit{sprints}, donde al término de cada uno se tendrá un avance del sistema.
\section{Factibilidad}

De acuerdo con Sommerville\cite{sommerville_software_2011}, un estudio de factibilidad es un estudio breve que responde tres preguntas clave:

\begin{enumerate}
    \item ¿El sistema contribuye a los objetivos generales de la organización? la respuesta es si, ya que la organización centra su misión en formar profesionales en ingeniería, tecnologías y ciencias de la computación, para lograrlo de mantenerse a actualizado en las tecnologías emergentes y ofrecer a sus estudiantes un mayor abanico de opciones para su desarrollo profesional.
    \item ¿Se puede implementar el sistema dentro del cronograma y el presupuesto utilizando las tecnologías actuales? claro que es posible como se muestra en la factibilidad tecnica se cuenta con las tecnologías para el desarrollo del producto y el esfuerzo es ajjstado para el equipo de desarrollo pero queda en los limites del tiempo establecido en el cronograma, ademas de ser desarrollado con una metodología agíl lo que ofrece productos funcionales por cada iteración.
    \item ¿Se puede integrar el sistema con otros sistemas que se utilizan? esto es perfectamente viable, el sistema puede ser adaptado a otros sistemas por la facilidad de estar disponible en la web tiene una gran opoprtinidad para comunicarse con otros sistemas disponibles en la organización, por ejemplo puede integrarse directamente en la asignatura de base de datos como una opción mas para el modelado de diagramas ER con la ventaja de tener el acercamiento a los modelos no relacionales de manera practica.
\end{enumerate}


Para responder estas preguntas se expone la factibilidad técnica, factibilidad económica, costos de desarrollo y conclusiones.

\subsection{Factibilidad técnica}\label{ref:factibilidad-tecnica}

De acuerdo con Pressman\cite{pressman_software_2005}, este estudio determina si el equipo de desarrollo cuenta con los recursos técnicos necesarios para la realización del sistema propuesto; esto se realiza considerando la disponibilidad de los recursos tanto de \textit{hardware}, \textit{software} y recurso humano.

\subsubsection*{Sistema operativo}

De acuerdo con Stallings\cite{stallings_operating_2012}, un sistema operativo es el \textit{software} principal o conjunto de programas de un sistema informático que gestiona los recursos de \textit{hardware} y provee servicios a los programas de aplicación de \textit{software}, ejecutándose en modo privilegiado respecto de los restantes.


Este es un elemento importante, ya que debe cumplir con las características de estabilidad, velocidad, seguridad y escalabilidad para soportar la instalación del sistema.


A continuación se presentan diferentes sistemas operativos que cumplen con las características mencionadas y que son suficientes para albergar el sistema:

\begin{itemize}
    \item Windows: es un grupo de varias familias de sistemas operativos gráficos patentados, son desarrollados y comercializados por Microsoft; cada familia atiende a un determinado sector de la industria informática\cite{wilson_about_2015}.
    \item GNU/Linux: es una familia de sistemas operativos tipo Unix de código abierto basados en el núcleo de Linux creado por Linus Torvalds\cite{love_linux_2010}.
\end{itemize}

El sistema operativo elegido para el desarrollo de la propuesta de solución es GNU/Linux porque es de libre acceso, es gratuito y los integrantes del equipo tienen experencia con este sistema.


\subsubsection*{Lenguaje de desarrollo}

De acuerdo con Aaby\cite{aaby_introduction_1996}, un lenguaje de programación es un lenguaje formal (es decir, un lenguaje con reglas gramaticales definidas) que le proporciona a una persona, en este caso el programador, la capacidad de escribir una serie de instrucciones o secuencias de órdenes en forma de algoritmos con el fin de controlar el comportamiento físico o lógico de una computadora, de manera que es posible obtener diversas clases de datos o ejecutar determinadas tareas.


Se valora que el lenguaje de programación para el desarrollo de la propuesta de solución debe tener soporte para conexión a base de datos, sea posible usarlo para el desarrollo, sea vigente (que esté en continua mejora), sea fácil de administrar y se integre con algún \textit{framework} web.


A continuación se presenta una lista de lenguajes de desarrollo que cumplen dichas características:

\begin{itemize}
    \item Java: es un lenguaje de programación de propósito general que está basado en clases, orientado a objetos y diseñado para tener la menor cantidad posible de dependencias de implementación; su objetivo es permitir seguir el  \textit{write once, run anywhere}\cite{joy_java_2000}.
    \item Python: es un lenguaje de programación interpretado, de alto nivel y de propósito general; la filosofía de diseño de Python enfatiza la legibilidad del código con su uso notable de espacios en blanco significativos; Sus construcciones de lenguaje y su enfoque orientado a objetos tienen como objetivo ayudar a los programadores a escribir código claro y lógico para proyectos de pequeña y gran escala\cite{van_rossum_python_2007}.
    \item C\#: es un lenguaje de programación multiparadigma de propósito general como imperativo, declarativo, funcional, genérico, orientado a objetos y orientado a componentes\cite{hejlsberg_c_2003}.
    \item JavaScript: es un lenguaje de programación interpretado, de alto nivel y de propósito general; se escribe dinámicamente; es compatible con múltiples paradigmas de programación, incluida la programación orientada a objetos\cite{noauthor_javascript_nodate}.
\end{itemize}

Los lenguajes de programación Python y JavaScript se han elegido para desarrollar la propuesta de solución; para más detalles revise la sección~\ref{ref:conclusiones-cap3}

\subsubsection*{Sistema gestor de base de datos}

De acuerdo con Connolly\cite{connolly_database_2005}, es un sistema de \textit{software} que permite a los usuarios definir, crear, mantener y controlar el acceso a la base de datos.


Este es un factor muy importante, ya que determinará cómo se almacenará la información del sistema, por lo tanto debe ser escalable, seguro, contar con soporte para grandes cantidades de información y soporte para conexión con distintos lenguajes de programación.


A continuación se presenta una lista de sistemas gestores de bases de datos que cumplen dichas características:

\begin{itemize}
    \item MySQL: es un sistema de gestión de bases de datos relacionales de código abierto\cite{dubois_mysql_1999}.
    \item MongoDB: es un gestor de bases de datos orientado a documentos y utiliza documentos similares a JSON con esquemas opcionales\cite{banker_mongodb_2011}.
\end{itemize}

El gestor de base de datos que se ha elegido para desarrollar la propuesta de solución es MongoDB; para más detalles revise la sección~\ref{ref:databases}.


La tabla~\ref{tab:hw_devices} muestra las características de las computadoras con los que el equipo de desarrollo dispone; la primera columna hace referencia al número de computadora del que se dispone; la siguiente hace referencia a qué elementos contiene, la tercera columna sus especificaciones y la cuarta columna el costo de cada computadora.

\begin{table}[H]
    \centering
    \begin{tabular}{|c|c|c|c|}
        \hline
        Equipos & Elementos & Especificaciones & Costo \\ \hline
        \multirow{3}{*}{Laptop 1} & Memoria RAM & 8 GB & \\
        & Almacenamiento & 500 GB HDD & 22 500.00 mxn  \\
        & Procesador & Intel Core i5 6ta gen. & \\ \hline
        \multirow{3}{*}{Laptop 2} & Memoria RAM & 8 GB & \\
        & Almacenamiento & 256 GB SSD & 32 000.00 mxn \\
        & Procesador & Intel Core i5 8va gen. & \\ \hline
    \end{tabular}
    \caption{Computadoras con las que se cuenta}
    \label{tab:hw_devices}
\end{table}


Con los datos que están en la tabla~\ref{tab:hw_devices}, se concluye que la tecnología para el desarrollo del sistema existe y se cuenta con los recursos de \textit{hardware} suficientes para iniciar con su implementación.
\subsection{Factibilidad económica}

De acuerdo con Pressman\cite{pressman_software_2005}, el punto de función es una ``unidad de medida'' para expresar la cantidad de funcionalidad comercial que un sistema de información (como producto) proporciona a un usuario; los puntos de función se utilizan para calcular una medición de tamaño funcional (FSM) de \textit{software}.

Como señala Pressman en su libro, se utiliza una métrica por puntos de función para realizar una estimación del costo total del proyecto, incluyendo los salarios de los desarrolladores que harán la implementación del sistema, así como los gastos por pagos de servicios que sean necesarios como se muestra en la tabla \ref{tab:function_point_metrics}. 

\subsubsection*{Métricas orientadas a la función}


Para este proyecto, se considera que todas las funciones identificadas son de complejidad media con excepción de las entradas que tienen la complejidad más alta del sistema.


\begin{table}[H]
	\centering
	\begin{tabular}{|l|l|l|l|l|l|}
	\hline
	\multirow{2}{*}{Parámetro} & \multirow{2}{*}{Cuenta} & \multicolumn{3}{|l|}{Factores de ponderación} & \multirow{2}{*}{Total} \\ \cline{3-5}
														 &                         & baja       	& media       & alta		       &                        \\ \hline
	Entradas                   & 6                       & 3            & 4           & 6              & 36                     \\ \hline
	Salidas                     & 5                       & 4            & 5           & 7              & 25                     \\ \hline
	Tablas                     & 1                       & 3            & 4           & 6              & 4                      \\ \hline
	Interfaces                 & 4                       & 7            & 10          & 15             & 40                     \\ \hline
	Consultas                  & 4                       & 5            & 7           & 10             & 28                     \\ \hline
	Conteo total               &                         &              &             &		             & 133                    \\ \hline
	\end{tabular}
	\caption{Cálculo de las métricas por puntos de función}
	\label{tab:function_point_metrics}
	\end{table}


	\textbf{Fi (i = 1..14)} son factores de ajuste de valor basados en las respuetas de las preguntas de la tabla \ref{tab:questions_adjusment}. Los valores pueden ir de 0 (no importante o aplicable) a 5 (absolutamente esencial).

	\begin{table}
		\begin{tabular}{|p{9cm}|c|}
		\hline
		Pregunta                                                                                                                 & Ponderación \\ \hline
		¿Requiere el sistema métodos de seguridad y recuperación fiables?                                                       & 3           \\ \hline
		¿Se requiere comunicación especializada?                                                                              & 5           \\ \hline
		¿Existen funciones de procesamiento distribuido?                                                                         & 2           \\ \hline
		¿Es crítico el rendimiento?                                                                                              & 4           \\ \hline
		¿Se ejecutará el sistema en un entorno operativo existente y fuertemente utilizado?                                      & 4           \\ \hline
		¿Requiere el sistema una entrada de datos interactiva?                                                                    & 5           \\ \hline
		¿Requiere la entrada de datos interactiva que las transacciones de entrada se lleven a cabo sobre múltiples operaciones? & 5           \\ \hline
		¿Se actualizan los archivos maestros de forma interactiva?                                                               & 3           \\ \hline
		¿Son complejas las entradas, salidas, archivos o consultas?                                                               & 4           \\ \hline
		¿Es complejo el procesamiento interno?                                                                                   & 4           \\ \hline
		¿Se ha diseñado el código para ser reutilizable?                                                                         & 4           \\ \hline
		¿Están incluidas en el diseño la instalación y conversión?                                                               & 3           \\ \hline
		¿Se ha diseñado el sistema para soportar múltiples instalaciones en diferentes organizaciones?                           & 4           \\ \hline
		¿Se ha diseñado el sistema para facilitar los cambios y para ser fácilmente utilizable?                                  & 4           \\ \hline
		\centering $\sum Fi=$                                                                                                  	 & 54          \\ \hline
		\end{tabular}
		\caption{Factores de ajuste}
		\label{tab:questions_adjusment}
		\end{table}


\subsubsection*{Puntos de función}

La fórmula para obtener los puntos de función con los factores de ajuste es la siguiente:

\begin{equation} \label{eq:cap4-00.01}
	\mathrm{PF} = \mathrm{conteo\ total} *  (0.65 + (0.01 * \sum F_i)) 
\end{equation}
	
De \eqref{eq:cap4-00.01} se deben sustituir los valores del conteo total y los factores de ajuste:

\begin{equation} \label{eq:cap4-00.02}
	\mathrm{PF} =  133 *  (0.65 + (0.01 * 54))
\end{equation}
\begin{equation} \label{eq:cap4-00.03}
	\mathrm{PF} = 158.27 \approx 159
\end{equation}
	
		
De lo anterior, aproximadamente se obtienen \textbf{159} puntos de función; una vez obtenidos utilizando la llamada ``Ball-Park'' o ``Estimación Indicativa'', que es la técnica de macro-estimación que se utiliza en situaciones de falta de información sobre el proyecto.


De acuerdo con Carper\cite{abran_applied_2006}, la siguiente ecuación determina el esfuerzo de desarrollo de un proyecto:


\begin{equation} \label{eq:cap4-01}
	\mathrm{Esfuerzo} = (\frac{\mathrm{PF}}{150})*\mathrm{PF}^{0.4} 
\end{equation}
donde \eqref{eq:cap4-01} $\mathrm{PF}$ son puntos de función; al sustituir los valores en \eqref{eq:cap4-01}:

\begin{equation} \label{eq:cap4-02}
	\mathrm{Esfuerzo} = (\frac{159}{150})*159^{0.4}
\end{equation}
\begin{equation} \label{eq:cap4-03}
	\mathrm{Esfuerzo} = 8.05 \; \mathrm{meses}
\end{equation}
	



Los 8.05 meses de \eqref{eq:cap4-03}; considerando un total de 40 horas a la semana de trabajo y 4.34 semanas por mes, el total de horas para el desarrollo y conclusión del proyecto se obtiene de esta manera:

\begin{equation} \label{eq:cap4-04}
	\mathrm{Tiempo\; de\; desarrollo} =  40 * 4.34 * 8.05 
\end{equation}

\begin{equation} \label{eq:cap4-05}
	\mathrm{Tiempo\; de\; desarrollo} =  1397.48 \approx 1398\;  \mathrm{horas}
\end{equation}

	



Por ejemplo, una sola persona trabajando en el desarrollo del proyecto debería invertir 1398 horas con una jornada de 8 horas diarias de lunes a viernes hasta su finalización, por lo que si un equipo de desarrollo es de 2 personas con un horario de lunes a viernes de 4 horas diarias, el proyecto concluiría en 8.05 meses.


\subsection{Costos de desarrollo}

De acuerdo con \textit{Software Guru}\cite{pedro_galvan_estudio_2020}, en una publicación que recopila los datos de salarios en el área de desarrollo de \textit{software} para febrero de 2020, un desarrollador con 0 a 2 años de experiencia, como es el caso de un estudiante, tiene en un salario de \$ 15 000 mensuales en una jornada completa, considerando que este proyecto contempla jornadas de medio tiempo (4 horas) de lunes a viernes, se reduce la cifra antes mencionada. Teniendo en cuenta estos datos y un periodo de 9 meses, que es el tiempo aproximado de duración del proyecto, el costo total por salarioss para el equipo de desarrollo está desglosado en la tabla \ref{tab:devs_salary}.

\begin{table}[H]
    \centering
    \begin{tabular}{|c|c|c|c|}
    \hline
        Concepto & Costo Aprox. Semanal & Costo Aprox. Mensual & Monto total \\ \hline
        Salario & 1850 & 7500 & 67500 \\ \hline
    \end{tabular}
    \caption{Costos del personal}
    \label{tab:devs_salary}
\end{table}


Considerado a 2 personas en el equipo de desarrollo y un periodo de 9 meses(se agrego un mes mas para el caso de estudio) utilizando los salarios de la tabla \ref{tab:devs_salary} tenemos que los gastos totales se obtienen con la siguiente fórmula:

\begin{equation} \label{eq:cap4-07}
	\mathrm{salarios} =  \mathrm{No\; de\; integrantes} * \mathrm{salario/mes} * \mathrm{tiempo\; de\; desarrollo}  {pesos mexicanos}
\end{equation}

sustituyendo los calores en \eqref{eq:cap4-07}

\begin{equation} \label{eq:cap4-08}
	\mathrm{salarios} =  2 * 7500 * 9  {pesos mexicanos}
\end{equation}


Esto da como resultado un total final de \$ 135 000 por los salarios de los 2 integrantes del equipo de desarrollo. Se tomaron en cuenta 9 meses para todos los gastos, un mes extra a lo obtenido en estimación para utilizarse en el caso de estudio del sistema una vez concluido.

Los gastos por pagos de licencia de \textit{software} quedan excluidos, ya que las tecnologías seleccionadas son libres o gratuitas, lo cual no supone un costo para su uso. De igual manera, esto se encuentra simplificado en la tabla \ref{tab:sw_licences}.

\begin{table}
    \centering
    \begin{tabular}{|c|c|c|c|}
    \hline
        Software & Licencia & Costo \\ \hline
        Visual Studio Code  & MIT & 0  \\ \hline
        Gunicorn(flask Server) & MIT & 0 \\ \hline
        MongoDB Atlas & Apache v2 & 0 \\ \hline
    \end{tabular}
    \caption{Costos por licencias de software}
    \label{tab:sw_licences}
\end{table}

Otros gastos necesarios son los pagos por servicios requeridos listados en la tabla \ref{tab:services_costs}.

\begin{table}
    \centering
    \begin{tabular}{|c|c|c|c|}
    \hline
        Concepto & Costo Mensual & Monto total \\ \hline
        Luz & 250 & 2250  \\ \hline
        Internet & 349 & 3141 \\ \hline
        Heroku hosting & 0 (free plan) & 0 \\ \hline
        Netlify hosting & 0 (free plan) & 0 \\ \hline
    \end{tabular}
    \caption{Costos por servicios}
    \label{tab:services_costs}
\end{table}

Habiendo realizado la suma de todas las cantidades antes mencionadas, el total final se obtine de la siguiente manera:

\begin{equation} \label{eq:cap4-09}
    \mathrm{servicio_totales} = \mathrm{servicios_por_mes} * \mathrm{tiempo_de_desarrollo} {pesos mexicanos}
    \mathrm{servicio_totales} = 5391 * 9 {pesos mexicanos}
    \mathrm{servicio_totales} = 48 519.00 {pesos mexicanos}
\end{equation}

\begin{equation} \label{eq:cap4-10}
    \mathrm{Gastos\; totales} = \mathrm{salarios} + \mathrm{servicios\; totales} + \mathrm{costos_de_equipo_de_compumto}
    \mathrm{Gastos\; totales} = 135 000 + 48 519 + 54 500
    \mathrm{Gastos\; totales} = 238 019 {pesos mexicanos}
\end{equation}

\begin{center}

	salarios = 135 000.00
	servicios = (2250 + 3141) * 9 = 48 519.00

	Gastos totales = 135 000.00 + 48 519.00 = 183 519.00 pesos mexicanos.

\end{center}



%\subsection{Factibilidad operativa}
\subsection{Conclusiones}


La factibilidad operativa permite predecir si es posible poner en marcha el sistema propuesto, aprovechando todos los beneficios que se ofrecen a todos los usuarios involucrados en ello. La herramienta va dirigida a los estudiantes que se encuentren en un primer acercamiento a los modelos de bases de datos entidad-relación o relacional desde un enfoque conceptual, buscando principalmente mostrarles una aproximación a los modelos no relacionales de bases de datos. El sistema propuesto cuenta con una interfaz intuitiva para que el usuario final, los estudiantes, puedan visualizar, crear y editar un diagrama ER y las opciones que esta les brinde de manera comprensible.


Teniendo en cuenta los motivos anteriormente explicados, el sistema propuesto tiene una alta probabilidad de aceptación por parte de los usuarios finales al encontrarse en un entorno en el que se trabaja con \textit{software} continuamente, además del beneficio que aporta al plan de estudios actual al ofrecer una forma práctica de ver aplicado los conceptos adquiridos en el curso de base de datos, el cual solo contempla un alcance hasta la normalización de bases de datos relacionales y tener una introducción a los modelos no relacionales (noSQL). Un estudiante que ha cursado dicha asignatura se dará cuenta que el tiempo disponible durante el curso es limitado por la cantidad de módulos que pretende cubrir y en muchas ocasiones los docentes deben prescindir de ciertos temas para completar el temario.

Con la implantación de la aplicación web que se está proponiendo, los estudiantes que cursen la asignatura de de base de datos tendrán la oportunidad de conocer una opción más en cuanto a tecnologías de almacenamiento de datos para implementar en sus propios sistemas. De igual manera, puede impulsarlos a generar propuestas para la apertura de una asignatura optativa si el interes por estos modelos de datos resulta interesante para ellos.

Teniendo en cuenta los puntos mencionados anteriormente, se concluye que el sistema propuesto tendrá un uso en la institución y un potencial beneficio para los estudiantes y los involucrados en ello.

\begin{enumerate}
    \item ¿El sistema contribuye a los objetivos generales de la organización? la respuesta es si, ya que la organización centra su misión en formar profesionales en ingeniería, tecnologías y ciencias de la computación, para lograrlo de mantenerse a actualizado en las tecnologías emergentes y ofrecer a sus estudiantes un mayor abanico de opciones para su desarrollo profesional.
    \item ¿Se puede implementar el sistema dentro del cronograma y el presupuesto utilizando las tecnologías actuales? claro que es posible como se muestra en la factibilidad tecnica se cuenta con las tecnologías para el desarrollo del producto y el esfuerzo es ajjstado para el equipo de desarrollo pero queda en los limites del tiempo establecido en el cronograma, ademas de ser desarrollado con una metodología agíl lo que ofrece productos funcionales por cada iteración.
    \item ¿Se puede integrar el sistema con otros sistemas que se utilizan? esto es perfectamente viable, el sistema puede ser adaptado a otros sistemas por la facilidad de estar disponible en la web tiene una gran opoprtinidad para comunicarse con otros sistemas disponibles en la organización, por ejemplo puede integrarse directamente en la asignatura de base de datos como una opción mas para el modelado de diagramas ER con la ventaja de tener el acercamiento a los modelos no relacionales de manera practica.
\end{enumerate}
\section{Análisis del sistema}
De acuerdo con Pressman\cite{pressman_software_2005}, las condiciones del mercado cambian con rapidez, clientes y usuarios finales necesitan cambios constantes por nuevas amenazas competitivas; por ello los profesionales deben enfocar la ingeniería de \textit{software} en forma que les permita mantenerse ágiles para definir procesos maniobrables, adaptativos y esbeltos que satisfagan las necesidades de los negocios modernos.


Una filosofía ágil para la ingeniería de \textit{software} pone el énfasis en cuatro aspectos clave: la importancia de los equipos con organización propia que tienen el control sobre el trabajo que realizan, la comunicación y colaboración entre los miembros del equipo, profesionales y sus clientes, el reconocimiento de que el cambio representa una oportunidad y la insistencia en la entrega rápida de \textit{software} que satisfaga al consumidor.


\subsection{Historias de usuario}\label{sec:historias-usuario}

De acuerdo con Scrum México\cite{scrum_mexico_scrum_2020}, las historias de usuario conforman la técnica por la que el usuario especifica de manera general los requerimientos que el sistema debe cumplir.


Normalmente estas redacciones se llevan a cabo en tarjetas de papel donde se describen brevemente las funciones que el producto final debe poseer, ya sean requisitos funcionales o no.


El tratamiento de las historias de usuario es flexible y dinámico, cada una de ellas es lo suficientemente detallada y delimitada para que el equipo de desarrollo implemente durante la duración del \textit{sprint}.


Es habitual que se siga una plantilla para estas tarjetas, como la que se expone a continuación:

\begin{itemize}
	\item Como \textbf{<Usuario>}
	\item Quiero \textbf{<algún objetivo>}
	\item Para \textbf{<motivo>}
\end{itemize}


Una de sus grandes ventajas, dado el caso de que un usuario no sea lo suficientemente detallista con la historia, es que esta se puede partir en historias más pequeñas antes de que el equipo empiece a trabajar en ella.


Este es un ejemplo de historia de usuario para el desarrollo:

\begin{itemize}
	\item Como usuario,
	\item quiero ingresar al sistema con mi correo y contraseña
	\item para tener acceso a sus funciones.
\end{itemize}


Otra forma de darle detalle a las historias de usuario es mediante el añadido de un criterio de aceptación; un criterio de aceptación es una prueba que será cierta cuando el equipo de desarrollo complete la descripción de la tarjeta.


A continuación se listan las principales historias de usuario que se consideraron para el desarrollo de la propuesta de solución; tenga en cuenta que algunas de ellas tienen criterios de aceptación, pero en otras no se consideró necesarias porque son explícitamente claras.


\noindent\rule{\textwidth}{1pt}
\begin{itemize}
	\item N.° 1
	\item Como usuario,
	\item quiero ingresar al sistema con mi correo y contraseña,
	\item para tener acceso a sus funciones.
	\item Criterios de aceptación:
	\begin{itemize}
		\item El usuario recibirá un correo electrónico de confirmación de su alta en el sistema con el correo y contraseña que ingresó para tenerlos de respaldo.
	\end{itemize}
\end{itemize}
\noindent\rule{\textwidth}{1pt}
\begin{itemize}
	\item N.° 2
	\item Como usuario,
	\item quiero recuperar mi contraseña en caso de olvidarla,
	\item para no perder el trabajo realizado en el sistema.
	\item Criterios de aceptación:
	\begin{itemize}
		\item El usuario podrá ingresar una nueva contraseña siempre y cuando recuerde el correo electrónico con el que se dio de alta en el sistema.
		\item Al ingresar una nueva contraseña, recibirá un correo de confirmación del cambio de contraseña y sus datos permanecerán intactos.
	\end{itemize}
\end{itemize}
\noindent\rule{\textwidth}{1pt}
\begin{itemize}
	\item N.° 3
	\item Como usuario del sistema, quiero darme de alta con una contraseña fácil de recordar,
	\item pero que esté segura en la base de datos,
	\item para no tener comprometidos los diagramas que genere en el sistema.

	\item Criterios de aceptación:
	\begin{itemize}
		\item Asegurarse que el usuario ingrese una contraseña de al menos 8 caracteres.
		\item Se le solicitará al usuario que ingrese 2 veces la misma contraseña para asegurarse que le es fácil recordarla y que efectivamente es la misma.
		\item Antes de guardar la contraseña, esta deberá pasar por un método que la haga ilegible para el usuario (algún algoritmo de digestión o cifrado).
	\end{itemize}
\end{itemize}
\noindent\rule{\textwidth}{1pt}
\begin{itemize}
	\item N.° 4
	\item Como usuario quiero crear un diagrama ER arrastrando y soltando elementos de una ``paleta'',
	\item para hacerlo de manera más fácil e intuitiva.
	\item Criterios de aceptación:
	\begin{itemize}
		\item El usuario podrá empezar un nuevo diagrama al seleccionar la opción de diagramador ER.
		\item Tendrá a su disposición una paleta con los elementos permitidos en un diagrama ER básico.
		\item Podrá arrastrar y soltar los elementos de la paleta a un área delimitada para empezar con el diseño de su diagrama.
	\end{itemize}
\end{itemize}
\noindent\rule{\textwidth}{1pt}
\begin{itemize}
	\item N.° 5
	\item Como usuario quiero guardar mi último trabajo realizado en el diagramador ER,
	\item para poder consultarlo en otro momento.
	\item Criterios de aceptación:
	\begin{itemize}
		\item Dispondrá de un botón para poder guardar en la base de datos el diagrama que esté creando o editando.
		\item Antes de almacenar el diagrama en el \textit{canvas} o zona de diagramado, se le mostrará un mensaje de confirmación para guardar su diagrama actual.
	\end{itemize}
\end{itemize}
\noindent\rule{\textwidth}{1pt}
\begin{itemize}
	\item N.° 6
	\item Como usuario me gustaría poder ver el último trabajo que realice
	\item cuando seleccione la opción ``Entidad-Relación'',
	\item para poder modificar el diseño.
\end{itemize}
\noindent\rule{\textwidth}{1pt}
\begin{itemize}
	\item N.° 7
	\item Como usuario quiero tener la opción de cargar un diagrama a partir de un archivo,
	\item para hacer modificación de dicho diagrama y guardarlo de ser necesario.
	\item Criterios de aceptación:
	\begin{itemize}
		\item El usuario tendra un botón ``cargar'' en el menú del diagramador ER para poder importar un archivo con extensión .json.
		\item Al importar el archivo, este pasará por un proceso de validación para asegurarse que es un archivo .json válido.
		\item Durante el proceso de validación, se verificará que el contenido del archivo es un diagrama compatible con la estructura de los generados por el diagramador ER.
		\item Al contener información compatible, se mostrará en la zona de diagramado el contenido del archivo.
	\end{itemize}
\end{itemize}
\noindent\rule{\textwidth}{1pt}
\begin{itemize}
	\item N.° 8
	\item Como usuario quiero descargar el diagrama que esté visible en la página web,
	\item para poder distribuirlo como yo desee.
	\item Criterios de aceptación:
	\begin{itemize}
		\item El usuario dispondrá de un botón ``Descargar'' en el diagramador ER para obtener un archivo con el contenido del diagrama visible en la zona de diagramado.
		\item El archivo generado será de extensión .json con la información necesaria para que el diagramador lo cargue en otro momento.
	\end{itemize}
\end{itemize}
\noindent\rule{\textwidth}{1pt}
\begin{itemize}
	\item N.° 9
	\item Como usuario quiero tener una forma de validar mi diagrama ER,
	\item porque es importante saber si el diagrama que estoy creando es un diagrama válido del modelo ER.
	\item Criterios de aceptación:
	\begin{itemize}
		\item El usuario tendrá disponible un botón que al darle clic iniciará un proceso de validación del diagrama actual en la zona de diagramado.
		\item Al término del proceso de validación, se le mostrará un mensaje al usuario indicando si el diagrama cumple o no las reglas del modelo ER.
	\end{itemize}
\end{itemize}
\noindent\rule{\textwidth}{1pt}
\begin{itemize}
	\item N.° 10
	\item Como usuario, en caso de tener un diagrama ER válido,
	\item me gustaría poder tranformar el diagrama ER en su versión del modelo relacional,
	\item para poder ver el equivalente del diagrama ER en el modelo Relacional.
	\item Criterios de aceptación:
	\begin{itemize}
		\item El usuario dispondrá de la opción de tranformar al modelo relacional solamente después de haber validado que el diagrama ER cumple con las reglas.
		\item Posterior a la validación, se le mostrará al usuario un mensaje de confirmación y un botón para disparar el proceso de transformación a su equivalente relacional.
		\item Al terminar el proceso de transformación equivalente, se le redirijira al menú ``Relacional'' donde podrá visualizar el equivalente al modelo relacional.
	\end{itemize}
\end{itemize}
\noindent\rule{\textwidth}{1pt}
\begin{itemize}
	\item N.° 11
	\item Como usuario, después de observar el diagrama relacional,
	\item quiero obtener las sentencias SQL equivalentes,
	\item para poder crear el esquema de base de datos relacional en un DBMS.
	\item Criterios de aceptación:
	\begin{itemize}
		\item Las sentencias SQL solo podrán ser descargadas en el menú ``Relacional'' a un archivo con extension .sql dando clic a un botón con la leyenda ``Descargar SQL''.
		\item Solo se obtendrán las sentencias SQL de un diagrama ER creado y/o validado por el sistema.
	\end{itemize}
\end{itemize}
%\noindent\rule{\textwidth}{1pt}
%\begin{itemize}
	%\item N.° 12
	%\item Como usuario, una vez observado el equivalente relacional del diagrama ER,
	%\item quiero iniciar el proceso de transformación al modelo no relacional
	%\item para poder observar el cambio entre modelos.
	%\item Criterios de aceptación:
	%\begin{itemize}
		%\item al dar clic al botón “Transformar a NoSQl”, el usuario iniciará el proceso para obtener el equivalente del modelo relacional al modelo NR.
		%\item al término del proceso de transformación, se le redirigirá al menú ``No Relacional'' donde observará el modelo NoSQL equivalente a su diagrama ER.
%	\end{itemize}
%\end{itemize}
\noindent\rule{\textwidth}{1pt}
\begin{itemize}
	\item N.° 12
	\item Como usuario quiero transformar mi diagrama ER en su equivalente modelo conceptual NoSQL,
	\item para poder observar el cambio entre modelos.
	\item Criterios de aceptación:
	\begin{itemize}
		\item El usuario dispondrá de la opción de tranformar al modelo NoSQL solamente después de haber validado que el diagrama ER cumple con las reglas.
		\item Posterior a la validación, se le mostrará al usuario un mensaje de confirmación y un botón para disparar el proceso de transformación al modelo conceptual NoSQL.
		\item Una vez validado el diagrama ER e iniciado el proceso para la transformación al modelo conceptual NoSQL, se le indicará al usuario que el proceso tardará un tiempo.
		\item Al término del proceso de transformación, se le redirigirá al menú ``No Relacional'' donde observará el modelo conceptual NoSQL equivalente a su diagrama ER.
	\end{itemize}
\end{itemize}
\noindent\rule{\textwidth}{1pt}
\begin{itemize}
	\item N.° 13
	\item Como usuario quiero obtener el \textit{script} desde el modelo conceptual NoSQL,
	\item para poder generar la base de datos en un gestor de base de datos NoSQL orientado a documentos.
	\item Criterios de aceptación:
	\begin{itemize}
		\item El usuario dispondrá de la opción de obtener el \textit{script} solamente después de haber validado que el diagrama ER cumple con las reglas.
		\item Posterior a la validación, se le mostrará al usuario un mensaje de confirmación y un botón para disparar el proceso de generación de \textit{scripts} para el gestor de base de datos orientado a documentos.
		\item Una vez empezado el proceso de generación de \textit{scripts}, se le indicará al usuario que el proceso tardará un tiempo.
		\item Al término del proceso de transformación, se le redirigirá al menú ``No Relacional'' donde observará los \textit{scripts} NoSQL.
	\end{itemize}
\end{itemize}

\noindent\rule{\textwidth}{1pt}
\begin{itemize}
	\item N.° 14
	\item Como usuario me gustaría tener un reporte técnico y
	\item quiero que la redacción sea legible y referenciada,
	\item para compartirlo en el futuro con equipos de desarrollo y ver la posibilidad de agregar nuevas funciones al sistema.
\end{itemize}
\noindent\rule{\textwidth}{1pt}




Teniendo en cuenta que se está trabajando con una metodología ágil, estas historias de usuario pueden aumentar o en su defecto dividirse en historias más pequeñas dependiendo de los criterios del equipo de desarrollo durante el proceso de la implementación de cada historia.




\subsection{Lista de producto}

De acuerdo a Trigas Gallego\cite{manuel_trigas_gallego_metodologiscrum_2020}, la lista de producto es una lista ordenada de todo lo que sería necesario en el producto y es la fuente de requisitos para cualquier cambio a realizarse en el mismo; enumera las características, funcionalidades, requisitos, mejoras y correcciones que constituyen cambios a realizarse en el producto para entregas futuras.

La tabla~\ref{tab:lista-producto} muestra la lista de producto para el proyecto; muestra el número de historia, la tarea a realizar, así como su encargado.


\begin{longtable}{ p{2cm} | p{10cm} | p{2cm} }
	\hline
	N.° de Historia de Usuario & Requerimiento/Tarea & Responsable \\[0.5cm]
	\hline
	\hline

	\endfirsthead

	\multicolumn{3}{c}{Continuación de tabla de lista de producto }\\
	\hline
	\hline
	\endhead

	\hline
	\hline
	\caption{Lista de producto}
	\endlastfoot

	% template row table
	% \centering & & & \\[0.5cm]
	% \hline

	\centering 14 & Investigación de bases de datos relacionales. & Eduardo/Omar \\[0.5cm]
	\hline
	\centering 14 & Redacción y selección de las tecnologías a utilizar para el desarrollo de la plataforma.  & Eduardo \\[0.5cm]
	\hline
	\centering 14 & Investigación de bases de datos relacionales.  & Eduardo/Omar \\[0.5cm]
	\hline
	\centering 14 & Redacción de bases de datos relacionales en el documento técnico.  & Eduardo \\[0.5cm]
	\hline
	\centering 14 & Investigación de bases de datos no relacionales.  & Eduardo/Omar \\[0.5cm]
	\hline
	\centering 14 & Redacción de bases de datos no relacionales en el documento técnico.  & Eduardo \\[0.5cm]
	\hline
	\centering 14 & Investigación y selección del modelo de base de datos no relacional a utilizar junto a las tecnologías a utilizar.  & Eduardo/Omar \\[0.5cm]
	\hline
	\centering 14 & Análisis y diseño de la aquitectura web.  & Eduardo/Omar \\[0.5cm]
	\hline
	\centering 1 & Desarrollo de la estructura básica del \textit{backend}.  & Omar \\[0.5cm]
	\hline
	\centering 1 & Desarrollo de la estructura básica del \textit{frontend}.  & Eduardo \\[0.5cm]
	\hline
	\centering 1 & Agregar servicio \textit{backend} para registrar un usuario. & Omar \\[0.5cm]
	\hline
	\centering 1 & Agregar formulario para captura de datos de registro de un usuario en el \textit{frontend}. & Eduardo \\[0.5cm]
	\hline
	\centering 2 & Agregar servicio \textit{backend} para recuperar contraseña del usuario. & Omar \\[0.5cm]
	\hline
	\centering 2 & Agregar servicio \textit{backend} para envío de correo al usuario registrado y de recuperación de contraseña. & Omar \\[0.5cm]
	\hline
	\centering 2 & Agregar vista con formulario para recuperación de contraseña del usuario en el \textit{frontend}. & Eduardo \\[0.5cm]
	\hline
	\centering 2 & Integración de los servicios de registro y recuperación de contraseña en el \textit{frontend}. & Eduardo \\[0.5cm]
	\hline
	\centering 3 & Agregar servicio \textit{backend} para hacer ilegible la contraseña del usuario en la base de datos. & Omar \\[0.5cm]
	\hline
	\centering 4 & Planteamiento de escenarios de los esquemas entidad-relación.  & Eduardo \\[0.5cm]
	\hline
	\centering 4 & Agregar a la interfaz gráfica de la aplicación web el menú ``Entidad-Relación''. & Eduardo \\[0.5cm]
	\hline
	\centering 4 & Agregar íconos de los elementos basicos de un diagrama ER en el diagramador. & Eduardo \\[0.5cm]
	\hline
	\centering 5 & Agregar servicio \textit{backend} para guardar un diagrama ER en formato JSON en la base de datos e integrarlo al \textit{frontend}. & Omar \\[0.5cm]
	\hline
	\centering 5 & Agregar servicio \textit{backend} para recuperar el diagrama guardado del usuario de la base de datos y regresarlo en formato JSON.  & Omar \\[0.5cm]
	\hline
	\centering 6 & Recuperar el último diagrama del usuario del \textit{backend} y monstrarlo en el \textit{frontend}. & Eduardo \\[0.5cm]
	\hline
	\centering 6 & Manejar el estado de la intefaz web para no perder el diagrama ER que está editando el usuario. & Eduardo \\[0.5cm]
	\hline
	\centering 6 & Definición de las reglas del modelo entidad-relación. & Eduardo \\[0.5cm]
	\hline
	\centering 4 & Implementar la edición de diagramas ER en el \textit{frontend}.  & Eduardo \\[0.5cm]
	\hline
	\centering 7 & Habilitar la carga de un archivo en la aplicación web.  & Omar \\[0.5cm]
	\hline
	\centering 7 & Agregar la función para validar el contenido del archivo .json y pintarlo en la zona de diagramado. & Eduardo \\[0.5cm]
	\hline
	\centering 8 & Agregar la descarga del diagrama visible en la zona de diagramado a un archivo .json. & Eduardo \\[0.5cm]
	\hline
	\centering 9 & Agregar botón de validar al \textit{frontend} y mostrar el \textit{loader} mientras se procesa el diagrama ER. & Eduardo/Omar \\[0.5cm]
	\hline
	\centering 9 & Agregar servicio \textit{backend} para la validación del diagrama entidad-relación. & Eduardo/Omar \\[0.5cm]
	\hline
	\centering 9 & implementación de algoritmo para validación del diagrama ER en el \textit{backend}. & Eduardo/Omar \\[0.5cm]
	\hline
	\centering 9 & Pruebas de captura de distintos diagramas entidad-relación.  & Eduardo/Omar \\[0.5cm]
	\hline
	\centering 9 & Pruebas para validar el algoritmo de validación. & Eduardo/Omar \\[0.5cm]
	\hline
	\centering 10 & Agregar servicio al \textit{backend} para transformación del esquema entidad-relación al modelo relacional.  & Omar \\[0.5cm]
	\hline
	\centering 10 & Implementación del algoritmo de transformación ER -> relacional & Eduardo/Omar \\[0.5cm]
	\hline
	\centering 10 & Agregar menú relacional a la intefaz gráfica. & Eduardo/Omar \\[0.5cm]
	\hline
	\centering 10 & Prueba de transformación de distintos diagramas ER al modelo relacional. & Eduardo/Omar \\[0.5cm]
	\hline
	\centering 10 & Visualización de la transformación del modelo ER al modelo relacional. & Eduardo \\[0.5cm]
	\hline
	\centering 14 & Revision de la redacción del reporte técnico para presentación de TT1  & Eduardo \\[0.5cm]
	\hline
	\centering 11 & Agregar servicio \textit{backend} para la descarga del archivo .sql con las sentencias equivalentes. & Eduardo/Omar \\[0.5cm]
	\hline
	\centering 11 & Pruebas de coherencia de las sentencias equivalentes en el DBMS. & Eduardo/Omar \\[0.5cm]
	\hline
	\centering 12 & Definición de las reglas de transformación al modelo NoSQL.  & Eduardo/Omar \\[0.5cm]
	\hline
	\centering 12 & Pruebas de distintos escenarios del modelo relacional al modelo NoSQL.  & Eduardo/Omar \\[0.5cm]
	\hline
	\centering 12 & Agregar servicio al \textit{backend} para transformación del esquema relacional al modelo NoSQL.  & Omar \\[0.5cm]
	\hline
	\centering 12 & Agregar servicio al \textit{backend} para guardar el modelo NoSQL en la base de datos.  & Omar \\[0.5cm]
	\hline
	\centering 12 & Agregar menú no relacional a la intefaz gráfica. & Omar \\[0.5cm]
	\hline
	\centering 12 & Implementación del algortimo de transformación de modelo relacional al modelo conceptual NoSQL. & Eduardo \\[0.5cm]
	\hline
	\centering 12 & Comprobación de la coherencia de la transformación entre modelos relacional a no relacional.  & Eduardo/Omar \\[0.5cm]
	\hline
	\centering 13 & Agregar servicio \textit{backend} para transformación del modelo ER al modelo no relacional. & Eduardo/Omar \\[0.5cm]
	\hline
	\centering 13 & Ajustar la interfaz del menú ER para mostrar mensaje de transformación al modelo NoSQL. & Omar \\[0.5cm]
	\hline
	\centering 13 & Manejar el estado del diagrama ER y redireccionar al menú no relacional al terminar la tranformación. & Eduardo \\[0.5cm]
	\hline
	\centering 13 & Pruebas de caso de estudio para verificar la correcta transformación y coherencia de los datos.  & Eduardo/Omar \\[0.5cm]
	\hline
	\centering 14 & Revisión de la redacción del reporte técnico para presentación de TT2 & Eduardo \\[0.5cm]

    \label{tab:lista-producto}
\end{longtable}

Se considera la tabla~\ref{tab:lista-producto} como la lista de producto con las tareas necesarias para cumplir con todas las historias de usuario mencionadas en la sección anterior, considerando que es posible que cambien conforme avancen los \textit{sprints} y así añadir nuevas tareas.
\section{Conclusiones}

La factibilidad operativa permite predecir si es posible poner en marcha el sistema propuesto, aprovechando todos los beneficios que se ofrecen a todos los usuarios involucrados en ello.


La herramienta va dirigida a estudiantes de nivel medio o nivel superior que tengan un primer acercamiento a los modelos de bases de datos entidad-relación o relacional desde un enfoque conceptual, como es el caso en ESCOM, en la asignatura de Base de Datos; el sistema propuesto cuenta con una interfaz intuitiva para que el usuario final, los estudiantes, visualice, crear y editar un diagrama ER y las opciones que esta les brinde de manera comprensible.


Por lo explicado anteriormente, el sistema propuesto tiene una alta probabilidad de aceptación por parte de los usuarios finales al encontrarse en un entorno en el que se trabaja con \textit{software} continuamente, además del beneficio que aporta al plan de estudios actual al ofrecer una forma práctica de ver aplicado los conceptos adquiridos en la asignatura de Bases de Datos, el cual solo contempla un alcance hasta la normalización de bases de datos relacionales y tener una introducción a los modelos no relacionales NoSQL.


Un estudiante que ha cursado dicha asignatura se dará cuenta que el tiempo disponible durante el curso es limitado por la cantidad de módulos que pretende cubrir y en muchas ocasiones los docentes deben prescindir de ciertos temas para completar el temario.


Con la implementación de la propuesta de solución, los estudiantes que cursen la asignatura de Bases de Datos tendrán la oportunidad de conocer una opción más en cuanto a tecnologías de almacenamiento de datos para implementar en sus propios sistemas; de igual manera, los puede impulsar a solicitar la apertura de una asignatura optativa sobre los modelos de datos NoSQL si hay interés por estos temas.


Se concluye que el sistema propuesto tendrá un uso en la institución y un potencial beneficio para los estudiantes y los involucrados.


A continuación están las respuestas (con preguntas incluidas) de la sección~\ref{ref:sec-factibilidad}:


\begin{enumerate}
    \item ¿El sistema contribuye a los objetivos generales de la organización?\\ Sí, ya que la misión en ESCOM es formar profesionales líderes en saberes de ingeniería, tecnología y ciencias de la computación con una visión globalizada; así como contribuir con investigación y desarrollo tecnológico para el crecimiento del país; por lo que la propuesta de solución contribuye directamente a la visión de la ESCOM.
    \item ¿Se puede implementar el sistema dentro del cronograma y el presupuesto utilizando las tecnologías actuales? \\Es posible, tal como se muestra en la sección~\ref{ref:factibilidad-tecnica}, se cuenta con las tecnologías para el desarrollo del producto y el esfuerzo es ajustado para el equipo de desarrollo, pero queda en los límites del tiempo establecido en el cronograma; además de ser desarrollado con una metodología ágil que ofrece productos funcionales por cada iteración.
    \item ¿Se puede integrar el sistema con otros sistemas que se utilizan?\\ Sí, el sistema es integrable a otros sistemas por estar disponible en la web; por ejemplo, puede integrarse directamente en la asignatura de Bases de Datos como una opción más para el modelado de diagramas ER con la ventaja de tener el acercamiento a los modelos no relacionales de manera práctica.
\end{enumerate}


Como se ha mencionado en el documento, se ha considerado Scrum como metodología para desarrollar la propuesta de solución, porque el proyecto requiere de entregas regulares de su avance para realizar modificaciones con ayuda de la retroalimentación constante del cliente, en este caso las directoras del proyecto.


Esto otorga beneficios como poder responder con flexibilidad, adaptación a los requisitos de cliente, estrechar la relación con el mismo y mantener al equipo motivado con pequeñas entregas funcionales del producto; además, tomando en cuenta la experiencia del equipo, esta forma de trabajo permite mostrar avances funcionales en el producto en un periodo de tiempo corto para realizar una evaluación y en caso de ser requerido se sugieran cambios.


De la sección de algoritmos, para la validez estructural de un diagrama entidad-relación se hace uso del trabajo de Dullea~\cite{dullea_analysis_2003} para el análisis de las relaciones unarias y binarias; además, se proponen restricciones para las entidades, atributos y relaciones del modelo entidad-relación básico.


Del mapeo modelo entidad-relación básico a modelo relacional se ha optado por usar la propuesta de Elmasri~\cite{ramez_elmasri_fundamentos_nodate}; para la obtención del esquema SQL se ha decidido parsear el modelo lógico relacional obtenido por el algoritmo anterior y se indica en la sección~\ref{sec:esquema-sql} las reglas básicas para implementar el algoritmo en el DBMS MySQL.


Para el algoritmo de mapeo entre el modelo entidad-relación y el GDM se ha optado por proponer consultas similares al trabajo de Chebotko\cite{chebotko_big_2015} para generar las consultas del GDM; asimismo, las relaciones en el modelo entidad-relación son referencias en el GDM y una consulta válida en el modelo entidad-relación es toda consulta que permita llegar a un atributo de una entidad $Y$ desde una entidad $X$ recorriendo las rutas del diagrama entidad-relación.


Para generar el modelo lógico orientado a documentos a partir del GDM se hace uso del algoritmo propuesto por Alfonso de la Vega, en el que se generan árboles de acceso de las consultas en el GDM para generar los documentos anidados en el modelo lógico orientado a documentos.


Finalmente, para obtener el esquema de sentencias en MongoDB se da una serie de pasos en la sección~\ref{sec:logico-documentos-fisico} para implementar el algoritmo.
