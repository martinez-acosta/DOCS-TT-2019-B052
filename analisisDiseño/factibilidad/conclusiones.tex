
%\subsection{Factibilidad operativa}
\subsection{Conclusiones}


La factibilidad operativa permite predecir de cierta forma si es posible poner en marcha el sistema propuesto, aprovechando todos los beneficios que se ofrecen a todos los usuarios involucrados en ello. La herramienta va dirigida a los estudiantes que se encuentren en un primer acercamiento a los modelos de bases de datos entidad-relación o relacional desde un enfoque conceptual, buscando principalmente mostrarles una aproximación a los modelos no relacionales de bases de datos. El sistema propuesto cuenta con una interfaz intuitiva para que el usuario final, los estudiantes, puedan visualizar, crear y editar un diagrama ER y las opciones que esta les brinde de manera comprensible.


Teniendo en cuenta los motivos anteriormente explicados, el sistema propuesto tiene una alta probabilidad de aceptación por parte de los usuarios finales al encontrarse en un entorno en el que se trabaja con \textit{software} continuamente, además del beneficio que aporta al plan de estudios actual al ofrecer una forma práctica de ver aplicado los conceptos adquiridos en el curso de base de datos, el cual solo contempla un alcance hasta la normalización de bases de datos relacionales y tener una introducción a los modelos no relacionales (noSQL). Un estudiante que ha cursado dicha asignatura se dará cuenta que el tiempo disponible durante el curso es limitado por la cantidad de módulos que pretende cubrir y en muchas ocasiones los docentes deben prescindir de ciertos temas para completar el temario.

Con la implantación de la aplicación web que se está proponiendo, los estudiantes que cursen la asignatura de de base de datos tendrán la oportunidad de conocer una opción más en cuanto a tecnologías de almacenamiento de datos para implementar en sus propios sistemas. De igual manera, puede impulsarlos a generar propuestas para la apertura de una asignatura optativa si el interes por estos modelos de datos resulta interesante para ellos.

Teniendo en cuenta los puntos mencionados anteriormente, se concluye que el sistema propuesto tendrá un uso en la institución y un potencial beneficio para los estudiantes y los involucrados en ello.

