
%\subsection{Factibilidad operativa}
\subsection{Conclusiones}


La factibilidad operativa permite predecir si es posible poner en marcha el sistema propuesto, aprovechando todos los beneficios que se ofrecen a todos los usuarios involucrados en ello.


La herramienta va dirigida a estudiantes de nivel medio o nivel superior que tengan un primer acercamiento a los modelos de bases de datos entidad-relación o relacional desde un enfoque conceptual, como es el caso en ESCOM, en la asignatura de Base de Datos; el sistema propuesto cuenta con una interfaz intuitiva para que el usuario final, los estudiantes, visualice, crear y editar un diagrama ER y las opciones que esta les brinde de manera comprensible.


Teniendo en cuenta los motivos anteriormente explicados, el sistema propuesto tiene una alta probabilidad de aceptación por parte de los usuarios finales al encontrarse en un entorno en el que se trabaja con \textit{software} continuamente, además del beneficio que aporta al plan de estudios actual al ofrecer una forma práctica de ver aplicado los conceptos adquiridos en la asignatura de Bases de Datos, el cual solo contempla un alcance hasta la normalización de bases de datos relacionales y tener una introducción a los modelos no relacionales NoSQL.


Un estudiante que ha cursado dicha asignatura se dará cuenta que el tiempo disponible durante el curso es limitado por la cantidad de módulos que pretende cubrir y en muchas ocasiones los docentes deben prescindir de ciertos temas para completar el temario.


Con la implementación de la propuesta de solución, los estudiantes que cursen la asignatura de Bases de Datos tendrán la oportunidad de conocer una opción más en cuanto a tecnologías de almacenamiento de datos para implementar en sus propios sistemas; de igual manera, los puede impulsar a solicitar la apertura de una asignatura optativa sobre los modelos de datos NoSQL si hay interés por estos temas.


Teniendo en cuenta los puntos mencionados anteriormente, se concluye que el sistema propuesto tendrá un uso en la institución y un potencial beneficio para los estudiantes y los involucrados.


Finalmente, a continuación están las respuestas (con preguntas incluidas) de la sección~\ref{ref:sec-factibilidad}:


\begin{enumerate}
    \item ¿El sistema contribuye a los objetivos generales de la organización?\\ Sí, ya que la misión en ESCOM es formar profesionales líderes en saberes de ingeniería, tecnología y ciencias, de la computación, con una visión globalizada; así como contribuir con investigación y desarrollo tecnológico para el crecimiento del país; por lo que la propuesta de solución contribuye directamente a la visión de la ESCOM.
    \item ¿Se puede implementar el sistema dentro del cronograma y el presupuesto utilizando las tecnologías actuales? \\Es posible, tal como se muestra en la sección~\ref{ref:factibilidad-tecnica}, se cuenta con las tecnologías para el desarrollo del producto y el esfuerzo es ajustado para el equipo de desarrollo, pero queda en los límites del tiempo establecido en el cronograma; además de ser desarrollado con una metodología ágil que ofrece productos funcionales por cada iteración.
    \item ¿Se puede integrar el sistema con otros sistemas que se utilizan?\\ Sí, el sistema es integrable a otros sistemas por estar disponible en la web; por ejemplo, puede integrarse directamente en la asignatura de Bases de Datos como una opción más para el modelado de diagramas ER con la ventaja de tener el acercamiento a los modelos no relacionales de manera práctica.
\end{enumerate}