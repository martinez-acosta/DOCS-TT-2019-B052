
%\subsection{Factibilidad operativa}
\subsection{Conclusiones}


La factibilidad operativa permite predecir si es posible poner en marcha el sistema propuesto, aprovechando todos los beneficios que se ofrecen a todos los usuarios involucrados en ello. La herramienta va dirigida a los estudiantes que se encuentren en un primer acercamiento a los modelos de bases de datos entidad-relación o relacional desde un enfoque conceptual, buscando principalmente mostrarles una aproximación a los modelos no relacionales de bases de datos. El sistema propuesto cuenta con una interfaz intuitiva para que el usuario final, los estudiantes, puedan visualizar, crear y editar un diagrama ER y las opciones que esta les brinde de manera comprensible.


Teniendo en cuenta los motivos anteriormente explicados, el sistema propuesto tiene una alta probabilidad de aceptación por parte de los usuarios finales al encontrarse en un entorno en el que se trabaja con \textit{software} continuamente, además del beneficio que aporta al plan de estudios actual al ofrecer una forma práctica de ver aplicado los conceptos adquiridos en el curso de base de datos, el cual solo contempla un alcance hasta la normalización de bases de datos relacionales y tener una introducción a los modelos no relacionales (noSQL). Un estudiante que ha cursado dicha asignatura se dará cuenta que el tiempo disponible durante el curso es limitado por la cantidad de módulos que pretende cubrir y en muchas ocasiones los docentes deben prescindir de ciertos temas para completar el temario.

Con la implantación de la aplicación web que se está proponiendo, los estudiantes que cursen la asignatura de de base de datos tendrán la oportunidad de conocer una opción más en cuanto a tecnologías de almacenamiento de datos para implementar en sus propios sistemas. De igual manera, puede impulsarlos a generar propuestas para la apertura de una asignatura optativa si el interes por estos modelos de datos resulta interesante para ellos.

Teniendo en cuenta los puntos mencionados anteriormente, se concluye que el sistema propuesto tendrá un uso en la institución y un potencial beneficio para los estudiantes y los involucrados en ello.

\begin{enumerate}
    \item ¿El sistema contribuye a los objetivos generales de la organización? la respuesta es si, ya que la organización centra su misión en formar profesionales en ingeniería, tecnologías y ciencias de la computación, para lograrlo de mantenerse a actualizado en las tecnologías emergentes y ofrecer a sus estudiantes un mayor abanico de opciones para su desarrollo profesional.
    \item ¿Se puede implementar el sistema dentro del cronograma y el presupuesto utilizando las tecnologías actuales? claro que es posible como se muestra en la factibilidad tecnica se cuenta con las tecnologías para el desarrollo del producto y el esfuerzo es ajjstado para el equipo de desarrollo pero queda en los limites del tiempo establecido en el cronograma, ademas de ser desarrollado con una metodología agíl lo que ofrece productos funcionales por cada iteración.
    \item ¿Se puede integrar el sistema con otros sistemas que se utilizan? esto es perfectamente viable, el sistema puede ser adaptado a otros sistemas por la facilidad de estar disponible en la web tiene una gran opoprtinidad para comunicarse con otros sistemas disponibles en la organización, por ejemplo puede integrarse directamente en la asignatura de base de datos como una opción mas para el modelado de diagramas ER con la ventaja de tener el acercamiento a los modelos no relacionales de manera practica.
\end{enumerate}