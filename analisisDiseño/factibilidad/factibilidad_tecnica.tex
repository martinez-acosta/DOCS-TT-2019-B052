\subsection{Factibilidad técnica}\label{ref:factibilidad-tecnica}

De acuerdo con Pressman\cite{pressman_software_2005}, este estudio determina si el equipo de desarrollo cuenta con los recursos técnicos necesarios para la realización del sistema propuesto; esto se realiza considerando la disponibilidad de los recursos tanto de \textit{hardware}, \textit{software} y recurso humano.

\subsubsection*{Sistema operativo}

De acuerdo con Stallings\cite{stallings_operating_2012}, un sistema operativo es el \textit{software} principal o conjunto de programas de un sistema informático que gestiona los recursos de \textit{hardware} y provee servicios a los programas de aplicación de \textit{software}, ejecutándose en modo privilegiado respecto de los restantes.


Este es un elemento importante, ya que debe cumplir con las características de estabilidad, velocidad, seguridad y escalabilidad para soportar la instalación del sistema.


A continuación se presentan diferentes sistemas operativos que cumplen con las características mencionadas y que son suficientes para albergar el sistema:

\begin{itemize}
    \item Windows: es un grupo de varias familias de sistemas operativos gráficos patentados, son desarrollados y comercializados por Microsoft; cada familia atiende a un determinado sector de la industria informática\cite{wilson_about_2015}.
    \item GNU/Linux: es una familia de sistemas operativos tipo Unix de código abierto basados en el núcleo de Linux creado por Linus Torvalds\cite{love_linux_2010}.
\end{itemize}

El sistema operativo elegido para el desarrollo de la propuesta de solución es GNU/Linux porque es de libre acceso, es gratuito y los integrantes del equipo tienen experencia con este sistema.


\subsubsection*{Lenguaje de desarrollo}

De acuerdo con Aaby\cite{aaby_introduction_1996}, un lenguaje de programación es un lenguaje formal (es decir, un lenguaje con reglas gramaticales definidas) que le proporciona a una persona, en este caso el programador, la capacidad de escribir una serie de instrucciones o secuencias de órdenes en forma de algoritmos con el fin de controlar el comportamiento físico o lógico de una computadora, de manera que es posible obtener diversas clases de datos o ejecutar determinadas tareas.


Se valora que el lenguaje de programación para el desarrollo de la propuesta de solución debe tener soporte para conexión a base de datos, sea posible usarlo para el desarrollo, sea vigente (que esté en continua mejora), sea fácil de administrar y se integre con algún \textit{framework} web.


A continuación se presenta una lista de lenguajes de desarrollo que cumplen dichas características:

\begin{itemize}
    \item Java: es un lenguaje de programación de propósito general que está basado en clases, orientado a objetos y diseñado para tener la menor cantidad posible de dependencias de implementación; su objetivo es permitir seguir el  \textit{write once, run anywhere}\cite{joy_java_2000}.
    \item Python: es un lenguaje de programación interpretado, de alto nivel y de propósito general; la filosofía de diseño de Python enfatiza la legibilidad del código con su uso notable de espacios en blanco significativos; Sus construcciones de lenguaje y su enfoque orientado a objetos tienen como objetivo ayudar a los programadores a escribir código claro y lógico para proyectos de pequeña y gran escala\cite{van_rossum_python_2007}.
    \item C\#: es un lenguaje de programación multiparadigma de propósito general como imperativo, declarativo, funcional, genérico, orientado a objetos y orientado a componentes\cite{hejlsberg_c_2003}.
    \item JavaScript: es un lenguaje de programación interpretado, de alto nivel y de propósito general; se escribe dinámicamente; es compatible con múltiples paradigmas de programación, incluida la programación orientada a objetos\cite{noauthor_javascript_nodate}.
\end{itemize}

Los lenguajes de programación Python y JavaScript se han elegido para desarrollar la propuesta de solución; para más detalles revise la sección~\ref{ref:conclusiones-cap3}

\subsubsection*{Sistema gestor de base de datos}

De acuerdo con Connolly\cite{connolly_database_2005}, es un sistema de \textit{software} que permite a los usuarios definir, crear, mantener y controlar el acceso a la base de datos.


Este es un factor muy importante, ya que determinará cómo se almacenará la información del sistema, por lo tanto debe ser escalable, seguro, contar con soporte para grandes cantidades de información y soporte para conexión con distintos lenguajes de programación.


A continuación se presenta una lista de sistemas gestores de bases de datos que cumplen dichas características:

\begin{itemize}
    \item MySQL: es un sistema de gestión de bases de datos relacionales de código abierto\cite{dubois_mysql_1999}.
    \item MongoDB: es un gestor de bases de datos orientado a documentos y utiliza documentos similares a JSON con esquemas opcionales\cite{banker_mongodb_2011}.
\end{itemize}

El gestor de base de datos que se ha elegido para desarrollar la propuesta de solución es MongoDB; para más detalles revise la sección~\ref{ref:databases}.


La tabla~\ref{tab:hw_devices} muestra las características de las computadoras con los que el equipo de desarrollo dispone; la primera columna hace referencia al número de computadora del que se dispone; la siguiente hace referencia a qué elementos contiene, la tercera columna sus especificaciones y la cuarta columna el costo de cada computadora.

\begin{table}[H]
    \centering
    \begin{tabular}{|c|c|c|c|}
        \hline
        Equipos & Elementos & Especificaciones & Costo \\ \hline
        \multirow{3}{*}{Laptop 1} & Memoria RAM & 8 GB & \\
        & Almacenamiento & 500 GB HDD & 22 500.00 mxn  \\
        & Procesador & Intel Core i5 6ta gen. & \\ \hline
        \multirow{3}{*}{Laptop 2} & Memoria RAM & 8 GB & \\
        & Almacenamiento & 256 GB SSD & 32 000.00 mxn \\
        & Procesador & Intel Core i5 8va gen. & \\ \hline
    \end{tabular}
    \caption{Computadoras con las que se cuenta}
    \label{tab:hw_devices}
\end{table}


Con los datos que están en la tabla~\ref{tab:hw_devices}, se concluye que la tecnología para el desarrollo del sistema existe y se cuenta con los recursos de \textit{hardware} suficientes para iniciar con su implementación.