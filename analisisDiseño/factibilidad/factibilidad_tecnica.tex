\subsection{Factibilidad técnica}

Mediante esta fase del estudio se determinará si el equipo de desarrollo cuenta con los recursos técnicos necesarios para la realización del sistema propuesto. Esto se realiza considerando la disponibilidad de los recursos tanto de \textit{hardware}, \textit{software} y recurso humano.

\subsubsection{Sistema operativo}

Este es un elemento importante, ya que debe cumplir con las características de estabilidad, velocidad, seguridad y escalabilidad para soportar la instalación del sistema.


A continuación se presentan diferentes alternativas de sistemas operativos que cumplen con las características mencionadas y que son suficientes para albergar el sistema:

\begin{itemize}
    \item Windows server 2019
    \item Red Hat Entrerprise 8
    \item Ubuntu server 19.04
\end{itemize}

\subsubsection{Lenguaje de desarrollo}

El lenguaje de desarrollo debe de cumplir con las siguientes características:

\begin{itemize}
    \item Soporte para conexión a base de datos.
    \item Facilidad para el desarrollo.
    \item En continua mejora.
    \item Fácil de administrar.
    \item Contar con algún \textit{framework} web.
\end{itemize}

A continuación se presenta una lista de lenguajes de desarrollo que cumplen dichas características:

\begin{itemize}
    \item Java
    \item Python
    \item C\#
    \item Ruby
\end{itemize}

\subsubsection{Sistema gestor de base de datos}

Este es un factor muy importante, ya que determinará cómo se almacenará la información del sistema, por lo tanto debe cumplir con las siguientes características:

\begin{itemize}
    \item Escalable.
    \item Seguro.
    \item Contar con soporte para grandes cantidades de información.
    \item Contar con soporte para conexión con distintos lenguajes de programación.
\end{itemize}


A continuación se presenta una lista de sistemas gestores de bases de datos que cumplen dichas características:

\begin{itemize}
    \item MySQL
    \item MariaDB
    \item Oracle database
    \item MongoDB Atlas
    \item DynamoDB
    \item Apache Cassandra
\end{itemize}

Las características de los equipos de cómputo con los que se dispone actualmente para el desarrollo del sistema se muestran en la tabla \ref{tab:hw_devices}.

\begin{table}
    \begin{tabular}{|c|c|c|}
        \hline
        Equipo & Elemento & Capacidad \\ \hline
        \multirow{3}{*}{Laptop 1} & Memoria RAM & 8 GB \\
        & Almacenamiento & 500 GB HDD \\
        & Procesador & Intel Core i5 6ta gen. \\ \hline
        \multirow{3}{*}{Laptop 2} & Memoria RAM & 8 GB \\
        & Almacenamiento & 256 GB SSD \\
        & Procesador & Intel Core i5 8va gen.\\ \hline
    \end{tabular}
    \caption{Equipo de cómputo}
    \label{tab:hw_devices}
\end{table}


Con los datos anteriormente mencionados, se concluye que la tecnología para el desarrollo del sistema existe y se cuenta con los recursos de \textit{hardware} suficientes para iniciar con su implementación.