\subsection{Costos de desarrollo}

De acuerdo con \textit{Software Guru}\cite{pedro_galvan_estudio_2020}, en una publicación que recopila los datos de salarios en el área de desarrollo de \textit{software} para febrero de 2020, un desarrollador con 0 a 2 años de experiencia, como es el caso de un estudiante de nivel superior, tiene en un salario de 15 000 mxn mensuales en una jornada completa.


La tabla~\ref{tab:devs_salary} muestra un desglose del costo total de salarios del personal del proyecto, empezando por el costo semanal aproximado por integrante, su costo mensual aproximado y el monto total.

\begin{table}[H]
    \centering
    \begin{tabular}{|c|c|c|c|}
    \hline
        Concepto & \begin{tabular}[c]{@{}l@{}}Costo semanal\\ aproximado\end{tabular} & \begin{tabular}[c]{@{}l@{}}Costo mensual\\ aproximado\end{tabular}& Monto total \\ \hline
        Salario & 1850 mxn & 7500 mxn & 67 500 mxn\\ \hline
    \end{tabular}
    \caption{Costos del personal}
    \label{tab:devs_salary}
\end{table}

Teniendo en cuenta que este proyecto contempla jornadas de medio tiempo (4 horas) de lunes a viernes por un periodo de 9 meses, que es el tiempo aproximado de duración del proyecto, el costo total por salarios para el equipo de desarrollo está desglosado en la tabla~\ref{tab:devs_salary}.


Considerado a 2 personas en el equipo de desarrollo y un periodo de 9 meses (se agregó un mes más para el caso de estudio) y utilizando los salarios de la tabla~\ref{tab:devs_salary}, tenemos que los gastos totales se obtienen con la siguiente fórmula:

\begin{equation} \label{eq:cap4-07}
	\mathrm{salarios} =  \mathrm{No.\; de\; integrantes} * \mathrm{salario/mes} * \mathrm{tiempo\; de\; desarrollo}  \mathrm{\;mxn}
\end{equation}

Sustituyendo los valores en \eqref{eq:cap4-07}

\begin{equation} \label{eq:cap4-08}
	\mathrm{salarios} =  2 * 7500 * 9  \mathrm{\;mxn}
\end{equation}


Esto da como resultado un total final de 135 000 mxn por los salarios de los 2 integrantes del equipo de desarrollo; se tomaron en cuenta 9 meses para todos los gastos, un mes extra a lo obtenido en estimación para utilizarse en el caso de estudio del sistema una vez concluido.


La tabla~\ref{tab:sw_licences} muestra los costos del \textit{software} usado en el proyecto y la tabla~\ref{tab:services_costs} los costos de servicios usados para el mismo. 

\begin{table}[H]
    \centering
    \begin{tabular}{|c|c|c|c|}
    \hline
        Software & Licencia & Costo \\ \hline
        Visual Studio Code  & MIT & 0  \\ \hline
        Gunicorn(flask Server) & MIT & 0 \\ \hline
        MongoDB Atlas & Apache v2 & 0 \\ \hline
    \end{tabular}
    \caption{Costos por licencias de software}
    \label{tab:sw_licences}
\end{table}


\begin{table}[H]
    \centering
    \begin{tabular}{|c|c|c|c|}
    \hline
        Concepto & Costo Mensual & Monto total \\ \hline
        Luz & 250 & 2250  \\ \hline
        Internet & 349 & 3141 \\ \hline
        Heroku hosting & 0 (free plan) & 0 \\ \hline
        Netlify hosting & 0 (free plan) & 0 \\ \hline
    \end{tabular}
    \caption{Costos por servicios}
    \label{tab:services_costs}
\end{table}


Los gastos por pagos de licencia de \textit{software} quedan excluidos, ya que las tecnologías seleccionadas son libres o gratuitas, lo cual no supone un costo para su uso; de igual manera, esto se encuentra simplificado en la tabla \ref{tab:sw_licences}; otros gastos necesarios son los pagos por servicios requeridos están listados en la tabla \ref{tab:services_costs}.

Habiendo realizado la suma de todas las cantidades antes mencionadas, el total final se obtine de la siguiente manera:

\begin{equation} \label{eq:cap4-09.00}
    \mathrm{servicio\; totales} = \mathrm{servicios\; por\; mes} * \mathrm{tiempo\; de\; desarrollo} \mathrm{\;mxn}
\end{equation}

\begin{equation} \label{eq:cap4-09.01}
    \mathrm{servicio\; totales} = 5391 * 9 \mathrm{\;mxn}
\end{equation}

\begin{equation} \label{eq:cap4-09.02}
    \mathrm{servicio\; totales} = 48 519.00 \mathrm{\;mxn}
\end{equation}

\begin{equation} \label{eq:cap4-10.00}
    \mathrm{Gastos\; totales} = \mathrm{salarios} + \mathrm{servicios\; totales} + \mathrm{\;costos\;de\;equipos}
\end{equation}

\begin{equation} \label{eq:cap4-10.01}
    \mathrm{Gastos\; totales} = 135 000 + 48 519 + 54 500 \mathrm{\;mxn}
\end{equation}

\begin{equation} \label{eq:cap4-10.02}
    \mathrm{Gastos\; totales} = 238 019 \mathrm{\;mxn}
\end{equation}

\begin{equation} \label{eq:cap4-11.00}
    \mathrm{salarios} = 135 000.00 \mathrm{\;mxn}
\end{equation}

\begin{equation} \label{eq:cap4-11.01}
    \mathrm{servicios} = (2250 + 3141) * 9 = 48 519.00
\end{equation}

\begin{equation} \label{eq:cap4-11.02}
    \mathrm{Gastos\; totales} = 135 000.00 + 48 519.00 = 183 519.00 \mathrm{\;mxn}
\end{equation}
