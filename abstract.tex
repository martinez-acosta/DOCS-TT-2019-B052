\begin{abstract}
En el presente trabajo se propone una aplicación web que brinde la funcionalidad de una herramienta CASE de acuerdo con reglas de validación establecidas para el modelo entidad-relación, permita el mapeo del modelo entidad-relación básico con consultas al modelo relacional o a un modelo conceptual NoSQL y permita generar sus esquemas de bases de datos, respectivamente; se da una introducción a los sistemas NoSQL, la problemática que pretende resolver este trabajo con la propuesta de solución; se justifica la aplicación en un alcance para un estudiante de medio superior o superior; se exponen los objetivos y limitaciones del trabajo a desarrollar; se compara la propuesta de solución con herramientas similares; se muestran los diferentes conceptos con los que el lector debe estar relacionado para comprender la solución que se propone; se hace un análisis de la metodología de diseño a usar; se estudia si es o no factible la propuesta de solución con los recursos disponibles por el equipo de desarrollo; se exponen los principales algoritmos a implementar; se muestran los diagramas usados para diseñar el sistema; se presenta la arquitectura a seguir en la propuesta de solución; finalmente, en el apéndice está un breve resumen sobre la notación UML en diagramas de clases por si el lector no está familiarizado con esta notación gráfica y una breve descripción del prototipo funcional de la propuesta de solución. 


\keywords{IPN, ESCOM, modelo conceptual, NoSQL, entidad-relación, modelo relacional, validación estructural, GDM, Generic Data Metamodel, orientado a documentos, bases de datos, Scrum, modelo lógico, modelo físico, SQL, MongoDB, Nuxt, Python.}
\end{abstract}

