\section{Modelo entidad-relación a Generic Data Metamodel}

%\begin{enumerate}
%    \item Por cada entidad se crea una clase entidad correspondiente en el GDM. 
%   \item Por cada atributo se crea un clase atributo correspondiente y se indica su tipo.
%   \item Por cada relación se crea una referencia correspondiente.
%    \item Con cada referencia se crea una \textit{query} básica.
%\end{enumerate}

El algoritmo necesita ser basado en querys como el de Chebotko.


input: modelo entidad-relacion


output: ¿modelo orientado a documentos o GDM?


¿Qué definición del modelo orientado a documentos se usará? ¿El del GDM o el de De Lima?


¿Cómo convertir las relaciones?
Hay reglas para convertir las relaciones en bloques del documento orientado a objetos.

