\section{Conclusiones}

Se consideran las siguientes puntos como trabajo a futuro:

\begin{itemize}
  \item La mutación de los elementos del diagrama entidad-relación básico con la ayuda de un menú contextual.
  \item La inclusión de atributos parciales para ejemplos de una complejidad mayor en los diagramas entidad-relación básicos.
  \item El manejo de atributos compuestos para ampliar las posibilidades para diagramas que acepte la aplicación.
  \item Robustecer la base de datos de la aplicación.
  \begin{itemize}
    \item Almacenar más de un diagrama entidad-relación básico por usuario.
    \item Guardar las consultas de acceso para cada diagrama entidad-relación básico.
  \end{itemize}
  \item Una forma más amigable para que el usuario cree consultas de acceso para los modelos NoSQL.
  \item Considerar otros modelos lógicos NoSQL como el orientado a columnas.
\end{itemize}

\section{Trabajo a futuro}
