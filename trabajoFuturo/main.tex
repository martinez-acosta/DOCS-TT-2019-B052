\section{Conclusiones}


El diagramado del modelo entidad-relación básico y la validación estructural se realizaron de acuerdo al trabajo de Dullea~\cite{dullea_analysis_2003}; lo más complicado de esta parte fue aprender a describir un modelo entidad-relación con la biblioteca para diagramado de GoJs~\cite{noauthor_gojs_nodate-1}.


Respecto a la transformación del modelo entidad-relación básico al modelo relacional para la generación del esquema de sentencias SQL bastó recurrir al algoritmo de 7 pasos de Elmasri~\cite{ramez_elmasri_fundamentos_nodate}.


Ahora bien, la dificultad de la transformación NoSQL desde un modelo entidad-relación básico recae no solo en el algoritmo de transformación, sino que también de desarrollar una forma de realizar consultas al modelo conceptual NoSQL o al modelo conceptual entidad-relación, porque de acuerdo con Mosquera~\cite{martinez-mosquera_modeling_2020}, entre otros autores, es necesario conocer cómo se realizarán las consultas de los datos.



No obstante, gracias a los algoritmos recientemente desarrollados y probados por Alfonso de la Vega y su equipo en Mortadelo~\cite{de_la_vega_mortadelo_2020}, se ha logrado implementar la transformación entre modelos definida en la propuesta de solución.


Debido a que el desarrollo de Mortadelo está muy enfocado a una ingeniería orientado a modelos en Java, se complicó implementar el Generic Data Metamodel en las tecnologías a usar seleccionadas, que son Python y JavaScript. Sin embargo, por la utilidad PyEcore~\cite{pyecore_pyecore_2020} se pudieron generar todas las clases correspondientes del modelo conceptual y del modelo lógico del GDM para que se implementaran los algoritmos correspondientes en las transformaciones entre modelos en Python, desarrollando los \textit{parsers} faltantes que no existen para las trasnformaciones entre modelos en dicho lenguaje.


Para la generación de sentencias en MongoDB se realizó generando su sentencia correspondiente en MongoDB de cada elemento del modelo orientado a documentos.


Finalmente, el usar los algoritmos de transformación del Generic Data Metamodel ha permitido al equipo tener la certeza de que se generan bases de datos orientados a documentos funcionales y probadas.




\section{Trabajo a futuro}

Se consideran las siguientes puntos como trabajo a futuro:

\begin{itemize}
  \item La mutación de los elementos del diagrama entidad-relación básico con la ayuda de un menú contextual.
  \item La inclusión de atributos parciales para ejemplos de una complejidad mayor en los diagramas entidad-relación básicos.
  \item El manejo de atributos compuestos para ampliar las posibilidades para diagramas que acepte la aplicación.
  \item Robustecer la base de datos de la aplicación.
  \begin{itemize}
    \item Almacenar más de un diagrama entidad-relación básico por usuario.
    \item Guardar las consultas de acceso para cada diagrama entidad-relación básico.
  \end{itemize}
  \item Una forma más amigable para que el usuario cree consultas de acceso para los modelos NoSQL.
  \item Considerar otros modelos lógicos NoSQL como el orientado a columnas.
\end{itemize}