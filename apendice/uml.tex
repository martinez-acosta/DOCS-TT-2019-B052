\section{Unified Modeling Language}

De acuerdo con Fowler\cite{fowler_brief_2003}, UML (\textit{Unified Modeling Language}) es una familia de notaciones gráficas, respaldada por un metamodelo único, que ayuda a describir y diseñar sistemas de \textit{software}, particularmente sistemas de \textit{software} descritos utilizando el estilo orientado a objetos.

\subsection{Diagramas de clases}

De acuerdo con Fowler\cite{fowler_brief_2003}, un diagrama de clases describe los tipos de objetos en el sistema y los diversos tipos de relaciones estáticas que existen entre ellos. 


Los diagramas de clase también muestran las propiedades y operaciones de una clase y las restricciones que se aplican a la forma en que se conectan los objetos.


UML usa el término característica como un término general que cubre las propiedades y operaciones de una clase.


\subsubsection*{Miembros}
UML proporciona mecanismos para representar los miembros de la clase, como atributos y métodos, así como información adicional sobre ellos.


\paragraph*{Visibilidad}
Para especificar la visibilidad de un miembro de la clase (es decir, cualquier atributo o método), se coloca uno de los siguientes signos delante de ese miembro:

\begin{enumerate}
    \item +	Público
    \item -	Privado
    \item \# Protegido
    \item /	Derivado (se puede combinar con otro)
    \item \~ \;	Paquete
\end{enumerate}

\paragraph*{Ámbitos}

UML especifica dos tipos de ámbitos para los miembros: instancias y clasificadores y estos últimos se representan con nombres subrayados.

\begin{enumerate}
    \item Los miembros clasificadores se denotan comúnmente como “estáticos” en muchos lenguajes de programación; su ámbito es la propia clase.
    \begin{enumerate}
        \item Los valores de los atributos son los mismos en todas las instancias.
        \item La invocación de métodos no afecta al estado de las instancias.
    \end{enumerate}
    \item Los miembros instancias tienen como ámbito una instancia específica.
\begin{enumerate}
    \item Los valores de los atributos pueden variar entre instancias.
    \item La invocación de métodos afecta al estado de las instancias (es decir, cambiar el valor de sus atributos).
\end{enumerate}
\end{enumerate}
Para indicar que un miembro posee un ámbito de clasificador, hay que subrayar su nombre; de lo contrario, se asume por defecto que tendrá ámbito de instancia.



La figura~\ref{img:uml-asociacion} es una asociación entre dos clases, la clase \textit{Person} y la clase \textit{Magazine}.

La figura~\ref{img:uml-agregacion} es una agregación entre dos clases, la clase \textit{Professor} y la clase \textit{Class} donde la cardinalidad es una a muchos.

La figura~\ref{img:uml-composicion} muestra una  asociación y composición entre clases, la clase Almacen está asociada a la clase Cuentas y con la clase Cliente hay una relación de composición.


\paragraph*{Relaciones}
Una relación es un término general que abarca los tipos específicos de conexiones lógicas que se encuentran en los diagramas de clases y objetos. UML presenta las siguientes relaciones:

\paragraph*{Enlace}
Un enlace es la relación más básica entre objetos.

\paragraph*{Asociación}



\begin{figure}[H] 
    \centering
    \includegraphics[width=0.65\textwidth]{uml/01.png}
    \caption{Asociación}
    \label{img:uml-asociacion}
\end{figure}

Una asociación como la de la figura~\ref{img:uml-asociacion} representa una familia de enlaces; una asociación binaria (entre dos clases) normalmente se representa con una línea continua; una misma asociación relaciona cualquier número de clases y una asociación que relacione tres clases se llama asociación ternaria.

A una asociación se le asigna un nombre y en sus extremos se hacen indicaciones, como el rol que desempeña la asociación, los nombres de las clases relacionadas, su multiplicidad, su visibilidad, y otras propiedades.

Hay cuatro tipos diferentes de asociación: bidireccional, unidireccional, agregación (en la que se incluye la composición) y reflexiva; las asociaciones unidireccional y bidireccional son las más comunes.


\subsubsection*{Agregación}


\begin{figure}[H] 
    \centering
    \includegraphics[width=0.65\textwidth]{uml/02.png}
    \caption{Agregación}
    \label{img:uml-agregacion}
\end{figure}


La agregación o agrupación como la de la figura~\ref{img:uml-agregacion} es una variante de la relación de asociación ``tiene un'': la agregación es más específica que la asociación; se trata de una asociación que representa una relación de tipo parte-todo o parte-de.


Al ser un tipo de asociación, una agregación es posible que tenga un nombre y las mismas indicaciones en los extremos de la línea; sin embargo, una agregación no debe incluir más de dos clases, debe ser una asociación binaria.


Una agregación es posible que se dé cuando una clase es una colección o un contenedor de otras clases, pero a su vez, el tiempo de vida de las clases contenidas no tienen una dependencia fuerte del tiempo de vida de la clase contenedora (de el todo). 


En UML, como está en la figura~\ref{img:uml-agregacion} se representa gráficamente con un rombo hueco junto a la clase contenedora con una línea que lo conecta a la clase contenida; todo este conjunto es, semánticamente, un objeto extendido que es tratado como una única unidad en muchas operaciones, aunque físicamente está hecho de varios objetos más pequeños.

\begin{figure}[H] 
    \centering
    \includegraphics[width=0.65\textwidth]{uml/03.png}
    \caption{Asociación rombo sin rellenar y composición rombo negro}
    \label{img:uml-composicion}
\end{figure}

\subsubsection*{Composición}

La representación en UML de una relación de composición es mostrada en la figura~\ref{img:uml-composicion} como un diamante rellenado del lado del la clase contenedora, es decir, al final de la línea que conecta la clase contenido con la clase contenedora.


\paragraph*{Diferencias entre Composición y Agregación}

\subparagraph*{Relación de Composición}

\begin{enumerate}
    \item Cuando se intenta representar un todo y sus partes.
    \item Cuando se elimina el contenedor, el contenido también es eliminado.
\end{enumerate}


\subparagraph*{Relación de Agrupación}

\begin{enumerate}
    \item Cuando se representan las relaciones en un \textit{software} o base de datos. 
    \item Cuando el contenedor es eliminado, el contenido usualmente no es destruido.
\end{enumerate}