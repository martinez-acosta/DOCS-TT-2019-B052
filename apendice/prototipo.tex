\section{Prototipo funcional}

El prototipo funcional de la propuesta de solución usa para la autenticación de usuarios \textit{web tokens}; se dispone del código fuente y un \textit{live demo} de la aplicación.


El \textit{front end} (Nuxt/JS/GoJS) lo puede consultar en \href{https://github.com/martinez-acosta/TT-2019-B052}{repositorio front}, el \textit{back end} (Python/Flask) en \href{https://github.com/omaraparicio07/api-tt-2019-b052}{repositorio back}, el documento de TT (Latex) en \href{https://github.com/martinez-acosta/DOCS-TT-2019-B052}{repositorio documento} y el \textit{live demo} está disponible en \href{https://serene-haibt-2239b4.netlify.app/}{dirección}.


El \textit{live demo} del \textit{front end} se está usando el \textit{deploy} automático en Netlify de la rama master y el código de desarrollo está en dev.


Asimismo, en la página de GitHub del proyecto de \textit{front end} es posible encontrar de manera breve un resumen del propósito de la aplicación web, estado actual del proyecto, tecnologías usadas y cosas que faltan por implementar.


El \textit{live demo} del \textit{back end} está usando el \textit{deploy} automático en Heroku de la rama master y el código de desarrollo está en las ramas \textit{feature}.


Para probar la aplicación, es necesario darse de alta en la aplicación, pero el equipo se ha tomado la molestia de crear credenciales temporales para los sinodales, directores y el maestro de seguimiento (las cuentas de usuario y contraseñas se hicieron llegar a cada uno de los involucrados); para los correos temporales de las cuentas se ha usado el servicio de: \href{https://temp-mail.org/es/}{temp-mail}.

\subsection*{Front end}

De acuerdo con el sitio de Netlify~\cite{netlify_netlify_nodate}, Netlify ofrece funcionalidad HTTPS para todos sus sitios, tiene una mitigación activa en contra de ataques DDoS, todo su tráfico está cifrado en redes TLS y los \textit{tokens} están cifrados.

De acuerdo con el sitio de JWT~\cite{jwt_web_2020}, un JSON Web Token (JWT) es un estándar abierto (RFC 7519) que define una forma compacta y autónoma para transmitir información de forma segura entre las partes como un objeto JSON.


Esta información se puede verificar y confiar porque está firmada digitalmente; los JWT se firman usando un clave denominada ``secreto'' (con el algoritmo HMAC) o un par de claves pública/privada usando RSA o ECDSA.

Aunque los JWT se pueden cifrar para proporcionar también secreto entre las partes, los \textit{jwt web tokens} se enfocan en \textit{tokens} firmados; los \textit{tokens} firmados pueden verificar la integridad de los reclamos que contiene, mientras que los \textit{tokens} cifrados ocultan esos reclamos de otras partes; cuando los \textit{tokens} se firman utilizando pares de claves públicas / privadas, la firma también certifica que solo la parte que posee la clave privada es la que la firmó.


La figura~\ref{img:prototipo-welcome} muestra la pantalla de bienvenida de la aplicación, donde el mensaje de iniciar sesión o registarse es visible si el usuario no ha iniciado sesión.

\begin{figure}[H]
    \centering
    \includegraphics[width=0.75\textwidth]{prototipo/welcome.png}
    \caption{Pantalla de bienvenida}
    \label{img:prototipo-welcome}
\end{figure}

La figura~\ref{img:prototipo-login} muestra la pantalla de inicio de sesión de la aplicación, donde se pide el correo y la contraseña para iniciar sesión.


\begin{figure}[H]
    \centering
    \includegraphics[width=0.75\textwidth]{prototipo/login.png}
    \caption{Pantalla de login}
    \label{img:prototipo-login}
\end{figure}
La figura~\ref{img:prototipo-signup} muestra el formulario de registro de la aplicación; para la validación de los campos se ha usado Vuelidate y muestra mensajes de error en caso de que algún campo esté rellenado incorrectamente.


\begin{figure}[H]
    \centering
    \includegraphics[width=0.75\textwidth]{prototipo/signup.png}
    \caption{Pantalla de alta de usuario}
    \label{img:prototipo-signup}
\end{figure}
La figura~\ref{img:prototipo-er} muestra la vista del diagramador entidad-relación básico, del lado izquierdo están los botones para guardar y cargar un diagrama en formato .json válido para la aplicación; en la segunda columna está la zona de diagramado para crear un diagrama entidad-relación y del lado derecho está la paleta de elementos donde están los elementos de un diagrama entidad-relación básico; en la parte de abajo está el JSON equivalente del diagrama que está en la zona de diagramado y a su derecha está un menú para modificar algunas propiedades de los elementos del diagramador.

\begin{figure}[H]
    \centering
    \includegraphics[width=0.50\textwidth]{prototipo/er.png}
    \caption{Pantalla del diagramador entidad-relación}
    \label{img:prototipo-er}
\end{figure}



Las figuras~\ref{img:prototipo-welcome},~\ref{img:prototipo-login},~\ref{img:prototipo-signup} y~\ref{img:prototipo-er} son capturas de pantalla del prototipo de la propuesta de solución, como es de notar el prototipo permite editar, guardar y cargar un diagrama entidad-relación básico.


\subsection*{Back end}


De acuerdo con el sitio de Heroku~\cite{heroku_heroku_2020}, la plataforma ofrece distintos mecanismos para la seguridad de los proyectos alojadas en ella, dentro de estas se encuentra el uso por defecto del protocolo HTTPS, que asegura el cifrado de los datos en la Internet.


De igual manera ofrece un constante escaneo de las aplicaciones en búsqueda de vulnerabilidades para mitigar los ataques DDoS, además de utilizar la infraestructura de la empresa Amazon, las cuales se encuentran acreditadas por diversos estándares de seguridad; esto da la confianza para que la aplicación Flask se ejecute de manera segura y tener salvaguardados los datos de los usuarios.


Los métodos o funciones implementadas en la aplicación flask constan de un CRUD (\textit{Create, Read, Update, Delete}) para el manejo de usuarios, métodos para el manejo de la sesión del usuario, así como la recuperación de contraseña entre otras que se describen a continuación:

\begin{itemize}
    \item create\_user: método para crear una nueva cuenta de usuario, recibe como parámetros en el cuerpo de la petición HTTP un \textit{email} y una contraseña.
    \item validate\_email: método que recibe como parámetro una cadena y la valida con una expresión regular para un correo electrónico.
    \item send\_email: método que recibe como parámetros una dirección de correo electrónico, contraseña, nombre de usuario y un valor booleano; envía un correo de bienvenida o recuperación de contraseña; el valor booleano sirve para identificar si se trata de un registro de usuario o una recuperación de contraseña.
    \item update\_user: método que recibe como parámetro un objeto de tipo \textit{user} con los valores que se desean asignar a un usuario ya existente.
    \item delete\_user: método que recibe como parámetro una cadena que corresponde al \textit{email} de un usuario existente y elimina todos los registros asociados a ese usuario.
    \item get\_user: método que recibe como parámetro una cadena que corresponde al \textit{email} de un usuario existente y devuelve un objeto de tipo \textit{user} con todos sus datos.
    \item get\_users: método que no recibe parámetros y que regresa un listado de todos los usuarios; para utilizar este método debe proporcionarse en el cuerpo de la petición HTTP un \textit{token} de autorización válido.
    \item token\_required: función privada para validar que en los encabezados de la petición HTTP se encuentre un atributo \textit{AUTHORIZATION} que es del tipo clave valor y contenga como valor un \textit{token} válido para el sistema; esta función sirve para decorar/anotar otras funciones y no tener que repetir esa validación para cada petición que requiera autorización de usuario.
    \item login: método que recibe como parámetro en el cuerpo de la petición HTTP un cadena con el \textit{email} y otra con la contraseña, realiza una búsqueda en la base de datos y valida si los datos corresponden a un usuario, en caso de ser encontrarlo, regresa un JWT para manejar la sesión del usuario en las futuras peticiones.
    \item save\_diagram: método que recibe parámetros en el cuerpo de la petición HTTP un objeto de tipo JSON con el contenido del diagrama ER y lo almacena en la base de datos.
    \item last\_diagram: método que no recibe parámetros y devuelve el diagrama ER de asociado al usuario en sesión.
\end{itemize}


De acuerdo con el sitio de MongoDB\cite{mongodb_mongodb_2020}, MongoDB es una plataforma de base de datos alojada en la nube que garantiza la disponibilidad, escalabilidad y cumplimiento de estándares de seguridad; por estos motivos que el almacenamiento de los datos está a cargo de este servicio, ofreciendo la conectividad con la aplicación Flask, ya que Heroku permite la conexión con los servicios de MongoDB Atlas; la configuración para la conexión no cambia si se están realizando pruebas en un ambiente local antes de exponer los últimos cambios al público.