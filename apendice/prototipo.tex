\section{Prototipo funcional}

Para probar el prototipo funcional de la propuesta de solución se puede consultar el código fuente y un \textit{live demo} de la aplicación.


El \textit{front end} (Nuxt/JS/GoJS) lo puede consultar en \href{https://github.com/martinez-acosta/TT-2019-B052}{repo front}, el \textit{back end} (Python/Flask) en \href{https://github.com/omaraparicio07/api-tt-2019-b052}{repo back}, el documento de TT (Latex) en \href{https://github.com/martinez-acosta/DOCS-TT-2019-B052}{repo documento} y el \textit{live demo} está disponible en \href{https://serene-haibt-2239b4.netlify.app/}{dirección}.


El \textit{live demo} del \textit{front end} se está usando el \textit{deploy} automático en Netlify de la rama master y el código de desarrollo está en dev. 


Asimismo, en la página de GitHub del proyecto de \textit{front end} es posible encontrar de manera breve un resumen del propósito de la aplicación web, estado actual del proyecto, tecnologías usadas y cosas que faltan por implementar.


El \textit{live demo} del \textit{back end} se está usando el \textit{deploy} automático en Heroku de la rama master y el código de desarrollo está en las ramas \textit{feature}.


Para probar la aplicación, es necesario darse de alta en la aplicación, pero el equipo se ha tomado la molestia de crear credenciales temporales para los sinodales, directores y el maestro de seguimiento y para los correos temporales de las cuentas se ha usado el servicio de: \href{https://temp-mail.org/es/}{temp-mail}.


De acuerdo con el sitio de Netlify~\cite{netlify_netlify_nodate}, ofrece funcional HTTPS para todos sus sitios, tiene una mitigación activa en contra de ataques DDoS, todo su tráfico está cifrado en redes TLS y \textit{tokens} están cifrados.

De acuerdo con el sitio de Heroku~\cite{heroku_heroku_2020}, 